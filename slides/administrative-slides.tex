\begin{frame}\frametitle{Administrative}
	\begin{itemize}
		\item	Assignment 2 (uncollected) at front
		\item	Assignment 3: available on Thursday
		\item	Assignment 4: distributed on 19 October
		\begin{itemize}
			\item	due on 26 October
		\end{itemize}
		\item	Project outline: due on Thursday
		\begin{itemize}
			\item	illustration/photo of unit
			\item	short description
			\item	physical principle used for the separation
			\item	list of references
		\end{itemize}		
		\item	Project focus: how is the unit sized
		\item	Side objective: capital and operating costs of the unit
	\end{itemize}
\end{frame}

\begin{frame}\frametitle{Midterm review}
	\begin{itemize}
		\item	About 30 messages received via website: 28 anonymous
		\item	General consensus: hard, too long, unfair, vague, unprepared
		\item	Easy way out: divide exam by 85 or 90 instead of 100 and ``move on''
		\item	The more I read the emails, the angrier(?) I became as well
			\begin{itemize}
				\item	Simply ``moving'' on is short-changing your education
				\item	Let's see why ...
			\end{itemize}
	\end{itemize}	
\end{frame}

\begin{frame}\frametitle{Midterm review}
	\begin{enumerate}
		\item	``{\color{myOrange}{It was hard}}''
		\begin{itemize}
			\item	It was mainly conceptual
			\item	You're not used to this (you are probably used to plug \& play)
			\item	Biggest issue: looking around for information in textbooks \emph{etc}
			\item	In December: you have to be able to say {\small ``that's easy''}
		\end{itemize}
		
		\item	``{\color{myOrange}{Questions were vague. It felt like there are multiple answers to the questions}}''
		\begin{itemize}
			\item	I see this comment on my course evaluations. I really want to improve; (or, confirm my suspicion: ``sometimes students don't really read the question'')
			\item	But you must tell me: what \textbf{exactly} is ``vague''?
			\item	Please use feedback form to tell me: quote the question and point it out.
			\item	(Let's go through the questions to see what is vague or unclear.)
			\item	{\color{myBlue}{Note that multiple answers are possible in some open-ended questions}}: engineers don't have ``one correct way'' implement something
		\end{itemize}
	\end{enumerate}
\end{frame}
	
\begin{frame}\frametitle{Midterm review}
	\begin{enumerate}
		\setcounter{enumi}{2}
		\item	``{\color{myOrange}{Too long}}''	
		\begin{itemize}
			\item	Can't compare it to final exam \textbf{time} allocation
			\item	My midterms usually last around 2 to 2.5 hours
			\item	15\% vs 45\% (final) is due to volume of material covered
			\item	If weight $\propto$ difficulty: then you're in for a nasty final exam!
			\item	Time management was mentioned to be important.
			\item	Number of questions in an exam is immaterial
			\begin{itemize}
				\item	Q1, Q2, Q6, Q8 and Q9: 41 marks [$<$ 30 minutes]
				\item	essentially ``free'' marks: concepts that must be at the top of your head, little thinking
				\item	Q4: separation factor required logical analysis [$\sim$ 10 minutes]
				\item	Q5: plug-and-chug [$\sim$ 10 minutes]
				\item	Q7: centrifuge: interpret and plug-and-chug [$\sim$ 10 minutes]
				\item	Total plug-and-chug: 20 marks
				\item	I don't care too much for plug-and-chug: if that's all you do well, you can be outsourced to a computer
			\end{itemize}
			\item	{\color{myBlue}{Surprising aspect: 65\% of students left early}}
			\begin{itemize}
				\item	lost patience or gave up
			\end{itemize}
		\end{itemize}
	\end{enumerate}
\end{frame}

\begin{frame}\frametitle{Midterm review}
	\begin{enumerate}
		\setcounter{enumi}{3}
		\item	``{\color{myOrange}{Unfair}}''
		\begin{itemize}
			\item	There was prior notice about question 3
			\item	Class was cancelled for questions
			\item	1.5 days is not too short
			\item	In practice: \textbf{\emph{hours}} to learn and apply
			\item	Most of the concepts required for Q3 were covered in \textbf{Tuesday}'s class and earlier
			\begin{itemize}
				\item	what is flux (covered 2 weeks ago)
				\item	where is flux measured: the permeate (covered Tuesday)
				\item	flux = $\displaystyle \frac{Q_P}{A}$ is obvious, but covered in \textbf{Thursday}'s class
				\item	only ``new'' material on Thursday: connecting modules in series; recognize the retentate cascades
			\end{itemize}
			\item	Unfair = ``not covered at all'' and ``beyond capability''
			\item	Email: ``{\color{myOrange}{The course this year is very different from the past years making it almost impossible to prepare well for it. All we have is the examples you did in class and the assignments.}}''
			\begin{itemize}
				\item	You should not prepare for something based on how you are going to be tested.
			\end{itemize}
			\item	Q3: purely mass balances (2nd year); subbing in equations (2nd year[?]); solving single non-linear equation (2nd/3rd year)
		\end{itemize}
	\end{enumerate}
\end{frame}

\begin{frame}\frametitle{Midterm review}
	\begin{enumerate}
		\setcounter{enumi}{4}
		\item	``{\color{myOrange}{Unprepared}}''
		\begin{itemize}
			\item	Message: ``{\color{myOrange}{i felt and i feel many others felt that attending the lectures and completing the homework assignments would not have prepared us well for this test ... evidently the test had very little similarity to class examples or assignments}}''
			\item	Q1, Q2, Q6: directly from course notes
			\item	Q3: example covered in class: we solved for area; this time we solve for retentate concentration and flow rate
			\item	Q4: Use (definition of separation factor) and (design equation for sedimentation): i.e. combine concepts learned in class/assignments
			\item	Q5: direct application of TSV (see assignment 1 and 2)
			\item	Q7: uses $\Sigma$ (assign 3 and covered in class) to calculate $Q_\text{cut}$
			\item	Q8: definition of flocculation (class: MIT video); membrane concepts (class): applications
			\item	Q9: application of cyclones (class): can you re-interpret what you've learned in a new context?
		\end{itemize}
	\end{enumerate}	
\end{frame}

\begin{frame}\frametitle{Question 3: let's break it down}
	Supposedly confusing, hard, worth too many marks. Let's address this:
	\begin{itemize}
		\item	``{\color{myGreen}{An asymmetric ultrafiltration membrane is used with the aim of separating dyes from a liquid stream and to achieve a more concentrated dye-water mixture}}''
		\begin{itemize}
			\item	Here's our \textbf{aim}: concentrating up a solute: ``the dye''
			\item	skip ahead to the questions: we are going to find the dye concentration, amongst other things
		\end{itemize}			
		\vspace{24pt}
		\item	{\color{myGreen}{The feed waste stream arrives at a flow rate of 2.2 $\text{m}^3.\text{hour}^{-1}$ with concentration of $1.2 \text{kg}\text{.m}^{-3}$}}
		\begin{itemize}
			\item	Some given information
		\end{itemize}
	\end{itemize}
\end{frame}

\begin{frame}\frametitle{Question 3: let's break it down}
	\begin{itemize}
		\item	{\color{myGreen}{The membrane's operating characteristic was calculated from various experiments:
				\[
					J_v = 0.04 \ln \left(\frac{15}{C}\right)
				\]
				where the bulk concentration $C$ has units of $\text{kg}\text{.m}^{-3}$ and flux is measured in $\text{m}^{3}\text{.hour}^{-1}\text{.m}^{-2}$.}}
		\item	Email: ``{\color{myOrange}{\small The description of $C$, the concentration term in the $J_v$ equation, was that it was the 'bulk concentration'. This confused many people that I talked with, including myself, who took that to mean you were telling us it was the inlet concentration.

			\vspace{5pt}
						If you give an empirical equation, make sure to either a) specify completely what the terms in the equation mean, or b) tell us explicitly that we have to decide what the term refers to.}}''
		\item	At 4:45 in video on 11 Oct class; and several other times later
		\item	Take a look back at slide 31: ``bulk'' and ``$J_v$'' explicitly defined
	\end{itemize}	
\end{frame}

\begin{frame}\frametitle{Question 3: let's break it down}
	\begin{itemize}
		\item	It is not an empirical equation: it is derived and has a logarithmic structure.
		
		\item	Recall from notes: \(\displaystyle\frac{J L_c}{D_{AB}} =  \displaystyle \frac{J}{h_w} = \displaystyle \ln\left(\frac{C_w }{C_f }\right) \)
		\item	{\color{myGreen}{If two membrane modules, each of area 25 $\text{m}^2$, are simply placed in series}}
		\begin{itemize}
			\item	connected in series: what connects to what?
			\item	draw a picture, if you haven't already
		\end{itemize}
		\vspace{2cm}
		\item	Now we are ready to answer the questions.
	\end{itemize}
\end{frame}

\begin{frame}\frametitle{Question 3: let's break it down}
	\begin{enumerate}
		\item	the dye concentration from the first membrane module?
		\item	the permeate flow rate from the first membrane module?
		\item	the dye concentration from the final membrane module?
		\item	the permeate flow rate from the final membrane module?
		\item	Then explain whether the above answers seem reasonable.
	\end{enumerate}
	Please show all calculations, assumptions and relevant details. (\emph{Yes, this question is on new material; it is not hard; just think logically.})
\end{frame}

\begin{frame}\frametitle{Solving question 3: on the board}
	
\end{frame}
