% For 2013: See "Pressure Swing Adsorption" book by Ruthven "/Users/kevindunn/Courses/4M3/References/Adsorption/Ruthven - Pressure swing adsorption.pdf" for general description of adsorption in chapter 2. Other parts useful?

\begin{comment}
	Perry's section 22.5.8

	EPA has compiled significant data on values of k and n for environmentally significant pollutants with typical activated carbons. Assuming equilibrium is reached, the isotherm provides the dose of carbon required for treatment. In a concurrent contacting process, the capacity is set by the required effluent concentration. In a countercurrent process, the capacity of the carbon is set by the untreated waste pollutant concentration. Thus countercurrent contacting is preferred.

	Activated carbon is available in powdered form (200-400 mesh) and granular form (10-40 mesh). The latter is more expensive but is easier to regenerate and easier to utilize in a countercurrent contactor. Powdered carbon is applied in well-mixed slurry-type contactors for detention times of several hours after which separation from the flow occurs by sedimentation. Often coagulation, flocculation, and filtration are required in addition to sedimentation. As it is difficult to regenerate, powdered carbon is usually discarded after use. Granular carbon is used in column contactors with EBCT of 30 minutes to 1 hour. Often several contactors are used in series, providing for full countercurrent contact. A single contactor will provide only partial countercurrent contact. When a contactor is exhausted, the carbon is regenerated either by a thermal method or by passing a solvent through the contactor. For waste-treatment applications where a large number of pollutants must be removed but the quantity of each pollutant is small, thermal regeneration is favored. In situations where a single pollutant in large quantity is removed by the carbon, solvent extraction regeneration can be used, especially where the pollutant can be recovered from the solvent and reused. Thermal regeneration is a complex operation. It requires removal of the carbon from the contactor, drainage of free water, transport to a furnace, heating under controlled conditions of temperature, oxygen, time, water vapor partial pressure, quenching, transport back to the reactor, and reloading of the column. Five to ten percent of the carbon is lost in this regeneration process due to burning and attrition during each regeneration cycle. Multiple hearth, rotary kiln, and fluidized bed furnaces have all been successfully used for carbon regeneration.

	Pretreatment prior to carbon adsorption is usually for removal of suspended solids. Often this process is used as tertiary treatment after primary and biological treatment. In either situation, the carbon columns must be designed to provide for backwash. Some solids will escape pretreatment, and biological growth will occur on the carbon, even with extensive pretreatment. Originally carbon treatment was viewed only as applicable for removal of toxic organics or those that are difficult to degrade biologically. Present practice applies carbon adsorption as a procedure for removal of all types of organics. It is realized that some biological activity will occur in virtually any activated carbon unit, so the design must be adjusted accordingly.

	\begin{frame}\frametitle{Representative example: solvent recovery}
		% Uhlmanns article on "Activated carbon", vol 6, p 787

		Air velocity                       0.2-0.4 m/s
		Air temperature                    20-40 C
		Bed height                         0.8-1.5 m
		Steam velocity                     0.1-0.2 m/s
		                                   Time cycle per adsorber
		Adsorption                         2-6 h
		Drying (hot air)                   0.2-0.5 h
		Cooling (cold air)                 0.2-0.5 h
		Solvent concentration              1-10 g/cm3
		Solvent adsorbed per cycle         10-25 wt\%
		Steam/solvent ratio                (2-5):1
		Energy                             500-600 kWh/t solvent
		Cooling water                      30-100 m3/t solvent
		Activated carbon                   0.5-1 kg/t solvent
	\end{frame}

	Note that the equilibrium section is where the isotherm applies: C_{A,S} = fn (C_A). We measure C_A, so we can calculate C_{A,S} = (kg adsorbate)/(adsorbent)
	Then if we know how much adsorbate I need to remove in a given time (theta_B)

	\begin{frame}\frametitle{More on the MTZ}
		% Details on this slide are from Schweitzer, p 3-23
		\begin{itemize}
			\item	Adsorbent information from vendors is published in terms of the dynamic or breakthrough capacity of a bed, for a given feed concentration.
			\item	It is clear from the MTZ concept, that the bed must at least exceed the length of the MTZ.
		\end{itemize}

		\vspace{12pt}
		MTZ is affected by
		\begin{itemize}
			\item	type of adsorbent
			\item	particle size distribution
			\item	fluid velocity through bed
			\item	operating temperature (adjusts isotherm, which affects MTZ shape)
			\item	concentration of contaminants
		\end{itemize}

		$t_b = \frac{as}{C_0 Q}\left(L-h\right)$
		\begin{itemize}
			\item	$L$ = bed length [m]
			\item	$h$ = half-length of the MTZ [m]
			\item	$Q$ = volumetric flow rate [$m^3$/s]
			\item	$t_b$ = breakthrough time [s]
			\item	$s$ = cross section of the adsorbent bed [$m^2$]
			\item	$a$ = volumetric adsorption capacity at $c_0$ [$m^3/m^3 ??????$] = $C_{A,S} [kg/kg]$
			\item	$C_0$ = adsorbate volume fraction $[m^3/m^3]  C_{A,feed} = kg/m^3$
		\end{itemize}
	\end{frame}

	\begin{frame}\frametitle{Fitting a model to data}
		Batch system example in Ghosh's and/or Geankoplis
		Example from UCT practical
	  For a given isotherm,  and feed concentration, calculate the separation factor at equilibrium
	\end{frame}

	\begin{frame}\frametitle{Laboratory testing}
		As mentioned prior, diffusion and equilibrium considerations must be taken into account
		\begin{itemize}
			\item	Diffusion modelling is hard: partial differential equations. {\color{myOrange}{Messy!}}
			\item	Just do a lab test and ``\emph{{\color{myGreen}{lump everything together}}}?''
		\end{itemize}
	\end{frame}

	Topics to consider for future:
	What is the purpose of the MTZ curves to troubleshoot or operate a packed bad
		Describe how Langmuir isotherms are favourable
		Space velocity.
	Multicomponent isotherms
	Describe pressure swing adsoprtion: see Seader, p 610 in their book for a good description
	Modelling of diffusion, heat transfer AND mass transfer
	\begin{frame}\frametitle{Potential theory}
		/Users/kevindunn/Sync/Figures/separations/adsorption/CR-v2-5ed-p991.png
	\end{frame}

\end{comment}

\begin{frame}\frametitle{References used (in alphabetical order)}
	\begin{itemize}
		\item	Geankoplis, ``Transport Processes and Separation Process Principles'', 4th edition, chapter 12
		\item	Ghosh, ``Principles of Bioseparation Engineering'', chapter 8
		\item	Johnston, ``Designing Fixed-Bed Adsorption Columns'', \emph{Chemical Engineering}, p 87-92, 1972
		\item	Lukchis, ``Adsorption Systems: Design by Mass-Transfer-Zone Concept'', \emph{Chemical Engineering}, 1973.
		\item	\href{http://accessengineeringlibrary.com/browse/perrys-chemical-engineers-handbook-eighth-edition}{Perry's Chemical Engineers' Handbook}, 8th edition, chapter 22
		\item	Richardson and Harker, ``Chemical Engineering, Volume 2'', 5th edition, chapter 17
		\item	Schweitzer, ``Handbook of Separation Techniques for Chemical Engineers'', chapter 3.1
		\item	Seader, Henley and Roper, ``Separation Process Principles'', 3rd edition, chapter 15
		\item	Uhlmann's Encyclopedia, ``Adsorption'', {\tiny \href{http://dx.doi.org/10.1002/14356007.b03\_09.pub2}{DOI:10.1002/14356007.b03\_09.pub2}}
		\item	Wankat, ``Separation Process Engineering'', chapter 16
	\end{itemize}
\end{frame}

\begin{frame}\frametitle{This section in context of the course}
	\begin{itemize}
		\item	Continuous operation
		\begin{itemize}
			\item	Sedimentation
			\item	Centrifuges, cyclones
			\item	Membranes (except periodically backflushed to regenerate)
			\item	Liquid-liquid extraction
		\end{itemize}

		\vspace{12pt}
		\item	Batch/cycled operation
		\begin{itemize}
			\item	filtration (e.g. plate and frame)
			\item	{\color{red}adsorption units}
			\item	drying units (\emph{next})
		\end{itemize}
	\end{itemize}

	\vspace{12pt}
	{\color{myOrange}Our goals}
	\begin{itemize}
		\item	understand what adsorbers look like and how they are operated
		\item	how to find the equilibrium isotherms for a new system
		\item	preliminary sizing of an adsorption unit
	\end{itemize}
\end{frame}

\begin{frame}\frametitle{Introduction to \textbf{{\color{purple}{sorption}}} processes}
	\begin{exampleblock}{\textbf{Sorption}}
		Components in a fluid phase, {\color{myGreen}{solutes}}, are selectively transferred to insoluble, (rigid) particles that are suspended in a vessel or packed in a column.
	\end{exampleblock}
	\begin{itemize}
		\item	{\color{purple}{(ad)sorbate}}: the (ad)sorbed solute that's usually of interest
		\item	{\color{purple}{(ad)sorbent}}: the (ad)sorbing agent, i.e. the MSA
		\item	Is there an ESA?
	\end{itemize}

	\vspace{12pt}
	\emph{Some sorption processes}:
	\begin{itemize}
		\item	a\textbf{b}sorption: gas into liquid phase [it is strictly speaking a sorption process, but not considered here (3M4)]
		\item	{\color{purple}{adsorption}}: molecules bond with a solid {\color{red}surface}
		\item	{\color{purple}{ion-exchange}}: ions displace dissimilar ions from solid phase
			\begin{itemize}
				\item	Water softening: $\text{Ca}^{2+}_\text{(aq)} + 2\text{NaR}_\text{(s)} \leftrightharpoons \text{CaR}_{2\text{(s)}} + 2\text{Na}^{+}_\text{(aq)}$
			\end{itemize}
		\item	{\color{purple}{chromatography}}: solutes move through column with an eluting fluid. Column is continuously regenerated.
	\end{itemize}
\end{frame}

\begin{frame}\frametitle{Sorption examples}
	\begin{exampleblock}{}
		We will focus on {\color{myGreen}{(ad)sorption}} for the next few classes.
	\end{exampleblock}

	\begin{columns}[t]
		\column{0.75\textwidth}
			Some well-known examples:
			\begin{itemize}
				\item	adsorption: charred wood products to improve water taste
				\item	adsorption: decolourize liquid with bone char
				\item	adsorption: those little white packets in boxes of electronics
				\item	ion-exchange: passing water through certain sand deposits removes salt
				\item	ion-exchange: synthetic polymer resins widely used to soften water
			\end{itemize}
			
		\column{0.35\textwidth}
			\begin{center}
				\includegraphics[width=0.8\textwidth]{\imagedir/separations/adsorption/flickr-184883876_bafa54631c.jpg}
			\end{center}
	\end{columns}
	
	\vspace{12pt}
	Industrial use of adsorption picked up with synthetic manufacturing of zeolites in the 1960s.
\end{frame}

\begin{frame}\frametitle{Adsorption examples}
	\begin{itemize}
		\item	Gas {\color{purple}{purification}}:
		\begin{itemize}
			\item	Volatile organics from a vent stream
			\item	Sulphur compounds from gas stream
			\item	Water vapour
			\item	Removal of $\text{CO}_2$ from natural gas [alternatives ?]
		\end{itemize}
		\item	{\color{purple}{Bulk separation}} in the gas phase:
		\begin{itemize}
			\item	$\text{O}_2$ from $\text{N}_2$ (adsorbed more strongly onto zeolites)
			\item	$\text{H}_2\text{O}$ from ethanol
			\item	High acetone quantities from air vent streams
		\end{itemize}
		\item	Liquid-liquid separation and purification:
		\begin{itemize}
			\item	Organics and toxic compounds from water
			\item	Sulphur compounds from water
			\item	Normal vs iso-paraffin separation
			\item	Separation of isomers: \emph{p}$\,$- vs \emph{m}-cresol
			\item	Fructose from dextrose separation
			\item	Gold in cyanide solutions
		\end{itemize}
		\vspace{-2.5cm}
		\begin{columns}[t]
			\column{0.75\textwidth}
			\column{0.20\textwidth}
				\begin{center}
					\includegraphics[width=0.8\textwidth]{\imagedir/separations/adsorption/P-cresol-spaceFilling.png}
				\end{center}
			\column{0.20\textwidth}
				\begin{center}
					\includegraphics[width=\textwidth]{\imagedir/separations/adsorption/M-cresol-spaceFill.png}
				\end{center}
		\end{columns}
		\begin{columns}[t]
			\column{0.75\textwidth}
			\column{0.20\textwidth}
				\begin{center}
					\emph{p}-cresol
				\end{center}
			\column{0.20\textwidth}
				\begin{center}
					\emph{m}-cresol
				\end{center}
		\end{columns}
		\hfill\see{Cresol figures from Wikipedia}

	\end{itemize}
\end{frame}

\begin{frame}\frametitle{Adsorbents}
	General principle (more details coming up soon)
	\begin{columns}[c]
		\column{0.60\textwidth}
			\begin{center}
				\includegraphics[width=\textwidth]{\imagedir/separations/adsorption/Adsorbent-pores-Seader-p569-3ed-mod.png}
			\end{center}
		\column{0.45\textwidth}
			Molecules attach to the {\color{red}particle's surface}: outside and on the pore walls

			\vspace{12pt}
			Main characterization: {\color{purple}{pore diameter}} of adsorbent

			\vspace{12pt}
			Mechanisms during adsorption
				\begin{itemize}
				\item	{\color{purple}{equilibrium interaction}}: solid-fluid interactions
				\item	{\color{purple}{kinetic}}: differences in diffusion rates
				\item	{\color{purple}{steric}}: pore structure hinders/retains molecules of a certain shape
			\end{itemize}
	\end{columns}
	\see{Modified from: Seader, 3ed, p 569}
\end{frame}

\begin{frame}\frametitle{Quick recap of some familiar concepts}
	\begin{itemize}
		\item	1m = 100cm = 1000mm = $10^6$\micron = $10^9$nm = $10^{10}$\AA
		\item	Hydrogen and helium atoms: $\approx 1$\AA
		\item	For a pore: \[\frac{\text{Internal surface area}}{\text{Pore volume}} = \frac{\pi d_p L}{\pi d_p^2 L/4} = \frac{4}{d_p}\]
		\item	$d_p$ = pore diameter: typically around 10 to 200 \AA
	\end{itemize}

	\vspace{12pt}Our main concern is solid's adsorption area per unit mass:
	\begin{itemize}
		\item	solids are about 30 to 85\% porous
		\item	typical values: 300 to 1200 $\text{m}^2$ per gram
		\item	area of hockey field = 91.4 $\times$ 55 m = 5027 $\text{m}^2$
	\end{itemize}
\end{frame}

\begin{frame}\frametitle{Adsorbents}
	Helpful to see what they look like to understand the principles:
	\begin{columns}[c]
		\column{0.60\textwidth}
			\begin{center}
				\includegraphics[width=0.9\textwidth]{\imagedir/separations/adsorption/Active_Al2O3-Wikipedia.png}
			\end{center}
			\vspace{-12pt}
			\see{\href{http://en.wikipedia.org/wiki/File:Active_Al2O3.jpg}{Wikipedia, Active\_Al2O3.jpg}}
		\column{0.40\textwidth}
			{\color{myGreen}{Activated alumina}}

			\begin{itemize}
				\item	made from from aluminum hydroxide
				\item	$\sim 300~\text{m}^2$ per gram
				\item	most widely used adsorbent
				\item	hydrophilic
				\item	pore diameter: 10 to 75 \AA
			\end{itemize}
	\end{columns}
\end{frame}

\begin{frame}\frametitle{Adsorbents}
	\begin{columns}[c]
		\column{0.50\textwidth}
			\begin{center}
				\includegraphics[width=0.9\textwidth]{\imagedir/separations/adsorption/DOI-{10.1016}{j.saa.2011.10.012}1-s2.0-S1386142511009012-gr3.jpg}
			\end{center}
			\vspace{-12pt}
			\see{\href{http://dx.doi.org/10.1016/j.saa.2011.10.012}{DOI:10.1016/j.saa.2011.10.012}}
		\column{0.50\textwidth}
			{\color{myGreen}{Activated carbon}}

			\begin{itemize}
				\item	partially oxidized coconut shells, nuts, wood, peat, bones, sewage sludge
				\item	difference hardnesses of adsorbent
				\item	400 to 1200 $\text{m}^2$ per gram
				\item	hydrophobic
				\item	pore diameter:  10 to over 50\AA
			\end{itemize}
			e.g. bone char: decolourizing syrups % Article by Johnston, Green book, "Designing Fixed-Bed Adsorption Columns.pdf"
	\end{columns}
\end{frame}

\begin{frame}\frametitle{Adsorbents}

	% See: http://cool.conservation-us.org/byorg/abbey/an/an20/an20-7/an20-702.html

	\begin{columns}[t]

		\column{0.60\textwidth}
			\vspace{-24pt}
			\begin{center}
				\includegraphics[width=0.9\textwidth]{\imagedir/separations/adsorption/Zeolites-Seader-3ed-p575.png}
			\end{center}
			\see{Seader, 3ed, p575}

			\see{Uhlmanns, p565}
			\vspace{-4pt}
			\begin{center}
				\includegraphics[width=\textwidth]{\imagedir/separations/adsorption/Zeolites-Uhlmanns-p565.png}
			\end{center}
		\column{0.40\textwidth}
			{\color{myGreen}{Zeolite lattices}}

			\vspace{12pt}
			\emph{Some examples}

			\vspace{6pt}
			$\text{K}_{12}[(\text{AlO}_2)_{12}(\text{SiO}_2)_{12}]$:

			\hfill {\color{myOrange}{drying gases  [2.9\AA]}}

			\vspace{12pt}
			$\text{Na}_{12}[(\text{AlO}_2)_{12}(\text{SiO}_2)_{12}]$:

			\hfill {\color{myOrange}{$\text{CO}_2$ removal  [3.8\AA]}}


			\vspace{12pt}
			$\text{Ca}_{43}[(\text{AlO}_2)_{86}(\text{SiO}_2)_{106}]$:

			\hfill {\color{myOrange}{air separation [8\AA]}}


			\vspace{18pt}
			Very specific pore diameters.
			\begin{itemize}
				\item	40 naturally occurring
				\item	$\sim$  150 synthesized % C+R v2, 5ed, p975
				\item	$\sim$	650 $\text{m}^2$ per gram
			\end{itemize}

	\end{columns}
\end{frame}

\begin{frame}\frametitle{Adsorbents}
	Zeolites; also called {\color{purple}molecular sieves}

	\begin{tabular}{cll}\\
		{\small \textbf{Window size}}	&	{\small \textbf{Adsorbs}...}	&	{\small \textbf{Dehydrates} ...}\vspace{6pt}\\ \hline
		3\AA			&	$\text{H}_2\text{O}$, $\text{NH}_3$		& 	unsaturated hydrocarbons\\
		4\AA			&  	$\text{H}_2\text{S}$, $\text{CO}_2$, $\text{C}_3\text{H}_6$		& 	saturated hydrocarbons\\
		5\AA			&	\emph{n}-paraffins from iso-paraffins   & \\
		8\AA			& 	iso-paraffins and olefins				&	\\
	\end{tabular}
	\vspace{5pt}
	\see{Johnston}%, Green book, "Designing Fixed-Bed Adsorption Columns.pdf"

	{\small Electrostatic fields exist inside the zeolite cage: strong interactions with polar molecules. Sieving not only based on shape/size exclusion.}

	\vspace{3pt}
	% Rousseau "21FC8B40D9C71DB16CAE2B81BA3244.pdf"
	\see{Rousseau, ``Handbook of Separation Technology''}

	\vspace{-5pt}
	\begin{tabular}{ll}\\
		\textbf{Adsorbent}				&	\textbf{Market size (1983)} \vspace{6pt}\\ \hline
			Activated carbon 			&	\$ 380 million $\longleftarrow$ 25\% for water treatment \\
			Molecular-sieve zeolites 	&	\$ 100 million  \\
			Silica gel					&	\$ 27 million  \\
			Activated alumina			&	\$ 26 million \\\hline
	\end{tabular}
\end{frame}

\begin{frame}\frametitle{Pore diameter characterization}
	\begin{center}
		\includegraphics[width=0.9\textwidth]{\imagedir/separations/adsorption/Pore-size-distribution-Seader-3ed-p574.png}
	\end{center}
	\see{Seader, 3ed, p574}
	{\scriptsize Determined using He and Hg porosimetry (see reference for details)}
	% MSC = molecular-sieve carbon (specially processed activated carbon to obtain smaller pores)
\end{frame}

\begin{frame}\frametitle{{\color{myOrange}{Example}}: Gold leaching and adsorption}
	\begin{columns}[c]
		\column{1.08\textwidth}
			\begin{itemize}
				\item	Crushed rock has gold particles exposed
				\item	Leaching: $4\text{Au}_\text{(s)} + 8\text{NaCN} + \text{O}_2 + 2 \text{H}_2\text{O} \rightleftharpoons  4 \text{Na[Au(CN)}_2\text{]} + 4 \text{NaOH}$
				\item	Adsorption: aurocyanide complex, $\text{Au(CN)}_2^{-}$, is adsorbed onto activated carbon
					\begin{itemize}
						\item	drives the equilibrium in the leaching step forward
						\item	separates the solid gold, $\text{Au}_\text{(s)}$, from the pulp (slurry)
						\item	obtain $C_\text{A,S} \sim 8000$ grams of Au per tonne of carbon
					\end{itemize}
			\end{itemize}
	\end{columns}
	\begin{columns}[t]
		\column{0.60\textwidth}
			\begin{itemize}
				\item	Desorption:
					\begin{itemize}
						\item	separate the highly concentrated gold-carbon pulp (screens/filter/ cyclones/sedimentation)
						\item	desorb the gold off the carbon with caustic contact
						\item	recycle the regenerated carbon
					\end{itemize}
			\end{itemize}
		\column{0.40\textwidth}
			\begin{center}
				\includegraphics[width=\textwidth]{\imagedir/separations/adsorption/Batch-adsorption-Seader-3ed-p610.png}
			\end{center}
	\end{columns}
\end{frame}

\begin{frame}\frametitle{Gold leaching: Johannesburg, RSA}
	\begin{center}
		\includegraphics[width=\textwidth]{\imagedir/separations/adsorption/Gold-CIP-circuit.png}
	\end{center}
	\begin{center}
		\includegraphics[width=.95\textwidth]{\imagedir/separations/adsorption/Gold-CIP-flows.png}
	\end{center}
	\vspace{-12pt}
	\see{\href{http://dx.doi.org/10.1590/S0104-66322007000400014}{Lima, 2007. Brazilian Journal of Chemical Engineering}}
\end{frame}

\begin{frame}\frametitle{H$_2$S and CO$_2$ pre-treatment adsorbers}
	\begin{center}
		\includegraphics[width=.95\textwidth]{\imagedir/separations/adsorption/flickr-4565083761_18fcf20d57_b.jpg}
	\end{center}
	\see{Flickr: http://www.flickr.com/photos/vmeprocess/4565083761}

	% Note the vertical orientation of the vessels
\end{frame}

\begin{frame}\frametitle{When to consider adsorption}
	Distillation, membranes, absorption, liquid-liquid extraction are sometimes viable alternatives.

	\vspace{12pt}
	But adsorption is considered when:
	\begin{itemize}
		\item	relative volatility between components is $<1.5$ (e.g. isomers)
		\item	large reflux ratios would be required % higher reflux improves separation between low and high key, at the expense of greater reboiler and condensor usage
		\item	too large area for a membrane
		\item	excessive temperatures or high pressure drops are to be avoided
		\item	high selectivity is required
		\item	feed is a very dilute stream of solute ({\color{purple}adsorbate})
	\end{itemize}
	But, some disadvantages:
	\begin{itemize}
		\item	only the surface of the adsorbent used
		\item	regeneration of MSA adsorbent required
		\item	MSA will break down mechanically over time as we move it around
		\begin{itemize}
			\item	{\color{myOrange}{we must pump it, filter it, and/or put it through cyclones to process it}}
		\end{itemize}
	\end{itemize}
\end{frame}

\begin{frame}\frametitle{Physical principles}
	\begin{exampleblock}{Adsorption releases heat, it's exothermic. Why?}
		\pause
		\iftoggle{instructor}{Loss of degrees of freedom of fluid: free energy is reduced, so $\Delta S \downarrow$}{Thermodynamics: ...}

		\iftoggle{instructor}{$\Delta G = \Delta H - T \Delta S \quad \Longrightarrow  \quad \Delta H = \Delta G + T \Delta S \quad \Longrightarrow  \quad \Delta H < 0$}{\vspace{18pt}}

		% Delta G is always negative for stuff to occur, so is Delta S (entropy), so Delta H is negative
	\end{exampleblock}

	\vspace{6pt}
	Two types of adsorption:
	\vspace{6pt}
	\begin{enumerate}
		\item	Physical adsorption ({\color{purple}{physisorption}}):
		\begin{itemize}
			\item	Low heat of adsorption released: $\Delta H_\text{ads} \sim$ 30 to 60 kJ/mol
			\item	Theory: {\color{purple}{van der Waals attractions}}
			\item	easily reversible
		\end{itemize}
		\item	Chemical adsorption ({\color{purple}{chemisorption}}):
		\begin{itemize}
			\item	High heat of adsorption released: $\Delta H_\text{ads} > 100$ kJ/mol
			\item	chemical bond formation, in the order of chemical bond strengths
			\item	leads to reaction products
			\item	more energy intensive to reverse
			\item	e.g.: catalysis, corrosion
		\end{itemize}
	\end{enumerate}
	\pause
	As adsorbate concentration increases:
	\begin{itemize}
		\item	single layers form, then multiple layers, then condensation
	\end{itemize}
\end{frame}

% These next 3 slides might seem out of place, but they work well here as an interjection, as well as introducing the concept of regeneration.
\begin{frame}\frametitle{Packed beds: adsorption and desorption steps}
	\begin{center}
		\includegraphics[width=.8\textwidth]{\imagedir/separations/adsorption/Packed-bed-CR-v2-5ed-p1028-1.png}
	\end{center}
	\vspace{-6pt}
	\see{Richardson and Harker, p 1028}

	{\color{purple}{Regeneration}} reverses the adsorbate-adsorbent equilibrium:
	\begin{enumerate}
		\item	\textbf{raise the temperature} to shift the equilibrium constant
		\item	\textbf{lower the pressure} (vapour-phase adsorbate)
		\item	displace the adsorbate with an alternative (e.g. steam)
	\end{enumerate}
	Regenerate is shorter duration when done in the reverse direction to loading.

	% An adsorption unit using thermal swing regeneration usually consists of two packed beds, one on-line and one regenerating, as shown in the figure. Regenerating consists of heating, and purging to remove adsorbate. The arrangement is flexible and robust. The desorption temperature depends on the properties of the adsorbent and the adsorbates. Manufacturers normally recommend the best regenerating temperature for their particular adsorbent. Exceeding this temperature may accelerate the ageing processes which cause pores to coalesce and capacity to be reduced. Too low a temperature may result in incomplete regeneration so that the effluent concentration in subsequent adsorption stages will be higher than its design value. Hot spots may develop in the operation of a fixed bed, so particular care has to be taken to control temperature when handling flammable materials.
	% The relatively poor conductivity of a packed bed makes it difficult to get the heat of regeneration into the bed, either from a jacket or from coils embedded in the packing. This is more easily achieved by preheating the purge stream. Even in the best conditions, it takes time for the temperature of the bed to rise to the required level. Thermal regeneration is normally associated with long cycle times, measured in hours. Such cycles require large beds and, since the adsorption wave occupies only a small part of the bed on-line, the utilisation of the total adsorbent in the unit is low.
	% It is good operating practice to regenerate a bed in the reverse direction to that followed during adsorption. This ensures that the adsorbent at the end of the bed, which controls the quality of the treated stream, is that which is most thoroughly regenerated. CARTER(44) has quantified the effect and showed that regeneration is achieved in a shorter time in this way.
\end{frame}

\begin{frame}\frametitle{Fluidized beds: for continual operation}
	\begin{columns}[t]
		\column{0.80\textwidth}
			\begin{center}
				\includegraphics[height=0.8\textheight]{\imagedir/separations/adsorption/Fluidized-bed-Uhlmanns-p556.png}
			\end{center}
			\vspace{-18pt}
			\see{Uhlmanns, p556}
		\column{0.30\textwidth}
			Materials of construction are important: carbon on carbon steel has a galvanic effect: leads to corrosion.

			Use stainless steel, or a lined vessel.

			\vspace{12pt}
			Cyclones used to recover adsorbent.

			\begin{itemize}
				\item	Adsorbent life: $\sim$ 100 cycles
				\item	Bleed off old adsorbent and continuously replenish
			\end{itemize}
	\end{columns}
\end{frame}

\begin{frame}\frametitle{(Fluidized bed?) example}
	Adsorption, Desorption and Recovery (ADR) plant in Burkina Faso
	\begin{center}
		\includegraphics[width=0.85\textwidth]{\imagedir/separations/adsorption/flickr-5043854546_4ff634cc5a_b.jpg}
	\end{center}
	\vspace{-12pt}
	\see{\href{http://www.flickr.com/photos/isurusen/5043854546/}{Flickr \#5043854546}}
	\emph{Zoom in on the high resolution photo to see details.}
\end{frame}

\begin{frame}\frametitle{Modelling the adsorption process}
	\begin{enumerate}
		\item	Diffusion
			\begin{itemize}
				\item	diffusion of the adsorbate in the bulk fluid (usually very fast)
				\item	diffusion of the adsorbate to the adsorbent surface through the boundary layer
				\item	diffusion of the adsorbate into the pore to an open site
				\begin{itemize}
					\item	steric (shape) effects may be an issue
				\end{itemize}
			\end{itemize}
		\item	Equilibrium considerations
			\begin{itemize}
				\item	adsorbate will attach to a vacant site
				\item	adsorbate will detach from an occupied site
			\end{itemize}
	\end{enumerate}

	\vspace{12pt}
	Mechanisms during adsorption
	\begin{itemize}
		\item	{\color{purple}{equilibrium interaction}}: solid-fluid interactions
		\item	{\color{purple}{kinetic}}: differences in diffusion rates (if multiple adsorbates)
		\item	{\color{purple}{steric}}: pore structure hinders/retains molecules of a certain shape
	\end{itemize}
\end{frame}

\begin{frame}\frametitle{Equilibrium modelling}
	\begin{exampleblock}{Why?}
		We ultimately would like to determine \textbf{how much adsorbent is required} to remove a given amount of adsorbate (e.g. impurity); particularly in batch processes.
	\end{exampleblock}

	\vspace{12pt}
	For now, assume we are only limited by equilibrium [we'll get there, we don't mind \emph{how long} {\small (due to kinetics of diffusion and mass transfer resistance) }it takes to get there]
	\begin{itemize}
		\item	Derive/Postulate a model relating bulk concentration to surface concentration of adsorbate
		\item	We call these equilibrium equations: ``isotherms''
		\item	{\color{purple}{Isotherm}}: relates amount of adsorbate on adsorbent ($C_\text{A,S}$) at different concentrations of adsorbate in the bulk ($C_\text{A}$), but at a fixed temperature.
	\end{itemize}
\end{frame}

\begin{frame}\frametitle{Equilibrium modelling: linear model}
	\begin{exampleblock}{Linear isotherm (Henry's law)}
		\[C_\text{A,S} = K C_\text{A}\]
		\[C_\text{A,S} = \frac{K P_\text{A}}{RT} = K' P_\text{A}\]
	\end{exampleblock}
	\begin{columns}[t]
		\column{1.1\textwidth}
			\begin{itemize}
				\item	{\small $C_\text{A,S} =$ concentration of adsorbate A on adsorbent surface \hfill $\left[\displaystyle \frac{\text{kg adsorbate}}{\text{kg adsorbent}}\right]$}
				\item	{\small $C_\text{A} =$ concentration of adsorbate A in the bulk fluid phase \hfill $\left[\displaystyle \frac{\text{kg adsorbate}}{\text{m}^3~ \text{fluid}}\right]$}
				\item	{\small $P_\text{A} =$ partial pressure of adsorbate A in the bulk fluid phase \hfill $\left[ \text{atm}\right]\,$}
				\item	{\small $K$ and $K'$ are temperature dependent equilibrium constants {\tiny (should be clear why)}}
				\item	{\small $R$ is the ideal gas constant}
				\item	{\small $T$ is the system temperature}
			\end{itemize}
	\end{columns}
	\vspace{12pt}
	\begin{itemize}
		\item	Few systems are this simple!
	\end{itemize}
\end{frame}

\begin{frame}\frametitle{Batch system example (\emph{previous midterm question})}
	You are to design a batch adsorber to remove an organic contaminant (A) from 400L of aqueous solution containing 0.05g/L of the contaminant. To facilitate this you do a bench scale experiment with 1L solution at the same concentration (0.05g/L) and 3g of an adsorbent. In the bench scale experiment you find that 96\% of the contaminant was removed. You need to remove 99\% of the contaminant in the full scale apparatus. You can assume that a linear isotherm applies.

	\vspace{12pt}

	For the full scale system:

	\begin{enumerate}
		\item	At the end of the batch, what will be the concentration of the solution in the adsorber and concentration of A on the adsorbent?
		\item	How much adsorbent do you need? {\color{myOrange}[Ans: 4.95 kg]}

	\end{enumerate}
\end{frame}

\begin{frame}\frametitle{Equilibrium modelling: Freundlich model}
	\begin{exampleblock}{Freundlich isotherm}
		\[C_\text{A,S} = K \left(C_\text{A}\right)^{1/m}\qquad \text{for $ 1 < m < 5$}\]
	\end{exampleblock}

	\vspace{12pt}
	\begin{itemize}
		\item	It is an empirical model, but it works well
	\end{itemize}
	\begin{center}
		\includegraphics[width=0.5\textwidth]{\imagedir/separations/adsorption/Freundlich-isotherm.png}
	\end{center}

	\begin{itemize}
		\item	Constants determined from a log-log plot
		\item	How would you go about setting up a lab experiment to collect data to calculate $K$?
		\item	Which way will the isotherm shift if temperature is increased?
	\end{itemize}
\end{frame}

\begin{frame}\frametitle{Equilibrium modelling: Langmuir isotherm}
	% Derivation from Fogler, p 422, chapter 10
	\begin{itemize}
		\item	we have a uniform adsorbent surface available {\tiny (all sites equally attractive)}
		\item	there are a total number of sites available for adsorbate A to adsorb to
		\item	$C_\text{T}$ = total sites available \hfill {\scriptsize $\left[\displaystyle\frac{\text{mol sites}}{\text{kg adsorbate}} \right]$}
		\pause
		\item	$C_\text{V}$ = vacant sites available \hfill {\scriptsize $\left[\displaystyle\frac{\text{mol sites}}{\text{kg adsorbate}} \right]$}
		\item	rate of adsorption = $k_\text{A} P_\text{A} C_\text{V}$ = proportional to number of collisions of A with site S
		\pause
		\item	$C_\text{A,S}$ = sites occupied by A \hfill {\scriptsize $\left[\displaystyle\frac{\text{mol sites}}{\text{kg adsorbate}} \right]$}
		\item	assuming 1 site per molecule of A, and only a monolayer forms
		\item	rate of desorption=  $k_{-\text{A}} C_\text{A,S}$ = proportional to number of occupied sites
		\item	net rate = $k_\text{A} P_\text{A} C_\text{V} - k_{-\text{A}} C_\text{A,S}$
	\end{itemize}
\end{frame}

\begin{frame}\frametitle{Equilibrium modelling: Langmuir isotherm}
	\vspace{-2cm}
	\begin{itemize}
		\item	Net rate = $k_\text{A} P_\text{A} C_\text{V} - k_{-\text{A}} C_\text{A,S}$
		\item	define $K_\text{A} = \displaystyle\frac{k_\text{A}}{k_{-\text{A}}}$
		\item	essentially an equilibrium constant: $\text{A} + \text{S} \rightleftharpoons \text{A}\cdot \text{S}$
		\item	at equilibrium, the net rate is zero
		\pause
		\item	implying $\displaystyle\frac{k_\text{A}C_\text{A,S}}{K_\text{A}} = k_\text{A} P_\text{A} C_\text{V}$
		\item	but total sites = $C_\text{T} = C_\text{V} + C_\text{A,S}$
		\item	so $\displaystyle\frac{k_\text{A}C_\text{A,S}}{K_A} = k_\text{A} P_\text{A} \left(C_\text{T} -  C_\text{A,S}\right)$
		\item	simplifying: $C_\text{A,S} = K_\text{A} P_\text{A} \left(C_\text{T} -  C_\text{A,S}\right)$
		\item	then \fbox{$C_\text{A,S} = \displaystyle \frac{K_\text{A}C_\text{T} P_\text{A}}{1 + K_\text{A} P_\text{A}} = \displaystyle \frac{K_1 P_\text{A}}{1 + K_2 P_\text{A}} = \displaystyle \frac{K_3 C_\text{A}}{1 + K_4 C_\text{A}}$}
		\item	Fit data using \href{http://en.wikipedia.org/wiki/Eadie\%E2\%80\%93Hofstee\_plot}{Eadie-Hofstee diagram} or nonlinear regression
		\item	Same structure as Michaelis-Menten model (bio people)
	\end{itemize}
	\vspace{-6.5cm}
	\begin{columns}[t]
		\column{0.70\textwidth}

		\column{0.40\textwidth}
			\begin{center}
				\includegraphics[width=\textwidth]{\imagedir/separations/adsorption/Langmuir-isotherm.png}
			\end{center}
	\end{columns}
\end{frame}

\begin{frame}\frametitle{Summary of isotherms}
	We aren't always sure which isotherm fits a given adsorbate-adsorbent pair:
	\begin{enumerate}
		\item	Perform a laboratory experiment to collect the data
		\item	Postulate a model (e.g. linear, or Langmuir)
		\item	Fit the model to the data
		\item	Good fit?
	\end{enumerate}

	\vspace{24pt}
	Other isotherms have been proposed:
	\begin{itemize}
		\item	BET (Brunauer, Emmett and Teller) isotherm
		\item	Gibb's isotherm: allows for a multilayer of adsorbate forming
	\end{itemize}

	These are far more flexible models (more parameters); e.g. Langmuir isotherm is a special case of the BET isotherm.
\end{frame}

\begin{frame}\frametitle{Further questions to try}
	\seefull{Adapted from Geankoplis question 12.2-1}

	2.5 $\text{m}^3$ of wastewater solution with 0.25 kg phenol/$\text{m}^3$ is mixed with 3.0 kg granular activated carbon until equilibrium is reached. Use the following isotherm, determined from lab values, to calculate the final equilibrium values of phenol extracted and percent recovery. Show the operating point on the isotherm. {\footnotesize Units of $C_\text{A}$ are [kg per $\text{m}^3$] and $C_\text{A,S}$ is in [kg solute per kg of activated carbon].}

	\vspace{1pt}

	{\small [{\color{myOrange}{Ans: $C_\text{A} \approx 0.10$ kg per $\text{m}^3$, $C_\text{A,S} \approx 0.12$ kg/kg, recovery = 58\%}}]}

	\begin{columns}[t]

		\column{0.70\textwidth}
			\vspace{-12pt}
			\begin{center}
				{\scriptsize\emph{~~~~~~~~~~Experimental isotherm data}}
				\includegraphics[width=\textwidth]{\imagedir/separations/adsorption/isotherm-Geankoplis-question-12-1-1.png}
			\end{center}
		\column{0.30\textwidth}
			\[C_\text{A,S} = \frac{0.145 C_\text{A}}{0.0174 + C_\text{A}}\]
	\end{columns}
\end{frame}

\begin{frame}\frametitle{Isotherms change at different temperatures}
	\begin{center}
		\includegraphics[width=\textwidth]{\imagedir/separations/adsorption/pressure-and-temperature-swing-Seader-p610-cleaned.png}
	\end{center}
	\vspace{-12pt}
	\see{Seader, 3ed, p610}
\end{frame}

\begin{frame}\frametitle{Understanding adsorption in packed beds (1 of 2)}
	\begin{center}
		% Lukchis article
		\includegraphics[width=\textwidth]{\imagedir/separations/adsorption/Lukchis-MTZ-illustration-part-1.png}
	\end{center}
	\vspace{-12pt}
	{\footnotesize $L$ = length; $\theta$ = time; $\theta_0$ = start-up time on a regenerated bed}
\end{frame}

\begin{frame}\frametitle{Understanding adsorption in packed beds (2 of 2)}
	\begin{center}
		% Lukchis article
		\includegraphics[width=\textwidth]{\imagedir/separations/adsorption/Lukchis-MTZ-illustration-part-2.png}
	\end{center}
	\vspace{-24pt}
	\see{Lukchis}
	\vspace{-18pt}
	\begin{columns}[t]
		\column{1.1\textwidth}
			\begin{itemize}
				\item	{\small $C_\text{A,S}$ = concentration of adsorbate on adsorbent}
				\item	{\small $C_\text{A,S}^e$ = concentration at equilibrium on the adsorbent {\color{myOrange}(equil loading)}}
				\item	{\small $C_\text{A,S}^0$ = concentration on the regenerated adsorbent at time 0}
				\item	{\small $\theta_b$ = breakthrough time: ``\emph{time to stop using the packed bed!\,}''}; usually when $C_\text{A} = 0.05 C_\text{A,F}$
				\item	{\small $\theta_e$ = the bed at equilibrium time; packed bed is completely used}

				\vspace{-5pt}
				\item	{\small $C_\text{A,S}$ values are not easy measured; outlet concentration $C_\text{A}$ is easy}
			\end{itemize}
	\end{columns}
\end{frame}

\begin{frame}\frametitle{Bed concentration just prior to breakthrough}
	\begin{center}
		\includegraphics[width=0.7\textwidth]{\imagedir/separations/adsorption/Bed-concentration-Ghosh-p144.png}
	\end{center}
	\vspace{-12pt}
	\see{Ghosh (adapted), p144}
	\vspace{6pt}
	{\small
		\begin{itemize}
			\item	{\color{purple}{MTZ}}: mass transfer zone is where adsorption takes place.% Assumed to be from 5 to 95\% of the concentration range.
			\item	It is S-shaped: indicates there is mass-transfer resistance and axial dispersion and mixing. Contrast to the ideal shape: is a perfectly vertical line moving through the bed
			\item	{\color{purple}{Equilibrium zone}}: this is where the isotherm applies!
			\item	{\color{purple}{Breakthrough}}: arbitrarily \textbf{defined as time} when either (a)  the lower limit of adsorbate detection, or (b) the maximum allowable adsorbate in effluent leaves the bed. Usually around 1 to 5\% of $C_\text{A,F}$.
		\end{itemize}
	}
\end{frame}

\begin{frame}\frametitle{Figures to help with the next example}
	\begin{center}
		\includegraphics[width=.9\textwidth]{\imagedir/separations/adsorption/adsorber-loading-Seader-p605.png}
	\end{center}
	\vspace{-24pt}
	\see{Seader, Henly, Roper, p 605}
\end{frame}

\begin{frame}\frametitle{Terminology}
	\begin{itemize}
		\item	{\color{purple}LES} = length of equilibrium section (increases as bed is used)
		\item	{\color{purple}LUB} = length of unused bed (decreases as bed is used up)
		\item	$L$ = total bed length = LES + LUB
		\item	No data available: use MTZ distance of 4ft
	\end{itemize}
\end{frame}

\begin{frame}\frametitle{Example (and some new theory applied)}
	An adsorbate in vapour is adsorbed in an experimental packed bed. The inlet contains $C_\text{A,F} = 600$ ppm of adsorbate. Data measuring the outlet concentration over time from the bed are plotted below:

	\begin{center}
		\includegraphics[width=0.8\textwidth]{\imagedir/separations/adsorption/example-Geankoplis-p768.png}
	\end{center}
	\vspace{-24pt}
	\see{Geankoplis, 4ed, p 768}
\end{frame}

\begin{frame}\frametitle{Example}
	\vspace{-12pt}
	\begin{columns}[t]
		\column{1.05\textwidth}
		\begin{enumerate}
			\item	Determine the {\color{purple}{breakthrough time}}, $\theta_b$. \pause \iftoggle{instructor}{[{\color{myOrange}{\small Ans: 3.65 hours}}]}{}
			\item	What would be the {\color{purple}{usable capacity}} of the bed \textbf{at time $\theta_b$} if we had an {\color{purple}{ideal wavefront}} (no mass transfer resistance)? \pause \iftoggle{instructor}{[{\color{myOrange}{Ans:}} the fractional area of $A_1$ = 3.65 / 6.9 = {\color{orange}{53\%}}]}{}
			\pause
				\begin{itemize}
					\item	Note plot area units = ``total time'', since ``height'' of y-axis = 1.0
					\item	Note: (area up to $\theta_b$) $\approx \theta_b$  when using a normalized y-axis
				\end{itemize}
			\begin{center}
				\includegraphics[width=0.55\textwidth]{\imagedir/separations/adsorption/example-Geankoplis-p768.png}
			\end{center}
			\vspace{-6pt}
			\item	How long does it take to reach this ideal capacity? {\color{myOrange}{$\approx$3.65 \text{hours}}}%, but if we want to be exact, it is $t_\text{used} = \displaystyle \int_0^{\theta_B}{\left(1-\frac{C_\text{A}}{C_\text{A,F}} \right)dt} \approx 3.65$

			\vspace{-4pt}
			{\tiny Ignore the tiny part missing from the integrated area.}
		\end{enumerate}
	\end{columns}
\end{frame}

\begin{frame}\frametitle{Example}
	\vspace{-12pt}
	\begin{columns}[t]
		\column{1.1\textwidth}
		\begin{enumerate}
			\setcounter{enumi}{3}
			\item	What actual fraction of the bed's capacity is used at $\theta_b$?
				\begin{center}
					\includegraphics[width=0.67\textwidth]{\imagedir/separations/adsorption/various-times-from-bed.png}
				\end{center}
				\vspace{-12pt}
				\begin{itemize}
					\item	The actual capacity used is the total shaded area = $A_1 + A_2$
					\item	This is called the {\color{purple}{stoichiometric capacity}} of the bed
					\item	Ideally, if there were no mass transfer resistance (i.e. spread in the breakthrough curve), then the
					\item	{\color{purple}{stoichiometric time}}, $\theta_S$, is defined as \textbf{time taken} for this \emph{actual capacity} to be used
					\item	$\theta_S$ is the point that breaks the MTZ into equal areas: in this case, $A_2$ \emph{vs} the unshaded area in previous diagram
					\vspace{2pt}
					\item	{\scriptsize $\theta_S = \displaystyle \int_0^{\infty}{\left(1-\frac{C_\text{A}}{C_\text{A,F}} \right)dt}$ = shaded area = $A_1 + A_2 = 3.65 + 1.55 = 5.20$ hrs}
					\item	So actual bed fraction used at $\theta_b$ is $\dfrac{5.2}{6.9}$ = 0.75 {\color{myOrange}{$\sim75\%$}}
					%$\theta_b = \dfrac{t_\text{used}}{\theta_S} = \dfrac{5.2}{6.9}$ = 0.75 {\color{myOrange}{$\sim75\%$}}
				\end{itemize}
		\end{enumerate}
	\end{columns}
\end{frame}

\begin{frame}\frametitle{Figures to help with the example}
	\begin{center}
		\includegraphics[width=\textwidth]{\imagedir/separations/adsorption/Bed-concentration-Ghosh-p144.png}
	\end{center}
	\vspace{-16pt}
	\begin{columns}[t]
		\column{1.1\textwidth}
		\begin{enumerate}
			\setcounter{enumi}{4}
			\item	If the lab-scale bed was originally 14cm long, what equivalent ``length'' is unused at time $\theta_b$?
				\begin{itemize}
					\item	intuitively: $14(1 - 0.75)$ = {\color{myOrange}{3.5 cm}}
					\item	LUB = length of unused bed = {\color{myOrange}{3.5 cm}}
					\item	LES = {\color{purple}{length of equilibrium section}} = the used up part = $14.0 - 3.5 = 10.5$~cm
				\end{itemize}
		\end{enumerate}
	\end{columns}
\end{frame}

\begin{frame}\frametitle{Example}
	\vspace{-12pt}
	\begin{columns}[t]
		\column{1.1\textwidth}
		\begin{enumerate}
			\setcounter{enumi}{5}

			\item	If we wanted a break-point time of $\theta_b = 7.5$ hours instead, how much longer should the bed be {\small (keeping the diameter and flow profile fixed)}?
			
			\begin{center}
				\begin{tabular}{c||c|r}
							&	\textbf{Current}			& \textbf{Desired} \\ \hline
					LES		&	$0.75 \times 14\,\text{cm} = 10.5$cm	&	21.6cm \\
					LUB 	&	$0.25 \times 14\,\text{cm} = \,\,\,3.5$cm	&	3.5cm\\ \hline
					Total 	&	14cm						&   25.1cm\\
				\end{tabular}
			\end{center}
			
			\begin{itemize}
				\item	Ratio LES lengths to breakthrough times: $ \dfrac{\text{LES}^\text{des}}{\text{LES}^\text{curr}} =  \dfrac{\theta_b^\text{des}}{\theta_b^\text{curr}}$
				\item	Length to get to breakthrough in 7.5 hours = $\text{LES}^\text{des} = 21.6$~cm
				\item	We have to add on the length of the unused bed = 4.1 cm from before (same diameter, same flow profile!)
				\item	So new bed length = LES + LUB = 21.6 + 4.1  = {\color{myOrange}{25.1 cm}}
				\item	LUB is the same length, provided all other conditions are the same
				\item	Then fraction actually used = $\dfrac{21.6}{24.5}$ = 0.88 {\scriptsize (compared to 0.75)}
			\end{itemize}
		\end{enumerate}
	\end{columns}
\end{frame}

\begin{frame}\frametitle{Bed mass balance}

	\begin{exampleblock}{Amount of material loaded into the bed up to $\theta_b$ in LES}
		\[Q_F \,\, C_\text{A,F} \,\, \theta_b = C_\text{A,S}^\text{e} \,\, \rho_B  \,\, A \, L_\text{LES} \]  %\left(1-\epsilon_b \right) <--- had this in previously, but if we use the bulk density, then it is accounted for already
	\end{exampleblock}
	
	\begin{tabular}{llr}\vspace{0pt}
		$Q_F $					& Feed flow rate					& $\left[ \dfrac{\text{m}^3}{\text{second}}\right]$ \\ 	\vspace{2pt}
		$C_\text{A,F} $			& Inlet concentration				& $\left[ \dfrac{\text{kg solute}}{\text{m}^3~\text{fluid}}\right]$ \\ 	\vspace{2pt}
		$\theta_b$				& Breakthrough time					& $\left[ \text{second}\right]\,$ \\ 	\vspace{2pt}
		$C_\text{A,S}^{e}$		& Eqbm adsorbed solute conc$^\text{n}$& $\left[ \dfrac{\text{kg solute}}{\text{kg adsorbent}}\right]$ \\ \vspace{2pt}
		$\rho_\text{B}$			& Adsorbent's {\color{purple}bulk density}			& $\left[ \dfrac{\text{kg adsorbent charged}}{\text{m}^3~\text{of occupied space}}\right]$ \\ 	\vspace{2pt}
		%$\epsilon_b$			& Bed voidage						& $\left[ \dfrac{\text{m$^3$ adsorbent}}{\text{m}^3~\text{of occupied space}} \right]$ \\  \vspace{2pt}
		$A L_\text{LES}$		& Bed volume = area $\times$ LES length	& $\left[ \text{m}^3~\text{of occupied space}\right]$
	\end{tabular}

	\vspace{6pt}
	{\small \color{myBlue}Add on LUB; determine volume adsorbent required = $A( L_\text{LES}+L_\text{LUB})$.}
	{\small \color{myOrange}Take porosity into account when calculating mass of adsorbent from the occupied volume.}
\end{frame}

\begin{frame}\frametitle{Modified from a previous exam}

	Trimethylethylene (TME) is being removed from an aqueous chemical plant waste stream on a \emph{continuous basis} (this is not a batch system). A bench scale system indicates that the adsorbent follows a Langmuir adsorption isotherm as:
	\[
		C_\text{A,S} = \frac{0.05C_\text{A}}{32.1 + C_\text{A}}
	\]
	where $C_\text{A,S}$ has units of [grams/grams], and the constant has units of 32.1\,ppm. In a tank we have an inlet flow of TME solution at 10L/min with density of 1000\,$\text{kg.m}^{-3}$. The TME enters at 100 ppm (parts per million, mass solute per $10^6$ mass solution) in the feed. The impurity is not detectable below 1\,ppm concentrations. The tank contains 15\,kg of initially fresh adsorbent which is retained in the tank. We wish to know:

	\begin{enumerate}
		\item	How much TME is adsorbed when the breakthrough concentration reaches 1 ppm? {\small\color{myOrange}[Ans: 22.66 g]}
		\item	How long it will take to reach this detectable outlet concentration? {\small\color{myOrange}[22.6 minutes]}
	\end{enumerate}
\end{frame}

\begin{frame}\frametitle{Regenerating the bed}
	\begin{exampleblock}{Aim}
		To remove adsorbate from the packed bed.
	\end{exampleblock}

	1: Temperature swing adsorption (TSA)
	\begin{itemize}
		\item	heat the bed: usually steam is used (due to high latent heat)
		\begin{itemize}
			\item	why add heat? (recall, heat is released during adsorption)
		\end{itemize}
		\item	creates a thermal wave through the packed bed
		\item	isotherm at higher temperature is shifted down
		\item	causes the adsorbate to be diluted in the stripping fluid
		\item	often leave some residual adsorbate behind, since time to completely strip adsorbent of it would be excessive
		\item	care must be taken with flammable adsorbates:
		\begin{itemize}
			\item	stripping temperatures are high
			\item	often near flammable limits
			\item	carbon beds have been known to catch fire
		\end{itemize}
	\end{itemize}
	\emph{See illustration on next page to understand TSA}
\end{frame}

\begin{frame}\frametitle{Regenerating the bed}
	2.	Pressure swing adsorption (PSA)
	\begin{itemize}
		\item	used when the ``product'' is the cleaned (stripped) fluid
		\item	add feed with adsorbate at high pressure (loads the adsorbate)
		\item	drop the pressure and the adsorbate starts to desorb
		\item	run two beds in parallel (one desorbing, the other adsorbing)
		\item	widely used for portable oxygen generation, $\text{H}_2\text{S}$ capture in refineries
	\end{itemize}
	\begin{center}
		\includegraphics[width=0.6\textwidth]{\imagedir/separations/adsorption/pressure-and-temperature-swing-Seader-p610.png}
	\end{center}
	\vspace{-14pt}
	\see{Seader, 3ed, p610}
\end{frame}

\begin{frame}\frametitle{Rotary devices: to avoid a separate regeneration}
	\begin{center}
		%\includegraphics[width=.7\textwidth]{\imagedir/separations/adsorption/Rotary-adsorber-Uhlmanns-p557.png}
		\includegraphics[width=\textwidth]{\imagedir/separations/adsorption/Rotary-devices-CR-v2-5ed-p1034.png}
	\end{center}
	\vspace{-12pt}
	\see{Richardson and Harker, p 1034}
	% Because of the difficulty of ensuring that the solid moves steadily and at a controlled rate with respect to the containing vessel, other equipment has been developed in which solid and vessel move together, relative to a fixed inlet for the feed and a fixed outlet for the product. Figure 17.29 shows the principle of operation of a rotary-bed adsorber used, for example, for solvent recovery from air on to activated-carbon. The activated-carbon is contained in a thick annular layer, divided into cells by radial partitions. Air can enter through most of the drum circumference and passes through the carbon layer to emerge free of solvent. The clean air leaves the equipment through a duct connected along the axis of rotation. As the drum rotates, the carbon enters a section in which it is exposed to steam. Steam flows from the inside to the outside of the annulus so that the inner layer of carbon, which determines the solvent content of effluent air, is regenerated as thoroughly as possible. Steam and solvent pass to condensers and the solvent recovered, either by decanting or by a process such as distillation. In the particular equipment shown, there is no separate provision for cooling the regenerated adsorbent; instead it is allowed to cool in contact with vapour-laden air and the adsorptive capacity may be lower as a result(16).	
\end{frame}

\begin{frame}\frametitle{Adsorption equipment: Sorbex column}
	{\scriptsize Bed remains stationary (minimizes adsorbent damage); fluid phase is pumped around.}
	\begin{center}
		\includegraphics[width=\textwidth]{\imagedir/separations/adsorption/Sorbex-Uhlmanns-p560.png}
	\end{center}
	\vspace{-12pt}
	\see{Uhlmanns, p 560}

	{\tiny a) Pump; b) Adsorbent chamber; c) \href{http://www.uop.com/products/equipment/aromatics-separation/}{Rotary valve}; d) Extract column; e) Raffinate column}

	% Over 100 commercial installations
	% It simulates counter-current operation,since true counter-current operation is not possible without axial mixing and damage to the adsorbent, this approach works very well.
	
	% The concentration profile shown in the figure actually moves down the adsorbent chamber. As the concentration profile moves, the points of injection and withdrawal of the net streams are moved along with it. This movement of the net streams is performed with a unique rotary valve developed by UOP specifi- cally for the Sorbex family of processes.

	% The flow distribution at each stage is achieved with specialized internals and grids, which are used to inject or withdraw liquid via the bed lines connecting the beds to the rotary valve. A typical unit has 12 beds and, in order to double the number of stages, two colums are then in series. At any given time, only four of the bed lines are active in carrying the net streams into and out of the adsorber. The rotary valve is used to periodically switch the position of the liquid feed and withdrawal points as the composition profile moves down the apparatus. It also sends the diluted extract and raffinate to the distillation columns and reprocesses the desorbent. A circulating pump provides a liquid flow from the bottom to the top of the column.


	% From Rousseau:

	% The problems associated with moving-bed processes have been solved by the development of the simulated moving bed.
	% This process, the bed is maintained stationary, and the feed and the displacement-liquid inlets and the two product outlets are moved as a function of time. In addition, the displacement liquid is continuously circulated by a pump from the bottom to the top of the bed. The most widely used version of this scheme is the Sorbex process, developed by Universal Oil Products, Inc. (UOP). A schematic diagram is given in Fig. 12.5-10. The lines shown as open are 2, 5, 9, and 12. After a short period of time each of the first three streams is moved to the next-higher-number point by the rotary valve. The less-adsorbed product line is switched from 12 to 1. During subsequent time periods the four lines are moved in the same manner. A more detailed description is given in Ref. 26.
	% Other versions of the simulated moving-bed process have been commercialized by Toray Industries,
	% This process can also be thought of as a variation of a displacement-purge cycle. In this
	% 27 28 29 Inc. ' and Mitsubishi Chemical Industries, Ltd.
	% These processes vary from the Sorbex technology in
	% details rather than in their basic concept.
	% Although simulated moving-bed processes have now reached wide commercial acceptance for a number
	% of separations, the need for a displacement liquid, which must subsequently be distilled from both product streams, constitutes a process complexity and an economic hindrance. For this reason, applications of these processes have been confined to those separations that are difficult or impossible to make by distillation.
	% The separation of fructose and glucose via simulated moving-bed adsorption is an example, like IsoSiv in the Total Isomerization Process for isoparaffin production, of the close coupling of reaction and separation systems. Glucose, which can be obtained from hydrolysis of starch, is isomerized to a near-equilibrium mixture of fructose and glucose (high-fructose corn syrup, HFCS). The resulting mixture can then be further enriched in fructose while also producing a glucose stream for recycle to isomerization. HFCS, because it rivals sucrose (cane sugar) in sweetness, is finding rapidly growing uses in beverages, candies, baking products, and so on.
\end{frame}

