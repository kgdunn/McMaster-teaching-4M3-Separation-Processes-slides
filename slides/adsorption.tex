References

* Wankat, Separation Process Engineering, Chapter 16
* Perry's, section 22

Perry's section 22.5.8

Adsorption This is the most widely used of the physical-chemical treatment processes. It is used primarily for the removal of soluble organics with activated carbon serving as the adsorbent. Most liquid-phase-activated carbon adsorption reactions follow a Freundlich Isotherm [Eq. (22-25)].

y = kc ^\frac{1/m}

(22-25)

where:            y = adsorbent capacity, mass pollutant/mass carbon
                      c = concentration of pollutant in waste, mass/volume
             k and m = empirical constants
EPA has compiled significant data on values of k and n for environmentally significant pollutants with typical activated carbons. Assuming equilibrium is reached, the isotherm provides the dose of carbon required for treatment. In a concurrent contacting process, the capacity is set by the required effluent concentration. In a countercurrent process, the capacity of the carbon is set by the untreated waste pollutant concentration. Thus countercurrent contacting is preferred.

Activated carbon is available in powdered form (200–400 mesh) and granular form (10–40 mesh). The latter is more expensive but is easier to regenerate and easier to utilize in a countercurrent contactor. Powdered carbon is applied in well-mixed slurry-type contactors for detention times of several hours after which separation from the flow occurs by sedimentation. Often coagulation, flocculation, and filtration are required in addition to sedimentation. As it is difficult to regenerate, powdered carbon is usually discarded after use. Granular carbon is used in column contactors with EBCT of 30 minutes to 1 hour. Often several contactors are used in series, providing for full countercurrent contact. A single contactor will provide only partial countercurrent contact. When a contactor is exhausted, the carbon is regenerated either by a thermal method or by passing a solvent through the contactor. For waste-treatment applications where a large number of pollutants must be removed but the quantity of each pollutant is small, thermal regeneration is favored. In situations where a single pollutant in large quantity is removed by the carbon, solvent extraction regeneration can be used, especially where the pollutant can be recovered from the solvent and reused. Thermal regeneration is a complex operation. It requires removal of the carbon from the contactor, drainage of free water, transport to a furnace, heating under controlled conditions of temperature, oxygen, time, water vapor partial pressure, quenching, transport back to the reactor, and reloading of the column. Five to ten percent of the carbon is lost in this regeneration process due to burning and attrition during each regeneration cycle. Multiple hearth, rotary kiln, and fluidized bed furnaces have all been successfully used for carbon regeneration.

Pretreatment prior to carbon adsorption is usually for removal of suspended solids. Often this process is used as tertiary treatment after primary and biological treatment. In either situation, the carbon columns must be designed to provide for backwash. Some solids will escape pretreatment, and biological growth will occur on the carbon, even with extensive pretreatment. Originally carbon treatment was viewed only as applicable for removal of toxic organics or those that are difficult to degrade biologically. Present practice applies carbon adsorption as a procedure for removal of all types of organics. It is realized that some biological activity will occur in virtually any activated carbon unit, so the design must be adjusted accordingly.