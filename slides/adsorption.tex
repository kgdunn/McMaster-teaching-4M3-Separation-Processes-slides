\begin{comment}
	Perry's section 22.5.8

	EPA has compiled significant data on values of k and n for environmentally significant pollutants with typical activated carbons. Assuming equilibrium is reached, the isotherm provides the dose of carbon required for treatment. In a concurrent contacting process, the capacity is set by the required effluent concentration. In a countercurrent process, the capacity of the carbon is set by the untreated waste pollutant concentration. Thus countercurrent contacting is preferred.

	Activated carbon is available in powdered form (200–400 mesh) and granular form (10–40 mesh). The latter is more expensive but is easier to regenerate and easier to utilize in a countercurrent contactor. Powdered carbon is applied in well-mixed slurry-type contactors for detention times of several hours after which separation from the flow occurs by sedimentation. Often coagulation, flocculation, and filtration are required in addition to sedimentation. As it is difficult to regenerate, powdered carbon is usually discarded after use. Granular carbon is used in column contactors with EBCT of 30 minutes to 1 hour. Often several contactors are used in series, providing for full countercurrent contact. A single contactor will provide only partial countercurrent contact. When a contactor is exhausted, the carbon is regenerated either by a thermal method or by passing a solvent through the contactor. For waste-treatment applications where a large number of pollutants must be removed but the quantity of each pollutant is small, thermal regeneration is favored. In situations where a single pollutant in large quantity is removed by the carbon, solvent extraction regeneration can be used, especially where the pollutant can be recovered from the solvent and reused. Thermal regeneration is a complex operation. It requires removal of the carbon from the contactor, drainage of free water, transport to a furnace, heating under controlled conditions of temperature, oxygen, time, water vapor partial pressure, quenching, transport back to the reactor, and reloading of the column. Five to ten percent of the carbon is lost in this regeneration process due to burning and attrition during each regeneration cycle. Multiple hearth, rotary kiln, and fluidized bed furnaces have all been successfully used for carbon regeneration.

	Pretreatment prior to carbon adsorption is usually for removal of suspended solids. Often this process is used as tertiary treatment after primary and biological treatment. In either situation, the carbon columns must be designed to provide for backwash. Some solids will escape pretreatment, and biological growth will occur on the carbon, even with extensive pretreatment. Originally carbon treatment was viewed only as applicable for removal of toxic organics or those that are difficult to degrade biologically. Present practice applies carbon adsorption as a procedure for removal of all types of organics. It is realized that some biological activity will occur in virtually any activated carbon unit, so the design must be adjusted accordingly.
\end{comment}

% 05 November 2012 (half a class only)
\begin{frame}\frametitle{Introduction to \textbf{{\color{purple}{sorption}}} processes}
	\begin{exampleblock}{\textbf{Sorption}}
		Components in a fluid phase, {\color{myGreen}{solutes}}, are selectively transferred to insoluble, (rigid) particles that are suspended in a vessel or packed in a column.
	\end{exampleblock}
	\begin{itemize}
		\item	{\color{purple}{(ad)sorbate}}: the (ad)sorbed solute that's usually of interest
		\item	{\color{purple}{(ad)sorbent}}: the (ad)sorbing agent, i.e. the MSA
		\item	Is there an ESA?
	\end{itemize}
	
	\vspace{12pt}
	\emph{Some sorption processes}:
	\begin{itemize}
		\item	a\textbf{b}sorption: gas into liquid phase [it is strictly speaking a sorption process, but not considered here (3M4)]
		\item	{\color{purple}{adsorption}}: molecules bond with a solid surface
		\item	{\color{purple}{ion-exchange}}: ions displace dissimilar ions from solid phase
			\begin{itemize}
				\item	Water softening: $\text{Ca}^{2+}_\text{(aq)} + 2\text{NaR}_\text{(s)} \longleftrightarrow \text{CaR}_{2\text{(s)}} + 2\text{Na}^{+}_\text{(aq)}$
			\end{itemize}
		\item	{\color{purple}{chromatography}}: solutes move through column with an eluting fluid. Column is continuously regenerated.
	\end{itemize}	
\end{frame}

\begin{frame}\frametitle{Sorption examples}
	\begin{exampleblock}{}
		We will focus on {\color{myGreen}{(ad)sorption}} for the next few classes.
	\end{exampleblock}
	
	\vspace{12pt}
	Some well-known examples:
	\begin{itemize}
		\item	adsorption: charred wood products to improve water taste
		\item	adsorption: decolourize liquid with bone char
		\item	adsorption: those little white packets in boxes of electronics
		\item	ion-exchange: passing water through certain sand deposits removes salt
		\item	ion-exchange: synthetic polymer resins widely used to soften water
	\end{itemize}
	
	Industrial use of adsorption picked up with molecular zeolites in the 1960s
\end{frame}

\begin{frame}\frametitle{Adsorbents}
	General principle (more details coming up soon)
	\begin{columns}[c]
		\column{0.60\textwidth}
			\begin{center}
				\includegraphics[width=\textwidth]{\imagedir/separations/adsorption/Adsorbent-pores-Seader-p569-3ed-mod.png}
			\end{center}
		\column{0.45\textwidth}
			Molecules attach to the particle's surfaces: outside and on the pore walls
			
			\vspace{12pt}
			Main characterization: {\color{purple}{pore diameter}} of adsorbent
			
			\vspace{12pt}
			Mechanisms during adsorption
				\begin{itemize}
				\item	{\color{purple}{equilibrium interaction}}: solid-fluid interactions
				\item	{\color{purple}{kinetic}}: differences in diffusion rates
				\item	{\color{purple}{steric}}: pore structure hinders/retains molecules of a certain shape
			\end{itemize}
	\end{columns}
	\see{Modified from: Seader, 3ed, p 569}
\end{frame}

\begin{frame}\frametitle{Quick recap of some familiar concepts}
	\begin{itemize}
		\item	1m = 100cm = 1000mm = $10^6$\micron = $10^9$nm = $10^{10}$\AA
		\item	Hydrogen and helium atoms: $\sim 1$\AA
		\item	For a pore: \[\frac{\text{Internal surface area}}{\text{Pore volume}} = \frac{\pi d_p L}{\pi d_p^2 L/4} = \frac{4}{d_p}\]
		\item	$d_p$ = pore diameter: typically around 10 to 200 \AA
	\end{itemize}
\end{frame}

\begin{frame}\frametitle{Adsorbents}
	Helpful to see what they look like to understand the principles:	
	\begin{columns}[c]
		\column{0.60\textwidth}
			\begin{center}
				\includegraphics[width=0.9\textwidth]{\imagedir/separations/adsorption/Active_Al2O3-Wikipedia.png}
			\end{center}
			\vspace{-12pt}
			\see{\href{http://en.wikipedia.org/wiki/File:Active_Al2O3.jpg}{Wikipedia}}
		\column{0.40\textwidth}
			{\color{myGreen}{Activated alumina}}
			
			\begin{itemize}
				\item	made from from aluminum hydroxide
				\item	$\sim 300~\text{m}^2$ per gram
				\item	most widely used adsorbent
				\item	hydrophilic
				\item	pore diameter: 10 to 75 \AA
			\end{itemize} 
	\end{columns}
\end{frame}

\begin{frame}\frametitle{Adsorbents}
	\begin{columns}[c]
		\column{0.50\textwidth}
			\begin{center}
				\includegraphics[width=0.9\textwidth]{\imagedir/separations/adsorption/DOI-{10.1016}{j.saa.2011.10.012}1-s2.0-S1386142511009012-gr3.jpg}
			\end{center}
			\vspace{-12pt}
			\see{\href{http://dx.doi.org/10.1016/j.saa.2011.10.012}{DOI:10.1016/j.saa.2011.10.012}}
		\column{0.50\textwidth}
			{\color{myGreen}{Activated carbon}}
			
			\begin{itemize}
				\item	partially oxidized coconut shells, nuts, wood, peat, bones, sewage sludge
				\item	difference hardnesses of adsorbent
				\item	400 to 1200 $\text{m}^2$ per gram
				\item	hydrophobic
				\item	pore diameter:  10 to over 50\AA 
			\end{itemize} 
			e.g. bone char: decolourizing syrups % Article by Johnston, Green book, "Designing Fixed-Bed Adsorption Columns.pdf"
	\end{columns}
\end{frame}

\begin{frame}\frametitle{Adsorbents}
	\begin{columns}[t]
		
		\column{0.60\textwidth}
			\vspace{-24pt}
			\begin{center}
				\includegraphics[width=0.9\textwidth]{\imagedir/separations/adsorption/Zeolites-Seader-3ed-p575.png}
			\end{center}
			\see{Seader, 3ed, p575}
			
			\see{Uhlmanns, p565}
			\vspace{-4pt}
			\begin{center}
				\includegraphics[width=\textwidth]{\imagedir/separations/adsorption/Zeolites-Uhlmanns-p565.png}
			\end{center}
		\column{0.40\textwidth}
			{\color{myGreen}{Zeolite lattices}}
			
			\vspace{12pt}
			\emph{Some examples}
			
			\vspace{6pt}
			$\text{K}_{12}[(\text{AlO}_2)_{12}(\text{SiO}_2)_{12}]$: 
			
			\hfill {\color{myOrange}{drying gases  [2.9\AA]}}
			
			\vspace{12pt}			
			$\text{Na}_{12}[(\text{AlO}_2)_{12}(\text{SiO}_2)_{12}]$: 
			
			\hfill {\color{myOrange}{$\text{CO}_2$ removal  [3.8\AA]}}
			
			
			\vspace{12pt}			
			$\text{Ca}_{43}[(\text{AlO}_2)_{86}(\text{SiO}_2)_{106}]$: 
			
			\hfill {\color{myOrange}{air separation [8\AA]}}
			
			
			\vspace{24pt}
			Very specific pore diameters.
			\begin{itemize}
				\item	40 naturally occurring
				\item	$\sim$  150 synthesized % C+R v2, 5ed, p975
			\end{itemize}
		
	\end{columns}
\end{frame}

% 07 November 2012
\begin{frame}\frametitle{Adsorbents}
	Molecular sieves (zeolites):
	
	\begin{tabular}{cll}\\
		\textbf{Size}	&	\textbf{Adsorbs} ... 					&	\textbf{Dehydrates} ...\vspace{6pt}\\ \hline 
		3\AA			&	$\text{H}_2\text{O}$, $\text{NH}_3$		& 	unsaturated hydrocarbons\\
		4\AA			&  	$\text{H}_2\text{S}$, $\text{CO}_2$, $\text{C}_3\text{H}_6$		& 	saturated hydrocarbons\\
		5\AA			&	\emph{n}-paraffins from iso-paraffins   & \\
		8\AA			& 	iso-paraffins and olefins				&	\\
	\end{tabular}
	\vspace{12pt}
	\see{Johnston}%, Green book, "Designing Fixed-Bed Adsorption Columns.pdf"

	\vspace{12pt}
	% Rousseau "21FC8B40D9C71DB16CAE2B81BA3244.pdf"
	Rousseau, ``Handbook of Separation Technology'': {\scriptsize market sizes in 1983}

	\begin{tabular}{ll}\\
		\textbf{Adsorbent}				&	\textbf{Market size} \vspace{6pt}\\ \hline 
			Activated carbon 			&	\$ 380 million $\longleftarrow$ 25\% for water treatment \\
			Molecular-sieve zeolites 	&	\$ 100 million  \\
			Silica gel					&	\$ 27 million  \\
			Activated alumina			&	\$ 26 million
	\end{tabular}
\end{frame}

\begin{frame}\frametitle{Pore diameter characterization}
	\begin{center}
		\includegraphics[width=0.9\textwidth]{\imagedir/separations/adsorption/Pore-size-distribution-Seader-3ed-p574.png}
	\end{center}
	\see{Seader, 3ed, p574}
	{\scriptsize Determined using He and Hg porosimetry (see reference for details)}
\end{frame}

\begin{frame}\frametitle{Adsorption examples}
	\begin{itemize}
		\item	Gas {\color{purple}{purification}}:
		\begin{itemize}
			\item	Volatile organics from a vent stream
			\item	Sulphur compounds from gas stream
			\item	Water vapour (we'll look at pressure swing adsorption)
			\item	Removal of $\text{CO}_2$ from natural gas [alternatives ?]
		\end{itemize}
		\item	{\color{purple}{Bulk separation}} in the gas phase:
		\begin{itemize}
			\item	$\text{O}_2$ from $\text{N}_2$ (adsorbed more strongly onto zeolites) 
			\item	$\text{H}_2\text{O}$ from ethanol
			\item	High acetone quantities from air vent streams
		\end{itemize}
		\item	Liquid-liquid separation and purification:
		\begin{itemize}
			\item	Organics and toxic compounds from water
			\item	Sulphur compounds from water
			\item	Normal vs iso-paraffin separation
			\item	Separation of isomers: \emph{p}$\,$- vs \emph{m}-cresol
			\item	Fructose from dextrose separation
			\item	Gold in cyanide solutions
		\end{itemize}
		\vspace{-2.5cm}
		\begin{columns}[t]
			\column{0.75\textwidth}				
			\column{0.20\textwidth}
				\begin{center}
					\includegraphics[width=0.8\textwidth]{\imagedir/separations/adsorption/P-cresol-spaceFilling.png}
				\end{center}
			\column{0.20\textwidth}
				\begin{center}
					\includegraphics[width=\textwidth]{\imagedir/separations/adsorption/M-cresol-spaceFill.png}					
				\end{center}				
		\end{columns}
		\begin{columns}[t]
			\column{0.75\textwidth}				
			\column{0.20\textwidth}
				\begin{center}
					\emph{p}-cresol
				\end{center}
			\column{0.20\textwidth}
				\begin{center}
					\emph{m}-cresol
				\end{center}
		\end{columns}
		\hfill\see{Cresol figures from Wikipedia}
		
	\end{itemize}	
\end{frame}

\begin{frame}\frametitle{Gold leaching and adsorption}	
	\begin{columns}[c]
		\column{1.08\textwidth}			
			\begin{itemize}
				\item	Crushed rock has gold particles exposed
				\item	Leaching: $4\text{Au} + 8\text{NaCN} + \text{O}_2 + 2 \text{H}_2\text{O} \longrightarrow  4 \text{Na[Au(CN)}_2\text{]} + 4 \text{NaOH}$
				\item	Adsorption: aurocyanide complex, $\text{Au(CN)}_2^{-}$, is adsorbed onto activated carbon
					\begin{itemize}
						\item	drives the equilibrium in the leaching step forward
						\item	separates the gold from the pulp (slurry)
						\item	obtain $C_\text{A,s}$ = 8000 grams of Au per tonne of carbon
					\end{itemize}
			\end{itemize}			
	\end{columns}
	\begin{columns}[t]
		\column{0.60\textwidth}
			\begin{itemize}
				\item	Desorption: 
					\begin{itemize}
						\item	separate the highly concentrated gold-carbon pulp (screens/filter)
						\item	desorb the gold off the carbon with caustic contact
						\item	recycle the regenerated carbon
					\end{itemize}
			\end{itemize}
		\column{0.40\textwidth}
			\begin{center}
				\includegraphics[width=\textwidth]{\imagedir/separations/adsorption/Batch-adsorption-Seader-3ed-p610.png}
			\end{center}
	\end{columns}		
\end{frame}

\begin{frame}\frametitle{When to consider adsorption}
	Distillation, membranes, absorption, liquid-liquid extraction are sometimes viable alternatives. 
	
	\vspace{12pt}
	But adsorption is considered when:
	\begin{itemize}
		\item	relative volatility between components is $<1.5$ (e.g. isomers)
		\item	large reflux ratios would be required
		\item	excessive temperatures or high pressure drops are to be avoided
		\item	too large area for a membrane
		\item	high selectivity is required
		\item	feed is a very dilute stream of solute (adsorbate) 
	\end{itemize}
	But, some disadvantages:
	\begin{itemize}
		\item	only the surface of the adsorbent used
		\item	regeneration of MSA adsorbent required
		\item	MSA will break down mechanically over time
		\begin{itemize}
			\item	{\color{myOrange}{we can pump it, filter it, and/or put it through cyclones}}
		\end{itemize}
	\end{itemize}
\end{frame}

\begin{frame}\frametitle{Quantifying the adsorbent}
	\seefull{Perry's, Ch 22}: A fixed bed of porous adsorbent material. Bulk density is 500 kg.m$^{-3}$, and the {\color{purple}{interparticle}} [between] void fraction is 0.40. The {\color{purple}{intraparticle}} [within] porosity is 0.50, with two-thirds of this in cylindrical pores of diameter 1.4 nm and the rest in much larger pores. \textbf{Find}:
	\begin{itemize}
		\item	surface area of the adsorbent
		\item	if solute has formed a complete {\color{purple}{monomolecular layer}} 0.3 nm thick inside the pores, determine the percent of the particle volume and the percent of the total bed volume filled with adsorbate.
	\end{itemize}
	
	\vspace{12pt}
	\emph{Solution}:
	Assume from surface area to volume ratio that the internal area is practically all in the small pores [ignore large pores]. 
	One gram of the adsorbent occupies 2cm$^3$ as packed and has 0.4cm$^3$ in small pores, which gives a surface area of 1150 m$^2$/gram {\scriptsize (university stadium field area $\sim 5000~\text{to}~8000 \text{m}^2$)}. 
	Based on the area of the annular region filled with adsorbate, the solute occupies 22.5\% of internal pore volume and 13.5\% of the total packed-bed volume.	
\end{frame}

\begin{frame}\frametitle{Physical principles}
	\begin{exampleblock}{Adsorption releases heat. Why?}
		\pause
		\iftoggle{instructor}{Loss of degrees of freedom of fluid: free energy is reduced, so $\Delta S \downarrow$}{Thermodynamics: ...}
		
		\iftoggle{instructor}{$\Delta G = \Delta H - T \Delta S \quad \Longrightarrow  \quad \Delta H = \Delta G + T \Delta S \quad \Longrightarrow  \quad \Delta H < 0$}{\vspace{18pt}}
	\end{exampleblock}
	
	\vspace{6pt}
	Two types of adsorption:
	\vspace{6pt}
	\begin{enumerate}
		\item	Physical adsorption ({\color{purple}{physisorption}}):
		\begin{itemize}
			\item	Low heat of adsorption released: $\Delta H_\text{ads} \sim$ 30 to 60 kJ/mol
			\item	Theory: {\color{purple}{van der Waals attractions}} 
			\item	easily reversible
		\end{itemize}
		\item	Chemical adsorption ({\color{purple}{chemisorption}}):
		\begin{itemize}
			\item	High heat of adsorption released: $\Delta H_\text{ads} > 100$ kJ/mol
			\item	chemical bond formation, in the order of chemical bond strengths
			\item	leads to reaction products 
			\item	more energy intensive to reverse
			\item	e.g.: catalysis, corrosion
		\end{itemize}
	\end{enumerate}
	\pause
	As adsorbate concentration increases:
	\begin{itemize}
		\item	single layers form, then multiple layers, then condensation
	\end{itemize}
\end{frame}

\begin{frame}\frametitle{Adsorption equipment}	
	Adsorption, Desorption and Recovery (ADR) plant in Burkina Faso
	\begin{center}
		\includegraphics[width=0.85\textwidth]{\imagedir/separations/adsorption/flickr-5043854546_4ff634cc5a_b.jpg}
	\end{center}
	\see{\href{http://www.flickr.com/photos/isurusen/5043854546/}{Flickr \#5043854546}}
\end{frame}

\begin{frame}\frametitle{Fluidized beds}
	\begin{columns}[t]
		\column{0.80\textwidth}
			\begin{center}
				\includegraphics[height=0.8\textheight]{\imagedir/separations/adsorption/Fluidized-bed-Uhlmanns-p556.png}
			\end{center}
			\vspace{-18pt}
			\see{Uhlmanns, p556}
		\column{0.30\textwidth}
			Materials of construction are important
			
			\vspace{12pt}
			Cyclones used to recover adbsorbent
	\end{columns}
	
\end{frame}

\begin{frame}\frametitle{Packed beds: adsorption and desorption steps}
	\begin{center}
		\includegraphics[width=\textwidth]{\imagedir/separations/adsorption/Packed-bed-CR-v2-5ed-p1028-1.png}
	\end{center}
	\see{Richardson and Harker, p 1028}
	
	Regeneration: puts the adsorbate in a new solvent or medium, usually at a higher concentration than it was at originally, or at least in a medium that is much easier to separate from.
\end{frame}

\begin{frame}\frametitle{Rotary devices}
	\begin{center}
		%\includegraphics[width=.7\textwidth]{\imagedir/separations/adsorption/Rotary-adsorber-Uhlmanns-p557.png}
		\includegraphics[width=\textwidth]{\imagedir/separations/adsorption/Rotary-devices-CR-v2-5ed-p1034.png}
	\end{center}
	\vspace{-12pt}
	\see{Richardson and Harker, p 1034}
\end{frame}

\begin{frame}\frametitle{Adsorption equipment: Sorbex column}	
	\begin{center}
		\includegraphics[width=\textwidth]{\imagedir/separations/adsorption/Sorbex-Uhlmanns-p560.png}
	\end{center}
	\see{Uhlmanns, p 560}
	
	{\scriptsize a) Pump; b) Adsorbent chamber; c) Rotary valve; d) Extract column; e) Raffinate column}
	
	% From Rousseau:
	
	% The problems associated with moving-bed processes have been solved by the development of the simulated moving bed.
	% This process, the bed is maintained stationary, and the feed and the displacement-liquid inlets and the two product outlets are moved as a function of time. In addition, the displacement liquid is continuously circulated by a pump from the bottom to the top of the bed. The most widely used version of this scheme is the Sorbex process, developed by Universal Oil Products, Inc. (UOP). A schematic diagram is given in Fig. 12.5-10. The lines shown as open are 2, 5, 9, and 12. After a short period of time each of the first three streams is moved to the next-higher-number point by the rotary valve. The less-adsorbed product line is switched from 12 to 1. During subsequent time periods the four lines are moved in the same manner. A more detailed description is given in Ref. 26.
	% Other versions of the simulated moving-bed process have been commercialized by Toray Industries,
	% This process can also be thought of as a variation of a displacement-purge cycle. In this
	% 27 28 29 Inc. ' and Mitsubishi Chemical Industries, Ltd.
	% These processes vary from the Sorbex technology in
	% details rather than in their basic concept.
	% Although simulated moving-bed processes have now reached wide commercial acceptance for a number
	% of separations, the need for a displacement liquid, which must subsequently be distilled from both product streams, constitutes a process complexity and an economic hindrance. For this reason, applications of these processes have been confined to those separations that are difficult or impossible to make by distillation.
	% The separation of fructose and glucose via simulated moving-bed adsorption is an example, like IsoSiv in the Total Isomerization Process for isoparaffin production, of the close coupling of reaction and separation systems. Glucose, which can be obtained from hydrolysis of starch, is isomerized to a near-equilibrium mixture of fructose and glucose (high-fructose corn syrup, HFCS). The resulting mixture can then be further enriched in fructose while also producing a glucose stream for recycle to isomerization. HFCS, because it rivals sucrose (cane sugar) in sweetness, is finding rapidly growing uses in beverages, candies, baking products, and so on.
	
	
\end{frame}

\begin{frame}\frametitle{TO INCLUDE}
	/Users/kevindunn/Sync/Figures/separations/adsorption/Screen Shot 2012-11-05 at 17.20.09 .png
	/Users/kevindunn/Sync/Figures/separations/adsorption/Screen Shot 2012-11-05 at 17.19.36 .png
	/Users/kevindunn/Sync/Figures/separations/adsorption/Screen Shot 2012-11-05 at 17.19.22 .png
	/Users/kevindunn/Sync/Figures/separations/adsorption/Screen Shot 2012-11-05 at 17.18.11 .png
	/Users/kevindunn/Sync/Figures/separations/adsorption/Screen Shot 2012-11-05 at 17.17.27 .png
\end{frame}

\begin{frame}\frametitle{Modelling the adsorption process}
	\begin{enumerate}
		\item	Diffusion
			\begin{itemize}
				\item	diffusion of the adsorbate in the bulk fluid (usually very fast)
				\item	diffusion of the adsorbate to the adsorbent surface
				\item	diffusion of the adsorbate into the pore to an open site
				\begin{itemize}
					\item	steric (shape) effects
				\end{itemize}
			\end{itemize}
		\item	Equilibrium
			\begin{itemize}
				\item	adsorbate will attach to a vacant site
				\item	adsorbate will detach from an occupied site
			\end{itemize}
	\end{enumerate}
	\todo{Picture here}
\end{frame}

\begin{frame}\frametitle{Equilibrium modelling}
	\begin{exampleblock}{Why?}
		We ultimately would like to determine \textbf{how much adsorbent is required} to remove a given amount of adsorbate (e.g. impurity); particularly in batch processes.
	\end{exampleblock}

	\vspace{12pt}
	For now, assume we are only limited by equilibrium (i.e diffusion resistance is negligible)	
	\begin{itemize}
		\item	Derive/Postulate a model relating bulk concentration to surface concentration of adsorbate
		\item	We call these equations/models: ``isotherms''
		\item	{\color{purple}{Isotherm}}: relates amount of adsorbate on adsorbent at different concentrations, but at a fixed temperature.
	\end{itemize}
\end{frame}

\begin{frame}\frametitle{Equilibrium modelling: linear model}
	\begin{exampleblock}{Linear isotherm (Henry's law)}
		\[C_\text{A,s} = K C_\text{A}\]
		\[C_\text{A,s} = \frac{K P_\text{A}}{RT} = K' P_\text{A}\]
	\end{exampleblock}
	\begin{columns}[t]
		\column{1.1\textwidth}
			\begin{itemize}
				\item	{\small $C_\text{A,s} =$ concentration of adsorbate A on adsorbent surface \hfill $\left[\displaystyle \frac{\text{kg adsorbate}}{\text{kg adsorbent}}\right]$}
				\item	{\small $C_\text{A} =$ concentration of adsorbate A in the bulk fluid phase \hfill $\left[\displaystyle \frac{\text{kg adsorbate}}{\text{m}^3~ \text{fluid}}\right]$}
				\item	{\small $P_\text{A} =$ partial pressure of adsorbate A in the bulk fluid phase \hfill $\left[ \text{atm}\right]\,$}
				\item	{\small $K$ and $K'$ are temperature dependent equilibrium constants {\tiny (should be clear why)}}
				\item	{\small $R$ is the ideal gas constant}
				\item	{\small $T$ is the system temperate}
			\end{itemize}
	\end{columns}
	\vspace{12pt}
	\begin{itemize}
		\item	Few systems are this simple!
	\end{itemize}
\end{frame}

\begin{frame}\frametitle{Equilibrium modelling: Freundlich model}
	\begin{exampleblock}{Freundlich}
		\[C_\text{A,S} = K \left(C_\text{A}\right)^{1/m}\qquad \text{for $ 1 < m < 5$}\] 
	\end{exampleblock}

	\vspace{12pt}
	\begin{itemize}
		\item	It is an empirical model, but it works well
	\end{itemize}
	\begin{center}
		\includegraphics[width=0.5\textwidth]{\imagedir/separations/adsorption/Freundlich-isotherm.png}
	\end{center}
	
	\begin{itemize}
		\item	Constants determined from a log-log plot
		\item	How would you go about setting up a lab experiment to collect data to calculate $K$?
		\item	Which way will the isotherm shift if temperature is increased?
	\end{itemize}
\end{frame}

\begin{frame}\frametitle{Equilibrium modelling: Langmuir isotherm}
	% Derivation from Fogler, p 422, Chapter 10
	\begin{itemize}
		\item	we have a uniform adsorbent surface available {\tiny (all sites equally attractive)}
		\item	there are a total number of sites available for adsorbate A to adsorb to
		\item	$C_\text{T}$ = total sites available \hfill {\scriptsize $\left[\displaystyle\frac{\text{mol sites}}{\text{kg adsorbate}} \right]$}
		\pause
		\item	$C_\text{V}$ = vacant sites available \hfill {\scriptsize $\left[\displaystyle\frac{\text{mol sites}}{\text{kg adsorbate}} \right]$}
		\item	rate of adsorption = $k_\text{A} P_\text{A} C_\text{V}$ = proportional to number of collisions of A with site S
		\pause
		\item	$C_\text{A,S}$ = sites occupied by A \hfill {\scriptsize $\left[\displaystyle\frac{\text{mol sites}}{\text{kg adsorbate}} \right]$}
		\item	assuming 1 site per molecule of A, and only a monolayer forms
		\item	rate of desorption=  $k_{-\text{A}} C_\text{A,S}$ = proportional to number of occupied sites
		\item	net rate = $k_\text{A} P_\text{A} C_\text{V} - k_{-\text{A}} C_\text{A,S}$  
	\end{itemize}
\end{frame}

\begin{frame}\frametitle{Equilibrium modelling: Langmuir isotherm}
	\begin{itemize}
		\item	Net rate = $k_\text{A} P_\text{A} C_\text{V} - k_{-\text{A}} C_\text{A,S}$  
		\item	define $K_\text{A} = \displaystyle\frac{k_\text{A}}{k_{-\text{A}}}$
		\item	essentially an equilibrium constant: $\text{A} + \text{S} \rightleftharpoons \text{A}\cdot \text{S}$		
		\item	at equilibrium, the net rate is zero 
		\pause
		\item	implying $\displaystyle\frac{k_\text{A}C_\text{A,S}}{K_\text{A}} = k_\text{A} P_\text{A} C_\text{V}$  
		\item	but total sites = $C_\text{T} = C_\text{V} + C_\text{A,S}$
		\item	so $\displaystyle\frac{k_\text{A}C_\text{A,S}}{K_A} = k_\text{A} P_\text{A} \left(C_\text{T} -  C_\text{A,S}\right)$  
		\item	simplifying: $C_\text{A,S} = K_\text{A} P_\text{A} \left(C_\text{T} -  C_\text{A,S}\right)$  
		\item	then \fbox{$C_\text{A,S} = \displaystyle \frac{K_\text{A}C_\text{T} P_\text{A}}{1 + K_\text{A} P_\text{A}} = \displaystyle \frac{K_1 P_\text{A}}{1 + K_2 P_\text{A}} = \displaystyle \frac{K_3 C_\text{A}}{1 + K_4 C_\text{A}}$}
		\item	Fit data using \href{http://en.wikipedia.org/wiki/Eadie\%E2\%80\%93Hofstee\_plot}{Eadie-Hofstee diagram} or nonlinear regression
		\item	Same structure as Michaelis-Menten model (bio people)
	\end{itemize}
\end{frame}

\begin{frame}\frametitle{Summary of isotherms}
	We aren't always sure which isotherm fits a given adsorbate-adsorbent pair:
	\begin{enumerate}
		\item	Postulate a model (e.g. linear, or Langmuir)
		\item	Perform a laboratory experiment to collect the data
		\item	Fit the model to the data
		\item	Good fit? 
	\end{enumerate}
	
	\vspace{24pt}
	Other isotherms have been proposed:
	\begin{itemize}
		\item	BET (Brunauer, Emmett and Teller) isotherm
		\item	Gibb's isotherm
	\end{itemize}
	
	These are far more flexible models (more parameters); e.g. Langmuir isotherm is a special case of the BET isotherm.
\end{frame}

\begin{frame}\frametitle{Understanding adsorption in packed beds}
	\begin{center}
		% Lukchis article
		\includegraphics[width=\textwidth]{\imagedir/separations/adsorption/Lukchis-MTZ-illustration-part-1.png}
	\end{center}
\end{frame}

\begin{frame}\frametitle{Understanding adsorption in packed beds}
	\begin{center}
		% Lukchis article
		\includegraphics[width=\textwidth]{\imagedir/separations/adsorption/Lukchis-MTZ-illustration-part-2.png}
	\end{center}
	\vspace{-12pt}
	\see{Lukchis}
	\vspace{-6pt}
	\begin{columns}[t]
		\column{1.1\textwidth}
			\begin{itemize}
				\item	{\small $X$ = concentration in the bulk of the bed = $C_A$ in previous notation}
				\item	{\small $X_e$ = concentration of adsorbate at equilibrium in the bulk= feed conc}
				\item	{\small $X_0$ = concentration of adsorbate in the regenerated adsorbent at time 0}
				\item	{\small $\theta_b$ = breakthrough time}
				\item	{\small $\theta_e$ = the bed at equilibrium time; adsorbent is completely used}
			\end{itemize}
	\end{columns}	
\end{frame}

\begin{frame}\frametitle{Bed concentration just prior to breakthrough}
	\begin{center}
		\includegraphics[width=0.9\textwidth]{\imagedir/separations/adsorption/Bed-concentration-Ghosh-p144.png}
	\end{center}
	\vspace{-12pt}
	\see{Ghosh (adapted), p144}
	\vspace{6pt}
	\begin{itemize}
		\item	{\color{purple}{MTZ}}: mass transfer zone is where adsorption takes place
		\item	{\color{purple}{Equilibrium zone}}: this is where the isotherm applies!
		\item	{\color{purple}{Breakthrough}}: arbitrarily defined as either (a)  the lower limit of adsorbate detection, or (b) the maximum allowable adsorbate in effluent. Around 1 to 5\% of $C_\text{A,feed}$.
	\end{itemize}
	
\end{frame}

% \begin{frame}\frametitle{Next steps}
% 	What is the purpose of the MTZ curves to troubleshoot or operate a packed bad
% 	Describe how Langmuir isotherms are favourable
% 	WES
% 	LUB
% 	LES
% 	Space velocity.
%   Usable capacity
%   Scale up from lab data (Geankoplis)
% \end{frame}
%
% \begin{frame}\frametitle{Fitting a model to data}
%	Batch system example in Ghosh's and/or Geankoplis
% 	Example from UCT practical
%   For a given isotherm,  and feed concentration, calculate the separation factor at equilibrium
% \end{frame}

 
% Topics to consider for future: 
% Multicomponent isotherms
% Describe pressure swing adsoprtion: see Seader, p 610 in their book for a good description
% Schweitzer book: any missing topics
% Modelling of diffusion, heat transfer AND mass transfer
% \begin{frame}\frametitle{Potential theory}
% 	/Users/kevindunn/Sync/Figures/separations/adsorption/CR-v2-5ed-p991.png
% \end{frame}

\begin{frame}\frametitle{References used (in alphabetical order)}
	\begin{itemize}
		\item	Geankoplis, ``Transport Processes and Separation Process Principles'', 4th edition, chapter 12
		\item	Ghosh, ``Principles of Bioseparation Engineering'', chapter 8
		\item	Johnston, ``Designing Fixed-Bed Adsorption Columns'', \emph{Chemical Engineering}, p 87-92, 1972
		\item	Lukchis, ``Adsorption Systems: Design by Mass-Transfer-Zone Concept'', \emph{Chemical Engineering}, 1973.
		\item	Perry's Chemical Engineers' Handbook, Chapter 22
		\item	Richardson and Harker, ``Chemical Engineering, Volume 2'', 5th edition, chapter 17	
		\item	Schweitzer, ``Handbook of Separation Techniques for Chemical Engineers'', Chapter 3.1
		\item	Seader, Henly and Roper, ``Separation Process Principles'', 3rd edition, chapter 15
		\item	Uhlmann's Encyclopedia, ``Adsorption'', {\tiny \href{http://dx.doi.org/10.1002/14356007.b03\_09.pub2}{DOI:10.1002/14356007.b03\_09.pub2}}
		\item	Wankat, ``Separation Process Engineering'', Chapter 16
	\end{itemize}
\end{frame}


