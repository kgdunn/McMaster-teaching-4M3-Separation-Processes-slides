\begin{comment}

References: C+R, UCT notes, Perry's

See UCT notes for ChE337Z

See Perry, chapter 21: first part

See ``Uhlmann - Solid-solid separation.pdf''

% Perry: http://accessengineeringlibrary.com/browse/perrys-chemical-engineers-handbook-eighth-edition/p200139d899721_8001#p200139d899821_18

Sieving is probably the most frequently used and abused method of analysis because the equipment, analytical procedure, and basic concepts are deceptively simple. In sieving, the particles are presented to equal-size apertures that constitute a series of go-no go gauges. Sieve analysis implies three major difficulties: (1) with woven-wire sieves, the weaving process produces three-dimensional apertures with considerable tolerances, particularly for fine-woven mesh; (2) the mesh is easily damaged in use; (3) the particles must be efficiently presented to the sieve apertures to prevent blinding.

For coarse separation, dry sieving is used, but other procedures are necessary for finer and more cohesive powders. The most aggressive agitation is performed with Pascal Inclyno and Tyler Ro-tap sieves, which combine gyratory and jolting movement, although a simple vibratory agitation may be suitable in many cases. With Air-Jet sieves, a rotating jet below the sieving surface cleans the apertures and helps the passage of fines through the apertures. The sonic sifter combines two actions, a vertical oscillating column of air and a repetitive mechanical pulse. Wet sieving is frequently used with cohesive powders

Perrys: Section 22.5.6:

Uhlmann, vol 8, p 81: ``Classifying Separations''

coarsest sieve, and the sieve column is vibrated either mechanically or electro-mechanically. The particles pass through the sieve openings, and the coarser ones are retained on the sieves. As soon as the particles cease to pass through the openings, the vibration is stopped. The amount of particles on each individual sieve is weighed, and the histogram of the density distribution is obtained by weight.
\end{comment}

% 18 September 2012 (reorganized slides after class; should flow better now in 2013; maybe add a bit more on sphericity; show an example?)
% In 2013: put the slides on grade efficiency (currently in the cyclone section) into this section.
% In 2013: make sure your definition of cut size is correct
% For 2013: See "/Users/kevindunn/Courses/4M3/References/Uhlmann - Classifying separations.pdf" for extra notes on cut-size, etc

\begin{frame}\frametitle{Some context}
	\begin{exampleblock}{}
		Three important characteristics of an individual solid particle are its composition, its size and its shape.
	\end{exampleblock}

	\vspace{12pt}
	{\color{myGreen}{Composition}}: affects density, conductivity, and other physical properties important to separating it.
	
	\vspace{24pt}
	We will consider shape and size characterization now.
	
\end{frame}

\begin{frame}\frametitle{Particle shape characterization}

	A particle may be regular shaped, e.g.
	\begin{itemize}
		\item	\emph{spherical} or \emph{cubic} objects are capable of precise definition using equations
	\end{itemize}

	\vspace{12pt}
	Irregular shaped: e.g. broken glass, sand, rock, most solids
	\begin{itemize}
		\item	properties of irregular shapes are expressed ito a regular shaped particle's characteristics
	\end{itemize}
	
	\vspace{12pt}
	So we will spend time characterizing spherical particles, then expressing other particles in terms of an\emph{equivalent spherical particle}
\end{frame}

\begin{frame}\frametitle{Particle shape: Sphericity}
	Why use a sphere?
	\begin{itemize}
		\item	it has the same shape from all angles
		\item	behaves the same way from all angles
	\end{itemize}

	Other particles behave less ideally; we define sphericity as one metric of a particle's shape:


	\[
		\psi = \frac{\text{surface area of sphere with same volume as particle}}{\text{surface area of particle}}
	\]

	\vspace{12pt}
	For all particles: \( 0 < \psi \leq 1 \)

	\vspace{12pt}
	To try: calculate the sphericity of a cube with side length = $c$ \\ {\scriptsize {\color{myGreen}{Answer: $\psi = 0.806$}}}
\end{frame}

\begin{frame}\frametitle{Other shape metrics}
	Find the diameter of a sphere that has the same \_\_\_\_\_\_ as the irregularly-shaped particle
	\begin{itemize}
		\item	volume
		\item	surface area
		\item	surface area per unit volume [filtration]
		\item	area in the projected direction of travel [drag diameter]
		\item	projected area, but in a position of maximum stability
		\item	settling velocity [Stokes' diameter]
		\item	will fit through the same size square aperture  [sieve diameter]
		%\item	Sauter mean diameter: http://en.wikipedia.org/wiki/Sauter_mean_diameter
	\end{itemize}
\end{frame}

\begin{frame}\frametitle{Shape metrics example}
	
	For example, how would you quantify yourself if measured by:
	\begin{enumerate}
		\item	Circumference around your waist?
		\item	Diameter of a sphere of the same surface area as your body?
		\item	Length of your longest chord (height)?
		
	\end{enumerate}
	
	\vspace{12pt}
	The measured values will have different meanings. e.g.
	\begin{enumerate}
		\item	used to size a life jacket 
		\item	(perhaps?) used to estimate heat losses through your skin
		\item	if you are buying a sleeping bag I suggest the last one.
	\end{enumerate}
	
	\see{Adapted from: George G. Chase, The University of Akron}	
\end{frame}

\begin{frame}\frametitle{Particle size characterization}
	So far we have assumed particles to be separated are of a single size. This is never true: there is always a size distribution.

	\vspace{12pt}
	{\color{myGreen}{Particle size}}: affects surface per unit volume (mass transfer), rate of settling in a fluid (separation), \emph{etc}

	\vspace{12pt}
	\begin{exampleblock}{{\color{myRed}{Aims}}}
		\vspace{12pt}
		How do we measure this distribution?

		\vspace{12pt}
		How do we describe (characterize) a size distribution?
		
		\vspace{12pt}
		What is the ``\textbf{{\color{Brown}{average}}}'' particle size?
	\end{exampleblock}	
\end{frame}

\begin{frame}\frametitle{Other reasons to consider particle size distributions?}
	\begin{itemize}
		\item	Understand your colleagues: ``After crushing, the feed may be ground in several stages from a size of 5 to 6 cm to a powder of 75 to 90 percent passing a 200-mesh sieve.''
		\item	Solid material handling industry: we must deal with distributions
		\begin{itemize}
			\item	What kind of industries are we referring to here?
			% foods, grains, pulp, sand, cement, coal, solid waste water treatment,
			% pretty much any industry that handles solids
			\item	e.g. activity of a powdered drug = $f$(particle size)
			\item	e.g. ``hiding power'' of a paint/pigment = $f$(particle size)
		\end{itemize}
	\end{itemize}

	\vspace{12pt}
	\begin{exampleblock}{}
		We will require this understanding for future sections: filtration, flow of fluids through packed beds, cyclones, centrifuges, membranes, and so on.
	\end{exampleblock}
\end{frame}

\begin{frame}\frametitle{Some typical particle sizes}
	We typically work in {\color{purple}microns}. 1 micron = 1 $\mu\text{m}$
	\begin{center}
		\includegraphics[height=0.80\textheight]{\imagedir/separations/screens/typical-powder-sizes-CRv2-5ed-p4.png}
	\end{center}
	\see{Richardson and Harker, 5ed, p4}
\end{frame}

\begin{frame}\frametitle{Standard screens}
	\begin{center}
		\includegraphics[width=0.95\textwidth]{\imagedir/separations/screens/standard-screen-mesh-10.jpg}
	\end{center}
	Mesh 10 screen = 2.00 mm opening = 10 openings per linear inch.
\end{frame}

\begin{frame}\frametitle{Standard screen sizes}
	{\small The US standard (Tyler series). Selected screens from Tyler sequence:}
	\vspace{4pt}
	\begin{columns}[c]
		\column{0.60\textwidth}
			\begin{tabular}{c|c}
				Mesh number & Square aperture opening (\micron) \\ \hline
				3.5		&	5600	\\
				$\vdots$& 	$\vdots$\\
				10		&	2000	\\
				$\vdots$& 	$\vdots$\\
				20		&	850		\\
				25		& 	710		\\
				30		& 	600		\\
				$\vdots$& 	$\vdots$\\
				140 	& 	106		\\
				170		& 	90		\\
				200		&	75		\\
				230 	&	63		\\
				$\vdots$& 	$\vdots$\\
				450 	& 	32 \\ \hline
			\end{tabular}
		\column{0.40\textwidth}
			\small
			Tyler standard:
			\begin{itemize}
				\item	e.g. 75\micron ~opening: called 200 mesh screen
				\item	i.e. apertures per inch = 200 mesh screen
				\vspace{12pt}
				\hrule
				\vspace{12pt}
				\item	Successive apertures decrease by factor of $\sim \sqrt[4]{2}$
				\vspace{12pt}
				\hrule
				\vspace{12pt}
				\item	Other standards: British I.M.M. and \\ U.S. A.S.T.M.
			\end{itemize}
	\end{columns}
\end{frame}

\begin{frame}\frametitle{Screens}
	Stack screens: apertures from largest on top to smallest
	\begin{center}
		\includegraphics[height=0.80\textheight]{\imagedir/separations/screens/Laborsiebmaschine_BMK-wikipedia.jpg}
	\end{center}
	\see{\href{http://en.wikipedia.org/wiki/File:Laborsiebmaschine\_BMK.jpg}{http://en.wikipedia.org/wiki/File:Laborsiebmaschine\_BMK.jpg}}
\end{frame}

\begin{frame}\frametitle{Screens}
	\begin{itemize}
		\item	Select top screen to (usually) have 100\% material passing through
		\item	Shaken for a predetermined time; or rate of screening levels out
		\item	Shake intensity must be balanced: not too aggressive to break particles apart
		\item	Smaller particles tend to stick to each other, so small size fractions inaccurate
		\item	One can have wet or dry screen systems
		\item	Wet screening: washes smaller particles off larger ones
	\end{itemize}
\end{frame}

% \begin{frame}\frametitle{Differential screen analysis}
% 	* Picture of a screen: oversize and undersize
% 	* Mass fractions
% 	* Cumulative plot
% 	* Take the derivative of it
% 	* Show an actual example: Seader, p 678
% 	* Plot the example
% 	* Show the derivative of it:
% 	* What is a suitable ``Average'' particle size ?
% 	** The most commonly occurring frequency
% 	** Weighted mean?
% 	* Might want to use a log scale over a wide size range
% \end{frame}

\begin{frame}\frametitle{Data analysis from a screen sample}
	\begin{tabular}{ccc|c|c}
		Mesh&Aperture [\micron]	& Mass retained [g]	& Avg size* & {\scriptsize Cuml. \% passing} \\ \hline
		14	& 1400		& 0		& -      &  100           \\
		16	& 1180		& 9.1	& 1290   &  98.1          \\
		18	& 1000		& 32.1	& 1090   &  91.6          \\
		20	& 850		& 39.8	& 925    &  83.5          \\
		30	& 600		& 235.4	& 725    &  35.5          \\
		40	& 425		& 89.1	& 513    &  17.4          \\
		50	& 300		& 54.4	& 363    &  6.3           \\
		70	& 212		& 22.0	& 256    &  1.8           \\
		100	& 150		& 7.2	& 181    &  0.4           \\
		140	& 106		& 1.2	& 128    &  0.1           \\
		Pan	& 0			& 0.5	& 53     &  0.0           \\ \hline
		Sum	&  			& \textbf{491}   &     			  \\ \hline
	\end{tabular}

	\vspace{12pt}
	$\ast$ \emph{\scriptsize average screen size used for differential plots}
\end{frame}

\begin{frame}\frametitle{Differential and Cumulative analysis}
	\begin{center}
		\includegraphics[width=\textwidth]{\imagedir/separations/screens/screen-result-analysis.png}
	\end{center}
	\hrule
	\vspace{6pt}
	Theory:
	\begin{columns}[t]
		\column{0.50\textwidth}
			{\color{myBlue}{\[f(x) = \displaystyle\frac{dF(x)}{dx}\]}}
			\vspace{-12pt}
			{\color{myBlue}{$x$} is the avg particle size (bin)}
			
		\column{0.50\textwidth}
			{\color{myGreen}{\[F(x) = \text{percent passing curve}\]}}
			\vspace{-12pt}
			{\color{myGreen}{\[1-F(x) = \text{percent retained curve}\]}}
	\end{columns}
	\vspace{18pt}
	\seefull{Seader, 3ed, p 676-677}; sometimes use a log scale on $x$-axis
\end{frame}

\begin{frame}\frametitle{Theoretical view}
	\begin{center}
		\includegraphics[width=\textwidth]{\imagedir/separations/screens/theoretical-size-analysis-Svarovsky4.png}
	\end{center}
	\vfill
	\see{Svarovsky, 4ed, p42}
\end{frame}

\begin{frame}\frametitle{Mean diameter calculations}
	A number of mean diameters can be calculated. These can be derived from the cumulative analysis plot:
	\begin{columns}[c]
		\column{0.70\textwidth}
			\begin{center}
				\includegraphics[width=\textwidth]{\imagedir/separations/screens/screen-result-differential-only.png}
			\end{center}
		\column{0.45\textwidth}
		\small
			\begin{itemize}
				\item	Arithmetic mean = 318 \micron
				\item	Volume mean diameter = 430 \micron
				\item	Surface mean diameter (Sauter mean diameter) = 565 \micron
				\item	Weight or mass-mean diameter = 666 \micron
			\end{itemize}
			
			\vspace{12pt}
			\footnotesize
			Our aim is not to calculate all these (the formulae are messy, and error prone). Your lab will have these already set up in their analysis software.
	\end{columns}

	\vspace{5pt}
	\tiny
	\begin{itemize}
		\item	Seader, 3ed, p 678 - 679 [for worked example]
		\item	Svarovsky, 4ed, p37 - 43 [for descriptions of many means]
	\end{itemize}
\end{frame}

\begin{frame}\frametitle{Which mean should I use?}
	\begin{itemize}
		\item	Rather use the distribution curve, if available
		\item	If one has to resort to a single number, use what is appropriate
		\begin{itemize}
			\item	volume mean diameter: used for packing estimation
			\item	surface mean diameter: used for skin friction, and mass transfer calculations
		\end{itemize}

		\item	The idea is that if two materials had the same ``mean diameter'', that they would behave the same way in the application being considered.
	\end{itemize}
\end{frame}

\begin{frame}\frametitle{Two distributions, same arithmetic mean}
	\begin{center}
		\includegraphics[height=0.85\textheight]{\imagedir/separations/screens/particle-distributions-two-cases.png}
	\end{center}
	\see{Svarovsky, 4ed, p59}
\end{frame}

% \begin{frame}\frametitle{Percent passing: cut size}
% 	\begin{exampleblock}{Now it should be clearer}
% 		``powder of 75 to 90 percent passing a 200-mesh sieve''
% 	\end{exampleblock}
% 
% 	\todo{show a cumulative histogram; draw a vertical line}
% \end{frame}

\begin{frame}\frametitle{Sampling a stream}
	Particle size measurements are strongly dependent on the sample taken. The ``golden'' rules of sampling:

	\begin{enumerate}
		\item	take sample from a moving stream: dry powders and slurry
		\item	sample \emph{whole} stream for many short periods (not part of stream for whole time)
	\end{enumerate}

	\vspace{12pt}
	There are books written on the topic of sampling. Consult an experienced person if important decisions rest on the sample taken.
\end{frame}

\begin{frame}\frametitle{References}
	\begin{enumerate}
		\item	Richardson and Harker, ``Chemical Engineering, Volume 2'', 5th edition, chapter 1
		\item	\href{http://accessengineeringlibrary.com/browse/perrys-chemical-engineers-handbook-eighth-edition}{Perry's Chemical Engineers' Handbook}, 8th edition, chapter 21.1
		\item	Seader, Henley and Roper, ``Separation Process Principles'', page 675 to 679 in 3rd edition
		\item	Svarovsky, ``Solid-Liquid Separation'', 4th edition.
	\end{enumerate}
\end{frame}