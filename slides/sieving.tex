References: C+R, UCT notes

See UCT notes for ChE337Z

See Perry, Chapter 21: first part

See ``Uhlmann - Solid-solid separation.pdf''

Perry: http://accessengineeringlibrary.com/browse/perrys-chemical-engineers-handbook-eighth-edition/p200139d899721_8001#p200139d899821_18

Sieving is probably the most frequently used and abused method of analysis because the equipment, analytical procedure, and basic concepts are deceptively simple. In sieving, the particles are presented to equal-size apertures that constitute a series of go–no go gauges. Sieve analysis implies three major difficulties: (1) with woven-wire sieves, the weaving process produces three-dimensional apertures with considerable tolerances, particularly for fine-woven mesh; (2) the mesh is easily damaged in use; (3) the particles must be efficiently presented to the sieve apertures to prevent blinding.

Sieves are often referred to their mesh size, which is a number of wires per linear unit. Electroformed sieves with square or round apertures and tolerances of ±2 μm are also available (ISO 3310, Test Sieves—Technical Requirements and Testing, 2000/2004: Part 1: Test Sieves of Metal Wire Cloth; 1999; Part 2: Test Sieves of Perforated Metal Plate; 1990; Part 3: Test Sieves of Electroformed Sheets).

For coarse separation, dry sieving is used, but other procedures are necessary for finer and more cohesive powders. The most aggressive agitation is performed with Pascal Inclyno and Tyler Ro-tap sieves, which combine gyratory and jolting movement, although a simple vibratory agitation may be suitable in many cases. With Air-Jet sieves, a rotating jet below the sieving surface cleans the apertures and helps the passage of fines through the apertures. The sonic sifter combines two actions, a vertical oscillating column of air and a repetitive mechanical pulse. Wet sieving is frequently used with cohesive powders

Perrys: Section 22.5.6:

Wastewater treatment is directed toward removal of pollutants with the least effort. Suspended solids are removed by either physical or chemical separation techniques and handled as concentrated solids.

Screens Fine screens such as hydroscreens are used to remove moderate-size particles that are not easily compressed under fluid flow. Fine screens are normally used when the quantities of screened particles are large enough to justify the additional units. Mechanically cleaned fine screens have been used for separating large particles. A few industries have used large bar screens to catch large solids that could clog or damage pumps or equipment following the screens.

Uhlmann, vol 8, p 81: ``Classifying Separations''

arsest sieve, and the sieve column is vibrated either mechanically or electro-mechanically. The particles pass through the sieve openings, and the coarser ones are retained on the sieves. As soon as the particles cease to pass through the openings, the vibration is stopped. The amount of particles on each individual sieve is weighed, and the histogram of the density distribution is ob- tained by weight.


\begin{frame}\frametitle{Administrative}
	Assignment 2 is posted
	\vspace{12pt}
	If handing in via Google Docs, please:
	\begin{itemize}
		\item	Add the TA's address: 
	\end{itemize}
\end{frame}

\begin{frame}\frametitle{Particle size characterization}
	So far we have assumed particles to be separated are of a single size. This is never true: there is always a size distribution.
	
	\vspace{12pt}
	\begin{exampleblock}{Aim}
		How do we measure this distribution? How do we describe (characterize) a size distribution? 
	\end{exampleblock}
\end{frame}