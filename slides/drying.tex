% Outline: 30 minutes
\begin{comment}
	Why remove water
	Terminology
	Equilibrium curves (C+R, Santiago, p7)

	Psychrometry: useful for determining humidity content (kg//kg) of air at a given temperature and given RH
	Bound vs unbound moisture
	Drying curve


	% Next class:
	Equipment for removing water
	Fluidized bed system
\end{comment}

\begin{frame}\frametitle{Administrative}
	\begin{itemize}
		\item	Assignment 5 is posted (3 questions so far); questions 4 and 5 posted by Tuesday afternoon
		\item	Assignment 5 is due in the Chem Eng drop box by Monday at 16:00, or earlier.
		\item	Assignment 4 will be available for pick up on Thursday and Friday.
		\item	Midterm will be available to pick up Tuesday, Thursday, Friday.
		\item	Please fill in a course evaluation: \href{https://evals.mcmaster.ca}{https://evals.mcmaster.ca}
		\item	Confused about grades? There's a grading spreadsheet online
		\begin{itemize}
			\item	Please do not use averages and symbols calculated by Avenue
		\end{itemize}
	\end{itemize}
	\vspace{12pt}
	\begin{itemize}
		\item	Course review on Friday, 30 November. {\color{myOrange} \hfill $\longleftarrow$ {\textbf{Final class}}}
	\end{itemize}
\end{frame}

\begin{frame}\frametitle{Background}
	We consider \textbf{drying of solid products} here.
	\begin{itemize}
		\item	Remove {\color{myOrange}{liquid phase}} from {\color{myOrange}{solid phase}} by an {\color{myOrange}{ESA = thermal energy}}
		\item	It is the final separating step in many processes
		\begin{itemize}
			\item	pharmaceuticals
			\item	foods
			\item	lumber, pulp and paper products
			\item	catalysts, fine chemicals
			\item	detergents
		\end{itemize}
	\end{itemize}
	Why dry?
	\begin{itemize}
		\item	packaging dry product is much easier than moist/wet product
		\item	reduces weight for shipping
		\item	preserves product from bacterial growth
		\item	stabilizes flavour and prolongs shelf-life in foods
		\item	provides desirable properties: e.g. flowability, crispiness
		\item	reduces corrosion: the ``corrosion triangle'': removes 1 of the 3
	\end{itemize}
\end{frame}

\begin{frame}\frametitle{The nature of water in solid material}
	p4-142 Schweitzer
	
\end{frame}

\begin{frame}\frametitle{Equipment}
	Multiple equipment types to dry material:
	\begin{itemize}
		\item	each have relative advantages and disadvantages
		\item	our purpose is not to cover their details
		\item	in practice: you would work in consultation with vendors
		\item	in practice: plenty of trade literature on the topics {\color{myGreen}{(SDL\emph{!})}}
	\end{itemize}
	\vspace{12pt}
	Some major distinctions though:
	\begin{itemize}
		\item	\textbf{{\color{myGreen}{mode of operation}}}: batch (low volume) \emph{vs} continuous
		\item	how the \textbf{{\color{myGreen}{heat is provided}}}:
		\begin{itemize}
			\item	{\color{purple}{direct heat}}: convective or adiabatic; provides heat and sweeps away moisture
			\item	{\color{purple}{indirect heat}}: non-adiabatic, i.e. by conduction or radiation; e.g microwave
		\end{itemize}
		\item	\textbf{{\color{myGreen}{degree of agitation}}}
		\begin{itemize}
			\item	stationary material
			\item	fluidized or mixed in some way
		\end{itemize}
	\end{itemize}
\end{frame}

\begin{frame}\frametitle{How to choose the equipment*}
	\begin{itemize}
		\item	Strongly dependent of feed presentation
		\begin{itemize}
			\item	is it: solid, slurry, paste, flowing powder, filter cake, fibrous, \emph{etc}
		\end{itemize}
		\item	Heating choice: temperature-sensitive if convective heat is directly applied
		\item	Agitation: 
		\begin{itemize}
			\item	produce fines (dust hazard) or fragile material
			\item	good mixing implies good heat distribution
			\item	stationary product: can form hot-spots in the solid
		\end{itemize}
	\end{itemize}
	
	General choices are between:
	\begin{enumerate}
		\item	shelf/tray dryers
		\item	continuous tunnels
		\item	rotary dryers
		\item	drums
		\item	spray dryers
		\item	fluidized beds
	\end{enumerate}
	\vspace{4pt}
	\hrule
	\vspace{4pt}
	* See Schweitzer; See Perry's
	%p4-168 Schwietzer, p4-161 and 4-162
\end{frame}

\begin{frame}\frametitle{Some equipment examples}
	\textbf{Rotating dryer}
	\begin{columns}[t]
		\column{0.60\textwidth}
			\begin{center}
				\includegraphics[width=\textwidth]{\imagedir/separations/drying/rotary-dryer-Schweitzer-4-162.png}
			\end{center}
		\column{0.40\textwidth}
			\begin{center}
				\includegraphics[width=\textwidth]{\imagedir/separations/drying/rotary-dryer-Schweitzer-p4-161-detail.png}
			\end{center}
	\end{columns}
	\see{Schweitzer, p 4-161 and 4-162}
\end{frame}

\begin{frame}\frametitle{Some equipment examples}
	\textbf{Fluidized bed dryer}
	\begin{columns}[t]
		\column{0.60\textwidth}
			\begin{center}
				\includegraphics[height=0.75\textheight]{\imagedir/separations/drying/fluid-bed-dryer-flow-patter-Schweitzer-p4-179.png}
			\end{center}
			\vspace{-12pt}
			\see{Schweitzer, p 4-179}
		\column{0.45\textwidth}
			\begin{itemize}
				\item	upward flowing air stream (elutriation)
				\item	turbulent mixing: good heat and mass transfer
				\item	uniform solid temperature
				\item	solids are gently treated
				\item	solids are retrieved via gravity and cyclones
				\item	fluidizing air is scrubbed before being vented
			\end{itemize}
	\end{columns}
\end{frame}

% Add in 2013
% \begin{frame}\frametitle{Some equipment examples}
% 	Atomizing drying
% 	e.g. for milk powder
% \end{frame}

\begin{frame}\frametitle{Terminology}
	\begin{center}
		\includegraphics[width=0.8\textwidth]{\imagedir/separations/drying/Water-phase-diagram-Drying-Wikipedia.png}
	\end{center}
\end{frame}

\begin{frame}\frametitle{Terminology}
	Enthalpy, humidity, sensible 
\end{frame}

\begin{frame}\frametitle{Mass transfer concepts}
	Bring liquid from interior of product to surface
	Vapourization of liquid at/near the surface
	Transport of vapour into the bulk gas phase

\end{frame}

\begin{frame}\frametitle{}%Psychrometric chart
	\vfill
	\begin{center}
		\includegraphics[width=\textwidth]{\imagedir/separations/drying/psychrometric-chart-Geankoplis.png}
	\end{center}
\end{frame}

\begin{frame}\frametitle{Heat transfer concepts}
	Heat transferred from bulk gas phase to solid phase: 
	\begin{itemize}
		\item	portion of it used to vapourize the liquid ({\color{purple}{latent heat}})
		\item	portion remains in the solid as ({\color{purple}{sensible heat}})
	\end{itemize}
	
	Santiago, slides S6: p 11 and 12 
\end{frame}

\begin{frame}\frametitle{Bringing mass and heat transfer together}
	
\end{frame}

\begin{frame}\frametitle{Fluidized beds}
	p 768 Seader, Henly Roper,
	p4-178 to 181 in Scweitzer
\end{frame}

\begin{frame}\frametitle{References used (in alphabetical order)}
	\begin{itemize}
		\item	Geankoplis, ``Transport Processes and Separation Process Principles'', 4th edition, chapter 09
		\item	Perry's Chemical Engineers' Handbook, Chapter 12
		\item	Richardson and Harker, ``Chemical Engineering, Volume 2'', 5th edition, chapter 16	
		\item	Schweitzer, ``Handbook of Separation Techniques for Chemical Engineers'', Chapter 4.10
		\item	Seader, Henly and Roper, ``Separation Process Principles'', 3rd edition, chapter 18
		\item	Uhlmann's Encyclopedia, ``Drying'', {\tiny \href{http://dx.doi.org/10.1002/14356007.b02\_04.pub2}{DOI:10.1002/14356007.b02\_04.pub2}}
	\end{itemize}
\end{frame}