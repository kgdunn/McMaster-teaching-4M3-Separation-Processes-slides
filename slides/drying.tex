% Outline: 30 minutes
\begin{comment}
	Why remove water
	Terminology
	Equilibrium curves (C+R, Santiago, p7)

	Psychrometry: useful for determining humidity content (kg//kg) of air at a given temperature and given RH
	Bound vs unbound moisture
	Drying curve


	% Next class:
	Equipment for removing water
	Fluidized bed system
\end{comment}

\begin{frame}\frametitle{Administrative}
	\begin{itemize}
		\item	Assignment 5 is posted (3 questions so far); questions 4 and 5 posted by Tuesday afternoon
		\item	Assignment 5 is due in the Chem Eng drop box by Monday, 03 December, at 16:00, or earlier.
		\item	Assignment 4 will be available for pick up on Thursday and Friday.
		\item	Midterm will be available to pick up Tuesday, Thursday, Friday.
		\item	Please fill in a course evaluation: \href{https://evals.mcmaster.ca}{https://evals.mcmaster.ca}
		\item	Confused about grades? There's a grading spreadsheet online
		\begin{itemize}
			\item	Please do not use averages and symbols calculated by Avenue
		\end{itemize}
	\end{itemize}
	\vspace{12pt}
	\begin{itemize}
		\item	Course review on Friday, 30 November. {\color{myOrange} \hfill $\longleftarrow$ {\textbf{Final class}}}
	\end{itemize}
\end{frame}

\begin{frame}\frametitle{Background}
	We consider \textbf{drying of solid products} here.
	\begin{itemize}
		\item	Remove {\color{myOrange}{liquid phase}} from {\color{myOrange}{solid phase}} by an {\color{myOrange}{ESA = thermal energy}}
		\item	It is the final separating step in many processes
		\begin{itemize}
			\item	pharmaceuticals
			\item	foods
			\item	crops, grains and cereal products
			\item	lumber, pulp and paper products
			\item	catalysts, fine chemicals
			\item	detergents
		\end{itemize}
	\end{itemize}
	Why dry?
	\begin{itemize}
		\item	packaging dry product is much easier than moist/wet product
		\item	reduces weight for shipping
		\item	preserves product from bacterial growth
		\item	stabilizes flavour and prolongs shelf-life in foods
		\item	provides desirable properties: e.g. flowability, crispiness
		\item	reduces corrosion: the ``corrosion triangle'': removes 1 of the 3
	\end{itemize}
\end{frame}

\begin{frame}\frametitle{The nature of water in solid material}
	% p4-142 Schweitzer
	% C+R v2, Ch16, p902
	\begin{columns}[c]
		\column{0.60\textwidth}
			
			\begin{center}
				At 20$^\circ\text{C}$
				
				\includegraphics[width=\textwidth]{\imagedir/separations/drying/equilibrium-moisture-contents-of-solids-C+Rv2,5ed,p902.png}
			\end{center}
			\vspace{-12pt}
			\see{Richardson and Harker, p 902}
		\column{0.50\textwidth}
			Material, when exposed to air with a certain humidity, will reach equilibrium with that air.
			
			\vspace{12pt}
			
			\begin{enumerate}
				\item	\textbf{Bound moisture}
				\begin{itemize}
					\item	adsorbed into material's capillaries and surfaces
					\item	or in cell walls of material
					\item	its vapour pressure is below water's partial pressure at this $T$
				\end{itemize}
				\item	\textbf{Free moisture}
				\begin{itemize}
					\item	water in excess of the above equilibrium water
				\end{itemize}
			\end{enumerate}
	\end{columns}
\end{frame}

\begin{frame}\frametitle{Drying: the heat and mass transfer view points}
	
	\begin{exampleblock}{}
		\begin{center}
			{\color{myGreen}{Both heat and mass transfer occur simultaneously}}
		\end{center}
	\end{exampleblock}
	
	\vspace{12pt}
	\begin{columns}[t]
		\column{0.50\textwidth}
			\textbf{Mass transfer}
			\begin{itemize}
				\item	Bring liquid from interior of product to surface
				\item	Vapourization of liquid at/near the surface
				\item	Transport of vapour into the bulk gas phase
			\end{itemize}
		\column{0.55\textwidth}
			\textbf{Heat transfer} from bulk gas phase to solid phase: 
			\begin{itemize}
				\item	portion of it used to vapourize the liquid ({\color{purple}{latent heat}})
				\item	portion remains in the solid as ({\color{purple}{sensible heat}})
			\end{itemize}
	\end{columns}
	\vspace{24pt}
	Key point: heat to vapourize the liquid is provided by the air stream
	%\todo{Figure here}
\end{frame}

\begin{frame}\frametitle{Terminology}
	\begin{center}
		\includegraphics[width=0.85\textwidth]{\imagedir/separations/drying/Phase_diagram_of_water.svg.png}
	\end{center}
	\see{\href{http://en.wikipedia.org/wiki/File:Phase\_diagram\_of\_water.svg}{Wikipedia File:Phase\_diagram\_of\_water.svg}}
\end{frame}

\begin{frame}\frametitle{Terminology}
	\begin{itemize}	
		\item	{\color{purple}{Partial pressure}}, recall, is the pressure due to water vapour in the water-air mixture
		\item	{\color{purple}{Vapour pressure}}, is the pressure exerted by (molecules of liquid water in the solid) on the gas phase in order to escape into the gas [a measure of volatility]
	\end{itemize}
	\begin{exampleblock}{}
		Moisture evaporates from a wet solid only when its vapour pressure exceeds the partial pressure
	\end{exampleblock}
	\begin{itemize}
		\item	Vapour pressure can be raised by heating the wet solid
	\end{itemize}
\end{frame}

% \item	Absolute humidity
% \item	Relative humidity (dimensionless?)
% \item	Wet bulb temperature

\begin{frame}\frametitle{Psychrometric chart}
	\vfill
	\begin{center}
		\includegraphics[width=\textwidth]{\imagedir/separations/drying/psychrometric-chart-Geankoplis.png}
	\end{center}
	\see{Geankoplis, p568; multiple internet sources have this chart digitized}
\end{frame}

\begin{frame}\frametitle{Terminology}
	\begin{itemize}
		\item	{\color{purple}{Dry bulb temperature}}: or just $T$ = ``\emph{temperature}'' {\tiny (nothing new here)}
		\begin{itemize}
			\item	the horizontal axis on the psychrometric chart
		\end{itemize}
		\item	{\color{purple}{Humidity}} = $\psi$ = mass of water vapour per kilogram of dry air
		\begin{itemize}
			\item	units are $ \left[ \displaystyle \frac{\text{kg water vapour}}{\text{kg dry air}}\right]$ 
			\item	called $H$ in many textbooks; always confused with enthalpy; so we will use $\psi$
			\item	units do not cancel, i.e. not dimensionless
			\item	the vertical axis on the psychrometric chart
		\end{itemize}
		\item	Maximum amount of water air can hold at a given $T$:
			\begin{itemize}
				\item	$\psi_S$ = {\color{purple}{saturation humidity}}
				\item	move up vertically to 100\% humidity
			\end{itemize}
		\item	{\color{purple}{Percentage humidity}} = $\displaystyle \frac{\psi}{\psi_S}\times 100$	
		\vspace{4pt}
		\item	{\color{purple}{Partial pressure}} we said is the pressure due to water vapour in the water-air mixture
		\vspace{4pt}
		\begin{itemize}
			\item	$\psi = \displaystyle \frac{\text{mass of water vapour}}{\text{mass of dry air}} = \displaystyle \frac{18.02}{28.97}\frac{p_A}{P-p_A}$
			\item	$p_A$ = partial pressure of water in the air
			\item	$P$ = total pressure = 101.325 kPa in this psychrometric chart
		\end{itemize}
	\end{itemize}
\end{frame}

\begin{frame}\frametitle{Terminology}
	\begin{itemize}
		\item	{\color{purple}{Dew point}}: the temperature to which you must cool the air/vapour mixture to just obtain saturation (100\% humidity), i.e. condensation just starts to occur.
	\end{itemize}
	
	\begin{center}
		\includegraphics[width=.8\textwidth]{\imagedir/separations/drying/psychrometric-chart-Geankoplis-dew-point.png}
	\end{center}
	
	\textbf{Example:} Air at 65\degC$\,$ and 10\% humidity has a dew point temperature of 25\degC. This parcel of air contains 0.021 kg of water per kilogram of dry air.
\end{frame}

\begin{frame}\frametitle{Terminology}
	\begin{itemize}
		\item	 {\color{purple}{Humid heat}}: amount of energy to raise 1kg of air and the water vapour it contains by 1\degC
		
		\begin{exampleblock}{}
			\[c_S = 1.005 + 1.88\psi\]
		\end{exampleblock}
		\begin{itemize}
			\item	$c_S$ has units $\left[\displaystyle \frac{\text{kJ}}{\text{(kg dry air)(K)}} \right]$
			\item	1.005 $\left[\displaystyle \frac{\text{kJ}}{\text{(kg dry air)(K)}} \right]$ is heat capacity of dry air
			\item	1.88 $\left[\displaystyle \frac{\text{kJ}}{\text{(kg water vapour)(K)}} \right]$ is heat capacity of water {\color{myRed}{\textbf{vapour}}}  % Verify this still
			\item	$\psi$ is the humidity $ \left[ \displaystyle \frac{\text{kg water vapour}}{\text{kg dry air}}\right]$ 
		\end{itemize}
	\end{itemize}
\end{frame}

\begin{frame}\frametitle{Administrative}
	\textbf{Assignment 5}
	\begin{itemize}
		\item	Due on Monday, 03 December, at 16:00 in Chem Eng dropbox
		\item	Or due electronically
		\item	Update to question 3.3.3 (or question 3.3(c))
			\begin{itemize}
				\item	{\color{myRed}{Solvent flow, $S = 27.5$ kg/hr}}, not 15 kg/hr
			\end{itemize}
	\end{itemize}
	\vspace{12pt}
	Please fill in a \textbf{course evaluation}:
	\begin{exampleblock}{}
		\begin{center}
			\href{https://evals.mcmaster.ca}{https://evals.mcmaster.ca}
		\end{center}
	\end{exampleblock}
	\begin{itemize}
		\item	Currently 32\% have filled it in
	\end{itemize}
	\begin{center}
		\includegraphics[width=0.35\textwidth]{\imagedir/separations/admin/course-evals-qr-code.png}
	\end{center}
\end{frame}

\begin{frame}\frametitle{Terminology: adiabatic saturation}
	Consider a stream of air at temperature $T$ and humidity $\psi$. It contacts fine water droplets long enough to reach equilibrium. The leaving gas has temperature $T_S$ and humidity $\psi_S$.

	\begin{center}
		\includegraphics[width=\textwidth]{\imagedir/separations/drying/adiabatic-saturation-temperature.png}
	\end{center}

	We expect outlet gas: $T_S < T$ and $\psi_S > \psi$
	
	\vspace{6pt}
	The energy to evaporate liquid water into the leaving air stream comes from the air.
\end{frame}

\begin{frame}\frametitle{Terminology: adiabatic saturation}
	Quantify it: do an enthalpy balance at $T_\text{ref} = T_S$ {\scriptsize (i.e. disregard water)}
	\begin{center}
		\includegraphics[width=0.4\textwidth]{\imagedir/separations/drying/adiabatic-saturation-temperature.png}
	\end{center}
	\vspace{-6pt}
	\begin{exampleblock}{Enthalpy of vapour phase entering:}
		\[c_S\left(T - T_S \right) + (\psi) (\Delta H_\text{vap})\]
	\end{exampleblock}
	\begin{exampleblock}{Enthalpy of vapour phase leaving:}
		\[c_S\left(T_S - T_S \right) + (\psi_S) (\Delta H_\text{vap})\]
	\end{exampleblock}
	
	\[	
		\displaystyle \frac{y\text{-axis change}}{x\text{-axis change}} = \frac{\color{red}{\boldsymbol \psi} \color{black} - \color{myGreen}{\boldsymbol \psi_S}}{\color{red}T \color{black}- \color{myGreen}{\boldsymbol T_S}} = - \frac{c_S}{\Delta H_\text{vap}} = - \frac{1.005 + 1.88 \color{red}{\boldsymbol \psi}}{\Delta H_\text{vap}}
	\]
	These are the diagonal sloped lines on the psychrometric chart: {\color{purple}{adiabatic saturation curves}}.
\end{frame}

\begin{frame}\frametitle{Adiabatic saturation temperature}
	\vfill
	\begin{center}
		\includegraphics[width=\textwidth]{\imagedir/separations/drying/psychrometric-chart-Geankoplis-adiabatic-saturation-temperature.png}
	\end{center}
\end{frame}

\begin{frame}\frametitle{Exercise}
	An air stream at 70\degC$\,$ and carrying 0.055 kg water per kg dry air is adiabatically contacted with liquid water until it reaches equilibrium. The process is continuous and operating at steady-state. Air feed is 1 kg dry air per minute.
	\todo{Add answers}
	\begin{enumerate}
		\item	What is the percentage humidity of the incoming air stream?
		\item	What is the percentage humidity of the air stream leaving?
		\item	What is the humidity [mass/mass] of the air stream leaving?
		\item	What is the temperature of the air stream leaving?
		\item	If the contacting takes place in a unit shown below, what is the mass of inlet make-up water required at steady-state operation?
	\end{enumerate}
	\begin{center}
		\includegraphics[width=0.6\textwidth]{\imagedir/separations/drying/adiabatic-saturation-temperature.png}
	\end{center}
\end{frame}

\begin{frame}\frametitle{Wet-bulb temperature}
	\todo{Relationship to Adiabatic temperature}
	\begin{center}
		\includegraphics[width=\textwidth]{\imagedir/separations/drying/Wikipedia-Sling_psychrometer.JPG}
	\end{center}
	\see{Wikipedia: \href{http://en.wikipedia.org/wiki/Wet-bulb\_temperature}{http://en.wikipedia.org/wiki/Wet-bulb\_temperature}}
\end{frame}

\begin{frame}\frametitle{Humid volume}
	
	Equivalent to the inverse density $1/\rho$ of moist air
	
	\vspace{12pt}
	Derived from the ideal-gas law and simplified here:
	\begin{exampleblock}{}
		\[ v_H = \left[ 2.83 \times 10^{-3} + 4.56 \times 10^{-3} \psi \right] T \quad \frac{\text{m}^3}{\text{kg moist air}} \]
	\end{exampleblock}
	\begin{itemize}
		\item	$\psi$ is humidity in kg water per kg dry air
		\item	$T$ is the temperature in K
	\end{itemize}
	
	\vspace{12pt}
	For example, 350K and $\psi = 0.026$ kg/kg, then 
	\[v_H = \left[2.83 \times 10^{-3} + 4.56 \times 10^{-3}(0.026)\right](350)  = 1.03\, \frac{\text{m}^3}{\text{kg moist air}}\]  
\end{frame}

\begin{frame}\frametitle{Drying profiles}
	Solids drying is phenomenally complex for different materials. Observe it experimentally:
	\begin{columns}[b]
		\column{0.50\textwidth}
			\begin{center}
				\includegraphics[width=\textwidth]{\imagedir/separations/drying/drying-profile-Seader-3ed-p751.png}
			\end{center}
		\column{0.55\textwidth}
			\begin{center}
				\includegraphics[width=\textwidth]{\imagedir/separations/drying/drying-profile-rate-Seader-3ed-p751.png}
			\end{center}
			\vspace{-7pt}
	\end{columns}

	\begin{itemize}
		\item	A $\rightarrow$ B: initial phase as solid heats up
		\item	B $\rightarrow$ C: constant-rate drying
		\item	C $\rightarrow$ D: falling-rate drying
	\end{itemize}
\end{frame}

\begin{frame}\frametitle{Drying profiles}	
	\begin{itemize}
		\item	Drying flux = $\displaystyle \frac{\text{mass of water removed}}{\text{(time)(area)}} = \displaystyle \frac{m_s}{A} \frac{dX}{dt}$
		\item	$X$ = mass of water
		\item	$A$ = surface area of solid exposed
		\item	$m_s$ = mass of dry solid
	\end{itemize}
	\vspace{12pt}
	We are most interested in the {\color{purple}{constant drying-rate}} period:
	\begin{itemize}
		\item	\emph{rate-limiting step}: mass transfer through boundary layer on the solid surface
		\item	the solid is able to provide water to the surface a fast rate
	\end{itemize}
\end{frame}

\begin{frame}\frametitle{Heat transfer coefficient estimation}
	\begin{itemize}
		\item	In constant-rate drying region the wet surface supplies adequate moisture.
		\item	So heat-transfer coefficients may be derived that are independent of solid type!
		\item	Derived from heat transfer convection equations (e.g. we ignore radiation)
		
		\begin{exampleblock}{}
			\[
				\displaystyle \frac{\text{heat transfer rate}}{\text{transfer surface area}}  = \text{flux} = \displaystyle \frac{\text{driving force}}{\text{resistance}} = \frac{h (T_\text{db} - T_\text{wb})}{\Delta H_\text{vap}}
			\]
		\end{exampleblock}
		\begin{itemize}
			\item	What is the driving force? $(T_\text{air} - T_\text{solid surface}) = (T_\text{db} - T_\text{wb})$
			\item	What is the resistance? $\Delta H_\text{vap} / h$
		\end{itemize}
	\end{itemize}
\end{frame}

\begin{frame}\frametitle{Some heat-transfer correlations}
	{\color{myRed}{Alert: awkward units!}}
	\begin{enumerate}
		\item	Parallel flow to surface:
		\begin{itemize}
			\item	Air between 45 to 150\degC
			\item	$G = 2\,450\text{~to~}29\,300\, \text{kg.hr}^{-1}\text{.m}^{-2}$
			\item	This corresponds to a velocity of $v$ = 0.61 to 7.6$\text{m.s}^{-1}$
			\item	$G = 3\,600\,\rho v_\text{avg}$ where $v$ and $\rho$ are in SI units
			\vspace{5pt}
			\item	$\mathbf{h = 0.0204G^{0.8}}\,\,[\text{W.m}^{-2}\text{.K}^{-1}]$
		\end{itemize}
		\vspace{24pt}
		\item	Perpendicular flow (impingement)
		\begin{itemize}
			\item	Air between 45 to 150\degC
			\item	$G = 3\,900\text{~to~}19\,500\, \text{kg.hr}^{-1}\text{.m}^{-2}$
			\item	This corresponds to a velocity of $v$ = 0.9 to 4.6$\text{m.s}^{-1}$
			\vspace{5pt}
			\item	$\mathbf{h = 1.17G^{0.37}}\,\,[\text{W.m}^{-2}\text{.K}^{-1}]$
		\end{itemize}
	\end{enumerate}
\end{frame}

\begin{frame}\frametitle{Equipment}
	Multiple equipment types to dry material:
	\begin{itemize}
		\item	each have relative advantages and disadvantages
		\item	our purpose is not to cover their details
		\item	in practice: you would work in consultation with vendors
		\item	in practice: plenty of trade literature on the topics {\color{myGreen}{(SDL\emph{!})}}
	\end{itemize}
	\vspace{12pt}
	Some major distinctions though:
	\begin{itemize}
		\item	\textbf{{\color{myGreen}{mode of operation}}}: batch (low volume) \emph{vs} continuous
		\item	how the \textbf{{\color{myGreen}{heat is provided}}}:
		\begin{itemize}
			\item	{\color{purple}{direct heat}}: convective or adiabatic; provides heat and sweeps away moisture
			\item	{\color{purple}{indirect heat}}: non-adiabatic, i.e. by conduction or radiation; e.g microwave
		\end{itemize}
		\item	\textbf{{\color{myGreen}{degree of agitation}}}
		\begin{itemize}
			\item	stationary material
			\item	fluidized or mixed in some way
		\end{itemize}
	\end{itemize}
\end{frame}

\begin{frame}\frametitle{How to choose the equipment*}
	\begin{itemize}
		\item	Strongly dependent of feed presentation
		\begin{itemize}
			\item	is it: solid, slurry, paste, flowing powder, filter cake, fibrous, \emph{etc}
		\end{itemize}
		\item	Heating choice: temperature-sensitive if convective heat is directly applied
		\item	Agitation: 
		\begin{itemize}
			\item	produce fines (dust hazard) or fragile material
			\item	good mixing implies good heat distribution
			\item	stationary product: can form hot-spots in the solid
		\end{itemize}
	\end{itemize}
	
	General choices are between:
	\begin{enumerate}
		\item	shelf/tray dryers
		\item	continuous tunnels
		\item	rotary dryers
		\item	drums
		\item	spray dryers
		\item	fluidized beds
	\end{enumerate}
	\vspace{4pt}
	\hrule
	\vspace{4pt}
	* See Schweitzer; See Perry's
	%p4-168 Schwietzer, p4-161 and 4-162
\end{frame}

\begin{frame}\frametitle{Some equipment examples}
	\textbf{Rotating dryer}
	\begin{columns}[t]
		\column{0.60\textwidth}
			\begin{center}
				\includegraphics[width=\textwidth]{\imagedir/separations/drying/rotary-dryer-Schweitzer-4-162.png}
			\end{center}
		\column{0.40\textwidth}
			\begin{center}
				\includegraphics[width=\textwidth]{\imagedir/separations/drying/rotary-dryer-Schweitzer-p4-161-detail.png}
			\end{center}
	\end{columns}
	\see{Schweitzer, p 4-161 and 4-162}
\end{frame}

\begin{frame}\frametitle{Some equipment examples}
	% In 2013: add figure from p 768 Seader, Henly Roper,
	\textbf{Fluidized bed dryer}
	% From  p4-178 to 181 in Scweitzer
	\begin{columns}[t]
		\column{0.60\textwidth}
			\begin{center}
				\includegraphics[height=0.75\textheight]{\imagedir/separations/drying/fluid-bed-dryer-flow-patter-Schweitzer-p4-179.png}
			\end{center}
			\vspace{-12pt}
			\see{Schweitzer, p 4-179}
		\column{0.45\textwidth}
			\begin{itemize}
				\item	upward flowing air stream (elutriation)
				\item	turbulent mixing: good heat and mass transfer
				\item	uniform solid temperature
				\item	solids are gently treated
				\item	solids are retrieved via gravity and cyclones
				\item	fluidizing air is scrubbed before being vented
			\end{itemize}
	\end{columns}
\end{frame}

%Add in 2013
%\begin{frame}\frametitle{Some equipment examples}
% 	Atomizing drying
% 	e.g. for milk powder
% \end{frame}

\begin{frame}\frametitle{Filter cake drying}
	\begin{columns}[b]
		\column{0.50\textwidth}
		\begin{center}
			\includegraphics[width=\textwidth]{\imagedir/separations/drying/filter-press-6322210751_0d18836361_o.jpg}
		\end{center}
		\see{\href{http://www.flickr.com/photos/cdeimages/6322210751}{Flickr, CC BY 2.0}}			
		\column{0.50\textwidth}
			\begin{center}
				\includegraphics[width=\textwidth]{\imagedir/separations/drying/filter-cake-flickr-6322211031_8082d6ee3b_o.jpg}
			\end{center}
			\see{\href{http://www.flickr.com/photos/cdeimages/6322211031}{Flickr, CC BY 2.0}}
	\end{columns}
	\vspace{12pt}
	Consider 100kg of cake, discharged at 30\% moisture (wet basis). Air to dry the cake at 75\degC$\,$ is used, 10\% relative humidity, with a velocity of 4 m/s. The aim is to achieve a 10\% (dry basis) cake which can be milled and packaged. Estimate the drying time.
\end{frame}

\begin{frame}\frametitle{Example}
	\begin{enumerate}
		\item	What is the humidity of the incoming air stream? {\scriptsize \color{myOrange}{[0.04 kg water/kg dry air]}}
		\item	What is the wet-bulb temperature of this air stream? {\scriptsize \color{myOrange}{[$\sim$41.3\degC]}}
		\item	What is the heat of vapourization of water at this temperature? {\scriptsize \color{myOrange}{[2401 kJ/kg]}}
		\begin{itemize}
			\item	2501 kJ/kg at 0\degC
			\item	2260 kJ/kg at 100\degC
		\end{itemize}
		\item	What is the humid volume of the drying air stream? {\scriptsize \color{myOrange}{[$T=348$K, $v_H = 1.048 \text{m}^3\text{/kg}$]}}
		\item	Estimate the heat transfer coefficient.
		\item	Substitute into the constant-drying rate expression to solve for drying time.
	\end{enumerate}
\end{frame}

\begin{frame}\frametitle{References used (in alphabetical order)}
	\begin{itemize}
		\item	Geankoplis, ``Transport Processes and Separation Process Principles'', 4th edition, chapter 09
		\item	Perry's Chemical Engineers' Handbook, Chapter 12
		\item	Richardson and Harker, ``Chemical Engineering, Volume 2'', 5th edition, chapter 16	
		\item	Schweitzer, ``Handbook of Separation Techniques for Chemical Engineers'', Chapter 4.10
		\item	Seader, Henly and Roper, ``Separation Process Principles'', 3rd edition, chapter 18
		\item	Uhlmann's Encyclopedia, ``Drying'', {\tiny \href{http://dx.doi.org/10.1002/14356007.b02\_04.pub2}{DOI:10.1002/14356007.b02\_04.pub2}}
	\end{itemize}
\end{frame}