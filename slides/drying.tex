% Outline: 30 minutes
\begin{comment}
	Why remove water
	Terminology
	Equilibrium curves (C+R, Santiago, p7)

	Psychrometry: useful for determining humidity content (kg//kg) of air at a given temperature and given RH
	Bound vs unbound moisture
	Drying curve


	% Next class:
	Equipment for removing water
	Fluidized bed system
\end{comment}

\begin{frame}\frametitle{Administrative}
	\begin{itemize}
		\item	Assignment 5 is posted (3 questions so far); questions 4 and 5 posted by Tuesday afternoon
		\item	Assignment 5 is due in the Chem Eng drop box by Monday, 03 December, at 16:00, or earlier.
		\item	Assignment 4 will be available for pick up on Thursday and Friday.
		\item	Midterm will be available to pick up Tuesday, Thursday, Friday.
		\item	Please fill in a course evaluation: \href{https://evals.mcmaster.ca}{https://evals.mcmaster.ca}
		\item	Confused about grades? There's a grading spreadsheet online
		\begin{itemize}
			\item	Please do not use averages and symbols calculated by Avenue
		\end{itemize}
	\end{itemize}
	\vspace{12pt}
	\begin{itemize}
		\item	Course review on Friday, 30 November. {\color{myOrange} \hfill $\longleftarrow$ {\textbf{Final class}}}
	\end{itemize}
\end{frame}

\begin{frame}\frametitle{Background}
	We consider \textbf{drying of solid products} here.
	\begin{itemize}
		\item	Remove {\color{myOrange}{liquid phase}} from {\color{myOrange}{solid phase}} by an {\color{myOrange}{ESA = thermal energy}}
		\item	It is the final separating step in many processes
		\begin{itemize}
			\item	pharmaceuticals
			\item	foods
			\item	crops, grains and cereal products
			\item	lumber, pulp and paper products
			\item	catalysts, fine chemicals
			\item	detergents
		\end{itemize}
	\end{itemize}
	Why dry?
	\begin{itemize}
		\item	packaging dry product is much easier than moist/wet product
		\item	reduces weight for shipping
		\item	preserves product from bacterial growth
		\item	stabilizes flavour and prolongs shelf-life in foods
		\item	provides desirable properties: e.g. flowability, crispiness
		\item	reduces corrosion: the ``corrosion triangle'': removes 1 of the 3
	\end{itemize}
\end{frame}

\begin{frame}\frametitle{The nature of water in solid material}
	% p4-142 Schweitzer
	% C+R v2, Ch16, p902
	\begin{columns}[c]
		\column{0.60\textwidth}
			
			\begin{center}
				At 20$^\circ\text{C}$
				
				\includegraphics[width=\textwidth]{\imagedir/separations/drying/equilibrium-moisture-contents-of-solids-C+Rv2,5ed,p902.png}
			\end{center}
			\vspace{-12pt}
			\see{Richardson and Harker, p 902}
		\column{0.50\textwidth}
			Material, when exposed to air with a certain humidity, will reach equilibrium with that air.
			
			\vspace{12pt}
			
			\begin{enumerate}
				\item	\textbf{Bound moisture}
				\begin{itemize}
					\item	adsorbed into material's capillaries and surfaces
					\item	or in cell walls of material
					\item	its vapour pressure is below water's partial pressure at this $T$
				\end{itemize}
				\item	\textbf{Free moisture}
				\begin{itemize}
					\item	water in excess of the above equilibrium water
				\end{itemize}
			\end{enumerate}
	\end{columns}
\end{frame}

\begin{frame}\frametitle{Drying: the heat and mass transfer view points}
	
	\begin{exampleblock}{}
		\begin{center}
			{\color{myGreen}{Both heat and mass transfer occur simultaneously}}
		\end{center}
	\end{exampleblock}
	
	\vspace{12pt}
	\begin{columns}[t]
		\column{0.50\textwidth}
			\textbf{Mass transfer}
			\begin{itemize}
				\item	Bring liquid from interior of product to surface
				\item	Vapourization of liquid at/near the surface
				\item	Transport of vapour into the bulk gas phase
			\end{itemize}
		\column{0.55\textwidth}
			\textbf{Heat transfer} from bulk gas phase to solid phase: 
			\begin{itemize}
				\item	portion of it used to vapourize the liquid ({\color{purple}{latent heat}})
				\item	portion remains in the solid as ({\color{purple}{sensible heat}})
			\end{itemize}
	\end{columns}
	\vspace{24pt}
	Key point: heat to vapourize the liquid is provided by the air stream
	%\todo{Figure here}
\end{frame}

\begin{frame}\frametitle{Terminology}
	\begin{center}
		\includegraphics[width=0.85\textwidth]{\imagedir/separations/drying/Phase_diagram_of_water.svg.png}
	\end{center}
	\see{\href{http://en.wikipedia.org/wiki/File:Phase\_diagram\_of\_water.svg}{Wikipedia File:Phase\_diagram\_of\_water.svg}}
\end{frame}

\begin{frame}\frametitle{Terminology}
	\begin{itemize}	
		\item	{\color{purple}{Partial pressure}}, recall, is the pressure due to water vapour in the water-air mixture
		\item	{\color{purple}{Vapour pressure}}, is the pressure exerted by (molecules of liquid water in the solid) on the gas phase in order to escape into the gas [a measure of volatility]
	\end{itemize}
	\begin{exampleblock}{}
		Moisture evaporates from a wet solid only when its vapour pressure exceeds the partial pressure
	\end{exampleblock}
	\begin{itemize}
		\item	Vapour pressure can be raised by heating the wet solid
	\end{itemize}
\end{frame}

% \item	Absolute humidity
% \item	Relative humidity (dimensionless?)
% \item	Wet bulb temperature

\begin{frame}\frametitle{Psychrometric chart}
	\vfill
	\begin{center}
		\includegraphics[width=\textwidth]{\imagedir/separations/drying/psychrometric-chart-Geankoplis.png}
	\end{center}
	\see{Geankoplis, p568; multiple internet sources have this chart digitized}
\end{frame}

\begin{frame}\frametitle{Terminology}
	\begin{itemize}
		\item	{\color{purple}{Dry bulb temperature}}: or just $T$ = ``\emph{temperature}'' {\tiny (nothing new here)}
		\begin{itemize}
			\item	the horizontal axis on the psychrometric chart
		\end{itemize}
		\item	{\color{purple}{Humidity}} = $\psi$ = mass of water vapour per kilogram of dry air
		\begin{itemize}
			\item	units are $ \left[ \displaystyle \frac{\text{kg water vapour}}{\text{kg dry air}}\right]$ 
			\item	called $H$ in many textbooks; always confused with enthalpy; so we will use $\psi$
			\item	units do not cancel, i.e. not dimensionless
			\item	the vertical axis on the psychrometric chart
		\end{itemize}
		\item	Maximum amount of water air can hold at a given $T$:
			\begin{itemize}
				\item	$\psi_S$ = {\color{purple}{saturation humidity}}
				\item	move up vertically to 100\% humidity
			\end{itemize}
		\item	{\color{purple}{Percentage humidity}} = $\displaystyle \frac{\psi}{\psi_S}\times 100$	
		\vspace{4pt}
		\item	{\color{purple}{Partial pressure}} we said is the pressure due to water vapour in the water-air mixture
		\vspace{4pt}
		\begin{itemize}
			\item	$\psi = \displaystyle \frac{\text{mass of water vapour}}{\text{mass of dry air}} = \displaystyle \frac{18.02}{28.97}\frac{p_A}{P-p_A}$
			\item	$p_A$ = partial pressure of water in the air
			\item	$P$ = total pressure = 101.325 kPa in this psychrometric chart
		\end{itemize}
	\end{itemize}
\end{frame}

\begin{frame}\frametitle{Terminology}
	\begin{itemize}
		\item	{\color{purple}{Dew point}}: the temperature to which you must cool the air/vapour mixture to just obtain saturation (100\% humidity), i.e. condensation just starts to occur.
	\end{itemize}
	
	\begin{center}
		\includegraphics[width=.8\textwidth]{\imagedir/separations/drying/psychrometric-chart-Geankoplis-dew-point.png}
	\end{center}
	
	\textbf{Example:} Air at 65\degC$\,$ and 10\% humidity has a dew point temperature of 25\degC. This parcel of air contains 0.021 kg of water per kilogram of dry air.
\end{frame}

\begin{frame}\frametitle{Terminology}
	\begin{itemize}
		\item	 {\color{purple}{Humid heat}}: amount of energy to raise 1kg of air and the water vapour it contains by 1\degC
		
		\begin{exampleblock}{}
			\[c_S = 1.005 + 1.88\psi\]
		\end{exampleblock}
		\begin{itemize}
			\item	$c_S$ has units $\left[\displaystyle \frac{\text{kJ}}{\text{(kg dry air)(K)}} \right]$
			\item	1.005 $\left[\displaystyle \frac{\text{kJ}}{\text{(kg dry air)(K)}} \right]$ is heat capacity of dry air
			\item	1.88 $\left[\displaystyle \frac{\text{kJ}}{\text{(kg water vapour)(K)}} \right]$ is heat capacity of water
			\item	$\psi$ is the humidity $ \left[ \displaystyle \frac{\text{kg water vapour}}{\text{kg dry air}}\right]$ 
		\end{itemize}
	\end{itemize}
\end{frame}

\begin{frame}\frametitle{Terminology: adiabatic saturation}
	Consider a stream of air at temperature $T$ and humidity $\psi$. It contacts fine water droplets long enough to reach equilibrium. The leaving gas has temperature $T_S$ and humidity $\psi_S$.
		
	\begin{center}
		\includegraphics[width=.95\textwidth]{\imagedir/separations/drying/saturation-temperature-Geankoplis-p570.png}
	\end{center}
	
	We expect outlet gas: $T_S < T$ and $\psi_S > \psi$
	
	\vspace{6pt}
	The energy to evaporate liquid water into the leaving air stream comes from the air.
\end{frame}

\begin{frame}\frametitle{Terminology: adiabatic saturation}
	Quantify it: do an enthalpy balance at $T_\text{ref} = T_S$ {\scriptsize (i.e. disregard water)}
	\begin{center}
		\includegraphics[width=0.4\textwidth]{\imagedir/separations/drying/saturation-temperature-Geankoplis-p570.png}
	\end{center}
	\vspace{-6pt}
	\begin{exampleblock}{Enthalpy of vapour phase entering:}
		\[c_S\left(T - T_S \right) + (\psi) (\Delta H_\text{vap})\]
	\end{exampleblock}
	\begin{exampleblock}{Enthalpy of vapour phase leaving:}
		\[c_S\left(T_S - T_S \right) + (\psi_S) (\Delta H_\text{vap})\]
	\end{exampleblock}
	
	\[	
		\displaystyle \frac{y\text{-axis change}}{x\text{-axis change}} = \frac{\psi - \psi_S}{T - T_S} = - \frac{c_S}{\Delta H_\text{vap}} = \frac{1.005 + 1.88 \psi}{\Delta H_\text{vap}}
	\]
	These are the diagonal sloped lines on the psychrometric chart: {\color{purple}{adiabatic saturation curves}}.
\end{frame}

\begin{frame}\frametitle{Exercise}
	An air stream at 70\degC$\,$ and carrying 0.055 kg water per kg dry air is adiabatically contacted with liquid water until it reaches equilibrium. The process is continuous and operating at steady-state.
	
	\begin{enumerate}
		\item	What is the percentage humidity of the incoming air stream?
		\item	What is the percentage humidity of the air stream leaving?
		\item	What is the humidity [mass/mass] of the air stream leaving?
		\item	What is the temperature of the air stream leaving?
		\item	If the contacting takes place in a unit on the previous slide, what is the mass of inlet make-up water required at steady-state operation?
	\end{enumerate}
\end{frame}


% \begin{frame}\frametitle{Drying profiles}
% 	
% \end{frame}

% \begin{frame}\frametitle{Equipment}
% 	Multiple equipment types to dry material:
% 	\begin{itemize}
% 		\item	each have relative advantages and disadvantages
% 		\item	our purpose is not to cover their details
% 		\item	in practice: you would work in consultation with vendors
% 		\item	in practice: plenty of trade literature on the topics {\color{myGreen}{(SDL\emph{!})}}
% 	\end{itemize}
% 	\vspace{12pt}
% 	Some major distinctions though:
% 	\begin{itemize}
% 		\item	\textbf{{\color{myGreen}{mode of operation}}}: batch (low volume) \emph{vs} continuous
% 		\item	how the \textbf{{\color{myGreen}{heat is provided}}}:
% 		\begin{itemize}
% 			\item	{\color{purple}{direct heat}}: convective or adiabatic; provides heat and sweeps away moisture
% 			\item	{\color{purple}{indirect heat}}: non-adiabatic, i.e. by conduction or radiation; e.g microwave
% 		\end{itemize}
% 		\item	\textbf{{\color{myGreen}{degree of agitation}}}
% 		\begin{itemize}
% 			\item	stationary material
% 			\item	fluidized or mixed in some way
% 		\end{itemize}
% 	\end{itemize}
% \end{frame}
% 
% \begin{frame}\frametitle{How to choose the equipment*}
% 	\begin{itemize}
% 		\item	Strongly dependent of feed presentation
% 		\begin{itemize}
% 			\item	is it: solid, slurry, paste, flowing powder, filter cake, fibrous, \emph{etc}
% 		\end{itemize}
% 		\item	Heating choice: temperature-sensitive if convective heat is directly applied
% 		\item	Agitation: 
% 		\begin{itemize}
% 			\item	produce fines (dust hazard) or fragile material
% 			\item	good mixing implies good heat distribution
% 			\item	stationary product: can form hot-spots in the solid
% 		\end{itemize}
% 	\end{itemize}
% 	
% 	General choices are between:
% 	\begin{enumerate}
% 		\item	shelf/tray dryers
% 		\item	continuous tunnels
% 		\item	rotary dryers
% 		\item	drums
% 		\item	spray dryers
% 		\item	fluidized beds
% 	\end{enumerate}
% 	\vspace{4pt}
% 	\hrule
% 	\vspace{4pt}
% 	* See Schweitzer; See Perry's
% 	%p4-168 Schwietzer, p4-161 and 4-162
% \end{frame}
% 
% \begin{frame}\frametitle{Some equipment examples}
% 	\textbf{Rotating dryer}
% 	\begin{columns}[t]
% 		\column{0.60\textwidth}
% 			\begin{center}
% 				\includegraphics[width=\textwidth]{\imagedir/separations/drying/rotary-dryer-Schweitzer-4-162.png}
% 			\end{center}
% 		\column{0.40\textwidth}
% 			\begin{center}
% 				\includegraphics[width=\textwidth]{\imagedir/separations/drying/rotary-dryer-Schweitzer-p4-161-detail.png}
% 			\end{center}
% 	\end{columns}
% 	\see{Schweitzer, p 4-161 and 4-162}
% \end{frame}
% 
% \begin{frame}\frametitle{Some equipment examples}
% 	% In 2013: add figure from p 768 Seader, Henly Roper,
% 	\textbf{Fluidized bed dryer}
% 	% From  p4-178 to 181 in Scweitzer
% 	\begin{columns}[t]
% 		\column{0.60\textwidth}
% 			\begin{center}
% 				\includegraphics[height=0.75\textheight]{\imagedir/separations/drying/fluid-bed-dryer-flow-patter-Schweitzer-p4-179.png}
% 			\end{center}
% 			\vspace{-12pt}
% 			\see{Schweitzer, p 4-179}
% 		\column{0.45\textwidth}
% 			\begin{itemize}
% 				\item	upward flowing air stream (elutriation)
% 				\item	turbulent mixing: good heat and mass transfer
% 				\item	uniform solid temperature
% 				\item	solids are gently treated
% 				\item	solids are retrieved via gravity and cyclones
% 				\item	fluidizing air is scrubbed before being vented
% 			\end{itemize}
% 	\end{columns}
% \end{frame}

% Add in 2013
% \begin{frame}\frametitle{Some equipment examples}
% 	Atomizing drying
% 	e.g. for milk powder
% \end{frame}

\begin{frame}\frametitle{References used (in alphabetical order)}
	\begin{itemize}
		\item	Geankoplis, ``Transport Processes and Separation Process Principles'', 4th edition, chapter 09
		\item	Perry's Chemical Engineers' Handbook, Chapter 12
		\item	Richardson and Harker, ``Chemical Engineering, Volume 2'', 5th edition, chapter 16	
		\item	Schweitzer, ``Handbook of Separation Techniques for Chemical Engineers'', Chapter 4.10
		\item	Seader, Henly and Roper, ``Separation Process Principles'', 3rd edition, chapter 18
		\item	Uhlmann's Encyclopedia, ``Drying'', {\tiny \href{http://dx.doi.org/10.1002/14356007.b02\_04.pub2}{DOI:10.1002/14356007.b02\_04.pub2}}
	\end{itemize}
\end{frame}