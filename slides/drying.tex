% Outline: 30 minutes

Why remove water
Terminology
Equilibrium curves (C+R, Santiago, p7)

Psychrometry: useful for determining humidity content (kg//kg) of air at a given temperature and given RH
Bound vs unbound moisture
Drying curve


% Next class:
Equipment for removing water
Fluidized bed system



\begin{frame}\frametitle{Background}
	http://en.wikipedia.org/wiki/Drying
\end{frame}

\begin{frame}\frametitle{Intro}
	p4-142 Scweitezer
	Santiago, p 4 and p5
\end{frame}

\begin{frame}\frametitle{Equipment}
	p4-168 Scwietzer
	p4-161 and 4-162
	Santiago's slides: p6
\end{frame}

\begin{frame}\frametitle{Terminology}
	
\end{frame}

\begin{frame}\frametitle{Mass transfer concepts}
	
\end{frame}

\begin{frame}\frametitle{Heat transfer concepts}
	
\end{frame}

\begin{frame}\frametitle{Bringing mass and heat transfer together}
	
\end{frame}

\begin{frame}\frametitle{Examples}
	
\end{frame}

\begin{frame}\frametitle{Fluidized beds}
	p 768 Seader, Henly Roper,
	p4-178 to 181 in Scweitzer
\end{frame}


\begin{frame}\frametitle{References used (in alphabetical order)}
	\begin{itemize}
		\item	Geankoplis, ``Transport Processes and Separation Process Principles'', 4th edition, chapter 09
		\item	Perry's Chemical Engineers' Handbook, Chapter 12
		\item	Richardson and Harker, ``Chemical Engineering, Volume 2'', 5th edition, chapter 16	
		\item	Schweitzer, ``Handbook of Separation Techniques for Chemical Engineers'', Chapter 4.10
		\item	Seader, Henly and Roper, ``Separation Process Principles'', 3rd edition, chapter 18
		\item	Uhlmann's Encyclopedia, ``Drying'', {\tiny \href{http://dx.doi.org/10.1002/14356007.b02\_04.pub2}{DOI:10.1002/14356007.b02\_04.pub2}}
	\end{itemize}
\end{frame}