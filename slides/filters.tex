% Sedimentation vs Filtration: Svarovsky book
% Selection between filters, centrifuges, sedimentation, etc     C+R v6, p 409, Fig 10.10
% Filter media       Table 10.2: C+R v6, p 411

\begin{frame}\frametitle{Filtration section}
	{\color{purple}{Filtration}}: a pressure difference that causes separation of solids from {\color{purple}{slurry}} by means of a {\color{purple}{porous medium}} (e.g. filter paper or cloth), which retains the solids and allows the {\color{purple}{filtrate}} to pass
	
	\begin{center}
		\includegraphics[width=\textwidth]{\imagedir/separations/filtration/terminology-Geakoplis-p905.png}
	\end{center}
	\see{Geankoplis, p 905}
\end{frame}

\begin{frame}\frametitle{References on filtration}
	\begin{itemize}
		\item	Geankoplis, ``Transport Processes and Separation Process Principles'', 4th edition, chapter YYYYY.
		\item	\href{http://accessengineeringlibrary.com/browse/perrys-chemical-engineers-handbook-eighth-edition}{Perry's Chemical Engineers' Handbook}, 8th edition, chapter YYYYY.
		\item	Seader, Henley and Roper, ``Separation Process Principles'', 3rd edition, chapter 19.
		\item	Uhlmann's Encyclopedia, ``Filtration 1. Fundamentals'', {\tiny \href{http://onlinelibrary.wiley.com/doi/10.1002/14356007.b02\_10.pub3/abstract}{DOI:10.1002/14356007.b02\_10.pub3}}
	\end{itemize}
\end{frame}

\begin{frame}\frametitle{Why filtration?}
	Example: alkaline protease, used as an additive in laundry detergent
	\begin{center}
		\includegraphics[width=\textwidth]{\imagedir/separations/filtration/MIT-OCW-Filtration-10-445-separation-processes-for-biochemical-products-summer-2005-lecture-10.png}
	\end{center}
	\see{MIT OCW, \href{http://ocw.mit.edu/courses/chemical-engineering/10-445-separation-processes-for-biochemical-products-summer-2005/lecture-notes/lecture_10.pdf}{Course 10-445, Separation Processes for Biochemical Products, 2005}}
\end{frame}

\begin{frame}\frametitle{Commercial units: rotary drum filter}
	\begin{center}
		\includegraphics[width=.9\textwidth]{\imagedir/separations/filtration/Seader-Rotary-drum-vacuum-filter-fig-19-13.jpg}
	\end{center}
	\vspace{-6pt}
	\see{Seader, modified from fig 19-13}
\end{frame}

\begin{frame}\frametitle{Commercial units: plate and frame}
	\begin{center}
		\includegraphics[width=\textwidth]{\imagedir/separations/filtration/alternating-plates-frame-press-flickr-6322212861_3281570406_b.jpg}
	\end{center}
\end{frame}

\begin{frame}\frametitle{Commercial units: plate and frame}
	\begin{center}
		\includegraphics[width=\textwidth]{\imagedir/separations/filtration/plate-frame-press-pushed-flickr-6322212987_ef5fb8a362_b.jpg}
	\end{center}
\end{frame}

\begin{frame}\frametitle{Commercial units: plate and frame (beer clarification)}
	\begin{center}
		\includegraphics[width=\textwidth]{\imagedir/separations/filtration/plate-and-frame-filter-press-flickr-8457498256_979ab6c31d.jpg}
	\end{center}
	% After a looser filtration (which removes yeast and other relatively large particulates, all the beers at Peabody Heights —Baltimore, Maryland's newest production brewery— are sterile-filtered through a Schenk plate-and-frame filter at 0.5 microns ... small enough to remove bacteria! (A micron is one-millionth of a meter.)	
\end{frame}

