% Sedimentation vs Filtration: Svarovsky book
% Selection between filters, centrifuges, sedimentation, etc     C+R v6, p 409, Fig 10.10
% Filter media       Table 10.2: C+R v6, p 411

\begin{frame}\frametitle{Filtration}
	\begin{center}
		\includegraphics[width=\textwidth]{\imagedir/separations/filtration/lab-filtration-flickr-518510770_9c52170842_b.png}
	\end{center}
\end{frame}

\begin{frame}\frametitle{Filtration section}
	{\color{purple}{Filtration}}: a pressure difference that causes separation of solids from {\color{purple}{slurry}} by means of a {\color{purple}{porous medium}} (e.g. filter paper or cloth), which retains the solids and allows the {\color{purple}{filtrate}} to pass
	
	\begin{center}
		\includegraphics[width=\textwidth]{\imagedir/separations/filtration/terminology-Geakoplis-p905.png}
	\end{center}
	\see{Geankoplis, p 905}
\end{frame}

\begin{frame}\frametitle{References on filtration}
	\begin{itemize}
		\item	Geankoplis, ``Transport Processes and Separation Process Principles'', 4th edition, chapter YYYYY.
		\item	\href{http://accessengineeringlibrary.com/browse/perrys-chemical-engineers-handbook-eighth-edition}{Perry's Chemical Engineers' Handbook}, 8th edition, chapter YYYYY.
		\item	Seader, Henley and Roper, ``Separation Process Principles'', 3rd edition, chapter 19.
		\item	Uhlmann's Encyclopedia, ``Filtration 1. Fundamentals'', {\tiny \href{http://onlinelibrary.wiley.com/doi/10.1002/14356007.b02\_10.pub3/abstract}{DOI:10.1002/14356007.b02\_10.pub3}}
	\end{itemize}
\end{frame}

\begin{frame}\frametitle{Why filtration?}
	Example: alkaline protease, used as an additive in laundry detergent
	\begin{center}
		\includegraphics[width=\textwidth]{\imagedir/separations/filtration/MIT-OCW-Filtration-10-445-separation-processes-for-biochemical-products-summer-2005-lecture-10.png}
	\end{center}
	\see{MIT OCW, \href{http://ocw.mit.edu/courses/chemical-engineering/10-445-separation-processes-for-biochemical-products-summer-2005/lecture-notes/lecture_10.pdf}{Course 10-445, Separation Processes for Biochemical Products, 2005}}
\end{frame}

\begin{frame}\frametitle{Commercial units: rotary drum filter}
	\begin{center}
		\includegraphics[width=.9\textwidth]{\imagedir/separations/filtration/Seader-Rotary-drum-vacuum-filter-fig-19-13.jpg}
	\end{center}
	\vspace{-6pt}
	\see{Seader, modified from fig 19-13}
\end{frame}

\begin{frame}\frametitle{Commercial units: plate and frame}
	\begin{center}
		\includegraphics[width=\textwidth]{\imagedir/separations/filtration/alternating-plates-frame-press-flickr-6322212861_3281570406_b.jpg}
	\end{center}
\end{frame}

\begin{frame}\frametitle{Commercial units: plate and frame}
	\begin{center}
		\includegraphics[width=\textwidth]{\imagedir/separations/filtration/plate-frame-press-pushed-flickr-6322212987_ef5fb8a362_b.jpg}
	\end{center}
\end{frame}

\begin{frame}\frametitle{Commercial units: plate and frame (beer clarification)}
	\begin{center}
		\includegraphics[width=\textwidth]{\imagedir/separations/filtration/plate-and-frame-filter-press-flickr-8457498256_979ab6c31d.jpg}
	\end{center}
	% After a looser filtration (which removes yeast and other relatively large particulates, all the beers at Peabody Heights —Baltimore, Maryland's newest production brewery— are sterile-filtered through a Schenk plate-and-frame filter at 0.5 microns ... small enough to remove bacteria! (A micron is one-millionth of a meter.)	
\end{frame}

\begin{frame}\frametitle{Questions to discuss}
	\begin{columns}[t]
		\column{0.60\textwidth}
			\begin{enumerate}
				\item	What characteristics of a filtration system will you use to {\color{myOrange}{\textbf{judge} the unit's performance}}?

				\vspace{36pt}
				\item	What factors can be used to {\color{myOrange}{\textbf{adjust}}} the units's performance?
			\end{enumerate}
		\column{0.40\textwidth}
			\begin{center}
				\includegraphics[width=\textwidth]{\imagedir/separations/filtration/flickr-4857096113_96e8686f7e_b.jpg}
			\end{center}
			Example: a rotary drum filter
	\end{columns}
\end{frame}

\begin{frame}\frametitle{Poiseuille's law}
	Recall from your fluid flow course that \textbf{laminar flow} in a pipe (no resistance):
	\begin{exampleblock}{}
		\[\dfrac{-\Delta P}{L_c} = \dfrac{32\,\, \mu\,\, v}{D^2} \label{P}
		\]
	\end{exampleblock}
	\[
		\begin{array}{rcll}
			-\Delta P&=& \text{pressure drop from start (high P) to end of tube}&[\text{Pa}]\\
			L_c	    &=& \text{length being considered}                     	&[\text{m}]\\
			\mu 	&=& \text{fluid viscosity}  							&[\text{Pa.s}]\\
			v   	&=& \text{fluid's velocity in the pipe}					&[\text{m.s}^{-1}]\\
			D 		&=& \text{pipe diameter} 								&[\text{m}]
		\end{array}
	\]	
\end{frame}

\begin{frame}\frametitle{Carmen-Kozeny equation through a bed of solids (cake)}
	$\dfrac{-\Delta P}{L_c} = \dfrac{32\,\, \mu\,\, v}{D^2} \label{P}$ {\small from which we get the Carmen-Kozeny equation:}
	\begin{exampleblock}{}
		$\dfrac{-\Delta P_C}{L_c} = k_1 \cdot \mu \cdot \dfrac{v}{\epsilon} \cdot \left( \dfrac{(1-\epsilon)S_0}{\epsilon} \right)^2$
	\end{exampleblock}
	\[
		\begin{array}{rcll}
			-\Delta P_C&=& \text{pressure drop through the cake}&[\text{Pa}]\\
			k_1	    &=& 4.17, \text{a constant}                     	&[-]\\
			\epsilon&=& \text{void fraction, or porosity {\color{myOrange}{typical values?}}}				&[-]\\
			S_0   	&=& \text{specific area per unit volume}			&[\text{m}^2\text{.m}^{-3} = \text{m}^{-1}]
		\end{array}
	\]
	\begin{itemize}
		\item	$S_0$ = specific surface area per unit volume is a property of the solids
		\item	Prove in the next assignment, for spheres, $S_0 = \dfrac{6}{d} = f(d)$
	\end{itemize}
\end{frame}
