% Sedimentation vs Filtration: Svarovsky book
% Selection between filters, centrifuges, sedimentation, etc     C+R v6, p 409, Fig 10.10
% Filter media Table 10.2: C+R v6, p 411
% MInimize costs: reduce Delta P and reduce A??

% 27 September 2013: Guest lecture, Claudia Chan
\begin{frame}\frametitle{Filtration}
	\begin{center}
		\includegraphics[width=\textwidth]{\imagedir/separations/filtration/lab-filtration-flickr-518510770_9c52170842_b.png}
	\end{center}
\end{frame}

\begin{frame}\frametitle{Filtration section}
	{\color{purple}{Filtration}}: a pressure difference that causes separation of solids from {\color{purple}{slurry}} by means of a {\color{purple}{porous medium}} (e.g. filter paper or cloth), which retains the solids and allows the {\color{purple}{filtrate}} to pass

	\begin{center}
		\includegraphics[width=\textwidth]{\imagedir/separations/filtration/terminology-Geakoplis-p905.png}
	\end{center}
	\see{Geankoplis, p 905}
\end{frame}

\begin{frame}\frametitle{References on filtration}
	\begin{itemize}
		\item	Geankoplis, ``Transport Processes and Separation Process Principles'', 4th edition, chapter 14.
		\item	\href{http://accessengineeringlibrary.com/browse/perrys-chemical-engineers-handbook-eighth-edition}{Perry's Chemical Engineers' Handbook}, 8th edition, chapter 18.
		\item	Seader, Henley and Roper, ``Separation Process Principles'', 3rd edition, chapter 19.
		\item	Uhlmann's Encyclopedia, ``Filtration 1. Fundamentals'', {\tiny \href{http://onlinelibrary.wiley.com/doi/10.1002/14356007.b02\_10.pub3/abstract}{DOI:10.1002/14356007.b02\_10.pub3}}
	\end{itemize}
\end{frame}

\begin{frame}\frametitle{Why filtration?}
	Example: alkaline protease, used as an additive in laundry detergent
	\begin{center}
		\includegraphics[width=\textwidth]{\imagedir/separations/filtration/MIT-OCW-Filtration-10-445-separation-processes-for-biochemical-products-summer-2005-lecture-10.png}
	\end{center}
	\see{MIT OCW, \href{http://ocw.mit.edu/courses/chemical-engineering/10-445-separation-processes-for-biochemical-products-summer-2005/lecture-notes/lecture_10.pdf}{Course 10-445, Separation Processes for Biochemical Products, 2005}}
\end{frame}

\begin{frame}\frametitle{Commercial units: rotary drum filter}
	\begin{center}
		\includegraphics[width=.9\textwidth]{\imagedir/separations/filtration/Seader-Rotary-drum-vacuum-filter-fig-19-13.jpg}
	\end{center}
	\vspace{-6pt}
	\see{Seader, modified from fig 19-13}
	\see{\href{http://www.youtube.com/watch?v=5Aj5owWp7WQ}{YouTube video}}
\end{frame}

\begin{frame}\frametitle{Commercial units: plate and frame}
	\begin{center}
		\includegraphics[width=\textwidth]{\imagedir/separations/filtration/plate-frame-press-pushed-flickr-6322212987_ef5fb8a362_b.jpg}
	\end{center}
\end{frame}

\begin{frame}\frametitle{Commercial units: plate and frame}
	\begin{center}
		\includegraphics[width=\textwidth]{\imagedir/separations/filtration/alternating-plates-frame-press-flickr-6322212861_3281570406_b.jpg}
	\end{center}
	\see{Also see \href{http://www.youtube.com/watch?feature=player_detailpage&v=btsXMiVtjcw}{YouTube video}}
\end{frame}

\begin{frame}\frametitle{Commercial units: plate and frame (beer clarification)}
	\begin{center}
		\includegraphics[width=\textwidth]{\imagedir/separations/filtration/plate-and-frame-filter-press-flickr-8457498256_979ab6c31d.jpg}
	\end{center}
	% After a looser filtration (which removes yeast and other relatively large particulates, all the beers at Peabody Heights —Baltimore, Maryland's newest production brewery— are sterile-filtered through a Schenk plate-and-frame filter at 0.5 microns ... small enough to remove bacteria! (A micron is one-millionth of a meter.)
\end{frame}

\begin{frame}\frametitle{Questions to discuss}
	\begin{columns}[t]
		\column{0.60\textwidth}
			\begin{enumerate}
				\item	What characteristics of a filtration system will you use to {\color{myOrange}{\textbf{judge} the unit's performance}}?

				\vspace{36pt}
				\item	What factors can be used to {\color{myOrange}{\textbf{adjust}}} the units's performance?
			\end{enumerate}
		\column{0.40\textwidth}
			\begin{center}
				\includegraphics[width=\textwidth]{\imagedir/separations/filtration/flickr-4857096113_96e8686f7e_b.jpg}
			\end{center}
			Example: a rotary drum filter
	\end{columns}
\end{frame}

\begin{frame}\frametitle{Poiseuille's law}
	Recall from your fluid flow course that \textbf{laminar flow} in a pipe (considering no resistance):
	\begin{exampleblock}{}
		\[\dfrac{-\Delta P}{L_c} = \dfrac{32\,\, \mu\,\, v}{D^2} \label{P}
		\]
	\end{exampleblock}
	\[
		\begin{array}{rcll}
			-\Delta P&=& \text{pressure drop from start (high P) to end of tube}&[\text{Pa}]\\
			L_c	    &=& \text{length being considered}                     	&[\text{m}]\\
			\mu 	&=& \text{fluid viscosity}  							&[\text{Pa.s}]\\
			v   	&=& \text{fluid's velocity in the pipe}					&[\text{m.s}^{-1}]\\
			D 		&=& \text{pipe diameter} 								&[\text{m}]
		\end{array}
	\]
\end{frame}

\begin{frame}\frametitle{Carmen-Kozeny equation through a bed of solids (cake)}
	$\dfrac{-\Delta P}{L_c} = \dfrac{32\,\, \mu\,\, v}{D^2} \label{P}$ {\small from which we derive the Carmen-Kozeny equation:}
	\begin{exampleblock}{}
		$\dfrac{-\Delta P_c}{L_c} = k_1 \cdot \mu \cdot \dfrac{v}{\epsilon} \cdot \left( \dfrac{(1-\epsilon)S_0}{\epsilon} \right)^2$
	\end{exampleblock}
	\[
		\begin{array}{rcll}
			-\Delta P_c&=& \text{pressure drop through the cake}&[\text{Pa}]\\
			k_1	    &=& 4.17, \text{a constant}                     	&[-]\\
			\epsilon&=& \text{void fraction, or porosity {\color{myOrange}{typical values?}}}				&[-]\\
			S_0   	&=& \text{specific area per unit volume}			&[\text{m}^2\text{.m}^{-3} = \text{m}^{-1}]
		\end{array}
	\]
	\begin{itemize}
		\item	$S_0$ = specific surface area per unit volume is a property of the solids
		\item	Prove in the next assignment, for spheres, $S_0 = \dfrac{6}{d} = f(d)$
	\end{itemize}
\end{frame}

\begin{frame}\frametitle{Solids balance}
	\begin{columns}[t]
		\column{0.65\textwidth}
			Mass of solids in the filter cake = {\color{brown}$A \,\,L_c\left(1-\epsilon\right)\rho_p$}

			\vspace{5pt}
			Mass of fluid trapped in the filter cake = {\color{myOrange}$A \,\,L_c\,\, \epsilon \,\,\rho_f \approx \text{small}$}

			\vspace{5pt}
			Mass of fluid in the filtrate = {\color{blue}$V \rho_f$}

			\vspace{5pt}
			{\scriptsize \color{red}What key assumptions are being made here?}

		\column{0.40\textwidth}
			\vspace{-48pt}
			\begin{center}
				\includegraphics[width=\textwidth]{\imagedir/separations/filtration/mass-balance-filtration.png}
			\end{center}

	\end{columns}
	\vspace{1pt}
	\[
		\begin{array}{rcll}
			\rho_p  &=& \text{solid particle density}				&[\text{kg.m}^{-3}]\\
			\rho_f  &=& \text{fluid density}						&[\text{kg.m}^{-3}]\\
			V		&=& \text{volume of filtrate collected} 		&[\text{m}^3]\\
			A		&=& \text{cross sectional area for filtration}  &[\text{m}^2]
		\end{array}
	\]
	Then define {\color{purple}slurry concentration}: \\
	%\vspace{12pt}
	\[ C_S = \dfrac{\text{mass of dry solids}}{\text{volume of liquid in slurry}}  \approx \dfrac{\text{mass of dry solids}}{\text{volume of filtrate}} = \dfrac{A L_c \left(1-\epsilon\right)\rho_p}{V} \]
\end{frame}

\begin{frame}\frametitle{Exercise}
	\begin{enumerate}
		\item	Calculate the mass of solids in the cake for the case when
		\begin{itemize}
			\item	$\rho_p = 3000 \text{kg.m}^{-3}$
			\item	$A = 8 \text{m}^2$
			\item	$L_c = 10 \text{cm}$
		\end{itemize}

		\item	Also calculate the mass of water in the cake.
	\end{enumerate}
\end{frame}

\begin{frame}\frametitle{Deriving the flow through the filter}
	Our standard equation for fluid flow:
	\begin{exampleblock}{}
		\[\dfrac{1}{A} \cdot \dfrac{dV}{dt} = v = \dfrac{Q}{A}\]
	\end{exampleblock}
	for a given velocity $v$, and volumetric feed flow rate, $Q$.

	\vspace{12pt}
	But, from the Carmen-Kozeny equation:
	\[
		\begin{array}{rcl}
			\dfrac{-\Delta P_c}{L_c} &=& \dfrac{k_1  \mu  v \left(1-\epsilon\right)^2 S_0^2}{\epsilon^3}\\
			\vspace{6pt}
			v &=& \dfrac{\left(-\Delta P_c\right) (\epsilon)^3}{(\mu) (k_1) ({\color{blue}L_c})  (1-\epsilon)^2 (S_0^2)}\\

			\vspace{6pt}
			&&\text{from our definition for $C_S$ we can solve for {\color{blue}$L_c$}} \\
			{\color{blue}L_c} &=&  \dfrac{C_S V}{A \left(1-\epsilon\right)\rho_p}\\
			\vspace{6pt}
			\dfrac{1}{A} \cdot \dfrac{dV}{dt} = v &=& \dfrac{\left(-\Delta P_c\right) (A) \cancel{\left(1-\epsilon\right)} {\color{myOrange}(\epsilon)^3 (\rho_p)}}{(\mu) (C_S) (V) \cancel{\left(1-\epsilon\right)} {\color{myOrange}(k_1) \left(1-\epsilon\right) (S_0^2)}}
			= \dfrac{-\Delta P_c}{\mu C_S V \,\, {\color{myOrange}\mathbf{ \alpha}}\,\,/ A}
		\end{array}
	\]
\end{frame}

\begin{frame}\frametitle{The general filtration equation}
	\begin{exampleblock}{}
		\[\dfrac{1}{A} \cdot \dfrac{dV}{dt} = \dfrac{-\Delta P_c}{\mu C_S V  \, \mathbf{ \alpha}\, / A}
			\]
	\end{exampleblock}
	\[
		\begin{array}{rcll}
			C_S  	&=& \text{slurry concentration}				&[\text{(kg dry solids)/(m}^{3}\text{ filtrate)}]\\
			\alpha	&=& \text{specific cake resistance} 		&[\text{m.kg}^{-1}]
		\end{array}
	\]
	All aspects of engineering obey this general law:
	\begin{exampleblock}{}
		\[
		\begin{array}{rcl}
			{\color{red}J = \text{flux} = \dfrac{\text{transfer rate}}{\text{transfer area}}} &{\color{red}=}& {\color{red}\dfrac{\text{driving force}}{\text{resistance}}}\\
			\vspace{6pt}
			\text{including the filtration equation:}\\
			\vspace{6pt}
			\dfrac{1}{A} \cdot \dfrac{dV}{dt} &=& \dfrac{-\Delta P_c}{\mu  C_S V \, \mathbf{ \alpha}\, / A} \\
			\dfrac{1}{A} \cdot \dfrac{dV}{dt} &=& {\color{myBlue}\dfrac{-\Delta P_c}{\mu R_c}}
		\end{array}
		\]
	\end{exampleblock}
	\vspace{-14pt}
	\[
		\begin{array}{rcll}
			R_c  &=& \text{resistance due to the cake = $\dfrac{C_S V \, \mathbf{ \alpha}}{A}$}				&[\text{m}^{-1}]\\
		\end{array}
	\]
\end{frame}

\begin{frame}\frametitle{Resistance due to the filter medium}
	\begin{columns}[t]
		\column{0.60\textwidth}
			In a similar way, we can define the filter {\color{myGreen}medium}'s resistance:
			\begin{exampleblock}{}
				\[\dfrac{1}{A} \cdot \dfrac{dV}{dt} = {\color{myGreen}\dfrac{-\Delta P_m}{\mu R_m}} \]
			\end{exampleblock}
		\column{0.40\textwidth}
			\vspace{-24pt}
			\begin{center}
				\includegraphics[width=\textwidth]{\imagedir/separations/filtration/filtration-pressures-resistances.png}
			\end{center}
	\end{columns}
	\[
		\begin{array}{rcll}
			-\Delta P_m		&=& \text{pressure drop across the medium}			&[\text{Pa}]\\
			R_m  			&=& \text{resistance due to the filter medium}		&[\text{m}^{-1}]\\
		\end{array}
	\]
	\textbf{Notes}:
	\begin{itemize}
		\item	From a practical standpoint, $R_m$ is empirical for the given filter
		\item	We wrap up all other minor resistances into $R_m$ also (e.g. pipe flow into/out of filter)
		\item	The flux through the {\color{myBlue}filter cake} is exactly the same as through the {\color{myGreen} medium}
		\item	After filtration gets started, we very often have ${\color{myGreen}R_m} \ll {\color{myBlue}R_c}$
	\end{itemize}
\end{frame}

\begin{frame}\frametitle{Bringing it all together}
	As with resistances in series (you learned in Physics I), we have:
	\begin{exampleblock}{}
		\[\dfrac{1}{A} \cdot \dfrac{dV}{dt} = \left({\color{myGreen}\dfrac{-\Delta P_m}{\mu R_m}} \,\,\text{add with}\,\, {\color{myBlue}\dfrac{-\Delta P_c}{\mu R_c}}\right) = \dfrac{-\Delta P_\text{tot}}{\mu \left({\color{myGreen}R_m} + {\color{myBlue}R_c} \right)}\]
	\end{exampleblock}
	this is called the {\color{purple} general filtration equation}.
	\[
		\begin{array}{rcll}
			R_c  &=& \text{resistance due to the cake}				&[\text{m}^{-1}]\\
			R_m  &=& \text{resistance due to the medium}			&[\text{m}^{-1}]\\
			-\Delta P_\text{tot}	&=& \text{total pressure drop} = -(\Delta P_c + P_m)	&[\text{Pa}]
		\end{array}
	\]
\end{frame}

\begin{frame}\frametitle{Questions}
	How would you use this equation?
	\begin{exampleblock}{}
		\[\dfrac{1}{A} \cdot \dfrac{dV}{dt} = \dfrac{-\Delta P_\text{tot}}{\mu \left({\color{myGreen}R_m} + {\color{myBlue}R_c} \right)}\]
	\end{exampleblock}

	\begin{enumerate}
		\item	to determine the medium resistance?
		\item	to determine the cake resistance?
		\item	to find the utility cost of operating the filter?
		\item	to predict the flow for a given filter when your boss wants a higher throughput?
	\end{enumerate}
\end{frame}

% 01 October 2013
\begin{frame}\frametitle{Cake filtration}
	\begin{columns}[t]
		\column{0.70\textwidth}
			\begin{center}
				\includegraphics[width=\textwidth]{\imagedir/separations/filtration/cake-filtration.png}
			\end{center}

		\column{0.45\textwidth}
			\begin{itemize}
				\item	most widely applied
				\item	functions only when particles have been trapped on the medium, forming a bed/cake
				\item	the cake is the filtering element, not the medium
				\item	medium's pores are larger than particles often
				\item	e.g. filter press, rotary vacuum drum filter
			\end{itemize}
	\end{columns}

\end{frame}

% \begin{frame}\frametitle{Deep bed filtration}
% 	\todo{PICTURE}
% 	\begin{itemize}
% 		\item	bed is provided, and is typically batch process
% 		\item	used for dilute suspensions, solids to be removed is small and particles are significantly smaller than pores
% 		\item	particles flow through long and tortuous pores where they are collected by mechanisms such as gravity, diffusion, inertia and attach to the filter medium by molecular and electrostatic forces - ex. sand swimming pool filter, clarifier for secondary effluent in wastewater treatment plant to remove TSS
% 		\item	??? cleaned by backwashing
% 	\end{itemize}
% \end{frame}

\begin{frame}\frametitle{Cross-Flow Filtration (TFF)}
	\begin{center}
		\includegraphics[width=1.1\textwidth]{\imagedir/separations/membranes/cross-flow-wikipedia.png}
	\end{center}
	\begin{itemize}
		\item	{\color{purple}{TFF}} = tangential flow filtration (parallel to medium)
		\item	used with gel-like or compressible substances
		\item	pores are smaller than particles
		\item	used when absolute exclusion is essential
		\item	inlet flow of suspension provides shear to limit cake build-up
		\item	reduces cake resistance, $R_c$
	\end{itemize}
\end{frame}

\begin{frame}\frametitle{Basic filtration equation recap}
	\[\dfrac{1}{A} \cdot \dfrac{dV}{dt} = \dfrac{-\Delta P_\text{tot}}{\mu \left({\color{myGreen}R_m} + {\color{myBlue}C_S V  \, \mathbf{ \alpha}\, / A} \right)} =  \dfrac{-\Delta P_\text{tot}}{\mu \left({\color{myGreen}R_m} + {\color{myBlue}R_c} \right)}
	\]
	\vspace{12pt}
	\[
		\begin{array}{rcll}
			V		&=& \text{volume of filtrate collected} 		&[\text{m}^3]\\
			A		&=& \text{cross sectional area for filtration}  &[\text{m}^2]\\
			-\Delta P_\text{tot}	&=& \text{total pressure drop} = -(\Delta P_c + P_m)	&[\text{Pa}]\\
			\mu 	&=& \text{fluid viscosity}  					&[\text{Pa.s}]\\
			C_S  	&=& \text{slurry concentration}					&[{\tiny \text{(kg dry solids)/(m}^{3}\text{ filtrate)}}]\\			
			R_m  &=& \text{resistance due to the medium}			&[\text{m}^{-1}]\\
			R_c  &=& \text{resistance due to the cake}				&[\text{m}^{-1}]\\
			\alpha	&=& \text{specific cake resistance} 			&[\text{m.kg}^{-1}]\\
			\alpha  &=& \dfrac{(k_1) \left(1-\epsilon\right) (S_0^2)}{(\epsilon)^3 (\rho_p)}
		\end{array}
	\]
	\begin{itemize}
		\item	which entries in the equation are a function of $t$?
		\item	which entries in the equation are a function of $V$?
	\end{itemize}
\end{frame}

\begin{frame}\frametitle{How do we determine cake resistance $\alpha$?}
	Recall:
	\[ \alpha = \dfrac{(k_1) \left(1-\epsilon\right)(S_0^2)}{(\epsilon)^3 (\rho_p)} \]
	\begin{itemize}
		\item	Measuring $S_0$ is difficult for irregular solids
		\item	$\epsilon$ changes depending on many factors (surface chemistry, upstream processing)
		\item	Is $\epsilon$ constant over time? What does it change with?
		\item	$\alpha$ will also change as a function of $\Delta P$
		\item	So we let $\alpha = \alpha_\text{0} \left(-\Delta P\right)^f$
	\end{itemize}

\end{frame}

\begin{frame}\frametitle{Constant pressure (batch) filtration}
	\begin{align*}
	   \dfrac{1}{A} \cdot \dfrac{d{\color{red}V}}{dt} &= \dfrac{-\Delta P_\text{tot}}{\mu \left(R_m + {\color{red}V} C_S \, \mathbf{ \alpha}\, / A \right)} =  \dfrac{-\Delta P_\text{tot}}{\mu \left(R_m + R_c \right)}\\
	   \intertext{Invert both sides, separate, divide by $A$}
	   A \cdot \dfrac{dt}{d{\color{red}V}} &= \dfrac{\mu \left(R_m + {\color{red}V}  C_S \, \mathbf{ \alpha}\, / A \right)}{-\Delta P_\text{tot}}\\
	   \dfrac{dt}{d{\color{red}V}} &= {\color{myOrange}\dfrac{\mu}{A\left(-\Delta P_\text{tot}\right)} R_m} + {\color{myBlue}\dfrac{\mu C_S \alpha  }{A^2 \left(-\Delta P_\text{tot}\right)}}{\color{red}V}\\
	   \dfrac{dt}{d{\color{red}V}} &= {\color{myOrange}B} + {\color{myBlue}K_p}{\color{red}V}\\
	   \int_{0}^{t}{dt} &= \int_{0}^{V}{B + K_pV}\\
	   t &= BV +  \dfrac{K_pV^2}{2}
	\end{align*}
	${\color{myOrange}B = \dfrac{\mu}{A\left(-\Delta P_\text{tot}\right)} R_m \quad [\text{s.m}^{-3}]} \qquad \text{and} \qquad {\color{myBlue}K_p = \dfrac{\mu C_S \alpha  }{A^2 \left(-\Delta P_\text{tot}\right)} \quad [\text{s.m}^{-6}]}$
\end{frame}

\begin{frame}\frametitle{Constant pressure (batch) filtration}
	\[ t = BV +  \dfrac{K_pV^2}{2} \]
	\vspace{12pt}

	\emph{Practical matters}:
	\begin{itemize}
		\item	how do we get constant $-\Delta P$?
		\item	which type of equipment has a constant pressure drop?
		\item	what is the expected relationship between $t$ and $V$?
		\item	Plot it for a given batch of slurry.
		\item	If $-\Delta P$ is constant, then $\epsilon$ is likely constant, and so is $\alpha$
	\end{itemize}
\end{frame}

\begin{frame}\frametitle{Example}
	A water-based slurry of mineral is being filtered under vacuum, with a controlled pressure drop of 38 kPa, through a filter paper of $0.07~\text{m}^2$. The slurry is at 24 kg solids per $\text{m}^3$ fluid. Use $\mu = 8.9 \times 10^{-4}~$ Pa.s and the following data:

	\begin{columns}[t]
		\column{0.9\textwidth}
			\vspace{-12pt}
			\begin{center}
				\begin{tabular}{c|c||c}
					$V~\text{[L]}$		&	$t~\text{[s]}$  & $t/V~\text{[s/L]}$\\ \hline
					0.5				 	& 	19 & 37\\
					1					& 	38 & 38\\
					2					& 	95 & 48\\
					3					& 	178 & 60\\
					4					&	280 & 70\\ \hline
				\end{tabular}
			\end{center}
			\begin{enumerate}
				\item	plot a rough sketch of $x=V$ against $y=t/V$
				\item	read off the intercept and slope
				\item	calculate medium resistance, $R_m$ \quad {\scriptsize \color{orange}[\emph{Ans}: $8.11 \times 10^{10} \text{m}^{-1}$]}
				\item	calculate specific cake resistance, $\alpha$ {\scriptsize \color{orange}[\emph{Ans}: $1.87 \times 10^{11} \text{m.kg}^{-1}$]}
				\item	cake resistance $R_c$ at $t=280$ s \qquad {\scriptsize \color{orange}[\emph{Ans}: $2.56 \times 10^{11} \text{m}^{-1}$]}
			\end{enumerate}
		\column{0.40\textwidth}
			\begin{center}
				\includegraphics[width=\textwidth]{\imagedir/separations/filtration/lab-filtration-flickr-518510770_9c52170842_b-portion.png}
			\end{center}
	\end{columns}
\end{frame}

\begin{frame}\frametitle{Pressure dependence of $\alpha$}
	Recall: $\alpha = \alpha_\text{0} \left(-\Delta P\right)^f$
	\begin{itemize}
		\item	We can find tables of $\alpha_\text{0}$ and $f$ for various solids
		\item	But they almost will never match our situation
	\end{itemize}
	
	
	\vspace{12pt}
	Finding the $\alpha$ value as a function of pressure drop:
	\begin{enumerate}
		\item	Repeat the example above at several different $-\Delta P$ levels
		\item	Calculate $\alpha$ at each $-\Delta P$ \hfill {\color{myOrange}(see example above)}
		\item	Plot $x=(-\Delta P)$ against $y = \alpha$ \hfill {\color{myOrange}(expected shape?)}
		\item	Take logs on both sides of equation:
		\[
			\ln{\left(\alpha\right)} = \ln{\left(\alpha_\text{0}\right)} + f \ln{\left(-\Delta P\right)}
		\]
	\end{enumerate}
\end{frame}

\begin{frame}\frametitle{Constant rate filtration}
	At a \textbf{constant rate}, i.e. $\frac{dV}{dt} = 0$
	\[ \dfrac{1}{A} \cdot \dfrac{dV}{dt} = \dfrac{V}{t} = {\color{myOrange}Q} = \dfrac{-\Delta P_\text{tot}}{\mu \left({\color{myGreen}R_m} + {\color{myBlue}C_S V  \, \mathbf{ \alpha}\, / A} \right)}	
	\]
	We can solve the equations for $-\Delta P$
	\begin{align*}x
		-\Delta P &= \dfrac{\mu R_m V}{A t} + \dfrac{\mu C_s \alpha V^2}{t \cdot A^2}\\
		-\Delta P &= \dfrac{\mu R_m {\color{myOrange}Q}}{A} + \dfrac{\mu C_s \alpha {\color{myOrange}Q^2}}{A^2}t
	\end{align*}
	\begin{itemize}
		\item	For many processes, the constant rate portion is fairly short
		\item	Occurs at the start of filtration, where a plot of $x = (t)$ against $y = (-\Delta P)$ is linear
		\item	Then we settle into constant pressure mode
		\item	That's the part we have focused on earlier
	\end{itemize}
\end{frame}

\begin{frame}\frametitle{Example continued}
	\[ t = BV +  \dfrac{K_pV^2}{2} 	\qquad {\color{myOrange}B = \dfrac{\mu}{A\left(-\Delta P_\text{tot}\right)} R_m } \quad \text{and} \quad {\color{myBlue}K_p = \dfrac{\mu C_S \alpha  }{A^2 \left(-\Delta P_\text{tot}\right)} } \]
	\begin{center}
		\includegraphics[width=\textwidth]{\imagedir/separations/filtration/alternating-plates-frame-press-flickr-6322212861_3281570406_b.jpg}
	\end{center}
	
\end{frame}

\begin{frame}\frametitle{Example continued}
	Suppose our prior lab experiments determined that $\alpha = 4.37\times 10^{9}\left(-\Delta P \right)^{0.3}$, with $-\Delta P$ in Pa and $\alpha$ in SI units.
	\vspace{12pt}
	Now we operate a plate and frame filter press at $-\Delta P$ of 67 kPa, with solids slurry content of $C_s = 300~\dfrac{\text{kg dry solids}}{\text{m}^3~\text{filtrate}}$. The cycle time that the filter actually operates is 45 minutes (followed by 15 minutes for cleaning). Based on a simple mass-balance, the company can calculate that $8.5~\text{m}^3$ of filtrate will be produced.
	
	\vspace{12pt}
	\begin{enumerate}
		\item	Calculate the area required.
		\item	If the slurry concentration had to double (still the same volume of filtrate), what would the required pressure drop have to be to maintain the same cycle time?
		\item	Create a plot of the volume of filtrate leaving the press as a function of time, $t$.
	\end{enumerate}	
\end{frame}

\begin{frame}\frametitle{Pressure requirements change with concentration of slurry}
	\begin{center}
		\includegraphics[width=\textwidth]{\imagedir/separations/filtration/example-pressure-increased-concentration.png}
	\end{center}
	\begin{itemize}
		\item	$A = 81~\text{m}^2$ stays fixed
		\item	$t = 2700~\text{s}$ stays fixed
	\end{itemize}
\end{frame}

\begin{frame}\frametitle{Volume of filtrate produced against time}
	\begin{center}
		\includegraphics[width=\textwidth]{\imagedir/separations/filtration/example-filtrate-volume-time.png}
	\end{center}
	\begin{itemize}
		\item	$A = 81~\text{m}^2$ stays fixed
	\end{itemize}	
\end{frame}



\begin{frame}\frametitle{Extend your knowledge}
	Research the following topics:
	\begin{itemize}
		\item	Variable rate filtration: pressure and flow rate both vary along the characteristic centrifugal pump curve
		\item	Filtering centrifuge (combine the topics from the last 2 weeks)
	\end{itemize}
\end{frame}

