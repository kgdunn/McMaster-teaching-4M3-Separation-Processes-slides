%!TEX root = ../4M3-course.tex

\begin{frame}\frametitle{Overview of Separation Processes}
	
	\begin{itemize}
		\item	Why study separation processes?
		\item	Economics of separation processes
		\item	Some everyday examples
		\item	Example flowsheet: Sugar production
		\item	Separating agents
		\item	Classification of separation processes		
	\end{itemize}
\end{frame}

\begin{frame}\frametitle{Why study separation processes?}
	\begin{itemize}
		\item	50\% to 90\% of capital expense on most flowsheets
		\item	60 to 100\% of the ongoing operating costs in chemical plants
		\item	\todo{See: King, Treybal, Seader intros}
		\item	Some important problems facing (the global) ``us'' are \textbf{separation problems}:
		\begin{itemize}
			\item	carbon capture and sequestration/storage (CCS) ... don't forget about methane
			\item	other air pollutants (e.g. cleaning small dust particles $\sim 5$\micron)
			\item	access to clean water/sanitation
		\end{itemize}
		
		\begin{exampleblock}{}
			These problems will be an important part of your career, and impact your life, as the world's population approaches 8, 9 and then 10 billion in our lifetime (expected around 2050 to 2080).
		\end{exampleblock}		
		
	\end{itemize}
\end{frame}

\begin{frame}\frametitle{World population: UN projections}
	\begin{center}
		\includegraphics[height=0.95\textheight]{\imagedir/teaching/{World-Population-1800-2100.png}}
	\end{center}
\end{frame}

\begin{frame}\frametitle{Everyday examples}
	Separation processes at home:
	\begin{itemize}
		\item	{\color{myGreen}{screening}}: sieve to strain water from pasta
		\item	{\color{myGreen}{absorption}}: washing dishes/hands (fat dissolves into non-polar branch)
		\item	{\color{myGreen}{liquid/liquid extraction}}: soak spices in oil to extract flavour 
		\item	\iftoggle{instructor}{\href{https://www.youtube.com/watch?v=uQTGKLbh9X4}{{\color{myGreen}{cyclone:}} vacuum cleaner}}{\href{https://www.youtube.com/watch?v=uQTGKLbh9X4}{cyclone:}}
		\item	\iftoggle{instructor}{{\color{myGreen}{filtering}}: vacuum cleaner; furnace filter}{{\color{myGreen}{filter}}:}		
		\item	\iftoggle{instructor}{{\color{myGreen}{leaching}}: coffee/espresso maker}{{\color{myGreen}{leaching}}:} % See Seader-3rd,p653
		\item	\iftoggle{instructor}{{\color{myGreen}{leaching}}: making tea}{{\color{myGreen}{leaching}}:}
		\item	\iftoggle{instructor}{{\color{myGreen}{adsorption}}: water filter}{{\color{myGreen}{adsorption}}:}
		\item	\iftoggle{instructor}{{\color{myGreen}{centrifugation}}: clothes washing machine}{{\color{myGreen}{centrifugation}}:}
		\item	\iftoggle{instructor}{{\color{myGreen}{phase change by heat addition}}: clothes drier}{{\color{myGreen}{phase change by heat addition}}:}
		\item	\iftoggle{instructor}{{\color{myGreen}{phase change by heat removal}}: dehumidifier}{{\color{myGreen}{phase change by heat removal}}:}
	\end{itemize}
\end{frame}
	
\begin{frame}\frametitle{Everyday examples}
	Separation processes in your body:
	\begin{itemize}
		\item	kidneys: separates waste from blood; reabsorbs water and salts back into blood
		\item	lungs: release of $\text{CO}_2$ from blood
		\item	liver: breaks down toxins, excreted into bile
		\item	gallbladder: concentrates bile
		\item	intestines: absorbed nutrients
		\item	spleen: removed old red blood cells
		\item	lymph nodes: filter foreign particles (e.g. cancers)
	\end{itemize}
\end{frame}

\begin{frame}\frametitle{Engineering example}
	A common, everyday substance: sugar
	\vspace{36pt}
	Video \href{http://www.youtube.com/watch?v=ZBOou6cahtw}{http://www.youtube.com/watch?v=ZBOou6cahtw}	
\end{frame}

\begin{frame}\frametitle{Sugar flowsheet (part 1)}
	\begin{center}
		\includegraphics[width=\textwidth]{\imagedir/separations/sugar-flowsheet/sugar-flowsheet-1.png}
	\end{center}
	\emph{Source}: C.J. King, Separation Processes
\end{frame}

\begin{frame}\frametitle{Sugar flowsheet (part 2)}
	\begin{center}
		\includegraphics[width=\textwidth]{\imagedir/separations/sugar-flowsheet/sugar-flowsheet-2.png}
	\end{center}
	\emph{Source}: C.J. King, Separation Processes
\end{frame}

\begin{frame}\frametitle{Current plan for 4M3 in 2012}
	\begin{center}
		\todo{Mind map here}
	\end{center}
\end{frame}

\begin{frame}\frametitle{Bioseparations}
	\begin{itemize}
		\item	Many of the topics we will cover are part of a pure bioseparations course
		\item	Often called ``downstream'' processing in the bio literature
		\begin{itemize}
			\item	i.e. all unit operations downstream of the bioreactors
		\end{itemize}
		\item	Unit operations include:
		\begin{itemize}
			\item	cell disruption: increase entropy!
			\item	centrifugation $\ast$
			\item	precipitation
			\item	adsorption and chromatography $\ast$
			\item	filtration $\ast$
			\item	membrane separation $\ast$
			\item	electrophoresis
		\end{itemize}
		\vspace{12pt}
		$\ast$ = a topic we will cover in \texttt{4M3}
	\end{itemize}
	\vspace{12pt}
	In this regard, you can see bioprocess separations are naturally designed and operated by chemical engineers.
\end{frame}

\begin{frame}\frametitle{How this course is structured}
	\begin{itemize}
		\item	We aim to consider a variety of separation systems
		\item	Solids, liquids and gases
		\item	Cover unit operations that rely on:
		\begin{itemize}
			\item	mechanical techniques to separate
			\item	mass transfer
			\item	heat transfer
			\item	phase creation or addition
		\end{itemize}
		\item	For each unit operation we consider
		\begin{itemize}
			\item	the physical principle that causes separation
			\item	how to size the unit and specify it; scale-up issues
			\item	issues that affect the unit's cost
			\item	troubleshoot problems with the unit; 
			\item	how to optimize it (e.g. use less energy, increase separation efficiency, modify an existing unit's purpose)
			
		\end{itemize}
	\end{itemize}
\end{frame}

\begin{frame}\frametitle{Topics}
	Separating agent: MSA and ESA
\end{frame}

\begin{frame}\frametitle{Separating agents}	
	\begin{exampleblock}
		{A material, force, or energy source applied to the feed for separation }
	\end{exampleblock}
	\vspace{12pt}
	i.e. what you add to get a separation
	\vspace{12pt}
	\begin{itemize}
		\item	heat
		\item	pressure
		\item	\iftoggle{instructor}{vacuum}{}
		\item	\iftoggle{instructor}{membrane}{}
		\item	\iftoggle{instructor}{filter media}{}
		\item	\iftoggle{instructor}{electric field}{}
		\item	\iftoggle{instructor}{flow}{}
		\item	\iftoggle{instructor}{temperature gradient}{}
		\item	\iftoggle{instructor}{concentration gradient}{}
		\item	\iftoggle{instructor}{gravitational field}{}
		\item	\iftoggle{instructor}{adsorbent}{}
		\item	\emph{there are many others}
	\end{itemize}
\end{frame}

\begin{frame}\frametitle{Tutorial question}
	* classification matrix
	
	
	
\end{frame}


