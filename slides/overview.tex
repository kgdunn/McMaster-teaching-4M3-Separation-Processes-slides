%!TEX root = ../4M3-course.tex

% 2013: Other cases studies to consider: 
% * Orange Juice: see Kasim and Nijab project report 2012: juice extractor, cyclones, centrifuge, evaporators (for concentration)
% * Nicole Rich-Portelli and Derek Seguin 2012 project
% * Any project that got over 55/60 on their grade should be a reasonable choice to consider

\begin{frame}\frametitle{Last class (06 September 2012)}
	\begin{itemize}
		\item	We covered the admin issues
		\item	Grading
		\item	And in particular what is appropriate group work
		\item	Midterm: leaning towards Friday, 12 October (before CSChE)
	\end{itemize}
\end{frame}

\begin{frame}\frametitle{Overview of Separation Processes}
	
	\begin{itemize}
		\item	Why study separation processes?
		\item	Economics of separation processes
		\item	Some everyday examples
		\item	Example flowsheet: Sugar production
		\item	Separating agents
		\item	Classification of separation processes		
	\end{itemize}
\end{frame}

\begin{frame}\frametitle{Why separate?}
	\begin{itemize}
		\item	Can't beat Nature: ``Second Law of Thermodynamics''
			\begin{itemize}
				\item	salt left in water
				\item	$\text{CO}_2$ pumped into the atmosphere
				\item	pollutants dumped into water
				\item	and even the kitchen sink
			\end{itemize}
	\end{itemize}
\end{frame}

\begin{frame}\frametitle{How to separate salts from water}
	\begin{itemize}		
		\item	\iftoggle{instructor}{\href{http://en.wikipedia.org/wiki/Electrodialysis}{electrodialysis}}{\href{http://en.wikipedia.org/wiki/Electrodialysis}{electrodialysis}}
		\item	\iftoggle{instructor}{\href{http://en.wikipedia.org/wiki/Electrodeionization}{electrodeionization}}{\href{http://en.wikipedia.org/wiki/Electrodeionization}{electrodeionization}}		
		\item	\pause\iftoggle{instructor}{evaporation through heating with condensation}{}
		\item	\iftoggle{instructor}{evaporation under vacuum}{}
		\item	\iftoggle{instructor}{freezing to form ice crystals}{}
		\item	\iftoggle{instructor}{reverse osmosis}{}
		\item	\iftoggle{instructor}{ion exchange}{}
		\item	\iftoggle{instructor}{apply pressure and force it through a membrane that delays salts}{}
	\end{itemize}
	Reference: \seefull{King, p 16}
	\vspace{12pt}
	
	Usually there are multiple ways to achieve a required separation.
\end{frame}

\begin{frame}\frametitle{Why study separation processes?}
	\begin{itemize}
		\item	50\% to 90\% of capital investment on petroleum and other chemical-reaction based flowsheets \see{King, p 15}
		\begin{itemize}
			\item	Expense often in proportion to the level of purity (called the separation factor) \see{Treybal, p 2}
		\end{itemize}
		\item	60 to 100\% of the ongoing operating costs in chemical plants  % Reference for this? Uhlmann overview PDF?
		\item	Some important problems facing (the global) ``us'' are \textbf{separation problems}:
		\begin{itemize}
			\item	carbon capture and sequestration/storage (CCS) ... don't forget about methane
			\item	other air pollutants (e.g. cleaning small dust particles $\sim 5$\micron)
			\item	access to clean water/sanitation
		\end{itemize}
		
		\begin{exampleblock}{}
			These problems will be an important part of your career, and impact your life, as the world's population approaches 8, 9 and then 10 billion in our lifetime (expected around 2050 to 2080).
		\end{exampleblock}		
		
	\end{itemize}
\end{frame}

\begin{frame}\frametitle{World population: UN projections}
	\begin{center}
		\includegraphics[height=0.95\textheight]{\imagedir/teaching/{World-Population-1800-2100.png}}
	\end{center}
\end{frame}

\begin{frame}\frametitle{Everyday examples}
	Separation processes at home:
	\begin{itemize}
		\item	{\color{myGreen}{screening}}: sieve to strain water from pasta
		\item	{\color{myGreen}{absorption}}: washing dishes/hands {\scriptsize (fat dissolves into non-polar branch)}
		\item	{\color{myGreen}{liquid/liquid extraction}}: soak spices in oil to extract flavour 		
		\item	\pause\iftoggle{instructor}{\href{https://www.youtube.com/watch?v=QpXQpRYxUCY.mp4}{{\color{myGreen}{cyclone:}} vacuum cleaner}}{\href{https://www.youtube.com/watch?v=QpXQpRYxUCY.mp4}{cyclone:}}
		\item	\iftoggle{instructor}{{\color{myGreen}{filtering}}: vacuum cleaner; furnace filter}{{\color{myGreen}{filter}}:}		
		\item	\iftoggle{instructor}{{\color{myGreen}{leaching}}: coffee/espresso maker}{{\color{myGreen}{leaching}}:} % See Seader-3rd,p653
		\item	\iftoggle{instructor}{{\color{myGreen}{leaching}}: making tea}{{\color{myGreen}{leaching}}:}
		\item	\iftoggle{instructor}{{\color{myGreen}{adsorption}}: water filter}{{\color{myGreen}{adsorption}}:}
		\item	\iftoggle{instructor}{{\color{myGreen}{centrifugation}}: clothes washing machine}{{\color{myGreen}{centrifugation}}:}
		\item	\iftoggle{instructor}{{\color{myGreen}{phase change by heat addition}}: clothes drier}{{\color{myGreen}{phase change by heat addition}}:}
		\item	\iftoggle{instructor}{{\color{myGreen}{phase change by heat removal}}: dehumidifier}{{\color{myGreen}{phase change by heat removal}}:}
	\end{itemize}
\end{frame}
	
\begin{frame}\frametitle{Everyday examples}
	Separation processes in your body:
	\begin{itemize}
		\item	kidneys: separates waste from blood; reabsorbs water and salts back into blood
		\item	lungs: release of $\text{CO}_2$ from blood
		\item	liver: breaks down toxins, excreted into bile
		\item	gallbladder: concentrates bile
		\item	intestines: absorb nutrients
		\item	spleen: removes old red blood cells
		\item	lymph nodes: filter foreign particles (e.g. cancers)
	\end{itemize}
\end{frame}

\begin{frame}\frametitle{Engineering example}
	A common, everyday substance: sugar \see{King, p 2 to 9}
	
	\vspace{36pt}
	Video \href{http://www.youtube.com/watch?v=ZBOou6cahtw}{http://www.youtube.com/watch?v=ZBOou6cahtw}	
\end{frame}

\begin{frame}\frametitle{Sugar flowsheet (part 1)}
	\begin{center}
		\includegraphics[width=\textwidth]{\imagedir/separations/sugar-flowsheet/sugar-flowsheet-1.png}
	\end{center}
	\emph{Source}: C.J. King, Separation Processes
\end{frame}

\begin{frame}\frametitle{Sugar flowsheet (part 2)}
	\begin{center}
		\includegraphics[width=\textwidth]{\imagedir/separations/sugar-flowsheet/sugar-flowsheet-2.png}
	\end{center}
	\emph{Source}: C.J. King, Separation Processes
\end{frame}

\begin{frame}\frametitle{Topics that you want to cover}
	Based on the class activity yesterday, from highest to lowest:
	\begin{itemize}
		\item	Reverse osmosis
		\item	Membranes, including (hemo)dialysis and pervaporation
		\item	Distillation
		\item	Centrifuges
		\item	Cyclones
		\item	Filtration (various types: regular, ultra-, nano-)
		\item	Juicing (has relationship to bioseparation steps)
		\item	Ion-exchange
		\item	Crystallization
		\item	Chromatography
		\item	Electrophoresis
		\item	Zeolites
		\item	Column-based operations: stripping, absorption, packed beds
		\item	Interesting: petro fracking, hydro-fracking, winnowing, demisters 
	\end{itemize}
\end{frame}

\begin{frame}\frametitle{Current plan for 4M3 in 2012}
	\begin{center}
		\includegraphics[width=\textwidth]{\imagedir/separations/overview-class/Separation-Processes-Mindmap.png}
	\end{center}
\end{frame}

\begin{frame}\frametitle{Bioseparations}
	\begin{itemize}
		\item	Many of the topics we will cover are part of a pure bioseparations course
		\item	Often called ``downstream'' processing in the bio literature
		\item	Only difference: they are operated under ``bio-compatible'' conditions: $T$, $P$, pH, aqueous media
		\begin{itemize}
			\item	i.e. all unit operations downstream of the bioreactors
		\end{itemize}
		\item	Unit operations include:
		\begin{itemize}
			\item	cell disruption: increase entropy!
			\item	centrifugation $\ast$
			\item	precipitation
			\item	adsorption and chromatography $\ast$
			\item	filtration $\ast$
			\item	membrane separation $\ast$
			\item	electrophoresis
		\end{itemize}
		\vspace{12pt}
		$\ast$ = a topic we will cover in \texttt{4M3}
	\end{itemize}
	\vspace{12pt}
	In this regard, you can see bioprocess separations are naturally designed and operated by chemical engineers.
\end{frame}

\begin{frame}\frametitle{How this course is structured}
	\begin{itemize}
		\item	We aim to consider a variety of separation systems
		\item	Solids and (liquids and gases) = fluids
		\item	Cover unit operations that rely on:
		\begin{itemize}
			\item	mechanical techniques to separate
			\item	mass transfer
			\item	phase creation or addition
			\item	heat transfer
		\end{itemize}
		\item	For each unit operation we consider
		\begin{itemize}
			\item	the physical principle that causes separation
			\item	how to size the unit and specify it; scale-up issues
			\item	issues that affect the unit's cost
			\item	troubleshoot problems with the unit; 
			\item	how to optimize it (e.g. use less energy, increase separation efficiency, modify an existing unit's purpose)
			
		\end{itemize}
	\end{itemize}
\end{frame}

\begin{frame}\frametitle{Tutorial question: another way of looking at separations}
	Fill in various separation processes in these 9 rectangles:
	\begin{center}
		\includegraphics[width=0.8\textwidth]{\imagedir/separations/overview-class/separation-matrix.png}
	\end{center}
\end{frame}


