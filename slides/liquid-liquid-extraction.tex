% References:
% Green book: Separation Techniques 1: Liquid-liquid systems, articles from "Chemical Engineering Magazine"
% Uhlmanns articles
% C+R v2, Chapter 13
% Perry's
% Existing slides from Santiago/Dickson/Ghosh

% 23 October 2012 (first 2 slides are from membranes; spent about 15 minutes on them)
\begin{comment}
\begin{frame}\frametitle{Slide 76: Some old and new terminology}
	
	Recall from ultrafiltration:
	\begin{itemize}
		\item	{\color{myRed}{$R = 1 - \displaystyle\frac{C_P}{C_F}$   }}
		
		\item	This {\color{purple}{rejection coefficient}} also applies to reverse osmosis.
		
		\item	A new term = {\color{purple}{cut}} = {\color{purple}{conversion}} = {\color{purple}{recovery}} = $\theta = \displaystyle \frac{Q_\text{P}}{Q_\text{F}}$ is between 40 and 50\% typically
	\end{itemize}	
\end{frame}

\begin{frame}\frametitle{Slide 78: Relaxing the assumption of $C_\text{R}$ = $C_\text{feed}$}
	\begin{enumerate}
		\item	Usually we specify the desired cut, $\theta = \displaystyle \frac{Q_\text{P}}{Q_\text{F}}$
		\item	$Q_\text{F} C_\text{F} = Q_\text{R} C_\text{R} + Q_\text{P} C_\text{P}$
		\item	$Q_\text{F} = Q_\text{R} + Q_\text{P}$
		\item	$1 = \displaystyle \frac{Q_\text{R}}{Q_\text{F}} + \theta$
		\item	$C_\text{F} = (1 - \theta)C_\text{R} + \theta C_\text{P}$ from equation (2) and (4)
		\item	$J_\text{solv}C_\text{P} = A_\text{solv}(\Delta P - \Delta \pi)C_\text{P}$ = salt flux leaving in permeate
		\item	$J_\text{salt} = A_\text{salt}(C_\text{R} - C_\text{P})$ = salt flux into membrane
	\end{enumerate}
	\hrule
	\begin{itemize}
		\item	Specify $C_\text{F}$ and $\theta$
		\item	Guess $C_\text{P}$ value [how?]
		\item	Calculate $C_\text{R}$ from equation 5
		\item	Calculate $J_\text{solv} C_\text{P}$ from equation 6, noting however that $J_\text{solv} = f(\pi_\text{R}, \pi_\text{P})$. So recalculate $\pi_\text{R}$ and $\pi_\text{P}$
		\item	Note then that equation 6 and 7 must be equal
		\item	Solve eqn 7 for $C_\text{P}$ and use that as a revised value to iterate.
	\end{itemize}	
\end{frame}
\end{comment}
	
% 23 October 2012: Liquid-liquid extraction section
\begin{frame}\frametitle{Liquid-liquid extraction (LLE)}
	\begin{center}
		\includegraphics[width=0.60\textwidth]{\imagedir/separations/liquid-liquid-extraction/flickr-3453475667_b781c8fa46_b-modified.jpg}
	\end{center}
	\vspace{-12pt}
	\see{\href{http://www.flickr.com/photos/39357285@N00/3453475667/}{Flickr\# 3453475667}}
\end{frame}

\begin{frame}\frametitle{Definitions}
	\begin{center}
		\includegraphics[width=\textwidth]{\imagedir/separations/liquid-liquid-extraction/terminology.png}
	\end{center}
	
	\begin{itemize}
		\item	{\color{purple}{solute}}: species we aim to recover (A) from the feed 
		\item	{\color{purple}{feed or ``feed solvent''}}: one of the liquids in the system {\tiny (``{\color{purple}{carrier}}'')}
		\item	{\color{purple}{solvent}}: MSA (by convention: the ``added'' liquid)
		\item	{\color{purple}{extract}}: solute mostly present in this layer = $y_\text{A}$
		\item	{\color{purple}{raffinate}}: residual solute in this layer = $x_\text{A}$
		\item	{\color{purple}{distribution}}: how the solute {\color{purple}{partitions}} itself = $D_\text{A} = \displaystyle \frac{y_\text{A}}{x_\text{A}} = \displaystyle \frac{y_\text{E}}{x_\text{R}}$
			\begin{itemize}
				\item	measure of affinity of solute
				\item	$D_\text{A} = \displaystyle \frac{\mu_R^0 - \mu_E^0}{RT} = \displaystyle \frac{\text{chemical potential difference}}{(R)(\text{temperature})}$	% Ghosh, p 92
				%\item	$D_\text{A}$ roughly constant at low concentrations of $x_\text{A}$ and $y_\text{A}$
				% where did you read this; does this make sense?
			\end{itemize}
	\end{itemize}	
\end{frame}

\begin{frame}\frametitle{Where/why LLE is used}
	Where?
	\begin{itemize}
		\item	Bioseparations
		\item	Nuclear (uranium recovery)
		\item	Mining: nickel/cobalt; copper/iron
		\item	Perfumes, fragrances and essential oils
		\item	Fine and specialty chemicals
	\end{itemize}
	Why?
	\begin{itemize}
		\item	Temperature sensitive products
		\item	High purity requirements
		\item	High-boiling point species in low quantity
		\item	Need to separate by species type (rather than relative volatility)
		\item	Close-boiling points, but high solubility difference
		\item	Azeotrope-forming mixtures
	\end{itemize}
\end{frame}

\begin{frame}\frametitle{Extractor types}
	\begin{center}
		\includegraphics[width=\textwidth]{\imagedir/separations/liquid-liquid-extraction/terminology.png}
	\end{center}
	\begin{enumerate}
		\item	Mixing/contacting:
		 	\begin{itemize}
		 		\item	turbulent contact between liquid phases
		 		\item	small droplet {\color{purple}{dispersion}} in a {\color{purple}{continuous}} phase
				\begin{itemize}
					\item	which phase is dispersed?
				\end{itemize}
		 		\item	mass-transfer between phases
		 		\item	limited by solute loading in solvent
		 	\end{itemize}
		\item	Phase separation: 
			\begin{itemize}
				\item	reverse of mixing step
				\item	drops coalesce 
				\item	relies on density difference
			\end{itemize}
		\item	Collection of phases leaving the unit
	\end{enumerate}
\end{frame}

\begin{frame}\frametitle{What are we aiming for?}
	\begin{exampleblock}{{\color{myRed}{Main aims}}}
		\begin{itemize}
			\item	High recovery of solute overall (low $x_R$ and high $y_E$)
			\item	Concentrated solute in extract (high $y_E$)
		\end{itemize}
	\end{exampleblock}
	\vspace{12pt}
	\emph{How to achieve this}?
	\begin{itemize}
		\item	Counter-current mixer-settlers in series
		\item	High interfacial area during mixing
		\item	Reduce mass-transfer resistance
		\item	Promote mass transfer
		\begin{itemize}
			\item	molecular diffusion
			\item	eddy diffusion  \hfill {\color{myOrange}{$\leftarrow$ orders of magnitude greater}}
		\end{itemize}
	\end{itemize}
\end{frame}

\begin{frame}\frametitle{Equipment for LLE}	
	\begin{enumerate}
		\item	Mixer-settlers
		\begin{itemize}
			\item	mix: impellers
			\item	mix: nozzles
			\item	mix: feeds meet directly in the pump
			\item	mix: geared-teeth devices
			\item	{\color{myOrange}{main aim: good contact; avoid droplets smaller than 2 \micron}}
			\item	settle: baffles, membranes
			\item	settle: ultrasound
			\item	settle: chemical treatment
			\item	settle: centrifuges
		\end{itemize}
		\item	Columns with:
			\begin{itemize}
				\item	(a) nothing or
				\item	(b) trays and/or
				\item	(c) packing and/or
				\item	(d) pulsating and/or
				\item	(e) agitation
			\end{itemize}
		\item	Rotating devices
	\end{enumerate}
	{\color{myRed}{Important point}}: LLE is an equilibrium-limited separation (as opposed to rate limited separations seen up to now).
\end{frame}

% 25 October 2012
\begin{frame}\frametitle{Mixer-settlers}
	\begin{center}
		\includegraphics[width=\textwidth]{\imagedir/separations/liquid-liquid-extraction/mixer-settler-CRv2-5ed-p745.png}
	\end{center}
	\see{Richardson and Harker, p 745}
	Common in mining industry: requirements $\sim$40000 L/min flows
\end{frame}

\begin{frame}\frametitle{Mixer-settlers}
	KnitMesh coalescer: consistency of ``steel wool''
	\begin{center}
		\includegraphics[width=\textwidth]{\imagedir/separations/liquid-liquid-extraction/KnitMesh-mixer-settler-CRv2-5ed-p747.png}
	\end{center}
	\see{Richardson and Harker, p 747}
\end{frame}

\begin{frame}\frametitle{Horizontal gravity settling vessel}
	\begin{center}
		\includegraphics[width=\textwidth]{\imagedir/separations/liquid-liquid-extraction/settler-only-Seader-3ed-p302}
	\end{center}
	\see{Seader, 3ed, p302}
\end{frame}

\begin{frame}\frametitle{Spray columns: separation principle is {\color{myRed}{gravity}}}
	\begin{center}
		\includegraphics[height=0.9\textheight]{\imagedir/separations/liquid-liquid-extraction/spray-towers-CRv2-5ed-p750.png}
	\end{center}
	\vspace{-14pt}
	\see{Richardson and Harker, p 751}
\end{frame}

\begin{frame}\frametitle{Tray columns}
	\begin{columns}[t]
		\column{0.70\textwidth}
			\vspace{-0.5cm}
			\begin{center}
				\includegraphics[height=0.9\textheight]{\imagedir/separations/liquid-liquid-extraction/baffle-plate-columns-CRv2-5ed-p749.png}
			\end{center}
			\vspace{-20pt}
			\see{Richardson and Harker, p 749}
		\column{0.40\textwidth}
			\begin{itemize}
				\item	coalescence on each tray
				\item	tray holes: $\sim$ 3mm
				\item	breaks gradient formation (axial dispersion)
			\end{itemize}
	\end{columns}
\end{frame}

\begin{frame}\frametitle{Rotating devices}
	\begin{center}
		\includegraphics[width=\textwidth]{\imagedir/separations/liquid-liquid-extraction/Graesser-contactor-Seader-3ed-p306.png}
	\end{center}
	\vspace{-12pt}
	\see{Seader, 3ed, p 306}
	\begin{itemize}
		\item	``white'' = lighter liquid
		\item	``grey'' = heavier liquid
	\end{itemize}
	Used when foams and emulsions would easily form: i.e. gentle mass transfer.
\end{frame}

\begin{frame}\frametitle{Linking up units (more on this later)}
	\begin{center}
		\includegraphics[width=\textwidth]{\imagedir/separations/liquid-liquid-extraction/multiple-contacts-mixer-settler-CRv2-5ed-p723.png}
	\end{center}
	\see{Richardson and Harker, p 723}
\end{frame}

\begin{frame}\frametitle{Integration with downstream units}
	\begin{center}
		\includegraphics[width=\textwidth]{\imagedir/separations/liquid-liquid-extraction/acetic-acid-Schweitzer-p1-257.png}
	\end{center}
	\vspace{-10pt}
	\see{Schweitzer, p 1-257}
\end{frame}

\begin{frame}\frametitle{Selecting a solvent}
	Schweitzer: ``The \textbf{choice of solvent} for a LLE process can often have a more significant impact on the process economics than any other design decision that has to be made''.
	
	\vspace{12pt}
	Which properties of a solvent influence our aims with LLE?
	\begin{itemize}
		\item	High distribution coefficient (selectivity) for solute 
		\pause
		\item	\iftoggle{instructor}{Low distribution coefficient for carrier}{}
		\item	\iftoggle{instructor}{Reasonable volatility difference with solute and carrier}{}
		\item	\iftoggle{instructor}{Reasonable surface tension: easy to disperse \textbf{and} coalesce}{}
		\item	\iftoggle{instructor}{High density difference: separates rapidly by gravity}{}
		\item	\iftoggle{instructor}{Stability to maximize its reuse}{}
		\item	\iftoggle{instructor}{Inert to materials of construction}{}
		\item	\iftoggle{instructor}{Low viscosity: maximizes mass transfer}{}
		\item	\iftoggle{instructor}{Safe: non-toxic, non-flammable}{}
		\item	\iftoggle{instructor}{Cheap, and easily available}{}
		\item	\iftoggle{instructor}{Compatible with carrier and solute: avoid contamination}{}
		\item	\iftoggle{instructor}{Doesn't foam, form emulsions, scum layers at interface}{}
	\end{itemize}
\end{frame}

\begin{frame}\frametitle{Calculating the distribution coefficient}
	\begin{columns}[t]
		\column{0.60\textwidth}
			Mass balance:
			\[ F x_F + S y_S = E y_E + R x_R \]
			\[ D = \frac{y_E}{x_R} \]

			\vspace{12pt}
			If $F = S = E = R$ and $y_s = 0$, then only measure $x_R$:
			\[ D = \frac{x_F}{x_R} - 1\]
		\column{0.40\textwidth}
			\begin{center}
				\includegraphics[width=1.2\textwidth]{\imagedir/separations/liquid-liquid-extraction/mass-balance-Schweitzer-p1-263.png}
			\end{center}
	\end{columns}
	\begin{itemize}
		\item	Capital letters refer to mass amounts
		\item	$y_\Box \,\, \leftarrow\,\,$  refers to mass fractions in solvent layer
		\item	$x_\Box \,\, \leftarrow\,\,$  refers to mass fractions in carrier and extract layers
	\end{itemize}
	Once $D$ is determined, we can obtain phase diagrams to understand how the process will operate.
	
	\vspace{12pt}
	Also: see Perry's for many values of $D$
\end{frame}

\begin{frame}\frametitle{Triangular phase diagrams: from laboratory studies}
	\begin{center}
		\includegraphics[width=0.60\textwidth]{\imagedir/separations/liquid-liquid-extraction/flickr-3453475667_b781c8fa46_b-modified.jpg}
	\end{center}
	\vspace{-12pt}
	\see{\href{http://www.flickr.com/photos/39357285@N00/3453475667/}{Flickr\# 3453475667}}
\end{frame}

\begin{frame}\frametitle{Using a triangular phase diagrams}
	\begin{center}
		\includegraphics[width=0.8\textwidth]{\imagedir/separations/liquid-liquid-extraction/Water-MIBK-acetone-from-youtube-gGYHXhcKM5s.png}
	\end{center}
	\vspace{-12pt}
	\see{\href{http://www.youtube.com/watch?v=gGYHXhcKM5s}{http://www.youtube.com/watch?v=gGYHXhcKM5s}}
\end{frame}

\begin{frame}\frametitle{Lever rule}
	\begin{columns}[t]
		\column{0.80\textwidth}
			\begin{center}
				\includegraphics[width=\textwidth]{\imagedir/separations/liquid-liquid-extraction/Water-MIBK-acetone-from-CRv2-5ed-p726-mod.png}
			\end{center}
			\see{Richardson and Harker, p 726}
		\column{0.30\textwidth}
			\textbf{Mix P and Q}
			
			\begin{itemize}
				\item	mixture = K
				\vspace{12pt}
				\item	$\displaystyle \frac{\text{PK}}{\text{KQ}} = \displaystyle\frac{\text{amount Q}}{\text{amount P}}$
				\vspace{12pt}
				\item	The converse applies also: when separating a settled mixture
				\item	Applies anywhere: even in the miscible region
			\end{itemize}
	\end{columns}
\end{frame}

\begin{frame}\frametitle{Q1: Using the lever rule}
	Which is a more or effective system?
	\begin{itemize}
		\item	S = pure solvent used 
		\item	F = feed concentration
	\end{itemize}
	
	\begin{center}
		\includegraphics[width=0.75\textwidth]{\imagedir/separations/liquid-liquid-extraction/solubility-effect-on-range-Seader-3ed-p312.png}
	\end{center}
	\iftoggle{instructor}{
		Answer: range of feed concentrations ($F$) is widest, i.e. more desirable for (a). Difference between (a) and (b): 
		\begin{itemize}
			\item	due to solvent choice 
			\item	due to different temperatures
			\item	due to pH modification, \emph{etc}
		\end{itemize}
	}{Answer: }
\end{frame}

\begin{frame}\frametitle{Q2: Using the lever rule}
	
	\begin{center}
		\includegraphics[width=0.9\textwidth]{\imagedir/separations/liquid-liquid-extraction/ternary-base-diagram.png}
	\end{center}
	Which is a more effective as a solvent: C or S ?
\end{frame}

\begin{frame}\frametitle{Q3: Using the lever rule}
	
	\begin{center}
		\includegraphics[width=0.85\textwidth]{\imagedir/separations/liquid-liquid-extraction/ternary-base-diagram-mixture.png}
	\end{center}
	Mix a feed stream, $F$, containing $C$ and $A$ (i.e. $x_f$) with a pure solvent stream $S$ (i.e. $y_S = 0$). Composition of the mixture?
\end{frame}

\iftoggle{instructor}{
	\begin{frame}\frametitle{Q3 {\color{myOrange}{\emph{solution}}}: Using the lever rule}		
		\begin{center}
			\includegraphics[width=0.85\textwidth]{\imagedir/separations/liquid-liquid-extraction/ternary-base-diagram-mixture-with-M.png}
		\end{center}
		Composition of the mixture? Trick question: \emph{we need more information} (e.g. amount of $F$ or $S$ must be given)
	\end{frame}
}{}

\begin{frame}\frametitle{Q4: Going to equilibrium}	
	\begin{center}
		\includegraphics[width=0.85\textwidth]{\imagedir/separations/liquid-liquid-extraction/ternary-base-diagram-mixture.png}
	\end{center}
	Let that mixture $M$ achieve equilibrium. What is the composition of the raffinate and extract?
\end{frame}

\iftoggle{instructor}{
	\begin{frame}\frametitle{Q4 {\color{myOrange}{\emph{solution}}}: Going to equilibrium}		
		\begin{center}
			\includegraphics[width=0.85\textwidth]{\imagedir/separations/liquid-liquid-extraction/ternary-base-diagram-mixture-with-M-R-E.png}
		\end{center}
		What is the composition of the raffinate and extract? \emph{Use the tie lines}.
	\end{frame}
}{}

\begin{frame}\frametitle{Q5: Altering flows}	
	\begin{center}
		\includegraphics[width=0.85\textwidth]{\imagedir/separations/liquid-liquid-extraction/ternary-base-diagram-mixture.png}
	\end{center}
	Same system, but now lower solvent flow rate {\tiny (to try save money!)}. What happens to (a) extract concentration and (b) solute recovery?
\end{frame}

\iftoggle{instructor}{
	\begin{frame}\frametitle{Q5 {\color{myOrange}{\emph{solution}}}: Altering flows}
		\begin{center}
			\includegraphics[width=0.85\textwidth]{\imagedir/separations/liquid-liquid-extraction/ternary-base-diagram-mixture-with-M-R-E-R-E.png}
		\end{center}
		(a) {\small \emph{extract concentration increases}}: (A at E*) $>$ (A at E): $y_{E^*}>y_{E}$
		
		\vspace{2pt}
		(b) {\small \emph{solute recovery drops}}: (A at R*) $>$ (A at R): $x_{R^*}>x_{R}$
	\end{frame}
}{}

%\begin{frame}\frametitle{The phase rule}
	%{\color{myGreen}{\iftoggle{instructor}{Recall the phase rule: $F = C - P + 2$}{}}}
%\end{frame}

% \begin{frame}\frametitle{Approach to sizing a unit}
% 	There are so many variations in LLE equipment: mixer-settlers; towers; rotating devices.
% 	
% 	\vspace{12pt}
% 	How do we design for them?
% 	
% 	\vspace{12pt}
% 	General approach:
% 	\begin{enumerate}
% 		\item	Use ternary diagrams to determine operating curves
% 		\item	Estimate number of ``theoretical plates''
% 		\item	Convert ``theoretical plates'' to actual equipment size
% 			\begin{itemize}
% 				\item	using mass transfer coefficients, and
% 				\item	concentration differences
% 			\end{itemize}
% 		\item	Take co- or counter-current operation into account
% 	\end{enumerate}
% 	\begin{itemize}
% 		\item	{\color{purple}{HETS = height equivalent to a theoretical stage}}
% 	\end{itemize}
% 	
% 	\vspace{12pt}
% 	Let's start slowly: a single stage, well-mixed.
% \end{frame}


\begin{frame}\frametitle{References}
	\begin{itemize}
		\item	Schweitzer, ``Handbook of Separation Techniques for Chemical Engineers'', Chapter 1.9
		\item	Seader, Henly and Roper, ``Separation Process Principles'', 3rd edition, chapter 8
		\item	Richardson and Harker, ``Chemical Engineering, Volume 2'', 5th edition, chapter 13
		\item	Geankoplis, ``Transport Processes and Separation Process Principles'', 4th edition, chapter 12.5
		\item	Ghosh, ``Principles of Bioseparation Engineering'', chapter 7
		\item	Uhlmann's Encyclopedia, ``Liquid-Liquid Extraction'',  \href{http://dx.doi.org/10.1002/14356007.b03\_06.pub2}{DOI:10.1002/14356007.b03\_06.pub2}
	\end{itemize}
\end{frame}

