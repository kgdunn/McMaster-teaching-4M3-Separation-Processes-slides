% References:
% Green book: Separation Techniques 1: Liquid-liquid systems, articles from "Chemical Engineering Magazine"
% Uhlmanns articles
% C+R v2, Chapter 13
% Perry's
% Existing slides from Santiago/Dickson/Ghosh

\begin{frame}\frametitle{Slide 76: Some old and new terminology}
	
	Recall from ultrafiltration:
	\begin{itemize}
		\item	{\color{myRed}{$R = 1 - \displaystyle\frac{C_P}{C_F}$   }}
		
		\item	This {\color{purple}{rejection coefficient}} also applies to reverse osmosis.
		
		\item	A new term = {\color{purple}{cut}} = {\color{purple}{conversion}} = {\color{purple}{recovery}} = $\theta = \displaystyle \frac{Q_\text{P}}{Q_\text{F}}$ is between 40 and 50\% typically
	\end{itemize}	
\end{frame}

\begin{frame}\frametitle{Slide 78: Relaxing the assumption of $C_\text{R}$ = $C_\text{feed}$}
	\begin{enumerate}
		\item	Usually we specify the desired cut, $\theta = \displaystyle \frac{Q_\text{P}}{Q_\text{F}}$
		\item	$Q_\text{F} C_\text{F} = Q_\text{R} C_\text{R} + Q_\text{P} C_\text{P}$
		\item	$Q_\text{F} = Q_\text{R} + Q_\text{P}$
		\item	$1 = \displaystyle \frac{Q_\text{R}}{Q_\text{F}} + \theta$
		\item	$C_\text{F} = (1 - \theta)C_\text{R} + \theta C_\text{P}$ from equation (2) and (4)
		\item	$J_\text{solv}C_\text{P} = A_\text{solv}(\Delta P - \Delta \pi)C_\text{P}$ = salt flux leaving in permeate
		\item	$J_\text{salt} = A_\text{salt}(C_\text{R} - C_\text{P})$ = salt flux into membrane
	\end{enumerate}
	\hrule
	\begin{itemize}
		\item	Specify $C_\text{F}$ and $\theta$
		\item	Guess $C_\text{P}$ value [how?]
		\item	Calculate $C_\text{R}$ from equation 5
		\item	Calculate $J_\text{solv} C_\text{P}$ from equation 6, noting however that $J_\text{solv} = f(\pi_\text{R}, \pi_\text{P})$. So recalculate $\pi_\text{R}$ and $\pi_\text{P}$
		\item	Note then that equation 6 and 7 must be equal
		\item	Solve eqn 7 for $C_\text{P}$ and use that as a revised value to iterate.
	\end{itemize}	
\end{frame}

\begin{frame}\frametitle{Liquid-liquid extraction (LLE)}
	\begin{center}
		\includegraphics[width=0.60\textwidth]{\imagedir/separations/liquid-liquid-extraction/flickr-3453475667_b781c8fa46_b-modified.jpg}
	\end{center}
	\vspace{-12pt}
	\see{\href{http://www.flickr.com/photos/39357285@N00/3453475667/}{Flickr\# 3453475667}}
\end{frame}

\begin{frame}\frametitle{Definitions}
	\begin{center}
		\includegraphics[width=\textwidth]{\imagedir/separations/liquid-liquid-extraction/terminology.png}
	\end{center}
	
	\begin{itemize}
		\item	{\color{purple}{solute}}: species we aim to recover (A) from the feed 
		\item	{\color{purple}{feed or ``feed solvent''}}: one of the liquids in the system {\tiny (``{\color{purple}{carrier}}'')}
		\item	{\color{purple}{solvent}}: MSA (by convention: the ``added'' liquid)
		\item	{\color{purple}{extract}}: solute mostly present in this layer = $y_\text{A}$
		\item	{\color{purple}{raffinate}}: residual solute in this layer = $x_\text{A}$
		\item	{\color{purple}{distribution}}: how the solute {\color{purple}{partitions}} itself = $D_\text{A} = \displaystyle \frac{y_\text{A}}{x_\text{A}}$
			\begin{itemize}
				\item	measure of affinity of solute
				\item	$D_\text{A} = \displaystyle \frac{\mu_R^0 - \mu_E^0}{RT} = \displaystyle \frac{\text{chemical potential difference}}{(R)(\text{temperature})}$	% Ghosh, p 92
				%\item	$D_\text{A}$ roughly constant at low concentrations of $x_\text{A}$ and $y_\text{A}$
				% where did you read this; does this make sense?
			\end{itemize}
	\end{itemize}	
\end{frame}

\begin{frame}\frametitle{Where/why LLE is used}
	Where?
	\begin{itemize}
		\item	Bioseparations
		\item	Nuclear (uranium recovery)
		\item	Mining: nickel/cobalt; copper/iron
		\item	Perfumes, fragrances and essential oils
		\item	Fine and specialty chemicals
	\end{itemize}
	Why?
	\begin{itemize}
		\item	Temperature sensitive products
		\item	High purity requirements
		\item	High-boiling point species in low quantity
		\item	Need to separate by species type (rather than relative volatility)
		\item	Close-boiling points, but high solubility difference
		\item	Azeotrope-forming mixtures
	\end{itemize}
\end{frame}

\begin{frame}\frametitle{Extractor types}
	\begin{center}
		\includegraphics[width=\textwidth]{\imagedir/separations/liquid-liquid-extraction/terminology.png}
	\end{center}
	\begin{enumerate}
		\item	Mixing/contacting:
		 	\begin{itemize}
		 		\item	turbulent contact between liquid phases
		 		\item	small droplet {\color{purple}{dispersion}} in a {\color{purple}{continuous}} phase
				\begin{itemize}
					\item	which phase is dispersed?
				\end{itemize}
		 		\item	mass-transfer between phases
		 		\item	limited by solute loading in solvent
		 	\end{itemize}
		\item	Phase separation: 
			\begin{itemize}
				\item	reverse of mixing step
				\item	drops coalesce 
				\item	relies on density difference
			\end{itemize}
		\item	Collection of phases leaving the unit
	\end{enumerate}
\end{frame}

\begin{frame}\frametitle{What are we aiming for?}
	\sout{Assume: solute dissolved in undesirable phase, e.g. solvent}
	\begin{itemize}
		\item	{\color{myGreen}{\iftoggle{instructor}{High recovery of solute in desired phase (phase)}{}}}
		\item	{\color{myGreen}{\iftoggle{instructor}{Concentrated solute in extract (water)}{}}}
	\end{itemize}
	\sout{\textbf{Note}: we can have that the solute/solvent is the desirable phase}
	
	\vspace{12pt}
	\emph{How to achieve this}?
	\begin{itemize}
		\item	Counter-current mixer-settlers in series
		\item	High interfacial area during mixing
		\item	Reduce mass-transfer resistance
		\item	Promote mass transfer
		\begin{itemize}
			\item	molecular diffusion
			\item	eddy diffusion  \hfill {\color{myOrange}{$\leftarrow$ orders of magnitude greater}}
		\end{itemize}
	\end{itemize}
\end{frame}

\begin{frame}\frametitle{Equipment for LLE}	
	\begin{enumerate}
		\item	Mixer-settlers
		\begin{itemize}
			\item	mix: impellers
			\item	mix: nozzles
			\item	mix: feeds meet directly in the pump
			\item	mix: geared-teeth devices
			\item	{\color{myOrange}{main aim: good contact; avoid droplets smaller than 2 \micron}}
			\item	settle: baffles, membranes
			\item	settle: ultrasound
			\item	settle: chemical treatment
			\item	settle: centrifuges
		\end{itemize}
		\item	Columns with:
			\begin{itemize}
				\item	(a) nothing or
				\item	(b) trays and/or
				\item	(c) packing and/or
				\item	(d) pulsating and/or
				\item	(e) agitation
			\end{itemize}
		\item	Rotating devices
	\end{enumerate}
\end{frame}

\begin{frame}\frametitle{Mixer-settlers}
	\begin{center}
		\includegraphics[width=\textwidth]{\imagedir/separations/liquid-liquid-extraction/mixer-settler-CRv2-5ed-p745.png}
	\end{center}
	\see{Richardson and Harker, p 745}
\end{frame}

\begin{frame}\frametitle{Mixer-settlers}
	KnitMesh coalescer: consistency of ``steel wool''
	\begin{center}
		\includegraphics[width=\textwidth]{\imagedir/separations/liquid-liquid-extraction/KnitMesh-mixer-settler-CRv2-5ed-p747.png}
	\end{center}
	\see{Richardson and Harker, p 747}
\end{frame}

\begin{frame}\frametitle{Spray columns}
	\begin{center}
		\includegraphics[height=0.9\textheight]{\imagedir/separations/liquid-liquid-extraction/spray-towers-CRv2-5ed-p750.png}
	\end{center}
	\vspace{-14pt}
	\see{Richardson and Harker, p 751}
\end{frame}

\begin{frame}\frametitle{Tray columns}
	\begin{columns}[t]
		\column{0.70\textwidth}
			\vspace{-0.5cm}
			\begin{center}
				\includegraphics[height=0.9\textheight]{\imagedir/separations/liquid-liquid-extraction/baffle-plate-columns-CRv2-5ed-p749.png}
			\end{center}
			\vspace{-20pt}
			\see{Richardson and Harker, p 749}
		\column{0.40\textwidth}
			\begin{itemize}
				\item	coalescence on each tray
				\item	tray holes: $\sim$ 3mm
				\item	breaks gradient formation (axial dispersion)
			\end{itemize}
	\end{columns}
\end{frame}

\begin{frame}\frametitle{Rotating devices}
	\begin{center}
		\includegraphics[width=0.7\textwidth]{\imagedir/separations/liquid-liquid-extraction/Graesser-contactor-Uhlmanns-v21-p272.png}
	\end{center}
	\vspace{-12pt}
	\see{Uhlmanns, v21, p272}
\end{frame}

\begin{frame}\frametitle{Linking up units (more on this later)}
	\begin{center}
		\includegraphics[width=\textwidth]{\imagedir/separations/liquid-liquid-extraction/multiple-contacts-mixer-settler-CRv2-5ed-p723.png}
	\end{center}
	\see{Richardson and Harker, p 723}
\end{frame}

\begin{frame}\frametitle{Integration with downstream units}
	\begin{center}
		\includegraphics[width=\textwidth]{\imagedir/separations/liquid-liquid-extraction/acetic-acid-Perrys-8ed-Fig-15-1.png}
	\end{center}
	\vspace{-10pt}
	\see{Perrys, 8ed, Ch 15}
\end{frame}

\begin{frame}\frametitle{Triangular phase diagrams: from laboratory studies}
	\begin{center}
		\includegraphics[width=0.60\textwidth]{\imagedir/separations/liquid-liquid-extraction/flickr-3453475667_b781c8fa46_b-modified.jpg}
	\end{center}
	\vspace{-12pt}
	\see{\href{http://www.flickr.com/photos/39357285@N00/3453475667/}{Flickr\# 3453475667}}
\end{frame}

\begin{frame}\frametitle{Using a triangular phase diagrams}
	\begin{center}
		\includegraphics[width=0.8\textwidth]{\imagedir/separations/liquid-liquid-extraction/Water-MIBK-acetone-from-youtube-gGYHXhcKM5s.png}
	\end{center}
	\vspace{-12pt}
	\see{\href{http://www.youtube.com/watch?v=gGYHXhcKM5s}{http://www.youtube.com/watch?v=gGYHXhcKM5s}}
\end{frame}

\begin{frame}\frametitle{Lever rule}
	\begin{columns}[t]
		\column{0.80\textwidth}
			\begin{center}
				\includegraphics[width=\textwidth]{\imagedir/separations/liquid-liquid-extraction/Water-MIBK-acetone-from-CRv2-5ed-p726-mod.png}
			\end{center}
			\see{Richardson and Harker, p 726}
		\column{0.30\textwidth}
			\textbf{Mix P and Q}
			
			\begin{itemize}
				\item	mixture = K
				\vspace{12pt}
				\item	$\displaystyle \frac{\text{PK}}{\text{KQ}} = \displaystyle\frac{\text{amount Q}}{\text{amount P}}$
				\vspace{12pt}
				\item	The converse applies also: when separating a settled mixture
				\item	Applies anywhere: even in the miscible region
			\end{itemize}
	\end{columns}
\end{frame}

\begin{frame}\frametitle{Phase rule}
	\emph{Next class}
\end{frame}

\begin{frame}\frametitle{References}
	\begin{itemize}
		%\item	Wankat, ``Separation Process Engineering'', 2nd edition, chapter 16
		%\item	Schweitzer, ``Handbook of Separation Techniques for Chemical Engineers'', Chapter 2.1 % <---- very well written; use this reference more in future years
		\item	Seader, Henly and Roper, ``Separation Process Principles'', 3rd edition, chapter 8
		\item	Richardson and Harker, ``Chemical Engineering, Volume 2'', 5th edition, chapter 13
		\item	Geankoplis, ``Transport Processes and Separation Process Principles'', 4th edition, chapter 12.5
		\item	Ghosh, ``Principles of Bioseparation Engineering'', chapter 7
		\item	Uhlmann's Encyclopedia, ``Liquid-Liquid Extraction'',  \href{http://dx.doi.org/10.1002/14356007.b03\_06.pub2}{DOI:10.1002/14356007.b03\_06.pub2}
	\end{itemize}
\end{frame}

