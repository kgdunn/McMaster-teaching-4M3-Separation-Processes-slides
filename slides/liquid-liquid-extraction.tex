% References:
% Green book: Separation Techniques 1: Liquid-liquid systems, articles from "Chemical Engineering Magazine"
% Uhlmanns articles
% C+R v2, Chapter 13
% Perry's
% Existing slides from Santiago/Dickson/Ghosh

% 23 October 2012 (first 2 slides are from membranes; spent about 15 minutes on them)
\begin{comment}
\begin{frame}\frametitle{Slide 76: Some old and new terminology}
	
	Recall from ultrafiltration:
	\begin{itemize}
		\item	{\color{myRed}{$R = 1 - \displaystyle\frac{C_P}{C_F}$   }}
		
		\item	This {\color{purple}{rejection coefficient}} also applies to reverse osmosis.
		
		\item	A new term = {\color{purple}{cut}} = {\color{purple}{conversion}} = {\color{purple}{recovery}} = $\theta = \displaystyle \frac{Q_\text{P}}{Q_\text{F}}$ is between 40 and 50\% typically
	\end{itemize}	
\end{frame}

\begin{frame}\frametitle{Slide 78: Relaxing the assumption of $C_\text{R}$ = $C_\text{feed}$}
	\begin{enumerate}
		\item	Usually we specify the desired cut, $\theta = \displaystyle \frac{Q_\text{P}}{Q_\text{F}}$
		\item	$Q_\text{F} C_\text{F} = Q_\text{R} C_\text{R} + Q_\text{P} C_\text{P}$
		\item	$Q_\text{F} = Q_\text{R} + Q_\text{P}$
		\item	$1 = \displaystyle \frac{Q_\text{R}}{Q_\text{F}} + \theta$
		\item	$C_\text{F} = (1 - \theta)C_\text{R} + \theta C_\text{P}$ from equation (2) and (4)
		\item	$J_\text{solv}C_\text{P} = A_\text{solv}(\Delta P - \Delta \pi)C_\text{P}$ = salt flux leaving in permeate
		\item	$J_\text{salt} = A_\text{salt}(C_\text{R} - C_\text{P})$ = salt flux into membrane
	\end{enumerate}
	\hrule
	\begin{itemize}
		\item	Specify $C_\text{F}$ and $\theta$
		\item	Guess $C_\text{P}$ value [how?]
		\item	Calculate $C_\text{R}$ from equation 5
		\item	Calculate $J_\text{solv} C_\text{P}$ from equation 6, noting however that $J_\text{solv} = f(\pi_\text{R}, \pi_\text{P})$. So recalculate $\pi_\text{R}$ and $\pi_\text{P}$
		\item	Note then that equation 6 and 7 must be equal
		\item	Solve eqn 7 for $C_\text{P}$ and use that as a revised value to iterate.
	\end{itemize}	
\end{frame}
\end{comment}	
% 23 October 2012: Liquid-liquid extraction section
\begin{frame}\frametitle{Liquid-liquid extraction (LLE)}
	\begin{center}
		\includegraphics[width=0.60\textwidth]{\imagedir/separations/liquid-liquid-extraction/flickr-3453475667_b781c8fa46_b-modified.jpg}
	\end{center}
	\vspace{-12pt}
	\see{\href{http://www.flickr.com/photos/39357285@N00/3453475667/}{Flickr\# 3453475667}}
\end{frame}

\begin{frame}\frametitle{Definitions}
	\begin{center}
		\includegraphics[width=\textwidth]{\imagedir/separations/liquid-liquid-extraction/terminology.png}
	\end{center}
	
	\begin{itemize}
		\item	{\color{purple}{solute}}: species we aim to recover (A) from the feed 
		\item	{\color{purple}{feed or ``feed solvent''}}: one of the liquids in the system {\tiny (``{\color{purple}{carrier}}'')}
		\item	{\color{purple}{solvent}}: MSA (by convention: the ``added'' liquid)
		\item	{\color{purple}{extract}}: {\color{myRed}{solvent (not solute) mostly present in this layer. $y_\text{E,A} =$~concentration of A, the solute, in extract.}}
		\item	{\color{purple}{raffinate}}: residual solute in this layer = $x_\text{R,A}$
		\item	{\color{purple}{distribution}}: how the solute {\color{purple}{partitions}} itself = $D_\text{A} = \displaystyle \frac{y_\text{E,A}}{x_\text{R,A}}$
			\begin{itemize}
				\item	measure of affinity of solute
				\item	$D_\text{A} = \displaystyle \frac{\mu_R^0 - \mu_E^0}{RT} = \displaystyle \frac{\text{chemical potential difference}}{(R)(\text{temperature})}$	% Ghosh, p 92
				%\item	$D_\text{A}$ roughly constant at low concentrations of $x_\text{A}$ and $y_\text{A}$
				% where did you read this; does this make sense?
			\end{itemize}
	\end{itemize}	
\end{frame}

\begin{frame}\frametitle{Where/why LLE is used}
	Where?
	\begin{itemize}
		\item	Bioseparations
		\item	Nuclear (uranium recovery)
		\item	Mining: nickel/cobalt; copper/iron
		\item	Perfumes, fragrances and essential oils
		\item	Fine and specialty chemicals
	\end{itemize}
	Why?
	\begin{itemize}
		\item	Temperature sensitive products
		\item	High purity requirements
		\item	High-boiling point species in low quantity
		\item	Need to separate by species type (rather than relative volatility)
		\item	Close-boiling points, but high solubility difference
		\item	Azeotrope-forming mixtures
	\end{itemize}
\end{frame}

\begin{frame}\frametitle{Extractor types}
	\begin{center}
		\includegraphics[width=\textwidth]{\imagedir/separations/liquid-liquid-extraction/terminology.png}
	\end{center}
	\begin{enumerate}
		\item	Mixing/contacting:
		 	\begin{itemize}
		 		\item	turbulent contact between liquid phases
		 		\item	small droplet {\color{purple}{dispersion}} in a {\color{purple}{continuous}} phase
				\begin{itemize}
					\item	which phase is dispersed?
				\end{itemize}
		 		\item	mass-transfer between phases
		 		\item	limited by solute loading in solvent
		 	\end{itemize}
		\item	Phase separation: 
			\begin{itemize}
				\item	reverse of mixing step
				\item	drops coalesce 
				\item	relies on density difference
			\end{itemize}
		\item	Collection of phases leaving the unit
	\end{enumerate}
\end{frame}

\begin{frame}\frametitle{What are we aiming for?}
	\begin{exampleblock}{{\color{myRed}{Main aims}}}
		\begin{itemize}
			\item	High recovery of solute overall (low $x_R$ and high $y_E$)
			\item	Concentrated solute in extract (high $y_E$)
		\end{itemize}
	\end{exampleblock}
	\vspace{12pt}
	\emph{How to achieve this}?
	\begin{itemize}
		\item	Counter-current mixer-settlers in series
		\item	High interfacial area during mixing
		\item	Reduce mass-transfer resistance
		\item	Promote mass transfer
		\begin{itemize}
			\item	molecular diffusion
			\item	eddy diffusion  \hfill {\color{myOrange}{$\leftarrow$ orders of magnitude greater}}
		\end{itemize}
	\end{itemize}
\end{frame}

\begin{frame}\frametitle{Equipment for LLE}	
	\begin{enumerate}
		\item	Mixer-settlers
		\begin{itemize}
			\item	mix: impellers
			\item	mix: nozzles
			\item	mix: feeds meet directly in the pump
			\item	mix: geared-teeth devices
			\item	{\color{myOrange}{main aim: good contact; avoid droplets smaller than 2 \micron}}
			\item	settle: baffles, membranes
			\item	settle: ultrasound
			\item	settle: chemical treatment
			\item	settle: centrifuges
		\end{itemize}
		\item	Columns with:
			\begin{itemize}
				\item	(a) nothing or
				\item	(b) trays and/or
				\item	(c) packing and/or
				\item	(d) pulsating and/or
				\item	(e) agitation
			\end{itemize}
		\item	Rotating devices
	\end{enumerate}
	{\color{myRed}{Important point}}: LLE is an equilibrium-limited separation (as opposed to rate-limited separations seen up to now).
\end{frame}

% 25 October 2012
\begin{frame}\frametitle{Mixer-settlers}
	\begin{center}
		\includegraphics[width=\textwidth]{\imagedir/separations/liquid-liquid-extraction/mixer-settler-CRv2-5ed-p745.png}
	\end{center}
	\see{Richardson and Harker, p 745}
	Common in mining industry: requirements $\sim$40000 L/min flows
\end{frame}

\begin{frame}\frametitle{Mixer-settlers}
	KnitMesh coalescer: consistency of ``steel wool''
	\begin{center}
		\includegraphics[width=\textwidth]{\imagedir/separations/liquid-liquid-extraction/KnitMesh-mixer-settler-CRv2-5ed-p747.png}
	\end{center}
	\see{Richardson and Harker, p 747}
\end{frame}

\begin{frame}\frametitle{Horizontal gravity settling vessel}
	\begin{center}
		\includegraphics[width=\textwidth]{\imagedir/separations/liquid-liquid-extraction/settler-only-Seader-3ed-p302}
	\end{center}
	\see{Seader, 3ed, p302}
\end{frame}

\begin{frame}\frametitle{Spray columns: separation principle is {\color{myRed}{gravity}}}
	\begin{center}
		\includegraphics[height=0.9\textheight]{\imagedir/separations/liquid-liquid-extraction/spray-towers-CRv2-5ed-p750.png}
	\end{center}
	\vspace{-14pt}
	\see{Richardson and Harker, p 751}
\end{frame}

\begin{frame}\frametitle{Tray columns}
	\begin{columns}[t]
		\column{0.70\textwidth}
			\vspace{-0.5cm}
			\begin{center}
				\includegraphics[height=0.9\textheight]{\imagedir/separations/liquid-liquid-extraction/baffle-plate-columns-CRv2-5ed-p749.png}
			\end{center}
			\vspace{-20pt}
			\see{Richardson and Harker, p 749}
		\column{0.40\textwidth}
			\begin{itemize}
				\item	coalescence on each tray
				\item	tray holes: $\sim$ 3mm
				\item	breaks gradient formation (axial dispersion)
			\end{itemize}
	\end{columns}
\end{frame}

\begin{frame}\frametitle{Tray columns with mechanical agitation}
	\begin{columns}[t]
		\column{0.60\textwidth}
			\begin{center}
				\includegraphics[width=\textwidth]{\imagedir/separations/liquid-liquid-extraction/rotating-contactor-Seader-3ed-p302.png}
			\end{center}
			
		\column{0.40\textwidth}
			\begin{itemize}
				\item	shearing to create dispersion
				\item	can have alternating layers of packing (coalescence)
				\item	some column designs pulsate $\Uparrow \Downarrow$
			\end{itemize}
	\end{columns}
	\see{Seader, 3ed, p302}
\end{frame}

% To do for 2013: add slides of centrifuge settlers
\begin{frame}\frametitle{Rotating devices}
	\begin{center}
		\includegraphics[width=\textwidth]{\imagedir/separations/liquid-liquid-extraction/Graesser-contactor-Seader-3ed-p306.png}
	\end{center}
	\vspace{-12pt}
	\see{Seader, 3ed, p 306}
	\begin{itemize}
		\item	``white'' = lighter liquid
		\item	``grey'' = heavier liquid
	\end{itemize}
	Used when foams and emulsions would easily form: i.e. gentle mass transfer.
\end{frame}

\begin{frame}\frametitle{Linking up units (more on this later)}
	\begin{center}
		\includegraphics[width=\textwidth]{\imagedir/separations/liquid-liquid-extraction/multiple-contacts-mixer-settler-CRv2-5ed-p723-modified.png}
	\end{center}
	\see{Richardson and Harker, p 723}
\end{frame}

\begin{frame}\frametitle{Integration with downstream units}
	\begin{center}
		\includegraphics[width=\textwidth]{\imagedir/separations/liquid-liquid-extraction/acetic-acid-Schweitzer-p1-257.png}
	\end{center}
	\vspace{-10pt}
	\see{Schweitzer, p 1-257}
\end{frame}

\begin{frame}\frametitle{Selecting a solvent}
	Schweitzer: ``The \textbf{choice of solvent} for a LLE process can often have a more significant impact on the process economics than any other design decision that has to be made''.
	
	\vspace{12pt}
	Which properties of a solvent influence our aims with LLE?
	\begin{itemize}
		\item	High distribution coefficient (selectivity) for solute 
		\item	Low distribution coefficient for carrier
		\item	Reasonable volatility difference with solute and carrier
		\item	Reasonable surface tension: easy to disperse \textbf{and} coalesce
		\item	High density difference: separates rapidly by gravity
		\item	Stability to maximize its reuse
		\item	Inert to materials of construction
		\item	Low viscosity: maximizes mass transfer
		\item	Safe: non-toxic, non-flammable
		\item	Cheap, and easily available
		\item	Compatible with carrier and solute: avoid contamination
		\item	Doesn't foam, form emulsions, scum layers at interface
	\end{itemize}
\end{frame}

\begin{frame}\frametitle{Calculating the distribution coefficient {\color{myOrange}{(in the lab only)}}}
	\begin{columns}[t]
		\column{0.60\textwidth}
			Mass balance:
			\[ F x_F + S y_S = E y_E + R x_R \]
			\[ D = \frac{y_E}{x_R} \]

			\vspace{12pt}
			If $F = S = E = R$ and $y_s = 0$, then only measure $x_R$:
			\[ D = \frac{x_F}{x_R} - 1\]
		\column{0.40\textwidth}
			\begin{center}
				\includegraphics[width=1.2\textwidth]{\imagedir/separations/liquid-liquid-extraction/mass-balance-Schweitzer-p1-263.png}
			\end{center}
	\end{columns}
	\begin{itemize}
		\item	Capital letters refer to mass amounts
		\item	$y_\Box \,\, \leftarrow\,\,$  refers to mass fractions in solvent layer
		\item	$x_\Box \,\, \leftarrow\,\,$  refers to mass fractions in carrier and extract layers
	\end{itemize}
	Once $D$ is determined, we can obtain phase diagrams to understand how the process will operate.
	
	\vspace{12pt}
	Also: see Perry's for many values of $D$
\end{frame}

\begin{frame}\frametitle{Triangular phase diagrams: from laboratory studies}
	\begin{center}
		\includegraphics[width=0.60\textwidth]{\imagedir/separations/liquid-liquid-extraction/flickr-3453475667_b781c8fa46_b-modified.jpg}
	\end{center}
	\vspace{-12pt}
	\see{\href{http://www.flickr.com/photos/39357285@N00/3453475667/}{Flickr\# 3453475667}}
\end{frame}

\begin{frame}\frametitle{Using a triangular phase diagrams}
	\begin{center}
		\includegraphics[width=0.8\textwidth]{\imagedir/separations/liquid-liquid-extraction/Water-MIBK-acetone-from-youtube-gGYHXhcKM5s.png}
	\end{center}
	\vspace{-12pt}
	\see{\href{http://www.youtube.com/watch?v=gGYHXhcKM5s}{http://www.youtube.com/watch?v=gGYHXhcKM5s}}
\end{frame}

\begin{frame}\frametitle{Lever rule}
	\begin{columns}[t]
		\column{0.80\textwidth}
			\begin{center}
				\includegraphics[width=\textwidth]{\imagedir/separations/liquid-liquid-extraction/Water-MIBK-acetone-from-CRv2-5ed-p726-mod.png}
			\end{center}
			\see{Richardson and Harker, p 726}
		\column{0.30\textwidth}
			\textbf{Mix P and Q}
			
			\begin{itemize}
				\item	mixture = K
				\vspace{12pt}
				\item	$\displaystyle \frac{\text{PK}}{\text{KQ}} = \displaystyle\frac{\text{amount Q}}{\text{amount P}}$
				\vspace{12pt}
				\item	The converse applies also: when separating a settled mixture
				\item	Applies anywhere: even in the miscible region
			\end{itemize}
	\end{columns}
\end{frame}

% 26 October 2012
\begin{frame}\frametitle{Q1: Using the lever rule}
	Which is a more \emph{flexible} system?
	\begin{itemize}
		\item	S = pure solvent used 
		\item	F = feed concentration point (more correctly it is $x_F$)
	\end{itemize}
	
	\begin{center}
		\includegraphics[width=0.75\textwidth]{\imagedir/separations/liquid-liquid-extraction/solubility-effect-on-range-Seader-3ed-p312.png}
	\end{center}
	
	\iftoggle{student}{
		Answer: \pause range of feed concentrations ($x_F$) is wider, i.e. more desirable, for \textbf{(a)}. Difference between (a) and (b): 
		\begin{itemize}
			\item	due to solvent choice 
			\item	due to different temperatures
			\item	due to pH modification, \emph{etc}
		\end{itemize}
	}{Answer: }
\end{frame}

\begin{frame}\frametitle{Q2: Using the lever rule}
	
	\begin{center}
		\includegraphics[width=0.8\textwidth]{\imagedir/separations/liquid-liquid-extraction/ternary-base-diagram.png}
	\end{center}
	{\color{myRed}{Which of C or S, when used as a solvent, will obtain a higher extract composition?}} {\color{myOrange}{Answer:}} S
\end{frame}

\begin{frame}\frametitle{Q3: Using the lever rule}
	
	\begin{center}
		\includegraphics[width=0.8\textwidth]{\imagedir/separations/liquid-liquid-extraction/ternary-base-diagram-mixture.png}
	\end{center}
	Mix a feed stream, $F$, containing $C$ and $A$ (i.e. $x_F$) with a pure solvent stream $S$ (i.e. $y_S = 0$). Composition of the mixture?
\end{frame}

\iftoggle{student}{
	\begin{frame}\frametitle{Q3 {\color{myOrange}{\emph{solution}}}: Using the lever rule}
		\begin{center}
			\includegraphics[width=0.85\textwidth]{\imagedir/separations/liquid-liquid-extraction/ternary-base-diagram-mixture-with-M.png}
		\end{center}
		Composition of the mixture? Trick question: \emph{we need more information} (e.g. amount of $F$ \textbf{and} $S$ must be given)
	\end{frame}
}{}

\begin{frame}\frametitle{Q4: Going to equilibrium}
	\iftoggle{student}{
		\begin{center}
			\includegraphics[width=0.85\textwidth]{\imagedir/separations/liquid-liquid-extraction/ternary-base-diagram-mixture-with-M.png}
		\end{center}
	}{
		\begin{center}
			\includegraphics[width=0.85\textwidth]{\imagedir/separations/liquid-liquid-extraction/ternary-base-diagram-mixture.png}
		\end{center}
	}
	Let that mixture $M$ achieve equilibrium. What is the composition of the raffinate and extract?
\end{frame}

\iftoggle{student}{
	\begin{frame}\frametitle{Q4 {\color{myOrange}{\emph{solution}}}: Going to equilibrium}
		\begin{center}
			\includegraphics[width=0.85\textwidth]{\imagedir/separations/liquid-liquid-extraction/ternary-base-diagram-mixture-with-M-R-E.png}
		\end{center}
		What is the composition of the raffinate and extract? \emph{Use the tie lines} [solid lines]; \emph{or interpolate between existing ones}.
	\end{frame}
}{}

\begin{frame}\frametitle{Q5: Altering flows}
	\iftoggle{student}{	
		\begin{center}
			\includegraphics[width=0.85\textwidth]{\imagedir/separations/liquid-liquid-extraction/ternary-base-diagram-mixture-with-M.png}
		\end{center}
	}{
		\begin{center}
			\includegraphics[width=0.85\textwidth]{\imagedir/separations/liquid-liquid-extraction/ternary-base-diagram-mixture.png}
		\end{center}
	}
	Same system, but now lower solvent flow rate {\tiny (to try save money!)}. What happens to (a) extract concentration and (b) solute recovery?
\end{frame}

\iftoggle{student}{
	\begin{frame}\frametitle{Q5 {\color{myOrange}{\emph{solution}}}: Altering flows}
		\begin{center}
			\includegraphics[width=0.85\textwidth]{\imagedir/separations/liquid-liquid-extraction/ternary-base-diagram-mixture-with-M-R-E-R-E.png}
		\end{center}
		(a) {\small \emph{extract concentration increases}}: (A at E*) $>$ (A at E): $y_{E^*}>y_{E}$
		
		\vspace{2pt}
		(b) {\small \emph{solute recovery drops}}: (A at R*) $>$ (A at R): $x_{R^*}>x_{R}$
	\end{frame}
}{}

\begin{frame}\frametitle{Q6: Composition of the mixture, $M$?}
	\begin{columns}[t]
		\column{0.70\textwidth}
			\begin{center}
				\includegraphics[width=1.2\textwidth]{\imagedir/separations/liquid-liquid-extraction/Seader-ternary-example-base.png}
			\end{center}
			\vfill
		\column{0.3\textwidth}
			\vspace{-1cm}
			\begin{center}
				\includegraphics[width=1.2\textwidth]{\imagedir/separations/liquid-liquid-extraction/mass-balance-Schweitzer-p1-263-new-notation.png}
			\end{center}
			{\scriptsize
			\begin{tabular}{ll}
				\textbf{Feed}		& 	\textbf{Solvent}\\ \hline
				$F$ = 250 kg		&	$S$ = 100 kg \\
				$x_{F,A} = 0.24$	&	$x_{S,A} = 0.0$\\
				$x_{F,C} = 0.76$	&	$x_{S,C} = 0.0$\\
				$x_{F,S} = 0.00$	&	$x_{S,S} = 1.0$\\\hline
			\end{tabular}}
	\end{columns}
	Answer: $M = \qquad\quad x_{M,A} = \qquad\quad x_{M,C} = \qquad\quad x_{M,S} = \qquad\quad$
\end{frame}

\iftoggle{student}{
	\begin{frame}\frametitle{Q6 {\color{myOrange}{\emph{solution}}}: Composition of the mixture, $M$?}
		\begin{columns}[t]
			\column{0.70\textwidth}
				\begin{center}
					\includegraphics[width=1.2\textwidth]{\imagedir/separations/liquid-liquid-extraction/Seader-ternary-example-base-with-mixture.png}
				\end{center}
				\vfill
			\column{0.3\textwidth}
				\vspace{-0.4cm}
				\begin{center}
					\includegraphics[width=1.2\textwidth]{\imagedir/separations/liquid-liquid-extraction/mass-balance-Schweitzer-p1-263-new-notation-red-M.png}
				\end{center}
				{\scriptsize
					\begin{tabular}{ll}
						\textbf{Feed}		& 	\textbf{Solvent}\\ \hline
						$F$ = 250 kg		&	$S$ = 100 kg \\
						$x_{F,A} = 0.24$	&	$x_{S,A} = 0.0$\\
						$x_{F,C} = 0.76$	&	$x_{S,C} = 0.0$\\
						$x_{F,S} = 0.00$	&	$x_{S,S} = 1.0$\\\hline
					\end{tabular}
					\vspace{12pt}
					\begin{exampleblock}{}
						\[M = F + S\]
					\end{exampleblock}			
				}
		\end{columns}
		Answer: $M = 350\text{kg};\,\, x_{M,A} = 0.17;\,\, x_{M,C} = 0.54;\,\, x_{M,S} = 0.29$
	\end{frame}
}{}

\begin{frame}\frametitle{Q7: Composition of the 2 phases leaving in equilibrium?}
	\begin{columns}[t]
		\column{0.70\textwidth}
			\begin{center}
				\includegraphics[width=1.2\textwidth]{\imagedir/separations/liquid-liquid-extraction/Seader-ternary-example-base.png}
			\end{center}
			\vfill
		\column{0.3\textwidth}
			\vspace{-1cm}
			\begin{center}
				\includegraphics[width=1.2\textwidth]{\imagedir/separations/liquid-liquid-extraction/mass-balance-Schweitzer-p1-263-new-notation.png}
			\end{center}
			{\scriptsize
				\begin{tabular}{ll}
					\textbf{Feed}		& 	\textbf{Solvent}\\ \hline
					$F$ = 250 kg		&	$S$ = 100 kg \\
					$x_{F,A} = 0.24$	&	$x_{S,A} = 0.0$\\
					$x_{F,C} = 0.76$	&	$x_{S,C} = 0.0$\\
					$x_{F,S} = 0.00$	&	$x_{S,S} = 1.0$\\\hline
				\end{tabular}
			}
	\end{columns}
	$R_1 = \qquad\quad x_{R_1,A} = \qquad\quad x_{R_1,C} = \qquad\quad x_{R_1,S} = \qquad\quad$ $E_1 = \qquad\quad x_{E_1,A} = \qquad\quad x_{E_1,C} = \qquad\quad x_{E_1,S} = \qquad\quad$ 
\end{frame}

\iftoggle{student}{
	\begin{frame}\frametitle{Q7 {\color{myOrange}{\emph{solution}}}: Composition of the 2 phases in equilibrium?}
		\begin{columns}[t]
			\column{0.70\textwidth}
				\begin{center}
					\includegraphics[width=1.2\textwidth]{\imagedir/separations/liquid-liquid-extraction/Seader-ternary-example-base-with-mixture-equilibrated.png}
				\end{center}
				\vfill
			\column{0.3\textwidth}
				\vspace{-0.4cm}
				\begin{center}
					\includegraphics[width=1.2\textwidth]{\imagedir/separations/liquid-liquid-extraction/mass-balance-Schweitzer-p1-263-new-notation-red-M.png}
				\end{center}
				{\scriptsize
					\begin{tabular}{ll}
						\textbf{Feed}		& 	\textbf{Solvent}\\ \hline
						$F$ = 250 kg		&	$S$ = 100 kg \\
						$x_{F,A} = 0.24$	&	$x_{S,A} = 0.0$\\
						$x_{F,C} = 0.76$	&	$x_{S,C} = 0.0$\\
						$x_{F,S} = 0.00$	&	$x_{S,S} = 1.0$\\\hline
					\end{tabular}
					\vspace{12pt}
					\begin{exampleblock}{}
						\[M = E_1 + R_1\]
					\end{exampleblock}
				}
		\end{columns}
		$R_1 = 222\text{kg};\quad x_{R_1,A} = 0.10;\quad x_{R_1,C} = 0.82;\quad x_{R_1,S} = 0.08$ $E_1 = 128\text{kg};\quad x_{E_1,A} = 0.33;\quad x_{E_1,C} = 0.06;\quad x_{E_1,S} = 0.61$ 
	\end{frame}
}{}

% 30 October 2012
\iftoggle{student}{
    % Handout: to print for class
	%\begin{frame}\frametitle{}
	%	\begin{center}
	%		\includegraphics[width=\textwidth]{\imagedir/separations/liquid-liquid-extraction/furfural-water-ethylene-glycol-ternary-phase-diagram-Seader-3ed-p153.png}
	%	\end{center}
	%\end{frame}
	
	\begin{frame}\frametitle{Phase diagram: furfural, water, ethylene glycol}
		\begin{columns}[b]
			\column{0.60\textwidth}
				\begin{center}
					\includegraphics[width=.95\textwidth]{\imagedir/separations/liquid-liquid-extraction/furfural-water-ethylene-glycol-ternary-phase-diagram-Seader-3ed-p153.png}
				\end{center}				
			\column{0.4\textwidth}
				\begin{center}
					\includegraphics[width=\textwidth]{\imagedir/separations/liquid-liquid-extraction/mass-balance-Schweitzer-p1-263-new-notation-red-M.png}
				\end{center}
				{\scriptsize
				\begin{tabular}{ll}
					\textbf{Feed}		& 	\textbf{Solvent}\\ \hline
					$F$ = 100 kg		&	$S$ = 200 kg \\
					$x_{F,A} = 0.45$	&	$x_{S,A} = 0.0$\\
					$x_{F,C} = 0.55$	&	$x_{S,C} = 0.0$\\
					$x_{F,S} = 0.00$	&	$x_{S,S} = 1.0$\\\hline
				\end{tabular}
				\begin{itemize}
					\item	A = ethylene glycol (solute)
					\item	C = water (carrier)
					\item	S = furfural (solvent)
				\end{itemize}}
		\end{columns}
		\textbf{AIM}: to remove ethylene glycol (solute) from water (carrier) into solvent (furfural)
		\begin{enumerate}
			\item	Calculate the mixture composition, {\color{myRed}{$M$}}
			\item	Calculate the equilibrium compositions in $E_1$ and $R_1$
		\end{enumerate}
		{\scriptsize \textbf{Note}: {\color{purple}{extract}} is defined as ``the solvent-rich stream leaving the system''}
	\end{frame}
	
	\begin{frame}\frametitle{}
		\begin{center}
			\includegraphics[width=\textwidth]{\imagedir/separations/liquid-liquid-extraction/furfural-water-ethylene-glycol-ternary-phase-diagram-Seader-3ed-p153-solution.png}
		\end{center}
	\end{frame}
	
	\begin{frame}\frametitle{{\color{myOrange}{Solution}}: Phase diagram: furfural, water, ethylene glycol}
		\begin{columns}[m]
			\column{0.70\textwidth}
				\begin{center}
					\includegraphics[width=.95\textwidth]{\imagedir/separations/liquid-liquid-extraction/furfural-water-ethylene-glycol-ternary-phase-diagram-Seader-3ed-p153-solution.png}
				\end{center}				
			\column{0.4\textwidth}
				{\scriptsize
				\begin{tabular}{ll}
					\textbf{Feed}		& 	\textbf{Solvent}\\ \hline
					$F$ = 100 kg		&	$S$ = 200 kg \\
					$x_{F,A} = 0.45$	&	$x_{S,A} = 0.0$\\
					$x_{F,C} = 0.55$	&	$x_{S,C} = 0.0$\\
					$x_{F,S} = 0.00$	&	$x_{S,S} = 1.0$\\\hline
				\end{tabular}
				\begin{itemize}
					\item	A = ethylene glycol solute
					\item	C = water (carrier)
					\item	S = furfural solvent
				\end{itemize}}
		\end{columns}
		\vspace{12pt}
		$M = 300\text{kg};\quad x_{M,A} = 0.15;\quad x_{M,C} = 0.18;\quad x_{M,S} = 0.67$ 
		$R_1 = ~82\text{kg};\quad x_{R_1,A} = 0.33;\quad x_{R_1,C} = 0.57;\quad x_{R_1,S} = 0.10$ 
		$E_1 = 218\text{kg};\quad x_{E_1,A} = 0.09;\quad x_{E_1,C} = 0.04;\quad x_{E_1,S} = 0.87$ 
	\end{frame}
}{}

\begin{frame}\frametitle{Link units in \emph{series}}
	\begin{center}
		\includegraphics[width=\textwidth]{\imagedir/separations/liquid-liquid-extraction/multiple-contacts-mixer-settler-CRv2-5ed-p723-modified.png}
	\end{center}
	\see{Richardson and Harker, p 723}
\end{frame}

\begin{frame}\frametitle{Q8: send raffinate from Q7 to second mixer-settler}

	\begin{columns}[t]
		\column{0.70\textwidth}
			\begin{center}
				\includegraphics[width=1.2\textwidth]{\imagedir/separations/liquid-liquid-extraction/Seader-ternary-example-base-with-mixture-equilibrated.png}
			\end{center}
		\column{0.4\textwidth}			
			\begin{center}
				\hspace{-3cm}
				\vspace{24pt}
				\includegraphics[width=1.4\textwidth]{\imagedir/separations/liquid-liquid-extraction/multiple-contacts-mixer-settler-CRv2-5ed-p723-coloured.png}
			\end{center}
	\end{columns}
	\vspace{12pt}
	\textbf{Question}: how much solvent should we use in the second stage?	
\end{frame}

%\todo{Come back to this one}
\iftoggle{student}{
	\begin{frame}\frametitle{Q8 {\color{myOrange}{\emph{solution}}}: send raffinate from Q7 to second mixer-settler}
		\begin{columns}[t]
			\column{0.70\textwidth}
				\begin{center}
					\includegraphics[width=1.2\textwidth]{\imagedir/separations/liquid-liquid-extraction/Seader-ternary-example-base-with-mixture-equilibrated-second-round.png}
				\end{center}
			\column{0.4\textwidth}			
				\begin{center}
					\hspace{-3cm}
					\vspace{24pt}
					\includegraphics[width=1.4\textwidth]{\imagedir/separations/liquid-liquid-extraction/multiple-contacts-mixer-settler-CRv2-5ed-p723-coloured.png}
				\end{center}
				Compare extract:
				\vspace{24pt}
				%\begin{itemize}
				%	\item	$x_\text{solute}$ from mixture B \_\_\_ $x_\text{solute}$ from D
				%\end{itemize}
				
				Compare volumes:
				\vspace{24pt}
				%\begin{itemize}
				%	\item	$x_\text{solute}$ from mixture B \_\_\_ $x_\text{solute}$ from D
				%\end{itemize}
		\end{columns}
		\vspace{12pt}
		{\scriptsize \textbf{Answer}: equilibrium from point $B$ (most solvent), $C$, $D$ (least solvent) will each be different. Trade-off: higher extraction vs lower recovery}
		%Answer: less solvent added: higher extract concentration, but raffinate will also contain more solute (i.e. lower recovery). We can't achieve both with units in series.	
		% Note that at point B, for example, E2 vs R2 ratio is about the same; but at D, we get a very low quantity of E2 with a higher concentration (conversely at R2 we get a lower solute concentration of A [desirable], but a very high quantity of raffinate])
	\end{frame}
}{}

	
\begin{frame}\frametitle{Course project}
	{\color{myGreen}{You should have all received feedback from me, via Google Docs}}
	
	\vspace{6pt}
	For the final report:
	\begin{itemize}
		\item	Please give a flowsheet of the overall process
		\item	Only focus on 1 separation unit operation
		\item	Some groups have 2 or 3 separation steps: only 10 pages!
		\item	You must focus on the {\color{myRed}{\textbf{separation}}} step in the flowsheet
		\item	Provide a detailed drawing of the unit
		\item	You must show the detailed design calculation for \textbf{sizing} the unit		
		\item	Choose a basis for sizing: e.g.
			\begin{itemize}
				\item	based on the inlet requirement(s), or
				\item	based on the outlet requirement(s)
			\end{itemize}
		\item	Brief discussion on capital costs and annual operating costs
		\begin{itemize}
			\item	maintenance 
			\item	ESA and/or MSA requirements
		\end{itemize}
	\end{itemize}
	\begin{exampleblock}{Report's length}
		10 pages maximum, please see \href{http://learnche.mcmaster.ca/4M3/Course_project_-_2012}{course website}
	\end{exampleblock}
\end{frame}

\begin{frame}\frametitle{Administrative issues: dates}
	\begin{tabular}{lll}
		\textbf{Unit}		& 	\textbf{Previous date} 	&	\textbf{Revised date}\\ \hline
		Assignment 4		&	01 November				&	06 November\\
		Project report		&  	09 November				&   16 November\\
		Assignment 5		&	16 November			    & 	23 November\\
		Take-home exam 		&  	30 November				& 	30 November
	\end{tabular}
	\begin{center}
		\includegraphics[width=0.8\textwidth]{\imagedir/separations/liquid-liquid-extraction/revised-calendar.png}
	\end{center}
\end{frame}

% 31 October 2012
\begin{frame}\frametitle{Series of co-current units}
	\begin{columns}[t]
		\column{0.40\textwidth}			
			\begin{center}
				\vspace{-12pt}
				{\color{myOrange}{$N=3$ in this illustration}}
				\includegraphics[width=\textwidth]{\imagedir/separations/liquid-liquid-extraction/cross-current-extraction-Schweitzer-p1-263.png}
			\end{center}
			\see{Schweitzer, p 1-263}
		\column{0.60\textwidth}
			\begin{itemize}
				\item	{\color{purple}{Recovery}} = fraction of solute recovered
					\begin{exampleblock}{}
						\[1 - \displaystyle\frac{(x_{R_N})(R_N)}{(x_F) (F)}\]
					\end{exampleblock}
				
				\vspace{6pt}
				\item	{\color{purple}{Concentration}} of overall extract = solute leaving in each extract stream, divided by total extract flow rate 
					\begin{exampleblock}{}
						\[\frac{\displaystyle \sum_n^N{(y_{E_n})(E_n)}}{ \displaystyle \sum_n^N{E_n}}\]
					\end{exampleblock}
			\end{itemize}
	\end{columns}	
\end{frame}

\begin{frame}\frametitle{Co-current \emph{vs} counter-current}
	\begin{columns}[t]
		\column{0.50\textwidth}
			{\color{myGreen}{Co-current ($N=2$ stages)}} 
		\column{0.50\textwidth}
			{\color{myGreen}{Counter-current ($N=2$ stages)}}
	\end{columns}
	\begin{columns}[t]
		\column{0.50\textwidth}
			\begin{center}
				\includegraphics[width=\textwidth]{\imagedir/separations/liquid-liquid-extraction/multiple-contacts-mixer-settler-CRv2-5ed-p723-modified.png}
			\end{center}
		\column{0.50\textwidth}
			\vspace{-12pt}
			\begin{center}
				\includegraphics[width=\textwidth]{\imagedir/separations/liquid-liquid-extraction/counter-current-2-units.png}
			\end{center}
	\end{columns}
	
	\begin{columns}[t]
		\column{0.50\textwidth}
			\begin{itemize}
				\item	We combine multiple extract streams
				\item	(Only 2 in illustration)
				\item	In general: $y_{E_1} > y_{E_2} > \ldots$
				\item	Fresh solvent added at each stage
			\end{itemize}
		\column{0.50\textwidth}
			\begin{itemize}
				\item	``Re-use'' the solvent, so
				\item	Far lower solvent flows
				\item	Recovery = 	\(1 - \displaystyle\frac{(x_{R_N})(R_N)}{(x_F) (F)}\)
				\item	Concentration = \(\displaystyle  y_{E_1}\)
			\end{itemize}
	\end{columns}	
	\vspace{6pt}
	You will have an assignment question to compare and contrast these two configurations
\end{frame}

\begin{frame}\frametitle{Some theory: Two \emph{counter-current} units}
	\begin{center}
		\includegraphics[width=0.5\textwidth]{\imagedir/separations/liquid-liquid-extraction/counter-current-2-units.png}
	\end{center}
	Just \textbf{{\color{myGreen}{consider $N=2$ stages}}} for now. Steady state mass balance:
	\begin{center}
		\begin{columns}[t]
			\column{0.30\textwidth}
				$F + E_2 = E_1 + R_1$
			\column{0.30\textwidth}
				$E_2 + R_2 = S + R_1$
		\end{columns}
	\end{center}
	Rearrange:
	\begin{center}		
		\begin{columns}[t]
		\column{0.30\textwidth}
				$F - E_1 = R_1 - E_2$
			\column{0.30\textwidth}
				$R_1 - E_2 = R_2 - S$
		\end{columns}
		\vspace{12pt}
		$(F - E_1)  =  (R_1 - E_2) = (R_2 - S) = P$
	\end{center}
	Note: each difference is equal to $P$ (look on the flow sheet where those \emph{differences} are).
\end{frame}

\begin{frame}\frametitle{Counter-current graphical solution: 2 units}
	\begin{center}
		\includegraphics[width=0.5\textwidth]{\imagedir/separations/liquid-liquid-extraction/counter-current-2-units.png}
	\end{center}
		
	Rearranging again:
	\[	F + P = E_1 \]
	\[	R_1 + P = E_2 \]
	\[	R_2 + P = S \]
		
	\emph{Interpretation}: \textbf{~~$P$ is a fictitious {\color{purple}{operating point}} on the ternary diagram} (from lever rule)
	\vspace{6pt}
	\begin{itemize}
		\item	$P$ connects $F$ and $E_1$ 
		\item	$P$ connects $R_1$ and $E_2$ 
		\item	$P$ connects $R_2$ and $S$
	\end{itemize}
\end{frame}

\begin{frame}\frametitle{Counter-current graphical solution: 2 units}
	\begin{columns}[t]
		\column{0.70\textwidth}
			\begin{center}
				\includegraphics[width=1.5\textwidth]{\imagedir/separations/liquid-liquid-extraction/Seader-ternary-example-counter-step1.png}
			\end{center}
		\column{0.40\textwidth}
			\begin{center}
				\includegraphics[width=\textwidth]{\imagedir/separations/liquid-liquid-extraction/counter-current-2-units.png}
			\end{center}
	\end{columns}
	\vspace{12pt}
	For example, let's require $x_{R_2,A} = 0.05$ {\scriptsize (solute concentration in raffinate)}. What is $y_{E_1,A}$ then {\scriptsize (concentration of solute in the extract)}?	
	\vfill
\end{frame}

\begin{frame}\frametitle{Counter-current graphical solution: 2 units}
	\begin{columns}[t]
		\column{0.70\textwidth}
			\begin{center}
				\includegraphics[width=1.5\textwidth]{\imagedir/separations/liquid-liquid-extraction/Seader-ternary-example-counter-step2.png}
			\end{center}
		\column{0.40\textwidth}
			\begin{center}
				\includegraphics[width=\textwidth]{\imagedir/separations/liquid-liquid-extraction/counter-current-2-units.png}
			\end{center}
			\textbf{Note}: the line connecting {\color{myBlue}{\textbf{$E_1$ to $R_2$ is not a tie line}}}. We use the lever rule and an overall mass balance $(F + S = E_1 + R_2)$ to solve for all flows and compositions of $F, S, E_1,$ and $R_2$.
	\end{columns}
	\vspace{12pt}
	$y_{E_1,A}\approx 0.38$ is found from an overall mass balance, through $M$.
	\vfill
\end{frame}

\begin{frame}\frametitle{Counter-current graphical solution: 2 units}
	\begin{columns}[t]
		\column{0.70\textwidth}
			\begin{center}
				\includegraphics[width=1.5\textwidth]{\imagedir/separations/liquid-liquid-extraction/Seader-ternary-example-counter-step3.png}
			\end{center}
		\column{0.40\textwidth}
			\begin{center}
				\includegraphics[width=\textwidth]{\imagedir/separations/liquid-liquid-extraction/counter-current-2-units.png}
			\end{center}
			\textbf{Recall}: 
			\vspace{-10pt}
			\[	F + P = E_1 \]
			\[	R_2 + P = S \]			
	\end{columns}
	\begin{exampleblock}{}
		Extrapolate through these lines until intersection at point {\color{myGreen}{$P$}}.
	\end{exampleblock}	
	\vspace{-6pt}
	{\small
		Minimal achievable $E_1$ concentration? \emph{mentally move point} {\color{myRed}{$M$}} \emph{towards} $S$. What happens to $P$? Alternative (simpler?) explanation on next slide.
	}
\end{frame}

\begin{frame}\frametitle{Counter-current graphical solution: maximual solvent flow}
	\begin{columns}[t]
		\column{0.70\textwidth}
			\begin{center}
				\includegraphics[width=1.5\textwidth]{\imagedir/separations/liquid-liquid-extraction/Seader-ternary-example-counter-step3-min-E1.png}
			\end{center}
		\column{0.40\textwidth}
			\begin{center}
				\includegraphics[width=\textwidth]{\imagedir/separations/liquid-liquid-extraction/counter-current-2-units.png}
			\end{center}
			\textbf{Recall}: 
			\vspace{-10pt}
			\[	F + P = E_1 \]
			\[	R_2 + P = S \]			
	\end{columns}
	\vspace{12pt}
	{\color{myOrange}{Subtle point}}: minimal achievable $E^\text{min}_1$ concentration: 
	{\small
		\begin{itemize}
			\item	occurs at a certain \emph{maximum} solvent flow rate indicated by {\LARGE{\color{myRed}{$\mathbf{\circ}$}}}
			\vspace{-6pt}
			\item	note that $R_2$ is {\color{myRed}{\sout{always}}} fixed (specified) in this example
		\end{itemize}
	}
\end{frame}

\begin{frame}\frametitle{Counter-current graphical solution: 2 units}
	\begin{columns}[t]
		\column{0.70\textwidth}
			\begin{center}
				\includegraphics[width=1.5\textwidth]{\imagedir/separations/liquid-liquid-extraction/Seader-ternary-example-counter-step4.png}
			\end{center}
		\column{0.40\textwidth}
			\begin{center}
				\includegraphics[width=\textwidth]{\imagedir/separations/liquid-liquid-extraction/counter-current-2-units.png}
			\end{center}
	\end{columns}
	\vspace{12pt}
	Once we have $E_1$, we can start: note that in stage 1 the $R_1$ and $E_1$ streams leave in equilibrium and can be connected with a tie line.
	\vfill
\end{frame}

\begin{frame}\frametitle{Counter-current graphical solution: 2 units}
	\begin{columns}[t]
		\column{0.70\textwidth}
			\begin{center}
				\includegraphics[width=1.5\textwidth]{\imagedir/separations/liquid-liquid-extraction/Seader-ternary-example-counter-step5.png}
			\end{center}
		\column{0.40\textwidth}
			\begin{center}
				\includegraphics[width=\textwidth]{\imagedir/separations/liquid-liquid-extraction/counter-current-2-units.png}
			\end{center}
			\textbf{Again recall}: 
			\vspace{-12pt}
			\[	R_1 + P = E_2 \]
	\end{columns}
	\vspace{12pt}
	Since we have point $P$ and $R_1$ we can bring the operating line back and locate point $E_2$
	\vfill
\end{frame}

\begin{frame}\frametitle{Counter-current graphical solution: 2 units}
	\begin{columns}[t]
		\column{0.70\textwidth}
			\begin{center}
				\includegraphics[width=1.5\textwidth]{\imagedir/separations/liquid-liquid-extraction/Seader-ternary-example-counter-step6.png}
			\end{center}
		\column{0.40\textwidth}
			\begin{center}
				\includegraphics[width=\textwidth]{\imagedir/separations/liquid-liquid-extraction/counter-current-2-units.png}
			\end{center}
	\end{columns}
	\vspace{12pt}
	The last unit in a cascade is a special case: we already know $R_{N=2}$, but we could have also calculated it from the tie line with $E_2$. We aim for some overshoot of $R_N$. {\scriptsize (Good agreement in this example.)}
\end{frame}

\begin{frame}\frametitle{In general: \emph{Counter-current} units}
	\begin{center}
		\includegraphics[width=\textwidth]{\imagedir/separations/liquid-liquid-extraction/Seader-3ed-counter-current-notation-p313.png}
	\end{center}
		\begin{center}
			\begin{columns}[t]
				\column{0.30\textwidth}
					\small $F + E_2 = E_1 + R_1$
				\column{0.30\textwidth}
					\small $E_2 + R_2 = E_3 + R_1$
				\column{0.30\textwidth}
					\small $E_n + R_n = E_{n+1} + R_{n-1}$
			\end{columns}
		\end{center}
		\vspace{-12pt}
		Rearrange:
		\begin{center}		
			\begin{columns}[t]
				\column{0.30\textwidth}
					\small $F - E_1 = R_1 - E_2$
				\column{0.30\textwidth}
					\small $R_1 - E_2 = R_2 - E_3$
				\column{0.30\textwidth}
					\small $R_{n-1} - E_n = R_{n} - E_{n+1}$
			\end{columns}
		\end{center}
		\vspace{-12pt}
		\begin{columns}[t]
			\column{1.2\textwidth}
				{			
				\small
				\[	
				\begin{array}{c}
					(F - E_1)  =  (R_1 - E_2) = \ldots = (R_{n-1} - E_n) = (R_{n} - E_{n+1}) = \ldots = (R_N - S) = \mathbf{P}
				\end{array}		
				\]}
		\end{columns}
		Notes:
		\begin{enumerate}
			\item	each difference is equal to $P$ (the difference between flows)
			\item	$E_n$ and $R_{n}$ are in equilibrium, leaving each stage {\color{myOrange}{[via tie line]}}
		\end{enumerate} 
\end{frame}

\begin{frame}\frametitle{Counter-current graphical solution}
	\begin{center}
		\includegraphics[width=\textwidth]{\imagedir/separations/liquid-liquid-extraction/Seader-3ed-operating-point-P-p316.png}
	\end{center}
	\vspace{-4cm}
	\begin{columns}[t]
		\column{0.60\textwidth}
		\column{0.40\textwidth}
			\begin{exampleblock}{}
				\hfill {\small $P$ is always on the raffinate}

				\hfill {\small envelope side of the diagram}
			\end{exampleblock}
	\end{columns}	
	\vspace{2.5cm}
	\begin{columns}[t]
		\column{0.50\textwidth}
		\vspace{-18pt}
		{\scriptsize
			\begin{enumerate}
				\item	We know $F$ and $S$; connect with a line and locate ``mixture'' $M$
				\item	Either specify $E_1$ or {\color{myRed}{$R_N$}} (we will always know one of them)
				\item	Connect a straight line through $M$ passing through the one specified
				\item	Solve for unspecified one {\color{myOrange}{[via tie line]}}
			\end{enumerate}}
		\column{0.50\textwidth}
		\vspace{-18pt}
		{\scriptsize				
			\begin{enumerate}
				\setcounter{enumi}{4}
				\item	Connect $S$ through {\color{myRed}{$R_N$}} and extrapolate
				\item	Connect $E_1$ through $F$ and extrapolate; cross lines at $P$
				\item	Locate $P$ by intersection of 2 lines
				\item	In general: connect $E_n$ and $R_n$ via {\color{myOrange}{equilibrium tie lines}}
			\end{enumerate}}
	\end{columns}
\end{frame}

\begin{frame}\frametitle{Relating the theoretical stages to an actual unit}
 	Assume we calculated $N\approx 6$, for example, as the theoretical stages required:

	\vspace{12pt}
	\textbf{Note:}
	\begin{itemize}
		\item	does not mean we require 6 mixer-settlers (though we could do that, but costly)
		\item	it means we need a \textbf{column} which has equivalent operation of 6 counter-current mixer-settlers that fully reach equilibrium
		\item	at this point we resort to correlations and vendor assistance
	\end{itemize}
\end{frame}

\begin{frame}\frametitle{For practice}
	\begin{columns}[t]
		\column{0.70\textwidth}
			\begin{center}
				\includegraphics[width=\textwidth]{\imagedir/separations/liquid-liquid-extraction/Seader-ternary-example-base.png}
			\end{center}			
		\column{0.30\textwidth}			
	\end{columns}
\end{frame}

\begin{frame}\frametitle{For practice}
	\begin{columns}[t]
		\column{0.70\textwidth}
			\begin{center}
				\includegraphics[width=\textwidth]{\imagedir/separations/liquid-liquid-extraction/Seader-ternary-example-base.png}
			\end{center}			
		\column{0.30\textwidth}			
	\end{columns}
\end{frame}

\begin{frame}\frametitle{References}
	\begin{itemize}
		\item	Schweitzer, ``Handbook of Separation Techniques for Chemical Engineers'', Chapter 1.9
		\item	Seader, Henly and Roper, ``Separation Process Principles'', 3rd edition, chapter 8
		\item	Richardson and Harker, ``Chemical Engineering, Volume 2'', 5th edition, chapter 13
		\item	Geankoplis, ``Transport Processes and Separation Process Principles'', 4th edition, chapter 12.5 and 12.6
		\item	Ghosh, ``Principles of Bioseparation Engineering'', chapter 7
		\item	Uhlmann's Encyclopedia, ``Liquid-Liquid Extraction'',  \href{http://dx.doi.org/10.1002/14356007.b03\_06.pub2}{DOI:10.1002/14356007.b03\_06.pub2}
	\end{itemize}
\end{frame}

\begin{comment}
	\begin{frame}\frametitle{The phase rule}
		{\color{myGreen}{\iftoggle{instructor}{Recall the phase rule: $F = C - P + 2$}{}}}
	\end{frame}
	
	\begin{frame}\frametitle{Approach to sizing a unit}
		There are so many variations in LLE equipment: mixer-settlers; towers; rotating devices.

		\vspace{12pt}
		How do we design for them?

		\vspace{12pt}
		General approach:
		\begin{enumerate}
			\item	Use ternary diagrams to determine operating curves
			\item	Estimate number of ``theoretical plates''
			\item	Convert ``theoretical plates'' to actual equipment size
				\begin{itemize}
					\item	using mass transfer coefficients, and
					\item	concentration differences
				\end{itemize}
			\item	Take co- or counter-current operation into account
		\end{enumerate}
		\begin{itemize}
			\item	{\color{purple}{HETS = height equivalent to a theoretical stage}}
		\end{itemize}

		\vspace{12pt}
		Let's start slowly: a single stage, well-mixed.
	\end{frame}
\end{comment}