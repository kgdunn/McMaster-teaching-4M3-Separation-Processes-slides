\begin{comment}

	/Users/kevindunn/Sync/Figures/examples/waste-water-treatment/flickr-4578619015_895f8dabbb_o-CC-BY-NC-SA-2.jpg
	See: http://www.flickr.com/photos/sawater/6434889743/  
	Watch video: http://www.youtube.com/watch?v=gVeW1rr8GEI&feature=related
	Irregular sedmination: http://www.youtube.com/watch?v=u6Y6d7oH6Vs&feature=relmfu

\end{comment}

% 11 September 2012
% 11 September 2012
\begin{frame}\frametitle{Mechanical separations}
	We will start with this topic
	\begin{itemize}
		\item	It's easy to understand!
		\item	Requires only a knowledge of basic physics (e.g. 1st year physics)
		\item	It introduces a number of important principles we will re-use later
		\item	Mechanical separations remain some of the most widely used steps in many flowsheets. Why?
		\begin{itemize}
			\item	reliable units
			\item	relatively inexpensive to maintain and operate
			\item	we can often achieve a very high \emph{separation factor} (that's desirable!)
		\end{itemize}
	\end{itemize}
\end{frame}

\begin{frame}\frametitle{Separation factor}
	As mentioned, we will introduce a number of important principles we will re-use later.
	
	\begin{exampleblock}{Separation factor}
		\begin{columns}[c]
			\column{0.40\textwidth}
				$S_{ij} = \displaystyle \frac{x_{i,1} / x_{j,1}}{x_{i,2} / x_{j,2}}$
			\column{0.40\textwidth}
				\begin{center}
					\includegraphics[width=\textwidth]{\imagedir/separations/overview-class/separation-factor.png}
				\end{center}
		\end{columns} 
	\end{exampleblock}
	
	\begin{itemize}
		\item	select $i$ and $j$ so that $S_{ij} \geq 1$
	\end{itemize}
	
	Based on this definition: we can see why solid-fluid separations often have high separation factors
\end{frame}

\begin{frame}\frametitle{Separating agents: MSA and ESA}
	\begin{exampleblock}
		{\small A material, force, or energy source applied to the feed for separation }
	\end{exampleblock}
	\vspace{6pt}
	i.e. what you add to get a separation. MSA = mass separating agent and ESA = energy separating agent
	\vspace{6pt}	
	\begin{itemize}
		\item	heat (ESA)
		\item	liquid solvent (MSA)
		\item	pressure (ESA)
		\item	\pause\iftoggle{instructor}{vacuum}{}
		\item	\iftoggle{instructor}{membrane}{}
		\item	\iftoggle{instructor}{filter media}{}
		\item	\iftoggle{instructor}{electric field}{}
		\item	\iftoggle{instructor}{flow}{}
		\item	\iftoggle{instructor}{temperature gradient}{}
		\item	\iftoggle{instructor}{concentration gradient}{}
		\item	\iftoggle{instructor}{gravitational field (natural, or artificially created)}{}
		\item	\iftoggle{instructor}{adsorbent}{}
		\item	\iftoggle{instructor}{absorbent}{}
	\end{itemize}
\end{frame}

\begin{frame}\frametitle{Units we will consider in depth}
	Under the title of ``Mechanical Separations'' we will consider:
	\begin{itemize}
		\item	free settling (sedimentation)
		\item	screening of particles 
		\item	centrifuges
		\item	cyclones
	\end{itemize}
	
	\vspace{12pt}
	There are also others that go in this category. Deserving a quick mention are:
	\begin{itemize}
		\item	magnetic separation
		\item	electrostatic precipitation
	\end{itemize}
\end{frame}

\begin{frame}\frametitle{{\color{myGreen}{Quick mention:}} Magnetic separation}
	\begin{itemize}
		\item	used mainly in the mineral processing industries
		\item	high throughputs: up to 3000 kg/hour per meter of rotating drum
		\item	e.g. remove iron from feed
		\item	Also used in food and drug industries at multiple stages to ensure product integrity		
	\end{itemize}
	
	\see{Sinnott, 4ed, v6, Ch10}
	\begin{center}
		\includegraphics[width=0.68\textwidth]{\imagedir/separations/overview-class/magnetic-separation.png}
	\end{center}
\end{frame}

\begin{frame}\frametitle{{\color{myGreen}{Quick mention:}} Electrostatic separators}
	
	\begin{itemize}
		\item	depends on differences in conductivity of the material
		\item	materials passes through a high-voltage field while on a rotating drum
		\item	the drum is earthed
		\item	some of the particles acquire a charge and adhere stronger to the drum surface
		\item	they are carried further than the other particles, creating a split
	\end{itemize}
	
	\see{Sinnott, 4ed, v6, Ch10}
	\begin{center}
		\includegraphics[width=0.50\textwidth]{\imagedir/separations/overview-class/electrostatic-separation.png}
	\end{center}
\end{frame}	
	
\begin{frame}\frametitle{What is sedimentation}
	Class demonstration with
	\begin{itemize}
		\item	concrete powder in water
		\item	drywall compound (calcium carbonate and other particles)
	\end{itemize}

	\vspace{12pt}
	DIY: add 2 to 3 tablespoons of vinegar to a glass of milk and stir	
\end{frame}

\begin{frame}\frametitle{Definitions}	
	\textbf{Sedimentation}
	\begin{quote}	
		Removal of suspended solid particles from a fluid (\textbf{liquid} or gas) stream by gravitational settling.
	\end{quote}
	Most common to use a \textbf{liquid} rather than gas phase.

	\vspace{12pt}
	Some semantics:
	
	\vspace{12pt}	
	{\color{myGreen}{Thickening}}: generally aims to increase the solids to higher concentration; higher throughput processes

	\vspace{12pt}
	{\color{myGreen}{Clarification}}: remove solids from a relatively dilute stream, usually aims for complete suspended-solids removal: units are deeper, and have provision for coagulation of feed.
	
	\seefull{Perry, 8ed, Ch18.5}	
\end{frame}

\begin{frame}\frametitle{References for this section}
	\begin{enumerate}
		\item	Schweitzer: Handbook of Separation Techniques
		\item	Svarovsky 4ed, Ch 5 and Ch 18
		\item	Geankoplis: 3ed and 4ed, Chapter 14
		\item	Perry's: 8ed, Chapter 18
		\item	Richardson \emph{et al}.: Volume 2, Chapter 3 and 5		
	\end{enumerate}
\end{frame}

\begin{frame}\frametitle{Where is it applied?}	
	Most commonly:
	\begin{itemize}
		\item	water treatment
		\item	and mineral processing applications
	\end{itemize}
	
	\vspace{12pt}
	But also chemical, pharmaceutical, nuclear, petrochemical processes use gravity settling to resolve emulsions or other liquid-liquid dispersions. \see{Svarovsky}	
\end{frame}

\begin{frame}\frametitle{Topics we will cover}
	\begin{itemize}
		\item	factors that influence sedimentation
		\item	designing a settler unit
		\item	costs of building and operating a settler unit
		\item	flocculation (coagulation)
	\end{itemize}
\end{frame}

\begin{frame}\frametitle{List any factors that influence sedimentation process}
	\begin{itemize}
		\item	diameter of the particles
		\pause
		\item	\iftoggle{instructor}{strength of gravitational field}{}
		\item	\iftoggle{instructor}{relative density of particle vs fluid}{}
		\item	\iftoggle{instructor}{density of fluid}{}
		\item	\iftoggle{instructor}{viscosity of fluid}{}
		\item	\iftoggle{instructor}{particle concentration (hindered)}{}
		\item	\iftoggle{instructor}{{\color{myRed}{no effect}}: diameter of the vessel}{}
		\item	\iftoggle{instructor}{{\color{myRed}{no effect}}: mass of particle}{}
	\end{itemize}
\end{frame}

\begin{frame}\frametitle{Ideal case: momentum balance on an unhindered particle}	
	Forces acting on a spherical particle in a fluid:
	\begin{enumerate}
		\item	\textbf{Gravity}: a constant downward force = $mg = V_p \rho_p g$		
		\item	\textbf{Buoyancy}: proportional to volume displaced = $V_p \rho_f g$
		\item	\textbf{Drag}: opposes the particle's motion (next slide)
		\item	\textbf{Particle-particle interactions} and Brownian motion: assumed zero for now
	\end{enumerate}	
	\[
		\begin{array}{rcl}
			V_p		&=& \text{particle's volume} = \displaystyle \frac{\pi D_p^3}{6} ~[\text{m}^3]\\
			\rho_p 	&=& \text{particle density}~[\text{kg.m}^{-3}]\\
			\rho_f  &=& \text{density of fluid}~[\text{kg.m}^{-3}]\\
			\mu_f   &=& \text{fluid's viscosity}~[\text{Pa.s}]\\
			g		&=& \text{gravitational constant} = 9.81~[\text{m.s}^{-2}] \\
			D_p 	&=& \text{particle's diameter}~[\text{m}]
		\end{array}
	\]
\end{frame}

\begin{frame}\frametitle{Drag force}
	\[
	 	F_\text{drag} = C_D A_p \frac{\rho_f v^2}{2}
	\]
	where
	\[
		\begin{array}{rcl}
			v	&=& \text{relative velocity between the particle and the fluid}~[\text{m.s}^{-1}] \\
			A_p &=& \text{projected area of particle in direction of travel}~[\text{m}^2] \\
			C_D	&=& \text{drag coefficient (it's assumed constant!)}~[-]
		\end{array}
	\]
\end{frame}

\begin{frame}\frametitle{Estimating the drag coefficient}
	It's a function of Reynolds number = Re $= \displaystyle \frac{D_p v \rho_f}{\mu_f}$ \\
	\see{Richardson and Barker, p 150-153}
	\begin{enumerate}
		\item	If Re $< 1$
				\[
					C_D = \frac{24}{\text{Re}}
				\]
				
		\item	If $1 < \text{Re} < \text{1000}$
				\[
					C_D = \frac{24}{\text{Re}}\left(1 + 0.15 \text{Re}^{0.687} \right)
				\]
		\item	If $\text{1000} < \text{Re} < 2\times10^5 $
				\[
					C_D = 0.44
				\]
		\item	If $\text{Re} > 2\times10^5 $
				\[
					C_D = 0.10
				\]			
	\end{enumerate}	
\end{frame}

\begin{frame}\frametitle{Drag coefficient as a function of Re}
	\begin{center}
		\includegraphics[width=\textwidth]{\imagedir/separations/sedimentation/drag-coefficient-Reynolds-number-function.png}
	\end{center}
	\seefull{Geankoplis, 3rd p818, 4th p921}
\end{frame}

\begin{frame}\frametitle{Momentum balance (Newton's second law)}	
	\[
	\begin{array}{rcccccl}
		\displaystyle m\frac{dv}{dt} &=&  F_\text{gravity} &-& F_\text{buoyancy} &-& F_\text{drag} = 0 \qquad \text{\color{myOrange}{at steady state}} \\		
		\\
		0 &=&  V_p \rho_p g &-& V_p \rho_f g &-& C_D A_p \displaystyle \frac{\rho_f v^2}{2} 
	\end{array}
	\]
	Substitute $V_p = \displaystyle \frac{\pi D_p^3}{6}$ and $A_p = \displaystyle \frac{\pi D_p^2}{4}$ for spherical particles and solve for $v$:
	
	\begin{exampleblock}{Terminal velocity of an unhindered particle}
		\[
			v = \sqrt{\frac{4\left(\rho_p - \rho_f \right)g D_p}{3 C_D \rho_f}}
		\]
	\end{exampleblock}
\end{frame}

\begin{frame}\frametitle{Stokes' law}
	
	Simplification of the above equation when Re $< 1$:
	
	\[
		v = \frac{\left( \rho_p - \rho_f \right) g D_p^2}{18 \mu_f}
	\]	
	
	\vspace{12pt}
	\textbf{Confirm it for yourself}: \emph{hint}: use the solution for a quadratic equation $ax^2 + bx + c = 0$
\end{frame}

\begin{frame}\frametitle{Solving the general equation for $v$}
	
	$v = fn(C_D)$, but $C_D = fn(\text{Re}) = fn(v)$
	
	\begin{enumerate}
		\item	Assume Re $< 1$ (Stokes' region)
		\item	Solve for $v$
		\item	Calculate Reynolds number, Re
		\item	Was Reynolds number region assumption true? If so: stop. 
		\item	If not, use new Re and recalculate $C_D$
		\item	Repeat from step 2 to 5 until convergence		
	\end{enumerate}
	
	\vspace{12pt}
	Alternative approach shown in Geankoplis.	
\end{frame}

\begin{frame}\frametitle{Why is the terminal velocity so important?}
	\begin{exampleblock}{Design criterion}
		Terminal velocity of the slowest particle is our limiting design criterion
	\end{exampleblock}
\end{frame}

\begin{frame}\frametitle{Hindered settling}
	Particles will not settle as perfect spheres at their terminal velocity under a variety of conditions:
	\begin{itemize}
		\item	if they are hindered by other particles
		\item	they are non-spherical
		\item	concentrated feeds: particles form clusters that tend to settle faster
		\item	concentrated feeds: modify the apparent density and viscosity of the fluid
		\item	upward velocity of displaced fluids
		\item	small particles are dragged in the wake of larger particles
		\item	ionized conditions can cause particle coagulation $\rightarrow$ larger diameters $\rightarrow$ faster settling
	\end{itemize}
	
	\vspace{12pt}
	Video: \href{http://www.youtube.com/watch?v=h8n3Nt4tPXU}{http://www.youtube.com/watch?v=h8n3Nt4tPXU}
	
	How to deal with this issue?
\end{frame}

\begin{frame}\frametitle{Video of sedimentation experiments}
	\href{http://www.youtube.com/watch?v=E9rHSLUr3PU}{http://www.youtube.com/watch?v=E9rHSLUr3PU}
\end{frame}

% 13 and 14 September 2012
\begin{frame}\frametitle{Settler terminology}
	A standard gravitational thickener:
	
	\begin{center}
		\includegraphics[width=\textwidth]{\imagedir/separations/sedimentation/gravity-thickener-Svarovsky.png}
	\end{center}
	\see{Svarovsky, 3ed, p141}	
\end{frame}

\begin{frame}\frametitle{Gravitational thickener terminology}
	\begin{center}
		\includegraphics[width=\textwidth]{\imagedir/separations/sedimentation/gravity-thickener-CRv2.png}
	\end{center}
	\see{Richardson and Harker, 5ed, v2, p256}	
\end{frame}

\begin{frame}\frametitle{Gravitational thickener top and side view}
	\begin{center}
		\includegraphics[width=0.69\textwidth]{\imagedir/separations/sedimentation/gravity-thickener-Perry.png}
	\end{center}
	\see{Perry's, 8ed, 18-71}	
\end{frame}

\begin{frame}\frametitle{Unhindered settling}
	\begin{itemize}
		\item	Takes place when settling occurs at a constant rate, independent of other particles.
		\item	Use the equations derived in last class to estimate settling velocity = $v$.
		\item	Draw an imaginary horizontal layer through the settler and observe the mass of solids passing across it per unit time, per unit area = {\color{myOrange}{mass flux}}.
		\item	The flux of solids is $\psi = C_0 v$, with units of $\displaystyle \frac{\text{kg solids}}{\text{m}^3~\text{feed}} \cdot \frac{\text{meters}}{\text{second}}$
		\item	$ \psi = C_0 v \quad \displaystyle \frac{\text{kg solids}}{\text{second}} \cdot \frac{1}{\text{meters}^2}$
		\item	$\psi$ = mass feed rate per unit area = loading rate
		\item	$\displaystyle \frac{1}{\psi}$ = unit area required per given amount of mass feed rate
	\end{itemize}
	\textbf{Note}: assuming no solids leave the overflow
\end{frame}

\begin{frame}\frametitle{Preliminary settler area estimate}
	The area required under these ideal conditions:
	\[
		A = \displaystyle \frac{Q C_0}{\psi} = \displaystyle \frac{Q C_0}{C_0 v} = \displaystyle \frac{Q}{v}
	\]
	where 
	\begin{itemize}
		\item	$Q$ = volumetric feed rate~ [$\text{m}^3$ of feed per unit time]
		\item	$C_0$ = concentration of solids in feed [$\text{kg of solids per~}\text{m}^{3}~\text{feed}$]
		\item	$v$ = settling velocity [meters per unit time]
	\end{itemize}	
\end{frame}

\begin{frame}\frametitle{Example}
 	A sample of material was settled in a graduated lab cylinder 300mm tall. The interface dropped from 500mL to 215mL on the graduations during a 4 minute period. 
	
	\begin{enumerate}
		\item	Give a preliminary estimate of the clarifier diameter required to treat a waste stream of 2100 L per minute. Over-design by a factor of 2, based on the settling rate, and account for about 7 $\text{m}^2$ of entry area used to eliminate turbulence in the entering stream.
		\item	If the feed concentration is {\color{myRed}{\xout{200}}}~3.5 kg per $\text{m}^3$ feed, what is the loading rate? Is it within the typical thickener range of 50 to 120 kg per day per square meter? \see{Perry, 8ed, p22-79}
	\end{enumerate}
	{\color{myGreen}{Answers}}:
	\begin{enumerate}
		\item	Settling rate = 171 mm per 4 minutes = 42.8 mm/min.\\Area = $\displaystyle \frac{2.1~\text{m}^3.\text{min}^{-1}}{\left(0.5\right)\left(42.8 \times 10^{-3} \text{m}.\text{min}^{-1}\right)} = 98 + 7 \text{m}^2$
		\item	$\psi = C_0 v = 3.5 \displaystyle\frac{\text{kg}}{\text{m}^3} \cdot 0.022 \displaystyle \frac{\text{m}}{\text{min}} \cdot \displaystyle\frac{60 \times 24 \text{min}}{\text{day}} = 106 \displaystyle \frac{\text{kg}}{\text{day}.\text{m}^2}$
	\end{enumerate}	
\end{frame}
	
\begin{frame}\frametitle{Setting zones during sedimentation}
	\begin{center}
		\includegraphics[width=0.75\textwidth]{\imagedir/separations/sedimentation/sedimentation-over-time.png}
	\end{center}
	\see{Svarovsky, 3rd ed, p 135}
\end{frame}

\begin{frame}\frametitle{Hindered settling}
	For a high concentration of particles we have hindered settling:
	\begin{itemize}
		\item	form clusters that settle faster
		\item	however, there is upward flow from displaced fluids
		\item	apparent density and viscosity of the fluid phase is increased
		\item	drag force increased due to other particles
	\end{itemize}
	
	\vspace{12pt}
	Approaches to dealing with this:
	\begin{enumerate}
		\item	Modify the density, viscosity and other terms in the momentum balance: use correction factors
		\item	Resort to lab tests on samples that closely match the actual feed material
		\begin{itemize}
			\item	use lab results to design the settler
		\end{itemize}
	\end{enumerate}
\end{frame}

\begin{frame}\frametitle{The effect of particle concentration}
	\begin{center}
		\includegraphics[width=\textwidth]{\imagedir/separations/sedimentation/sedimentation-rate-conc-vs-time.png}
	\end{center}
	
	More concentrated solutions take longer to settle; sometimes see clearer supernatants with concentrated solutions: small particles are pulled down in wake of larger particles.
\end{frame}

\begin{frame}\frametitle{How can we accelerate settling?}
	\begin{itemize}
		\item	modify the particle shape: spherical vs needle shape (usually not possible)
		\item	modify the fluid viscosity and density 
		\begin{itemize}
			\item	not practical in most cases
			\item	e.g. used to separate diamonds in a process called ``dense medium separation''
		\end{itemize}
		\item	raking or stirring: creates free channels for particles to settle in
		\item	flocculation: a way of modifying particle shape
	\end{itemize}
\end{frame}

\begin{frame}\frametitle{Flocculation}
	MIT video on water cleaning: \href{http://www.youtube.com/watch?v=5uuQ77vAV\_U}{http://www.youtube.com/watch?v=5uuQ77vAV\_U}
	
	\vspace{7cm}
	{\tiny Please take notes from the video}
\end{frame}

\begin{frame}\frametitle{Flocculation}
	Small particles (around $< 40 \mu m$) and some biologically active particles will take unreasonably long times to settle, if at all.
	\vspace{12pt}	
	Flocculated particles cluster together and settle at higher rates
	\begin{itemize}
		\item	impossible to predict shape and hence settling rate
		\item	used in clarifiers, where clear supernatant is desired
	\end{itemize}
	\vspace{12pt}
	Flocculation is commonly ``included'' with the sedimentation step:
	\begin{center}
		\includegraphics[width=0.55\textwidth]{\imagedir/separations/sedimentation/flocculation-Perry-ch22.png}
	\end{center}
	\see{Perry, 8ed, Ch22}
	\begin{itemize}
		\item	important to not disrupt the flocs after contacting with flocculant: 30 seconds to 2 minutes contact time
	\end{itemize}
\end{frame}

\begin{frame}\frametitle{Sludge interface experiments}
	Since flocculant and concentration effects cannot be derived from theory, resort to lab settling tests.
	\begin{center}
		\includegraphics[width=\textwidth]{\imagedir/separations/sedimentation/sedimentation-batch-lab-experiment-Svarovsky-updated.png}
	\end{center}
\end{frame}

\begin{frame}\frametitle{Sludge interface experiments}
	\begin{exampleblock}{Interesting point}
		For a given feed, if you have the settling curve for one height, you have it for other heights also. Ratio 0A' : 0A'' is constant everywhere.
	\end{exampleblock}
	
	\begin{center}
		\includegraphics[width=0.6\textwidth]{\imagedir/separations/sedimentation/settling-rates-different-heights.png}
	\end{center}
	\see{Richardson and Harker, 5ed, v2}
\end{frame}

\begin{frame}\frametitle{Sludge interface experiments}
	\begin{itemize}
		\item	initial constant rate of settling is observed
		\item	a critical point is reached: point of inflection
		\item	slow compression of the solids after this point
	\end{itemize}
	\vspace{12pt}
	At least 2 procedures in the literature to design settlers from settling curves:
	\begin{itemize}
		\item	Talmage and Fitch: tends to overdesign the area
		\item	Coe and Clevenger: underdesign of the thickener area 
	\end{itemize}
	\see{Svarovsky, 4ed, p 180}

	\vspace{12pt}	
	{\color{myGreen}{In practice: we will rely on outside consultants and civil engineers, most likely, to size and design the unit. Else see the references at end for more details.}}
\end{frame}

\begin{frame}\frametitle{Settler design: shape, length, width}
	\begin{enumerate}
		\item	What width and depth should the settler be?
		\item	How long should the particles be in the settler? Does residence time matter?
	\end{enumerate}
	
	Perry's, section 22.5.6:
	\begin{itemize}
		\item	sedimentation tank can be rectangular or circular
		\begin{itemize}
			\item	rectangular: effluent weirs at the end 
			\item	circular: around the periphery 
		\end{itemize}
		\item	main concern: uniform flow in the tank (no short-circuits)		
		\item	removal efficiency = f(hydraulic flow pattern in tank)
		\begin{itemize}
			\item	incoming flow must be dissipated before solids can settle
			\item	evenly distributed; minimal disruption to existing fluid
			\item	overflow and underflow draw collected without creating hydraulic currents
			\item	solids are removed by scraping, and hydraulic flow
		\end{itemize}
	\end{itemize}
\end{frame}

\begin{frame}\frametitle{Concept: the ideal rectangular settling basin}
	\begin{center}
		\includegraphics[width=\textwidth]{\imagedir/separations/sedimentation/rectangular-tank-sedimentation.png}
	\end{center}
	{\scriptsize
	\begin{itemize}
		\item	Inlet zone: feed is assumed to be uniformly distributed across the tank's cross-section (if viewed from the top)
		\item	Settling zone: where particles move downwards towards the sludge area; particles also move horizontally due to fluid flow
		\item	Outlet zone: the supernatant/clarified liquid is collected along the basin's cross section and removed in the {\color{myGreen}{\emph{overflow}}}
		\item	Sludge zone: where the solids collect and are removed in the tank's {\color{myGreen}{\emph{underflow}}}
	\end{itemize}
	}
	\hrule
	Ensure horizontal fluid velocity (i.e. residence time) is slow enough that particles at their terminal velocity, $v$, will reach the sludge zone and settle out.
\end{frame}

\begin{frame}\frametitle{The ideal rectangular settling basin}
	\begin{center}
		\includegraphics[width=0.6\textwidth]{\imagedir/separations/sedimentation/effect-of-vessel-depth-Svarovsky.png}
	\end{center}
	Changing depth has no effect in a rectangular basin \see{Svarovsky, 4ed, p170}
\end{frame}

\begin{frame}\frametitle{Concept: the ideal circular settling basin}
	
	\begin{center}
		\includegraphics[width=\textwidth]{\imagedir/separations/sedimentation/circular-tank-sedimentation.png}
	\end{center}
	\begin{itemize}
		\item	Same zones as before
		\item	Fluid velocity is a function of radial distance
		\item	As before, ensure residence time is long enough for particles to reach the sludge zone
	\end{itemize}
\end{frame}

\begin{frame}\frametitle{Settler design rules of thumb: size}
	
	For wastewater treatment the \textbf{main design criterion}: solids percentage in underflow
	\begin{itemize}
		\item	A volume and mass balance on solids and liquids is then used to find the liquid overflow rate
		\item	surface overflow rate (SOR) $\sim 40 \text{m}^3$ per day per $\text{m}^2$ for primary units
		\item	secondary units as low as 12 up to $30 \text{m}^3$ per day per $\text{m}^2$
		\item	minimum depth of sedimentation tanks is around 3.0 m 
		\item	circular sedimentation: minimum diameter of 6.0 m 
		\item	length to width ratio of 5:1
	\end{itemize} 
\end{frame}

\begin{frame}\frametitle{Settler design rules of thumb: residence time}
	\begin{itemize}
		\item	gravity sedimentation tanks normally provide for 2 hour retention of solids, based on average flow
		\item	longer times for light solids, or in winter times
		\item	organic solids generally will not compact to more than 5 to 10\%
		\item	inorganic solids will compact up to 20 or 30\%
		\item	why important: we have to design sludge pumps to remove the solids: high concentration solids require diaphragm pumps		
	\end{itemize}
\end{frame}

\begin{frame}\frametitle{Some thickener designs}
	\begin{center}
		\includegraphics[width=\textwidth]{\imagedir/separations/sedimentation/thickener-design-Svarovsky.png}
	\end{center}
	\see{Svarovsky, 3ed, p172}
	
	\vspace{12pt}
	Can replace the steel with a sloped concrete slab.
\end{frame}

\begin{frame}\frametitle{Feed area: feedwell}
	\begin{itemize}
		\item	aim to minimize turbulence from entry velocity
		\item	avoid disruption to existing settling
		\item	avoid breaking up existing flocs
		\item	must not get clogged
	\end{itemize}
	
	\begin{center}
		\includegraphics[width=0.9\textwidth]{\imagedir/separations/sedimentation/Fitch-feedwell-Svarovsky.png}
	\end{center}
	\see{Svarovsky, 3ed, p174}
\end{frame}

\begin{frame}\frametitle{Capital costs considerations}
	Svarovsky 3rd, p179: cost = $a x^b$
	\begin{itemize}
		\item	$x$ = tank diameter between 10 and 225 ft
		\item	$a=147$ and $b=1.38$ for thickeners
	\end{itemize}
	
	\seefull{Perry, 8ed, section 18.6}
	\begin{itemize}
		\item	Installation costs will be at least 3 to 4 times the actual equipment costs.
		\item	Equipment items must include:
		\begin{itemize}
			\item	rakes, drivehead and motors
			\item	walkways and bridge (center pier) and railings
			\item	pumps, piping, instrumentation and lift mechanisms
			\item	overflow launder and feed 
		\end{itemize}
		Installation is affected by:
		\begin{itemize}			
			\item	site surveying
			\item	site preparation and excavation 
			\item	reinforcing bar placement 
			\item	backfill			
		\end{itemize}		
	\end{itemize}	
\end{frame}

\begin{frame}\frametitle{Operating costs}
	These are mostly insignificant
	\begin{itemize}
		\item	e.g. 60 m (200 ft) diameter thickener, torque rating = 1.0 MN.m: requires $\sim$ 12 kW
		\item	due to slow rotating speed: peripheral speed is about 9 m/min 
		\item	implies low maintenance costs
		\item	little attention from operators after start-up
		\item	chemicals for flocculation (if required), frequently dwarfs all other operating costs \see{Perry, 8ed, Ch18.6}
	\end{itemize}
\end{frame}

\begin{frame}\frametitle{Further self-study}
	\begin{itemize}
		\item	Designs with peripheral inlets (submerged-orifice flow control) and either center-weir outlets or peripheral-weir outlets adjacent to the peripheral-inlet channel.
		\item	Deep cone thickener
		\item	Lamella (inclined plate or tubes): often for gas-solid applications 
				%\vspace{12pt}
				%\todo{Link in YouTube video} %AzGH6ObX_Uk or FLmzCkFa9VA
	\end{itemize}
\end{frame}

\begin{frame}\frametitle{Further references to explore}	
	\begin{itemize}
		\item	Talmage and Fitch, 1955, ``Determining Thickener Unit Areas'', \emph{Ind. Eng. Chem.},\textbf{47}, 38-41, \href{http://dx.doi.org/10.1021/ie50541a022}{\small DOI:10.1021/ie50541a022}
		
		\item	Fitch, 1965, ``Current theory and thickener design'', \emph{Ind. Eng. Chem.}, \textbf{57}, p 18-28, \href{http://dx.doi.org/10.1021/ie50682a006}{\small DOI:10.1021/ie50682a006}
		
		\item	Svarovsky, ``Solid Liquid Separation'', 3rd or 4th edition. Particularly thorough regarding the settler's mechanical accessories: pumps, scrapers, \emph{etc}.
	\end{itemize}
	%* Ind Eng Chem, v46, 1164-1172, 1954
	%* Review of sedimentation theory, Chem Eng, v66, p 75-80, 1959
\end{frame}

\begin{frame}\frametitle{Practice questions}
	\begin{enumerate}
		\item	Calculate the minimum area and diameter of a circular thickener to treat 720 $\text{m}^3$ per hour of slurry containing 20\micron~particles of silica, whose density is about 2600 kg.$\text{m}^{-3}$. The particles are suspended in water at a concentration of 650~kg.$\text{m}^{-3}$. The slurry cannot be tested in a lab. Use an over-design factor of 1.5 on the settling velocity.
		\item	If it is desired to have an underflow of 1560 kg solids per $\text{m}^{3}$ underflow; what is the underflow volumetric flow rate if total separation of solids occurs?
		\item	Calculate the separation factor.		
	\end{enumerate}	
\end{frame}
