\begin{comment}

	/Users/kevindunn/Sync/Figures/examples/waste-water-treatment/flickr-4578619015_895f8dabbb_o-CC-BY-NC-SA-2.jpg

	See: http://www.flickr.com/photos/sawater/6434889743/  

	Watch video: http://www.youtube.com/watch?v=gVeW1rr8GEI&feature=related

	Irregular sedmination: http://www.youtube.com/watch?v=u6Y6d7oH6Vs&feature=relmfu

	Flocculation: http://www.youtube.com/watch?v=M2gRyYR482A&feature=related

	MIT video on water cleaning: http://www.youtube.com/watch?v=5uuQ77vAV_U&feature=related

	Lab video on sedimentation: http://www.youtube.com/watch?v=E9rHSLUr3PU&feature=related


In wastewater treatment: Perry's section 22.5.6

Gravity Sedimentation Slowly settling particles are removed with gravity sedimentation tanks. For the most part,  A sedimentation tank can be rectangular or circular. The important factor affecting its removal efficiency is the hydraulic flow pattern through the tank. The energy contained in the incoming-wastewater flow must be dissipated before the solids can settle. The wastewater flow must be distributed properly through the sedimentation volume for maximum settling efficiency. After the solids have settled, the settled effluent should be collected without creating serious hydraulic currents that could adversely affect the sedimentation process. Effluent weirs are placed at the end of rectangular sedimentation tanks and around the periphery of circular sedimentation tanks to ensure uniform flow out of the tanks. Once the solids have settled, they must be removed from the sedimentation-tank floor by scraping and hydraulic flow. Conventional sedimentation tanks have sludge hoppers to collect the concentrated sludge and to prevent removal of excess volumes of water with the settled solids. Cross-sectional diagrams of conventional sedimentation tanks are shown in Figs. 22-32 and 22-33.

Design criteria for gravity sedimentation tanks normally provide for 2-h retention based on average flow, with longer retention periods used for light solids or inert solids that do not change during their retention in the tank. Care should be taken that sedimentation time is not too long; otherwise, the solids will compact too densely and affect solids collection and removal. Organic solids generally will not compact to more than 5 to 10 percent. Inorganic solids will compact up to 20 or 30 percent. Centrifugal sludge pumps can handle solids up to 5 or 6 percent, while positive-displacement sludge pumps can handle solids up to 10 percent. With solids above 10 percent the sludge tends to lose fluid properties and must be handled as a semi-solid rather than a fluid. Circular sedimentation tanks have steel truss boxes with angled sludge scrapers on the lower side. As the sludge scrapers rotate, the solids are pushed toward the sludge hopper for removal on a continuous or semicontinuous basis. The rectangular sedimentation tanks employ chain-and-flight sludge collectors or rail-mounted sludge collectors. When floating solids can occur in primary sedimentation tanks, surface skimmers are mounted on the sludge scrapers so that the surface solids are removed at regular intervals.


Figure 22-32. Schematic diagram of a circular sedimentation tank.

Figure 22-33. Schematic diagram of a rectangular sedimentation tank.
The surface overflow rate (SOR) for primary sedimentation is normally held close to 40.74 m 3/(m 2·day) [1000 gal/(ft 2·day)] for average flow rates, depending upon the solids characteristics. Lowering the SOR below 40.74 m 3/(m 2·day) does not produce improved effluent quality in proportion to the reduction in SOR. Generally, the minimum depth of sedimentation tanks is 3.0 m (10 ft), with circular sedimentation tanks having a minimum diameter of 6.0 m (20 ft) and rectangular sedimentation tanks having length-to-width ratios of 5:1. Chain-and-flight limitations generally keep the width of rectangular sedimentation tanks to increments of 6.0 m (20 ft) or less. While hydraulic overflow rates have been limited on the effluent weirs, operating experience has indicated that the recommended limit of 186 m 3/ (m·day) [15,000 gal/(ft·day)] is lower than necessary for good operation. A circular sedimentation tank with a single-edge weir provides adequate weir length and is easier to adjust than one with a doublesided weir. More problems appear to be created from improper adjustment of the effluent weirs than from improper length.


And Perry's section 22.5.7

edimentation Tanks These tanks are an integral part of any activated-sludge system. It is essential to separate the suspended solids from the treated liquid if a high-quality effluent is to be produced. Circular sedimentation tanks with various types of hydraulic sludge collectors have become the standard secondary sedimentation system. Square tanks have been used with commonwall construction for compact design with multiple tanks. Most secondary sedimentation tanks use center-feed inlets and peripheral-weir outlets. Recently, efforts have been made to employ peripheral inlets with submerged-orifice flow controllers and either center-weir outlets or peripheral-weir outlets adjacent to the peripheral-inlet channel.

\end{comment}

% 11 September 2012
\begin{frame}\frametitle{What is sedimentation}
	Class demonstration with
	\begin{itemize}
		\item	concrete powder in water
		\item	drywall compound (calcium carbonate and other particles)
	\end{itemize}

	\vspace{12pt}
	DIY: add 2 to 3 tablespoons of vinegar to a glass of milk and stir	
\end{frame}

\begin{frame}\frametitle{Definitions}	
	\textbf{Sedimentation}
	\begin{quote}	
		Removal of suspended solid particles from a fluid (\textbf{liquid} or gas) stream by gravitational settling.
	\end{quote}
	Most common to use a \textbf{liquid} rather than gas phase.

	\vspace{12pt}
	Some semantics:
	
	\vspace{12pt}	
	{\color{myGreen}{Thickening}}: generally aims to increase the solids to higher concentration; higher throughput processes

	\vspace{12pt}
	{\color{myGreen}{Clarification}}: remove solids from a relatively dilute stream, usually aims for complete suspended-solids removal: units are deeper, and have provision for coagulation of feed.
	
	\seefull{Perry, 8ed, Ch18.5}	
\end{frame}

\begin{frame}\frametitle{References for this section}
	\begin{enumerate}
		\item	Schweitzer: Handbook of Separation Techniques
		\item	Svarovsky 4ed, Ch 5 and Ch 18
		\item	Geankoplis: 3ed and 4ed, Chapter 14
		\item	Perry's: 8ed, Chapter 18
		\item	Richardson \emph{et al}.: Volume 2, Chapter 3 and 5		
	\end{enumerate}
\end{frame}

\begin{frame}\frametitle{Where is it applied?}	
	Most commonly:
	\begin{itemize}
		\item	water treatment
		\item	and mineral processing applications
	\end{itemize}
	
	\vspace{12pt}
	But also chemical, pharmaceutical, nuclear, petrochemical processes use gravity settling to resolve emulsions or other liquid-liquid dispersions. \see{Svarovsky}	
\end{frame}

\begin{frame}\frametitle{Topics we will cover}
	\begin{itemize}
		\item	factors that influence sedimentation
		\item	designing a settler unit
		\item	costs of building and operating a settler unit
		\item	flocculation (coagulation)
	\end{itemize}
\end{frame}

\begin{frame}\frametitle{List any factors that influence sedimentation process}
	\begin{itemize}
		\item	diameter of the particles
		\pause
		\item	\iftoggle{instructor}{strength of gravitational field}{}
		\item	\iftoggle{instructor}{relative density of particle vs fluid}{}
		\item	\iftoggle{instructor}{density of fluid}{}
		\item	\iftoggle{instructor}{viscosity of fluid}{}
		\item	\iftoggle{instructor}{particle concentration (hindered)}{}
		\item	\iftoggle{instructor}{{\color{myRed}{no effect}}: diameter of the vessel}{}
		\item	\iftoggle{instructor}{{\color{myRed}{no effect}} :mass of particle}{}
	\end{itemize}
\end{frame}

\begin{frame}\frametitle{Ideal case: momentum balance on an unhindered particle}	
	Forces acting on a spherical particle in a fluid:
	\begin{enumerate}
		\item	\textbf{Gravity}: a constant downward force = $mg = V_p \rho_p g$		
		\item	\textbf{Buoyancy}: proportional to volume displaced = $V_p \rho_f g$
		\item	\textbf{Drag}: opposes the particle's motion (next slide)
		\item	\textbf{Particle-particle interactions} and Brownian motion: assumed zero for now
	\end{enumerate}	
	\[
		\begin{array}{rcl}
			V_p		&=& \text{particle's volume} = \displaystyle \frac{\pi D_p^3}{6} ~[\text{m}^3]\\
			\rho_p 	&=& \text{particle density}~[\text{kg.m}^{-3}]\\
			\rho_f  &=& \text{density of fluid}~[\text{kg.m}^{-3}]\\
			\mu_f   &=& \text{fluid's viscosity}~[\text{Pa.s}]\\
			g		&=& \text{gravitational constant} = 9.81~[\text{m.s}^{-2}] \\
			D_p 	&=& \text{particle's diameter}~[\text{m}]
		\end{array}
	\]
\end{frame}

\begin{frame}\frametitle{Drag force}
	\[
	 	F_\text{drag} = C_D A \frac{\rho_f v^2}{2}
	\]
	where
	\[
		\begin{array}{rcl}
			v	&=& \text{relative velocity between the particle and the fluid}~[\text{m.s}^{-1}] \\
			A_p &=& \text{projected area of particle in direction of travel}~[\text{m}^2] \\
			C_D	&=& \text{drag coefficient (it's assumed constant!)}~[-]
		\end{array}
	\]
\end{frame}

\begin{frame}\frametitle{Estimating the drag coefficient}
	It's a function of Reynolds number = Re $= \displaystyle \frac{D_p v \rho_f}{\mu_f}$ \\
	\see{Richardson and Barker, p 150-153}
	\begin{enumerate}
		\item	If Re $< 1$
				\[
					C_D = \frac{24}{\text{Re}}
				\]
				
		\item	If $1 < \text{Re} < \text{1000}$
				\[
					C_D = \frac{24}{\text{Re}}\left(1 + 0.15 \text{Re}^{0.687} \right)
				\]
		\item	If $\text{1000} < \text{Re} < 2\times10^5 $
				\[
					C_D = 0.44
				\]
		\item	If $\text{Re} > 2\times10^5 $
				\[
					C_D = 0.10
				\]			
	\end{enumerate}	
\end{frame}

\begin{frame}\frametitle{Drag coefficient as a function of Re}
	\begin{center}
		\includegraphics[width=\textwidth]{\imagedir/separations/sedimentation/drag-coefficient-Reynolds-number-function.png}
	\end{center}
	\seefull{Geankoplis, 3rd p818, 4th p921}
\end{frame}

\begin{frame}\frametitle{Momentum balance (Newton's second law)}	
	\[
	\begin{array}{rcccccl}
		\displaystyle m\frac{dv}{dt} &=&  F_\text{gravity} &-& F_\text{buoyancy} &-& F_\text{drag} = 0 \qquad \text{\color{myOrange}{at steady state}} \\		
		\\
		0 &=&  V_p \rho_p g &-& V_p \rho_f g &-& C_D A_p \displaystyle \frac{\rho_f v^2}{2} 
	\end{array}
	\]
	Substitute $V_p = \displaystyle \frac{\pi D_p^3}{6}$ and $A_p = \displaystyle \frac{\pi D_p^2}{4}$ for spherical particles and solve for $v$:
	
	\begin{exampleblock}{Terminal velocity of an unhindered particle}
		\[
			v = \sqrt{\frac{4\left(\rho_p - \rho_f \right)g D_p}{3 C_D \rho_f}}
		\]
	\end{exampleblock}
\end{frame}

\begin{frame}\frametitle{Stokes' law}
	
	Simplification of the above equation when Re $< 0.2$:
	
	\[
		v = \frac{\left( \rho_p - \rho_f \right) g D_p^2}{18 \mu_f}
	\]	
	
	\vspace{12pt}
	\textbf{Confirm it for yourself}: \emph{hint}: use the solution for a quadratic equation $ax^2 + bx + c = 0$
\end{frame}

\begin{frame}\frametitle{Solving the general equation for $v$}
	
	$v = fn(C_D)$, but $C_D = fn(\text{Re}) = fn(v)$
	
	\begin{enumerate}
		\item	Assume Re $< 1$ (Stokes' region); calculate $C_D$
		\item	Solve for $v$
		\item	Calculate Reynolds number, Re
		\item	Was Reynolds number region assumption true? If so: stop. 
		\item	If not, use new Re and recalculate $C_D$
		\item	Repeat from step 2 to 5 until convergence		
	\end{enumerate}
	
	\vspace{12pt}
	Alternative approach shown in Geankoplis.	
\end{frame}

\begin{frame}\frametitle{Why is the terminal velocity so important?}
	\begin{exampleblock}{Design criterion}
		Terminal velocity of the slowest particle is our limiting design criterion
	\end{exampleblock}
\end{frame}

\begin{frame}\frametitle{Hindered settling}
	Particles will not settle as perfect spheres at their terminal velocity under a variety of conditions:
	\begin{itemize}
		\item	if they are hindered by other particles
		\item	they are non-spherical
		\item	they are in a dilute suspension; concentrated particles form clusters that tend to settle faster
		\item	concentrated feeds: modify the apparent density and viscosity of the fluid
		\item	upward velocity of displaced fluids
		\item	small particles are dragged in the wake of larger particles
		\item	ionized conditions can cause particle coagulation $\rightarrow$ larger diameters $\rightarrow$ faster settling
	\end{itemize}
	
	\vspace{12pt}
	Video: \href{http://www.youtube.com/watch?v=h8n3Nt4tPXU}{http://www.youtube.com/watch?v=h8n3Nt4tPXU}
	
	How to deal with this issue?
\end{frame}

\begin{frame}\frametitle{Video of sedimentation experiments}
	\href{http://www.youtube.com/watch?v=E9rHSLUr3PU}{http://www.youtube.com/watch?v=E9rHSLUr3PU}
\end{frame}

% 13 September 2012
\begin{comment}

\begin{frame}\frametitle{Sedimentation: lab data}
	\begin{center}
		\includegraphics[width=0.75\textwidth]{\imagedir/separations/sedimentation/sedimentation-over-time.png}
	\end{center}
	\see{Svarovsky, 3rd ed, p 135}
\end{frame}

\begin{frame}\frametitle{Settler design}
	\begin{enumerate}
		\item	What width and depth should the settler be?
		\item	How long should the particles be in the settler? Does residence time matter?
	\end{enumerate}
\end{frame}

\begin{frame}\frametitle{Settler depth}
	\begin{itemize}
		\item	Does not affect the settler's performance: only residence time and terminal velocity of slowest particle are important
	\end{itemize}
	
	\[
		\text{area} = \frac{Q_c}{V_m}
	\]
	
	\todo{See figure 5.3 in Svarovsky4, p 170}
	
	\vspace{12pt}
	These tanks are designed on the basis of retention time, surface overflow rate, and minimum depth. \see{Perry 8ed, Ch22.5}
\end{frame}

\begin{frame}\frametitle{How can we accelerate settling}
	\begin{itemize}
		\item	raking or stirring: creates free channels for particles to settle in
		\item	flocculation
	\end{itemize}
\end{frame}

\begin{frame}\frametitle{How well have we done?}
	\begin{itemize}
		\item	separation efficiency
		\item	mass fraction of solids in overflow
		\item	dryness/moisture of solid fraction
	\end{itemize}
\end{frame}

\begin{frame}\frametitle{Type 1: unhindered settling}
	Settling is at a linear rate.
	
	\vspace{12pt}
	
	\textbf{Example}: a sample of material was settled in a graduated lab cylinder 300mm tall. The interface dropped from 500mL to 215mL on the scale during a 4 minute period. What diameter clarifier would be required to treat a waste stream of 145 L per minute if it enters through a pipe at a velocity of 330 mm per minute? 	
	Over-design the unit by a factor of 2.
	
	
	
	hat is the cross sectional area required for a settler where a sample of material
	
	Settler example in Geankoplis with residence time 

	Also see example in C+R v2, Example 5.1 , p265

	Perry, Ch 22.5.6:  
	%* provide for 2-h retention based on average flow (not too much longer else solids are too compact)
	%* surface overflow rate (SOR) for primary sedimentation is normally held close to 40.74 m 3/(m 2 .day) [1000 gal/(ft 2·day)] for average flow rates, depending upon the solids characteristics. Lowering the SOR below 40.74 m 3/(m 2·day) does not produce improved effluent quality in proportion to the reduction in SOR. Generally, the minimum depth of sedimentation tanks is 3.0 m (10 ft), with circular sedimentation tanks having a minimum diameter of 6.0 m (20 ft) and rectangular sedimentation tanks having length-to-width ratios of 5:1. Chain-and-flight limitations generally keep the width of rectangular sedimentation tanks to increments of 6.0 m (20 ft) or less. While hydraulic overflow rates have been limited on the effluent weirs, operating experience has indicated that the recommended limit of 186 m 3/ (m.day) [15,000 gal/(ft·day)] is lower than necessary for good operation. A circular sedimentation tank with a single-edge weir provides adequate weir length and is easier to adjust than one with a double sided weir. More problems appear to be created from improper adjustment of the effluent weirs than from improper length.
	
	%Secondary units: Surface overflow rates have been slowly reduced from 33 m 3/(m 2·day) [800 gal/(ft 2·day)] to 24 m 3/(m 2·day) to 16 m 3/(m 2·day) [600 gal/(ft 2·day) to 400 gal/(ft 2·day)] and even to 12 m 3 (m 2·day) [300 gal/(ft 2·day)] in some instances, based on average raw-waste flows. 
	
	
	Perry, Ch 22.5.8:  Loading rates on thickeners range from 50-122 kg/m 2/d (10-25 lb/ft 2/d) for primary sludge to 12-45 kg/m 2/d (2.5-9 lb/ft 2/d). Solids detention time is 0.5 days in summer to several days in winter.
\end{frame}

\begin{frame}\frametitle{Sludge interface experiments}
	
	\begin{exampleblock}{Interesting point}
		For a given feed stream, if you have the settling curve for one height, you have it for other heights also. Ratio 0A' : 0A'' is constant everywhere.
	\end{exampleblock}
	
	\begin{center}
		\includegraphics[width=0.6\textwidth]{\imagedir/separations/sedimentation/settling-rates-different-heights.png}
	\end{center}
	
	
\end{frame}

\begin{frame}\frametitle{Capital costs considerations}
	Svarovsky 3rd, p179: cost = $a x^b$
	\begin{itemize}
		\item	$x$ = tank diameter between 10 and 225 ft
		\item	$a=147$ and $b=1.38$ for thickeners
	\end{itemize}
	
	\seefull{Perry, 8ed, section 18.6}
	\begin{itemize}
		\item	Installation costs will be at least 3 to 4 times the actual equipment costs.
		\item	Equipment items must include:
		\begin{itemize}
			\item	rakes, drivehead and motors
			\item	walkways and bridge (center pier) and railings
			\item	pumps, piping, instrumentation and lift mechanisms
			\item	overflow launder and feed 
		\end{itemize}
		Installation is affected by:
		\begin{itemize}			
			\item	site surveying
			\item	site preparation and excavation, 
			\item	reinforcing bar placement, 
			\item	backfill			
		\end{itemize}		
	\end{itemize}	
\end{frame}

\begin{frame}\frametitle{Operating costs}
	These are mostly insignificant
	\begin{itemize}
		\item	e.g. 60 m (200 ft) diameter thickener, torque rating = 1.0 MN.m: require $\sim$ 12 kW
		\item	due to slow rotating speed: peripheral speed is about 9 m/min 
		\item	implies low maintenance costs
		\item	little attention from operators after start-up
		\item	chemicals for flocculation (if required), frequently dwarfs all other operating costs \see{Perry, 8ed, Ch18.6}
	\end{itemize}
\end{frame}

\begin{frame}\frametitle{Feed mechanisms}
	to control turbulence
	to mix in flocculants
	to control entry velocity
\end{frame}

\begin{frame}\frametitle{Further self-study}
	\begin{itemize}
		\item	Deep cone thickener
		\item	Lamella (inclined plate or tubes): often for gas-solid applications 
				\vspace{12pt}
				\todo{Link in YouTube video} %AzGH6ObX_Uk or FLmzCkFa9VA
	\end{itemize}
\end{frame}

\begin{frame}\frametitle{Flocculation}
	MIT video
\end{frame}

\begin{frame}\frametitle{CCD}
	
\end{frame}

\begin{frame}\frametitle{Other references to check still}
	* Ind Eng Chem, v46, 1164-1172, 1954
	* Review of sedimentation theory, Chem Eng, v66, p 75-80, 1959
	* Fitch, Current theory and thickener design, Ind Eng Chem, v 57, p 18, 1965
	
\end{frame}

\end{comment}