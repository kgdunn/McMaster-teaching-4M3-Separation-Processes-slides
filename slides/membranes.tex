% 02 October 2012
\begin{frame}\frametitle{Membranes}
	On a loose sheet of paper, please list/describe 5 topics related to membranes that you want to learn about in the next 5 classes.
	
	\vspace{12pt}
	For example:
	\begin{itemize}
		\item	the equations to model flow of fluid through the a membrane
		\item	
		\item	
		\item	
		\item	
		\item	
	\end{itemize}
	
	\vspace{12pt}
	e.g. recall interesting ideas from Henk Koops' talk; check the internet; talk with the person next to you
\end{frame}

\begin{frame}\frametitle{Introduction to membranes}
	Please refer to Henk Koops' slides/video from 28 September 2012 on \href{http://learnche.mcmaster.ca/4M3/Membranes,_dialysis,_reverse_osmosis,_filters_and_bioseparations_-_2012}{the course website}
\end{frame}

\begin{frame}\frametitle{Why use membranes?}
	Some really difficult separations:
	\begin{itemize}
		\item	finely dispersed solids; density close to liquid phase; gelatinous particles
		\item	dissolved salts
		\item	non-volatile organics (e.g. \href{http://en.wikipedia.org/wiki/Humic_acid}{humic substances})
		\item	biological materials: sensitive to the environment
			\begin{itemize}
				\item	cannot centrifuge 
				\item	cannot sediment
			\end{itemize}
	\end{itemize}
	
	\vspace{12pt}
	\begin{exampleblock}{It is usually worth asking:}
		\begin{center}
			How does nature separate?
		\end{center}
		\scriptsize
		\begin{itemize}
			\item	energy efficient
			\item	effective
			\item	maybe slow?
		\end{itemize}
	\end{exampleblock}		
\end{frame}

\begin{frame}\frametitle{Why use membranes?}
	Relatively new separation step {\small (``new'' meaning since 1960 to 1980s)}
	\begin{itemize}
		\item	often saves energy costs over alternative separations
		\begin{itemize}
			\item	ambient temperature operation
		\end{itemize}
		\item	often easier to operate and control
	\end{itemize}
	
	
	\begin{columns}[c]
		\column{0.40\textwidth}
			\begin{center}
				\includegraphics[width=\textwidth]{\imagedir/separations/membranes/membrane-module-Henk-Koops-talk.png}
			\end{center}
			\see{Henk Koops' slides, GE Water and Process Technologies}
		\column{0.60\textwidth}
			Modules:
			\begin{itemize}
				\item	feed stream split into parallel units
				\item	easier to maintain and replace parts
				\item	can be expanded as needs grow
			\end{itemize}			
	\end{columns}
\end{frame}

\begin{frame}\frametitle{Challenges in membrane design}
	Challenges:
	\begin{itemize}
		\item	withstanding high pressure differences but still have thin membrane
		\item	increasing selectivity (separation factor) for specific application areas
		\item	dealing with fouling and cleaning
		\item	uniformity of pore sizes
		\item	temperature stability (e.g. steam sterilization)
	\end{itemize}
\end{frame}

\begin{frame}\frametitle{Market size}
	\begin{center}
		\includegraphics[width=\textwidth]{\imagedir/separations/membranes/membrane-market-size-Perrys.png}
	\end{center}
	\see{Perry's: Chapter 20, 8ed}
\end{frame}

\begin{frame}\frametitle{Let's formalize some terminology}
	
	%\todo{Mindmap of terms}
	\begin{center}
		\includegraphics[width=\textwidth]{\imagedir/separations/membranes/some-terminology-Seader-based.png}
	\end{center}
	\see{Seader, Henley and Roper, p 501}
\end{frame}

\begin{frame}\frametitle{More terminology}
	
	{\color{myGreen}{semipermeable}}: partially permeable, e.g. your skin allows certain size particles in, but not others

	\vspace{12pt}
	{\color{myGreen}{mass separating agent}}: the membrane itself
	
	\vspace{12pt}	
	{\color{myGreen}{energy separating agent}}: the applied pressure (pressure drop)
	
	\vspace{12pt}
	{\color{myGreen}{porosity}} = $\displaystyle \frac{\text{area of open pores}}{\text{total surface area}}$ 
		%\item	permeate: the material passing through the membrane from feed to outlet side
		%\item	retentate: the material retained on the feed-side of the membrane
		%\item	solute: most often retained on the inside (feed side) of the membrane and deposited on the membrane wall
		%\item	solvent: the liquid phase that carries the solute
		%\item	gel effect / buildup: the solute will build up on the membrane wall and form a gel
		%\item	cross-flow:
		%\item	sweep:
		%\item	kilodalton
		%\item	MWCO: C+Rv2ed5, p 440 and 441: A parameter often quoted in manufacturer’s literature is the nominal molecular weight cut-off (MWCO) of a membrane. This is based on studies of how solute molecules are rejected by membranes. A solute will pass through a membrane if it is sufficiently small to pass through a pore, if it does not significantly interact with the membrane and if it does not interact with other, larger solutes. It is possible to define a solute rejection coefficient R by: R = 1 − (C_p/C_f ) where C_f is the concentration of solute in the feed stream and C_p is the concentration of solute in the permeate. See fig 8.3 in C+Rv2ed5
		%\item	Pervaporation
		%\item	Reverse osmosis
		%\item	Permeability
		%\item	permeance
\end{frame}

\begin{frame}\frametitle{What is flux?}
	\begin{exampleblock}{}
		\centering
		\small The (volumetric) or (molar) or (mass) flow per unit time for 1 unit of area
	\end{exampleblock}
	\vspace{12pt}
	\begin{columns}[t]
		\column{0.80\textwidth}
			\begin{itemize}
				\item	$J$ = flux = $\displaystyle \frac{\text{transfer rate}}{\text{transfer area}}$
				\item	e.g. 42 $\text{mol.s}^{-1}\text{.m}^{-2}$
				\item	never simplify the units: write 13 $\text{m}^3.\text{s}^{-1}\text{.m}^{-2}$
				\item	{\color{myRed}{do not write}} 13 $\text{m}\text{.s}^{-1}$
			\end{itemize}
		\column{0.35\textwidth}
			\begin{center}
				\includegraphics[width=\textwidth]{\imagedir/separations/membranes/flux-concept.png}
			\end{center}
	\end{columns}
	\begin{exampleblock}{General principle}
		For a given unit area, we want the highest flux possible (at the lowest possible cost)
	\end{exampleblock}	
\end{frame}

\begin{frame}\frametitle{Membrane classification}
	
	\begin{center}
		\includegraphics[width=\textwidth]{\imagedir/separations/membranes/membrane-application-areas-CRv2-5ed-p438.png}
	\end{center}
	
	\see{Richardson and Harker, p 438}
\end{frame}

\begin{frame}\frametitle{Transport through a membrane}
	\begin{exampleblock}{Why study theoretical models?}
		All forms of membrane applications rely to some extent on the same equation \textbf{structure}. The details will change.
	\end{exampleblock}

	\vspace{12pt}
	
	Will allow us to:
	\begin{itemize}
		\item	troubleshoot problems with the process
		\item	predict expected impact of improvements/changes to the process
		\item	used for crudely sizing the unit (order or magnitude estimates)
	\end{itemize}
	
	%Geankop: Fig 7.1.3 	
	%Why have sweep: increase $k_c2$ coefficient
	%Film on wall effect
	%Diffusion if $\Delta C$
	%Pressure $\Delta P$
	%Asymmetric and symmetric membrane equations
	%Kozeny-Carmen equation
	%Surface adsorption or materials and fouling
\end{frame}

\begin{frame}\frametitle{Examples you will be able to solve}
	\begin{enumerate}
		\item	how long should we operate unit at constant \( \Delta P \) to achieve desired separation?
		\item	what is the mass transfer coefficient through the lab membrane?
		\item	what pressure drop (and therefore pump size) do I expect?
		\item	how many cassettes does this application require?
	\end{enumerate}
\end{frame}

\begin{frame}\frametitle{The general equation}
	\[
		\displaystyle \frac{\text{transfer rate}}{\text{transfer area}}  = \text{Flux} = \displaystyle \frac{(\text{permeability})(\text{driving force})}{\text{thickness}} = \displaystyle \frac{\text{driving force}}{\text{resistance}} 
	\]
\end{frame}

\begin{frame}\frametitle{Microfiltration}
	\begin{columns}[t]
		\column{0.60\textwidth}
			\begin{itemize}
				\item	0.1 \micron ~to 10 \micron
				\item	conventional filters: not effective below $\sim 5$~\micron
				\item	microfiltration membranes: generally symmetric pores
				\item	porosity as high as $\epsilon = 0.8$
				\item	driving force: pressure difference
				\item	application areas:
					\begin{itemize}
						\item	yeast cells
						\item	wine/beer/juice clarification
						\item	large bacteria
					\end{itemize}
			\end{itemize}
		\column{0.40\textwidth}
			\begin{center}
				\includegraphics[width=0.95\textwidth]{\imagedir/separations/membranes/symmetric-pore-structure-Henk-Koops-talk.png}
			\end{center}
			\see{Henk Koops' slide}
	\end{columns}	
\end{frame}

\begin{frame}\frametitle{General modelling equation applied}
	\begin{exampleblock}{}
		\begin{columns}[c]
			\column{0.30\textwidth}
				\[
					\text{Flux} = J = \displaystyle \frac{\Delta P}{\mu \left(R_m + R_c \right)}
				\]
			\column{0.70\textwidth}
				\begin{center}
					\includegraphics[width=0.53\textwidth]{\imagedir/separations/membranes/solute-build-up-CRv2-5ed-p446.png}
				\end{center}
		\end{columns}		
	\end{exampleblock}	
	\vspace{6pt}
	\begin{tabular}{lll}
		$J$			&	[$\text{kg}.\text{s}^{-1}\text{.m}^{-2}$] 			& permeate flux \\
		$\Delta P$	&	$[Pa] = [\text{kg}.\text{m}^{-1}\text{.s}^{-2}]$ 	& generally $0.1 \text{~to~} 0.5 \times 10^6$ Pa \\
		$\mu$ 		&  	$[\text{kg}.\text{m}^{-1}\text{.s}^{-1}]$			& permeate viscosity\\
		$R_m$ 		&  	$[\text{m}^2.\text{kg}^{-1}]$						& resistance due to the membrane (small)\\
		$R_c$ 		&  	$[\text{m}^2.\text{kg}^{-1}]$						& resistance due to the cake build up (large)\\
	\end{tabular}
\end{frame}

\begin{frame}\frametitle{Flow patterns for microfiltration}
	% The particle-containing fluid to be filtered is pumped at a velocity in the range 1–8 m/s parallel to the face of the membrane and with a pressure difference of (MPa) across the membrane. T
	\begin{columns}[t]
		\column{0.50\textwidth}
			{\color{myBlue}{Dead-end flow}}
			\begin{itemize}
				\item	only for very low concentration feeds
				\item	else becomes rapidly clogged
			\end{itemize}
		\column{0.50\textwidth}
			{\color{myBlue}{Cross-flow}}
			\begin{center}
				\includegraphics[width=\textwidth]{\imagedir/separations/membranes/cross-flow-wikipedia.png}
			\end{center}
			\begin{itemize}
				\item	TFF = tangential flow filtration
				\item	main purpose?
				%\item	$\Delta P = \displaystyle \frac{P}{}$
			\end{itemize}
	\end{columns}
	
\end{frame}

\begin{frame}\frametitle{References}
	% Notes from Dickson
	\begin{itemize}
		\item	Wankat, ``Separation Process Engineering'', 2nd edition, chapter 16
		\item	Seader, Henly and Roper, ``Separation Process Principles'', 3rd edition, chapter 14
		\item	Richardson and Harker, ``Chemical Engineering, Volume 2'', 5th edition, chapter 8
		\item	Geankoplis, ``Transport Processes and Separation Process Principles'', 4th edition, chapter 7 (theory) and chapter 13
		\item	Ghosh, ``Principles of bioseparation engineering'', chapter 11
		%* Ridgeway talk on internet
		%* Membrane bioreactors: http://www.gewater.com/products/equipment/mf\_uf\_mbr/mbr.jsp
	\end{itemize}
\end{frame}

\begin{comment}

\begin{frame}\frametitle{Application areas}
	\begin{itemize}
		\item	blood cleaning and oxygenation
		\item	wastewater treatment
		\item	desalination
		\item	reverse osmosis
		\item	MBR
		\item	high flux membranes with carbon nanotubes (Shol and Johnson)
	\end{itemize}
\end{frame}
	
\begin{frame}\frametitle{Ultrafiltration}
	% eqn 20-84 Perry
	
	\begin{itemize}
		\item	recovery of proteins
		\item	separate permanent emulsions: oil phase will not pass
		\item	fine colloidal particles: e.g. paint/dyes in wastewater
		\item	large molecules in retentate
		\item	See Wankat, p 574
	\end{itemize}
\end{frame}

\begin{frame}\frametitle{Transport through a membrane}
	Fig 3 in green book: velocity function for ultrafiltration
\end{frame}

\begin{frame}\frametitle{Wastewater applications}
	Perry's Chapter 22.5.8

	Membrane Processes These processes use a selectively permeable membrane to separate pollutants from water. Most of the membranes are formulated from complex organics that polymerize during membrane preparation. This allows the membrane to be tailored to discriminate by molecular size or by degree of hydrogen bonding potential. Ultrafiltration membranes discriminate by molecular size or weight, while reverse osmosis membranes discriminate by hydrogenbonding characteristics. The permeability of these membranes is low: from 0.38-3.8 m/day (10-100 gal/d/ft 2). The apparatus in which they are used must provide a high surface area per unit volume.

	Membrane Bioreactors (MBRs): Membrane bioreactors are a technology that combines biological degradation of waste products with membrane filtratio
\end{frame}

\begin{frame}\frametitle{Principles used to separate: DISCUSS THESE AS NEEDED, ie. when you are introducing microfiltration vs dialysis}
	* Principle(s) used:
	 - diffusion differences (different solubilities)
	 - size differences
	 - electric attraction
\end{frame}

\begin{frame}\frametitle{Water treatment}
	http://www.youtube.com/watch?v=YlMGZWmh\_Mw
	Desalination (with some energy recovery): http://www.youtube.com/watch?v=M3mpJysa6zQ
\end{frame}
	
\end{comment}



