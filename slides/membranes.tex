
References:

Wankat, Separation Process Engineering, Chapter 16

Membrane bioreactors: http://www.gewater.com/products/equipment/mf_uf_mbr/mbr.jsp


\begin{frame}\frametitle{Wastewater applications}
	Perry's Chapter 22.5.8
	
	Membrane Processes These processes use a selectively permeable membrane to separate pollutants from water. Most of the membranes are formulated from complex organics that polymerize during membrane preparation. This allows the membrane to be tailored to discriminate by molecular size or by degree of hydrogen bonding potential. Ultrafiltration membranes discriminate by molecular size or weight, while reverse osmosis membranes discriminate by hydrogenbonding characteristics. The permeability of these membranes is low: from 0.38–3.8 m/day (10–100 gal/d/ft 2). The apparatus in which they are used must provide a high surface area per unit volume.

	Membrane Bioreactors (MBRs): Membrane bioreactors are a technology that combines biological degradation of waste products with membrane filtratio
\end{frame}