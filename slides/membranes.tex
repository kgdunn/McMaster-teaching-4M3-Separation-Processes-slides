\begin{frame}\frametitle{Membranes}
	On a loose sheet of paper, please list/describe 5 topics related to membranes that you want to learn about in the next 5 classes.
	
	\vspace{12pt}
	For example:
	\begin{itemize}
		\item	the equations to model flow of fluid through the a membrane
		\item	
		\item	
		\item	
		\item	
	\end{itemize}
	
	\vspace{12pt}
	e.g. recall interesting ideas from Henk Koops' talk; check the internet; talk with the person next to you
\end{frame}

\begin{frame}\frametitle{Why use membranes?}
	Relatively new separation step (``new'' meaning since 1960 to 1980s)
	\begin{itemize}
		\item	often saves energy costs over alternative separations
		\item	modularized: easier to maintain and replace parts
		\item	modules: smaller footprint
		\item	modules: feed stream is split into parallel units
		\item	easier to control (operate)
	\end{itemize}
	\todo{image of modules vs one larger unit}
	Challenges:
	\begin{itemize}
		\item	withstanding high pressure differences but still thin
		\item	increasing selectivity for specific applications
	\end{itemize}
\end{frame}

\begin{frame}\frametitle{Terminology}
	See paper notes
	
	Add fig 14.1 Seader
	
	membrane: 
	* semipermeable
	* not breakdown, dissolve or disintegrate
	
	* Separating agent: membrane itself
	* Principle(s) used:
	 - diffusion differences (different solubilities)
	 - size differences
	 - electric attraction
	 - 
	
\end{frame}

\begin{frame}\frametitle{Transport through a membrane}
	Geankop: Fig 7.1.3 
	
	Why have sweep: increase k_c2 coefficient
\end{frame}



\begin{frame}\frametitle{Wastewater applications}
	Perry's Chapter 22.5.8
	
	Membrane Processes These processes use a selectively permeable membrane to separate pollutants from water. Most of the membranes are formulated from complex organics that polymerize during membrane preparation. This allows the membrane to be tailored to discriminate by molecular size or by degree of hydrogen bonding potential. Ultrafiltration membranes discriminate by molecular size or weight, while reverse osmosis membranes discriminate by hydrogenbonding characteristics. The permeability of these membranes is low: from 0.38–3.8 m/day (10–100 gal/d/ft 2). The apparatus in which they are used must provide a high surface area per unit volume.

	Membrane Bioreactors (MBRs): Membrane bioreactors are a technology that combines biological degradation of waste products with membrane filtratio
\end{frame}

\begin{frame}\frametitle{References}
	References:
	* Notes from Dickson
	* Wankat, Separation Process Engineering, Chapter 16
	* Seader
	* Coulson and Richardson
	* Geankoplis
	* Ghosh
	* Seader and Henly
	* Ridgeway talk on internet
	* Membrane bioreactors: http://www.gewater.com/products/equipment/mf_uf_mbr/mbr.jsp
	
\end{frame}