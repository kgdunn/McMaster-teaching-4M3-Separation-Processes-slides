% 2013: Cover gas-gas separation: ultrafiltration section? CO2 separation from hydrocarbons; see project of Luu and Tang 2012, and there was another group also.  Panchal and Giraldini project from 2012: CO2 removal from natural gas.

% For the above: create a project topic that covers this.

% Nice bio application: Nicole Rich-Portelli and Derek Seguin 2012 project

% Investigate http://arstechnica.com/science/2013/09/microbial-battery-could-turn-wastewater-into-electricity/

% How they are made?
% What do they look like
% Application areas

% 08 October 2012

\begin{frame}\frametitle{Administrative issues}
	\begin{itemize}
		\item	Assignment 2 solutions are posted now
		\item	Project topics are posted. Due by 12 November 2013
		\begin{enumerate}
			\item	Treatment of dissolved solids in fracking wastewater
			\item	Removing $\text{CO}_2$ from a gas phase stream of mixed hydrocarbons
			\item	Removing non-valuable particulate solids (dust) from a gas-phase stream
			\item	\emph{Challenge Project}: design and operation of a device or method to create drinking-quality water in a region of hardship
			\begin{itemize}
				\item	water is not easily accessible, and is contaminated
				\item	electricity is not readily available 
				\item	consumers of the water would have little/low money to pay for your water
				\item	the device/method must not require technical sophistication to operate
			\end{itemize}
		\end{enumerate}
	\end{itemize}
\end{frame}

\begin{frame}\frametitle{Membranes}
	\begin{columns}[c]
		\column{0.50\textwidth}
			\begin{center}
				\includegraphics[width=\textwidth]{\imagedir/separations/membranes/flickr-2057883807_e05aceecb8_o.jpg}
			\end{center}
			\see{Flickr: 21182585@N07/2057883807}
		\column{0.50\textwidth}
			\begin{center}
				\includegraphics[width=\textwidth]{\imagedir/separations/membranes/flickr-ultrafiltration-3574729377_938b4ecaf7_b}
			\end{center}
			\see{Flickr: 21182585@N07/3574729377}
	\end{columns}
\end{frame}

\begin{frame}\frametitle{Quick activity}
	On a sheet of paper write 
	\begin{itemize}
		\item	bullet points \emph{and/or} 
		\item	draw a diagram \emph{and/or} 
		\item	describe 
	\end{itemize}
	\vspace{12pt}
	``what you know about membranes''
\end{frame}

\begin{frame}\frametitle{References}
	\begin{itemize}
		\item	\href{http://accessengineeringlibrary.com/browse/perrys-chemical-engineers-handbook-eighth-edition}{Perry's Chemical Engineers' Handbook}, 8th edition, chapter 20.
		\item	Wankat, ``Separation Process Engineering'', 2nd edition, chapter 16.
		\item	Schweitzer, ``Handbook of Separation Techniques for Chemical Engineers'', chapter 2.1. % <---- very well written; use this reference more in future years
		\item	Seader, Henley and Roper, ``Separation Process Principles'', 3rd edition, chapter 14.
		\item	Richardson and Harker, ``Chemical Engineering, Volume 2'', 5th edition, chapter 8.
		\item	Geankoplis, ``Transport Processes and Separation Process Principles'', 4th edition, chapter 7 (theory) and chapter 13.
		\item	Ghosh, ``Principles of Bioseparation Engineering'', chapter 11.
		\item	Uhlmann's Encyclopedia, ``Membrane Separation Processes, 1. Principles'',  \href{http://dx.doi.org/10.1002/14356007.a16_187.pub3}{DOI:10.1002/14356007.a16\_187.pub3}
		%* Membrane bioreactors: http://www.gewater.com/products/equipment/mf\_uf\_mbr/mbr.jsp
	\end{itemize}
\end{frame}

\begin{frame}\frametitle{Why use membranes?}
	Some really difficult separations:
	\begin{itemize}
		\item	finely dispersed solids; density close to liquid phase; gelatinous particles
		\item	dissolved salts must be removed
		\item	non-volatile organics (e.g. \href{http://en.wikipedia.org/wiki/Humic_acid}{humic substances})
		\item	biological materials: sensitive to the environment
		\item	biological materials: aseptic operation is required
			\begin{itemize}
				\item	cannot centrifuge
				\item	cannot sediment
			\end{itemize}
	\end{itemize}

	\vspace{12pt}
	\begin{exampleblock}{It is usually worth asking:}
		\begin{center}
			How does nature separate?
		\end{center}
		\scriptsize
		\begin{itemize}
			\item	energy efficient
			\item	effective
			\item	maybe slow?
		\end{itemize}
	\end{exampleblock}
\end{frame}

\begin{frame}\frametitle{Why use membranes?}
	Relatively new separation step {\small (``new'' meaning since 1960 to 1980s)}
	\begin{itemize}
		\item	often saves energy costs over alternative separations
		\begin{itemize}
			\item	ambient temperature operation
		\end{itemize}
		\item	often easier to operate and control
	\end{itemize}

	\begin{columns}[c]
		\column{0.40\textwidth}
			\begin{center}
				\includegraphics[width=\textwidth]{\imagedir/separations/membranes/membrane-module-Henk-Koops-talk.png}
			\end{center}
		\column{0.60\textwidth}
			Modules:
			\begin{itemize}
				\item	feed stream split into parallel units
				\item	easier to maintain and replace parts
				\item	can be expanded as needs grow
			\end{itemize}
	\end{columns}
	\see{Henk Koops' slides, GE Water and Process Technologies}
\end{frame}

\begin{frame}\frametitle{Challenges in membrane design}
	Challenges that still remain:
	\begin{itemize}
		\item	withstanding high pressure differences but still have a thin membrane
		\item	dealing with fouling and cleaning
		\item	increasing selectivity (separation factor) for specific application areas
		\item	uniformity of pore sizes
		\item	temperature stability (e.g. steam sterilization)
	\end{itemize}
\end{frame}

\begin{frame}\frametitle{Market size}
	\begin{center}
		\includegraphics[width=\textwidth]{\imagedir/separations/membranes/membrane-market-size-Perrys.png}
	\end{center}
	\see{Perry's: chapter 20, 8ed}
\end{frame}

\begin{frame}\frametitle{Let's formalize some terminology}

	%\todo{Mindmap of terms}
	\begin{center}
		\includegraphics[width=\textwidth]{\imagedir/separations/membranes/some-terminology-Seader-based.png}
	\end{center}
	\see{Seader, Henley and Roper, p 501}
\end{frame}

\begin{frame}\frametitle{More terminology}

	{\color{purple}{semipermeable}}: partially permeable, e.g. your skin allows certain size particles in, but not others

	\vspace{12pt}
	{\color{purple}{mass separating agent}}: the membrane itself

	\vspace{12pt}
	{\color{purple}{energy separating agent}}: the applied pressure (pressure drop)

	\vspace{12pt}
	{\color{purple}{porosity}} = $\displaystyle \frac{\text{area of open pores}}{\text{total surface area}}$
		%\item	permeate: the material passing through the membrane from feed to outlet side
		%\item	retentate: the material retained on the feed-side of the membrane
		%\item	solute: most often retained on the inside (feed side) of the membrane and deposited on the membrane wall
		%\item	solvent: the liquid phase that carries the solute
		%\item	gel effect / buildup: the solute will build up on the membrane wall and form a gel
		%\item	kilodalton
		%\item	MWCO: C+Rv2ed5, p 440 and 441: A parameter often quoted in manufacturer’s literature is the nominal molecular weight cut-off (MWCO) of a membrane. This is based on studies of how solute molecules are rejected by membranes. A solute will pass through a membrane if it is sufficiently small to pass through a pore, if it does not significantly interact with the membrane and if it does not interact with other, larger solutes. It is possible to define a solute rejection coefficient R by: R = 1 − (C_p/C_f ) where C_f is the concentration of solute in the feed stream and C_p is the concentration of solute in the permeate. See fig 8.3 in C+Rv2ed5
		%\item	Pervaporation
		%\item	Reverse osmosis
\end{frame}

\begin{frame}\frametitle{What is flux?}
	\begin{exampleblock}{}
		\centering
		\small The (volumetric) or (molar) or (mass) flow per unit time for 1 unit of area
	\end{exampleblock}
	\vspace{12pt}
	\begin{columns}[t]
		\column{0.80\textwidth}
			\begin{itemize}
				\item	$J$ = flux = $\displaystyle \frac{\text{transfer rate}}{\text{transfer area}}$
				\item	e.g. 42 $\text{mol.s}^{-1}\text{.m}^{-2}$
				\item	never simplify the units: write 13 $\left(\text{m}^3.\text{s}^{-1}\right)\text{.m}^{-2}$
				\item	you may, and probably should, omit the brackets: 13 $\text{m}^3.\text{s}^{-1}\text{.m}^{-2}$
				\item	{\color{myRed}{do not write}} 13 $\text{m}\text{.s}^{-1}$
			\end{itemize}
		\column{0.35\textwidth}
			\begin{center}
				\includegraphics[width=\textwidth]{\imagedir/separations/membranes/flux-concept.png}
			\end{center}
	\end{columns}
	\begin{exampleblock}{General principle}
		For a given unit area, we want the highest flux possible (at the lowest possible cost)
	\end{exampleblock}
\end{frame}

\begin{frame}\frametitle{Membrane classification}

	\begin{center}
		\includegraphics[width=\textwidth]{\imagedir/separations/membranes/membrane-application-areas-CRv2-5ed-p438.png}
	\end{center}

	\see{Richardson and Harker, p 438}
\end{frame}

\begin{frame}\frametitle{Transport through a membrane}
	\begin{exampleblock}{Why study theoretical models?}
		All forms of membrane applications rely to some extent on the same equation \textbf{structure}. The details will change.
	\end{exampleblock}

	\vspace{12pt}

	Will allow us to:
	\begin{itemize}
		\item	troubleshoot problems with the process
		\item	predict expected impact of improvements/changes to the process
		\item	used for crudely sizing the unit (order of magnitude estimates)
	\end{itemize}

\end{frame}

\begin{frame}\frametitle{Examples you will be able to solve}
	\begin{enumerate}
		\item	how long should we operate unit at constant \( \Delta P \) to achieve desired separation?
		\item	what is the mass transfer coefficient through the lab membrane?
		\item	what pressure drop (and therefore pump size) do I expect?
		\item	how many cassettes (area) does this application require?
	\end{enumerate}
\end{frame}

\begin{frame}\frametitle{The general equation}
	\begin{exampleblock}{}
		\[
			\displaystyle \frac{\text{transfer rate}}{\text{transfer area}}  = \text{flux} = \displaystyle \frac{(\text{permeability})(\text{driving force})}{\text{thickness}} = \displaystyle \frac{\text{driving force}}{\text{resistance}}
		\]
	\end{exampleblock}


	\vfill
	{\color{myOrange}{Symbolically:}}
	\[
		\displaystyle \rho_f\frac{Q_p}{A} = \displaystyle\frac{\rho_f}{A}\cdot\frac{dV}{dt} =  J = \displaystyle \frac{(\text{permeability})(\text{driving force})}{L} = \displaystyle \frac{\text{driving force}}{R}
	\]
	\begin{itemize}
		\item	 $\text{\color{purple}{permeance}} = \displaystyle \frac{\text{\color{purple}{permeability}}}{L} = \displaystyle \frac{1}{\text{\color{purple}{resistance}}} = \displaystyle \frac{1}{R} = \text{{\color{purple}{\tiny ``mass transfer coeff''}}}$
		\item	permeance: easier to measure
		\item	permeance units: depend on choice of (driving force) and $J$
		\item	resistance = $f(\text{thickness, viscosity, porosity, pore size})$
		\item	we will specifically define resistance in each case
	\end{itemize}

\end{frame}

\begin{frame}\frametitle{Microfiltration}
	\begin{columns}[t]
		\column{0.65\textwidth}
			\vspace{-20pt}
			\begin{itemize}
				\item	0.1 \micron ~to 10 \micron ~retained mainly by sieving mechanism
				\item	conventional filters: not effective below $\sim 5$~\micron
				\item	microfiltration membranes: generally symmetric pores
				\item	\href{http://en.wikipedia.org/wiki/Polysulfone}{polysulfone} membrane
				\item	(surface) porosity as high as $\epsilon = 0.8$
				\item	driving force = ${\color{red}\Delta P}$: 100 to 500 kPa % C+Rv2;5ed;p443
				\item	high fluxes at low {\color{purple}{TMP}} (trans-membrane pressure)
				\item	application areas:
					\begin{itemize}
						\item	yeast cells harvesting
						\item	wine/beer/juice clarification
						\item	bacteria and virus removal
						\item	air filtration
						\item	cytology: concentrate up cells
					\end{itemize}
			\end{itemize}
			% no more text allowed
		\column{0.40\textwidth}
			\begin{center}
				\includegraphics[width=0.95\textwidth]{\imagedir/separations/membranes/symmetric-pore-structure-Henk-Koops-talk.png}
			\end{center}
			\see{Henk Koops' slides}
	\end{columns}
\end{frame}

\begin{frame}\frametitle{General modelling equation applied}
	\vspace{-12pt}
	\begin{columns}[c]
		\column{0.60\textwidth}
			\[
				\begin{array}{rl}
					\displaystyle\frac{\rho_f}{A}\cdot\frac{dV}{dt} = \text{Flux} &= J = \dfrac{{\color{red}\Delta P}}{\mu \left(R_m \ell_M + R_c L_c \right)} \\
					\\
					\displaystyle\frac{\rho_f}{A}\cdot\frac{dV}{dt} = \text{Flux} &= J = \dfrac{{\color{red}\Delta P}}{\mu \left(R'_m + R'_c \right)}
				\end{array}
			\]
		\column{0.40\textwidth}
			\vspace{-12pt}
			\begin{center}
				\includegraphics[width=\textwidth]{\imagedir/separations/membranes/solute-build-up-CRv2-5ed-p446.png}
			\end{center}
	\end{columns}
	
	\vspace{-24pt}
	\begin{tabular}{cll}
		$J$			&	[$\text{kg}.\text{s}^{-1}\text{.m}^{-2}$] 				& permeate flux \\
		$\mu$ 		&  	$[\text{kg}.\text{m}^{-1}\text{.s}^{-1}]$				& permeate viscosity\\
		${\color{red}\Delta P}$	&	$[\text{Pa}] = [\text{kg}.\text{m}^{-1}\text{.s}^{-2}]$ & TMP varies for different applications\\
		$R_m$ 		&  	$[\text{m}.\text{kg}^{-1}]$								& resistance through membrane (small)\\
		$R_c$ 		&  	$[\text{m}.\text{kg}^{-1}]$								& resistance through cake (large)\\
		$\ell_m$ 	&   $[\text{m}]$											& membrane thickness\\
		$L_c$ 		&   $[\text{m}]$											& effective cake thickness\\
		$\rho_f$ 	& 	$[\text{kg.m}^{-3}]$ 									& fluid density
	\end{tabular}
	\vspace{12pt}
	\see{Illustration from Richardson and Harker, Ch8}
\end{frame}

\begin{frame}\frametitle{Flow patterns for microfiltration}
	\begin{columns}[t]
		\column{0.40\textwidth}
			{\color{myBlue}{Dead-end flow}}
			\begin{center}
				\includegraphics[width=.7\textwidth]{\imagedir/separations/membranes/solute-build-up-CRv2-5ed-p446.png}
			\end{center}
			\begin{itemize}
				\item	only for very low concentration feeds
				\item	else becomes rapidly clogged
				\item	air filtration and virus removal applications
			\end{itemize}
			\begin{exampleblock}{}
					\[
						J = \displaystyle \frac{\Delta P}{\mu \left(R'_m + R_c L_c\right)}
					\]
			\end{exampleblock}

		\column{0.60\textwidth}
			{\color{myBlue}{Cross-flow (TFF)}}
			\begin{center}
				\includegraphics[width=1.1\textwidth]{\imagedir/separations/membranes/cross-flow-wikipedia.png}
			\end{center}
			\begin{itemize}
				\item	{\color{purple}{TFF}} = tangential flow filtration
				\item	main purpose?
				\pause
					\begin{itemize}
						\item	microfiltration: tends to have cake build up
						\item	cross-flow induces shearing to erode cake
						\item	reduces cake resistance, $R'_c$
						\item	${\color{red}\Delta P} = \displaystyle \frac{P_\text{in} + P_\text{out}}{2} - P_\text{P}$
					\end{itemize}
			\end{itemize}
	\end{columns}
\end{frame}

\begin{frame}\frametitle{Dead-end flow vs cross-flow geometries}
	\begin{columns}[t]
		\column{0.40\textwidth}
			{\color{myBlue}{Dead-end flow}}
			\begin{itemize}
				\item	cake thickness increases with time: $L_c(t)$
				\item	implies cake resistance changes with time: $R'_c(t)$
				\item	so for a constant $\Delta P$, implies $J(t)$ falls off
			\end{itemize}
			\vspace{12pt}
			\begin{exampleblock}{}
					\[
						J = \displaystyle \frac{\Delta P}{\mu \left(R'_m + R_c L_c\right)}
					\]
			\end{exampleblock}
		\column{0.60\textwidth}
			{\color{myBlue}{Cross-flow (TFF)}}
			\begin{center}
				\includegraphics[width=\textwidth]{\imagedir/separations/membranes/cross-flow-wikipedia.png}
			\end{center}
			\vspace{-12pt}
			\begin{itemize}
				\item	fluid velocity: 1 to 8 m.s$^{-1}$ tangentially
				\item	keeps mass transfer resistance low
				\item	for a given $\Delta P$: TFF allows us to obtain higher fluxes than dead-end (usually $\Delta P$ is 100 to 500 kPa)
				\item	cannot take lab test results with a filter cloth dead-end and apply it to cross-flow situation
			\end{itemize}
	\end{columns}
\end{frame}

\begin{frame}\frametitle{Cross-flow flowsheet}
	\begin{center}
		\includegraphics[width=\textwidth]{\imagedir/separations/membranes/flowsheet-for-microfiltration-CRv2-5ed-p444.png}
	\end{center}

	\vspace{-12pt}
	How to pressurize the unit?
	\begin{enumerate}
		\item	\small Supply feed at pressure; valve at retentate to adjust/control $\Delta P$
		\item	Draw a vacuum at permeate and pull material through membrane
	\end{enumerate}
	% Also see Ghosh: p 212
	\see{Illustration from Richardson and Harker, Ch8}
\end{frame}

\begin{frame}\frametitle{Dealing with fouling}
	\begin{center}
		\includegraphics[width=0.85\textwidth]{\imagedir/separations/membranes/cross-flow-patters-CRv2-5ed-p444.png}
	\end{center}
	\vspace{-12pt}
	\see{Richardson and Harker, Ch8}
\end{frame}

\begin{comment}

\begin{frame}\frametitle{A preliminary design}
	\begin{exampleblock}{Main aim}
		Determine the \textbf{size} of a membrane for a required \textbf{flow rate} of permeate.
	\end{exampleblock}
	We have a reasonable budget to purchase equipment, and membrane samples from suppliers.

	\vspace{12pt}
	How would you set up your lab experiment(s) to get the information required?
	\pause
	\begin{itemize}
		\item	\iftoggle{instructor}{$J = \displaystyle \rho_f	 \frac{Q_p}{A} = \displaystyle \frac{\Delta P}{\mu \left(R_m \ell_m + R_c L_c\right)} = \frac{\Delta P}{\mu \left(R'_m + R'_c\right)}$}{}

		\vspace{6pt}
		\item	\iftoggle{instructor}{$R'_m$: estimated using pure solvent through membrane at $\Delta P$}{}

		\vspace{6pt}
		\item	\iftoggle{instructor}{$R'_c = R_c L_c$: obtained from a plot of $J_i$ vs $\Delta P_i$}{}
			\iftoggle{instructor}{\begin{itemize}
				\item	set different $\Delta P_i$; then measure corresponding $J_i$ once steady
				\item	find $J_i$ (interpolate) that gives required $Q_p$ by varying $A$
			\end{itemize}}{}
	\end{itemize}
\end{frame}

\begin{frame}\frametitle{Factors to improve flux}
	\begin{itemize}
		\item	increase pressure difference
		\item	regular backflush
		\item	choose alternative membrane structure
		\item	feed concentration kept low
		\item	shear rate (velocity in cross-flow): reduces $R'_c = R_c L_c$
		\item	increase temperature of feed
		\item	nature of the solids deposited: affects resistance $R_c$
	\end{itemize}
\end{frame}

% 05 October 2012: course evaluation; in-class pop-quiz and the example to end microfiltration
\begin{frame}\frametitle{Pop-quiz question}
	A microfiltration membrane operating with pure feed of water produces a flux of 0.06 $\text{kg}.\text{s}^{-1}\text{.m}^{-2}$ when operated with a TMP of 30 kPa.

	\begin{enumerate}
		\item	What is the resistance due to the membrane? Specify the units.
		\item	If operated with a protein-water mixture at a 20 kPa pressure difference, a flux of $216 \times 10^{-6}$ $\text{kg}.\text{s}^{-1}\text{.m}^{-2}$ is measured at steady state. What is the resistance due to cake build-up? Specify the units.
	\end{enumerate}
\end{frame}

\begin{frame}\frametitle{Estimating the cake resistance, $R_c$}
	\begin{itemize}
		\item	$R'_{c,v} = R_{c,v} L_c = R_{c,v} \displaystyle\frac{V_\text{cake}}{A_\text{membrane}}$
		\item	$R_{c,v} = R_{c} \cdot \rho_f$ and similarly $R_{m,v} = R_{m} \cdot \rho_f$
		\item	\textbf{Important note}: $R'_{c,v}$ emphasizes that this is a resistance only when $J_v = \displaystyle\frac{J}{\rho_f}$, which has units $\left[\left(\text{m}^3.\text{s}^{-1}\right)\text{.m}^{-2}\right]$
		\item	Carman relationship: $R_{c,v} = 180 \left( \displaystyle \frac{1-e}{e^3} \right)\left(\displaystyle \frac{1}{D_p^2} \right)$
		\item	$\epsilon$ = porosity of the cake
		\item	$D_p$ = \href{http://en.wikipedia.org/wiki/Sauter_mean_diameter}{Sauter mean particle diameter} [m]
		\item	$L_c$ = estimated cake thickness [m]
		\item	$R'_{c,v}$ has units of $[\text{m}^{-1}]$
		\item	$R_{c,v}$ has units of $[\text{m}^{-2}]$
	\end{itemize}
\end{frame}

\begin{frame}\frametitle{Microfiltration example}
	The previous lab experiment to determine mass-transfer resistance is preferred. But we can estimate it.
	\vspace{12pt}
	\begin{exampleblock}{Water microfiltration}
		\begin{itemize}
			\item	Constant $\Delta P = 50\text{kPa}$ applied in cross-flow membrane set up
			\item	Membrane area = 50cm$^2$ = 0.005m$^2$
			\item	Pure water at this $\Delta P$ produced a flux of 1.0 $\text{kg}.\text{s}^{-1}\text{.m}^{-2}$
			\item	Feed at this same TMP produced a flux of 0.065 $\text{kg}.\text{s}^{-1}\text{.m}^{-2}$ permeate
			%\item	Cake porosity estimated at $\epsilon = 0.55$ for particles of $D_p$ = 2\micron
			\item	What is the estimated thickness of the cake build-up if the average particle size diameter is 2\micron?
		\end{itemize}
	\end{exampleblock}
	\vfill
	Practical use of this example?
\end{frame}

% 09 October 2012: UF and nano-filtration
% Delayed: covered assignment 3 solutions and course evaluation feedback in the class
\begin{frame}\frametitle{Ultrafiltration (UF)}
	\begin{columns}[c]
		\column{0.65\textwidth}
			\vspace{-20pt}
			\begin{itemize}
				\item	5 nm to 100 nm (0.1 \micron) particles are retained
				\item	1 to 1000 kDa particles are retained (move to using molecular weight)
				\begin{itemize}
					\item	1 dalton = 1 atomic mass unit
					\item	1 kilodalton = 1000 dalton = 1000~g/mol
					\item	particles with lower molecular weight, e.g. most solvents, pass through
				\end{itemize}
				\item	pore sizes: 1 to 20nm
				\item	typical fluxes: $J_v = 0.01 \text{~to~} 0.5 \text{~m}^3\text{.m}^{-2}.\text{hr}^{-1}$\\$J_v = ~~\,10 \text{~to~} \,50 \text{~L}\text{.m}^{-2}.\text{hr}^{-1}$ ({\color{purple}{LMH}})
				\item	asymmetric structure
				\item	almost always operated in {\color{purple}{TFF}}
			\end{itemize}
			% no more text allowed
		\column{0.40\textwidth}
			\begin{center}
				\includegraphics[width=\textwidth]{\imagedir/separations/membranes/asymmetrical-membrane-Perrys-fig-20-62.png}
			\end{center}
			\vspace{-12pt}
			\see{Perry's 8ed; Ch20.4}
	\end{columns}
\end{frame}

\begin{frame}\frametitle{Ultrafiltration applications}
	UF: loosely considered: {\color{myGreen}{``cross-flow filtration at molecular level''}}
	\begin{itemize}
		\item	Recovery of proteins and high molecular weight materials ({\color{purple}{solute}})
		\item	Permanent emulsions: e.g. oil phase will not pass
		\item	Fine colloidal particles: e.g. paint/dyes
		\item	Large molecules of interest might remain in retentate; permeate discarded
		\item	e.g. albumin (egg white) concentration
		\item	e.g whey processing: %Green book chapter
		\begin{itemize}
			\item	UF first, followed by reverse osmosis (RO)
			\item	valuable proteins retained by UF
			\item	permeate sent to RO to concentrate smaller molecule sugars and salts
			\item	this concentrated permeate: used for ethanol and lactic acid production
		\end{itemize}
	\end{itemize}
\end{frame}

\begin{frame}\frametitle{Ultrafiltration (UF)}

	\begin{itemize}
		\item	driving force = $\Delta P$ of 0.1 to 1.0 MPa % Perry and C+Rv2
		\item	``tight'', low-permeability side faces the TFF to retain particles
		\item	this skin layer is about 10\micron~ thick; provides selectivity
		\item	open, high-permeability side mainly for mechanical strength % Perrys
	\end{itemize}

	%\todo{Sieving curve terminology}
	\begin{columns}[t]
		\column{0.70\textwidth}
			\vspace{-12pt}
			\begin{center}
				\includegraphics[width=\textwidth]{\imagedir/separations/membranes/rejection-coefficient-C+Rv2-5ed-p441.png}
			\end{center}
			\vspace{-12pt}
			\see{Richardson and Harker, Ch8}
		\column{0.30\textwidth}

			$R = 1 -\displaystyle \frac{C_\text{permeate}}{C_\text{feed}}$

			$R = 1- \displaystyle \frac{C_p}{C_f} = 1 -S$

			\vspace{12pt}
			{\color{purple}{MWCO}}: molecular weight where $R=0.9$

			\vspace{12pt}
			i.e. 10\% of that molecular weight passes through to the permeate
	\end{columns}
\end{frame}

\begin{frame}\frametitle{Transport phenomena in UF}
	\begin{itemize}
		\item	solute (i.e. particles) carried towards membrane by solvent
		\item	$J = \displaystyle\frac{\Delta P}{R_m + R_{cp}}$
		\item	$R_m$ = membrane resistance [$\text{m.s}^{-1}$ if $J$ is mass flux]
		\item	$R_{cp}$ = resistance due to ``concentration polarization''
		\item	$R_{cp}$ effectively is the resistance due to solute boundary layer
		\item	Mass concentration $C_f$ (in retentate), steadily increasing to $C_w$ (wall)
		\item	Units of $C$ are kg solute per m$^3$ solvent
	\end{itemize}
	\begin{center}
		\includegraphics[width=\textwidth]{\imagedir/separations/membranes/solute-build-up.png}
	\end{center}
	% Perry: Component Transport Transport through membranes can be considered as mass transfer in series: (1) transport through a polarization layer above the membrane that may include static or dynamic cake layers, (2) partitioning between the upstream polarization layer and membrane phases at the membrane surface, (3) transport through the membrane, and (4) partitioning between the membrane and downstream fluid.
\end{frame}

\begin{frame}\frametitle{Transport phenomena in UF}
	\begin{itemize}
		\item	Experimental evidence agrees well with theory ... to a point.
		\item	Increasing $\Delta P$ leads to compacting this layer, increasing $C_w$
		\item	So diminishing returns from increasing $\Delta P$
		\item	Also, there is a strong concentration gradient
		\item	Diffusion \emph{away} from membrane due to concentration gradients
		\item	Eventually solute forms a colloidal gel on the membrane, $C_g$
		\item	Adjusting pressure has little/no effect anymore
	\end{itemize}
	\begin{center}
		\includegraphics[width=.6\textwidth]{\imagedir/separations/membranes/UF-flux-pressure-velocity-Green-Chemical-Engineering-book.png}
	\end{center}
	\vspace{-12pt}
	\see{Chemical Engineering Magazine, 8 May 1978}
\end{frame}

% 11 October 2012: UF and nano-filtration
\begin{frame}\frametitle{Transport phenomena in UF}
	\begin{itemize}
		\item	Solute flux towards membrane: $\displaystyle \frac{J\cdot C}{\rho_f} = J_v C$
		\\
		\item	Solute flux out of membrane: $J_v C_\text{permeate} \approx 0$  {\tiny if membrane retains solute}
	\end{itemize}
	\begin{exampleblock}{}
		Net transport of solute = $J(C - C_p)$
	\end{exampleblock}

	\begin{tabular}{cll}
		$J_v$		&	$\left[\displaystyle\frac{\text{m}^3 \text{~solvent}}{\text{m}^2 \text{.s}}\right]$	& permeate volumetric flux \\
		$C$ 		&  	$\left[\displaystyle\frac{\text{kg} \text{~solute}}{\text{m}^3 \text{~solvent}}\right]$	& solute mass concentration in bulk\\
		$C_p \approx 0$ &  	$\left[\displaystyle\frac{\text{kg} \text{~solute}}{\text{m}^3 \text{~solvent}}\right]$	& solute mass concentration in permeate\\
	\end{tabular}
	\vfill
	{\tiny Space for picture}
\end{frame}

\begin{frame}\frametitle{Diffusion term}
	\begin{itemize}
		\item	Solute {\color{purple}{diffusion}} away from membrane
		\begin{exampleblock}{}
			\[
				J_\text{v,diffusion} = -D_{AB} \displaystyle \frac{1}{\rho_f}\frac{dC}{dy}
			\]
		\end{exampleblock}
	\end{itemize}

	\begin{tabular}{cll}
		$D_{AB}$				&	$\left[\displaystyle\frac{\text{m}^3 \text{~solvent}}{\text{m.s}}\right] = \left[\text{m}^2\text{.s}^{-1}\right]$	& diffusion of solute in solvent \\
		$J_\text{v,diffusion}$ 	&  	$\left[\displaystyle\frac{\text{m}^3 \text{~solvent}}{\text{m}^2 \text{.s}}\right]$	& solvent volumetric flux\\
	\end{tabular}
	\begin{center}
		\includegraphics[width=0.9\textwidth]{\imagedir/separations/membranes/solute-build-up.png}
	\end{center}
	\vspace{-6pt}
	\begin{itemize}
		\item	See \href{http://en.wikipedia.org/wiki/File:DiffusionMicroMacro.gif}{animation on Wikipedia}
	\end{itemize}
\end{frame}

\begin{frame}\frametitle{Transport at steady state}
	At steady state: diffusion back equals transfer through membrane
	\[
		\begin{array}{rl}
		 	\displaystyle  \frac{J(C - C_p)}{\rho_f}  &= -D_{AB} \displaystyle \frac{1}{\rho_f}\frac{dC}{dy} \\
			\displaystyle -\frac{J}{D_{AB}} \int_{0}^{L_c}{dy} &= \displaystyle \int_{C_w}^{C_f}{\frac{dC}{C-C_p}} \\ \\
			\displaystyle\ln\left(\frac{C_w - C_p}{C_f - C_p}\right) &= \displaystyle\frac{J L_c}{D_{AB}} = \displaystyle \frac{J}{h_w}
		\end{array}
	\]
	\vspace{6pt}
	\begin{center}
		\includegraphics[width=\textwidth]{\imagedir/separations/membranes/solute-build-up.png}
	\end{center}
	% Perry: Component Transport Transport through membranes can be considered as mass transfer in series: (1) transport through a polarization layer above the membrane that may include static or dynamic cake layers, (2) partitioning between the upstream polarization layer and membrane phases at the membrane surface, (3) transport through the membrane, and (4) partitioning between the membrane and downstream fluid.
\end{frame}

\begin{frame}\frametitle{UF: mass-transfer {\color{myOrange}{key points}}}
	Assuming $C_p \approx 0$
	\begin{exampleblock}{}
		\[
			\displaystyle\frac{J L_c}{D_{AB}} =  \displaystyle \frac{J}{h_w} = \displaystyle \ln\left(\frac{C_w }{C_f }\right)
		\]
	\end{exampleblock}
	where $h_w$ is a  mass-transfer coefficient, with units of $\text{m.s}^{-1}$
	\begin{itemize}
		\item	there are correlations for $h_w= f(\text{velocity, temperature, channel diameter, viscosity})$
		\item	when gelling occurs, $C_w = C_g$ at the wall
		\item	the effect of increasing $\Delta P$ is
		\begin{itemize}
			\item	increase in solute flux towards boundary layer
			\item	diffusion increases to oppose it
			\item	net effect: almost zero (see earlier plot)
			\item	experiments mostly agree with this theory
		\end{itemize}
		\item	there is a limiting flux $J_\text{lim} = f(C_w, C_f, h_w)$
		\item	at higher feed concentrations, lower fluxes if we are at/near the gel polarization state (gelling)
		\item	typical diffusivities: $1 \times 10^{-9}$ (fast!) to $1 \times 10^{-11} \text{~m}^2\text{.s}^{-1}$
	\end{itemize}
\end{frame}

\begin{frame}\frametitle{Example question}
	An ultrafiltration application is required to treat a waste stream that has 0.5~$\text{kg}\text{.m}^{-3}$ waste in the feed. The desired solute concentrate must be 20~$\text{kg}^3\text{.m}^{-3}$.

	\vspace{12pt}
	Pilot plant studies show the flux can be expressed as
	\[
		J = 0.02 \ln \left(\frac{25}{C_f} \right)
	\]
	in units of $\text{m}^3.\text{hour}^{-1}.\text{m}^{-2}$. Due to fouling the flux from this membrane system never exceeds 0.05 $\text{m}^3.\text{hour}^{-1}.\text{m}^{-2}$.

	\vspace{12pt}
	What is the limiting final concentration, $C_f$? What is the interpretation of it?
\end{frame}

\begin{frame}\frametitle{Geometries for ultrafiltration (recap)}
	\begin{columns}[t]
		\column{0.60\textwidth}
			\textbf{Tubes in a shell}
				\begin{itemize}
					\item	membrane on a porous support
					\item	cleaned with soft sponge balls
				\end{itemize}
			\begin{center}
				\includegraphics[width=\textwidth]{\imagedir/separations/membranes/geometry-5-Wankat-2ed-p540.png}
			\end{center}
		\column{0.40\textwidth}
			\textbf{Plate and frame}
				\begin{itemize}
					\item	batch operation
				\end{itemize}
				\begin{center}
					\includegraphics[width=\textwidth]{\imagedir/separations/membranes/flat-sheet-module-CRv2-5ed-p456.png}
				\end{center}
	\end{columns}
	\begin{itemize}
		\item	All these units bought as complete module from supplier
		\item	In fixed sizes; so need to be combined (next section)
		\item	Also as cassettes, tubes and flat sheets run in TFF to increase flux. % Perry:
	\end{itemize}
	\see{Illustrations from Wankat, 2ed, Ch 16}
	%figure for future use: /Users/kevindunn/Sync/Figures/separations/membranes/application-area-geometry-Perrys-8ed-Table-20-25.png
\end{frame}

\begin{frame}\frametitle{Geometries for ultrafiltration (recap)}
	\begin{columns}[t]
		\column{0.60\textwidth}
			\textbf{Spiral wound}

				\begin{itemize}
					\item	\iftoggle{instructor}{high surface area per unit volume}{}
					\item	\iftoggle{instructor}{high turbulence, reducing mass transfer resistance}{}
				\end{itemize}
			\vfill
			\begin{center}
				\includegraphics[width=\textwidth]{\imagedir/separations/membranes/geometry-4-Wankat-2ed-p540.png}
			\end{center}
			\see{Illustrations from Wankat, 2ed, Ch 16}
		\column{0.50\textwidth}
			\textbf{Hollow fibre membranes}
			\begin{itemize}
				\item	largest area to volume ratio
				\item	fibre inside diameter = 500 to 1100 \micron~ for UF
				\item	UF: feed inside tube, with thin membrane skin on the inside
			\end{itemize}
			\begin{center}
				\includegraphics[width=.8\textwidth]{\imagedir/separations/membranes/geometry-6-Wankat-2ed-p540.png}
			\end{center}
	\end{columns}
\end{frame}

\begin{frame}\frametitle{}
	\begin{center}
		\includegraphics[width=\textwidth]{\imagedir/separations/membranes/hollow-fibres-CRv2-5ed-p457.png}
	\end{center}
	\see{Richardson and Harker, Ch8}
\end{frame}

\begin{frame}\frametitle{Sequencing membrane modules}
	\begin{columns}[t]
		\column{0.50\textwidth}
			\textbf{Parallel}
			\begin{itemize}
				\item	most common configuration
				\item	allows increase in throughput
			\end{itemize}
		\column{0.50\textwidth}
			\textbf{Series}
			\begin{itemize}
				\item	used to achieve a desired separation factor (concentration)
				\item	high pressure drop across series circuit
				\item	cannot recover pressure (energy separating agent)
			\end{itemize}
	\end{columns}
	\begin{center}
		\includegraphics[width=\textwidth]{\imagedir/separations/membranes/geometry-1-Wankat-2ed-p540.png}
	\end{center}
	\see{Wankat, Ch16}
\end{frame}

\begin{frame}\frametitle{Example of an installation}
	\includegraphics[width=\textwidth]{\imagedir/separations/membranes/ide-tech.com-1D13_0-low-res.jpg}
	\begin{itemize}
		\item	Larnaca, Cyprus
		\item	SWRO membrane, i.e. desalination
		\item	21.5 million $\text{m}^3$ per year
		\item	parallel and series
	\end{itemize}
	\see{ide-tech.com}
\end{frame}

\begin{frame}\frametitle{Operating UF units}
	% Perry:
	\begin{itemize}
		\item	Continuous operation provides lower-cost operation
		\item	Batch operation: seldom used, except for start up (see next)
		\item	Biologicals: require batch processing to meet regulatory requirements
		\item	High solids in feed? Require multiple-pass: simply recycle
	\end{itemize}

	%Applications with high solids require multiple-pass operation to obtain a significant conversion ratio, but water treatment can achieve high conversion in one pass.
\end{frame}

\begin{frame}\frametitle{Recycle operation: ``feed plus bleed''}
	\begin{center}
		\includegraphics[width=\textwidth]{\imagedir/separations/membranes/recycle-flowsheet.png}
	\end{center}
	\see{Modified from Richardson and Harker, Ch8}
	\begin{itemize}
		\item	Initially close retentate valve (batch mode operation)
		\item	Fluxes slowly reduce
		\item	Open retentate valve and operate at steady state
	\end{itemize}
\end{frame}

% 16 October 2012
\begin{frame}\frametitle{Class example}
	We need to treat 50 $\text{m}^3\text{.day}^{-1}$ of waste containing a solute at 0.5~$\text{kg}\text{.m}^{-3}$. The desired solute concentrate must be 20~$\text{kg}\text{.m}^{-3}$. The plant operates 20 hours per day.

	\vspace{12pt}
	At steady-state the feed-plus-bleed circuit operates with a flux
	\[
		J = 0.02 \ln \left(\frac{25}{C_f} \right)
	\]
	in units of $\text{m}^3.\text{hour}^{-1}.\text{m}^{-2}$.

	\vspace{12pt}
	If each membrane module is 30 m$^2$:
	\begin{itemize}
		\item	how many membrane modules are required?
		\item	series or parallel?
	\end{itemize}
\end{frame}

\begin{frame}\frametitle{Covered in class on 11 October}
	\begin{center}
		\includegraphics[width=\textwidth]{\imagedir/separations/membranes/White-board-11-Oct-2012}
	\end{center}
\end{frame}

% 2013: check if you can say \frac{Q_0}{Q_R1} = \frac{Q_R1}{Q_R2}  (i.e the ratio of the feed to retentate in unit 1 is the same as the corresponding ratio from unit 2)

% 16 October 2012
\begin{frame}\frametitle{Class example (11 Oct)}
	We need to treat 50 $\text{m}^3\text{.day}^{-1}$ of waste containing a solute at 0.5~$\text{kg}\text{.m}^{-3}$. The desired solute concentrate must be 20~$\text{kg}\text{.m}^{-3}$. The plant operates 20 hours per day.

	\vspace{12pt}
	At steady-state the feed-plus-bleed circuit operates with a flux
	\[
		J = 0.02 \ln \left(\frac{25}{C_f} \right)
	\]
	in units of $\text{m}^3.\text{hour}^{-1}.\text{m}^{-2}$.

	\vspace{12pt}
	If each membrane module is 30 m$^2$:
	\begin{itemize}
		\item	how many membrane modules are required?
		\item	series or parallel?
	\end{itemize}
\end{frame}

\begin{frame}\frametitle{Multiple units in series (will be in Assignment 4)}
	\begin{center}
		\includegraphics[width=\textwidth]{\imagedir/separations/membranes/feed-and-bleed-in-series.png}
	\end{center}

	Now consider the previous example. Find the optimal areas, $A_1$ and $A_2$ for the membranes.
\end{frame}

\begin{frame}\frametitle{Reverse osmosis}
	\begin{itemize}
		\item	The \textbf{most requested} topic on your list of learning objectives
		\item	Membrane market by \$ size
			\begin{enumerate}
				\item	Dialysis
				\item	Reverse osmosis (water treatment)
			\end{enumerate}
		\item	What is osmosis? [Greek = ``push'']
		\item	Then we look at reverse osmosis (RO)
		\item	Applications of RO
		\item	Modelling RO
	\end{itemize}
\end{frame}

\begin{frame}\frametitle{Reverse osmosis principle}
	\vfill
	\begin{center}
		\includegraphics[width=\textwidth]{\imagedir/separations/membranes/reverse-osmosis-initial.png}
	\end{center}
\end{frame}

\begin{frame}\frametitle{Reverse osmosis principle}
	\vfill
	\begin{center}
		\includegraphics[width=\textwidth]{\imagedir/separations/membranes/reverse-osmosis-equilibrium.png}
	\end{center}
\end{frame}

\begin{frame}\frametitle{Reverse osmosis principle}
	\vfill
	\begin{center}
		\includegraphics[width=\textwidth]{\imagedir/separations/membranes/reverse-osmosis-developed.png}
	\end{center}
\end{frame}

\begin{frame}\frametitle{(Reverse) Osmosis principle}
	\begin{itemize}
		\item	Assume solute barely passes through membrane ($C_p \approx 0$)
		\item	Solvent passes freely
		\item	Chemical potential drives pure solvent (water) to dilute the solute/solvent (mixture).
		\item	This \emph{solvent flux} continues until equilibrium is reached
			\begin{itemize}
				\item	solvent flow to the left equals solvent flow to the right
				\item	results in a pressure difference (head)
				\item	called the \emph{osmotic pressure} = $\pi$ [Pa]
				\item	a thermodynamic property $\neq f(\text{membrane})$
				\item	a thermodynamic property $= f(\text{fluid and solute properties})$
			\end{itemize}
	\end{itemize}
\end{frame}

\begin{frame}\frametitle{(Reverse) Osmosis principle}
	\begin{columns}[t]
		\column{0.60\textwidth}
			\begin{itemize}
				\item	Osmosis in action:
				\begin{itemize}
					\item	trees and plants to bring water to the cells in upper branches
					\item	killing snails by placing salt on them
					\item	why freshwater fish die in salt water and \emph{vice versa}
					\item	try at home: place peeled potato in very salty water
				\end{itemize}
				\item	If you exceed osmotic pressure you reverse the solvent flow
				\item	Called ``reverse osmosis''
				\item	Net driving force = \underline{$\qquad\qquad\qquad$}
			\end{itemize}
		\column{0.40\textwidth}
			\vspace{-1cm}
			\begin{center}
				\includegraphics[width=0.6\textwidth]{\imagedir/separations/membranes/johnny_automatic_tree-open-clipart-7411.png}
			\end{center}
			\begin{center}
				\includegraphics[width=\textwidth]{\imagedir/separations/membranes/reverse-osmosis-developed.png}
			\end{center}
	\end{columns}
\end{frame}

\begin{frame}\frametitle{Typical values of osmotic press}
	\begin{exampleblock}{}
		\[
			\pi \approx \displaystyle \frac{nRT}{V_m} = CRT
		\]
	\end{exampleblock}
	\begin{tabular}{cll}
		$\pi$		&	[atm] 													& osmotic pressure \\
		$n$ 		&  	[mol]													& mols of \textbf{ions}: e.g. $\text{Na}^{+}$ and $\text{Cl}^{-}$\\
		$R$			&	$[\text{m}^3\text{.atm.K}^{-1}\text{.mol}^{-1}]$ 		& gas law constant: $8.2057 \times 10^{-5}$\\
		$V_m$ 		&  	$[\text{m}^{3}]$										& volume of solvent associated with solute\\
		$T$ 		&  	[K]														& temperature\\
		$C$ 		&  	[mol of ions per m$^{3}$]								& generic concentration
	\end{tabular}

	\vspace{12pt}
	\textbf{Example}

	\vspace{6pt}
	Prove to yourself: 0.1 mol of NaCl dissolved in 1 L of water at 25$^\circ$C is \textbf{4.9 atm}!
	\begin{itemize}
		\item	that's almost 500 kPa
		\item	or almost 5m of head for 5.8 g NaCl in a litre of water
	\end{itemize}
\end{frame}

\begin{frame}\frametitle{Other osmotic values}
	The previous equation is an approximation.

	\vspace{12pt}
	Some actual values:
	\begin{tabular}{ll}\\ \hline
	 	\textbf{Substance}  		& \textbf{Osmotic pressure [atm]} \\ \hline
		Pure water					& 0.0\\
		0.1 mol NaCl in 1 L water 	& 4.56\\
		2.0 mol NaCl in 1 L water	& 96.2\\
		Seawater [3.5 wt\% salts]	& 25.2\\
	\end{tabular}

	\begin{itemize}
		\item	Driving force in membrane separation is pressure difference
		\item	$\Delta P = \pi$ implies we only counteract the osmotic pressure
		\item	Reverse osmosis: increase $\Delta P > \pi$
		\item	So the net useful driving force applied: $\Delta P - \pi$
		\item	Ultrafiltration $\Delta P$ was 0.1 to 1.0 MPa typically
		\item	RO: typical $\Delta P$ values: 2.0 MPa to 8.0 MPa, even 10.5 MPa  % 7.0 in Seader, p 532; 8.0 in C+Rv2;5ed;p454; 10.5 from Perry's 8ed; p20-45
	\end{itemize}
\end{frame}

\begin{frame}\frametitle{Let's be a little more accurate}
	\begin{itemize}
		\item	The solute (salt) passes through the membrane to the permeate side
		\item	$C_p \neq 0$
		\item	There is an osmotic pressure, $\pi_\text{perm}$ back into the membrane.
		\item	Correct, net driving force = $\Delta P - \Delta \pi$
		\begin{itemize}
			\item	$\Delta P$ is the usual TMP we measure
			\item	$\Delta \pi = \pi_\text{feed} - \pi_\text{perm}$
			\item	$\Delta \pi = C_\text{ions,feed}RT_\text{feed} - C_\text{ions,perm}RT_\text{perm}$
			\item	Even more correctly: $\Delta \pi = C_\text{ions,wall}RT_\text{wall} - C_\text{ions,perm}RT_\text{perm}$
		\end{itemize}
	\end{itemize}
\end{frame}

\begin{frame}\frametitle{Widest application for RO: desalination}
	Some quotes:
	\begin{itemize}
		\item	``\emph{McIlvaine forecasts that world RO equipment and membrane sales will reach \$5.6 billion (USD) in 2012, compared to \$3.8 billion in 2008 (actual).}''
		\item	``\emph{Depleting water supplies, coupled with increasing water demand, are driving the global market for desalination technology, which is expected to reach \$52.4 billion by 2020, up 320.3\% from \textbf{\$12.5 billion in 2010}. According to a recent report from energy research publisher SBI Energy, membrane technology reverse osmosis will see the largest growth, reaching \$39.46 billion by 2020.}''
	\end{itemize}
\end{frame}

\begin{frame}\frametitle{Industrial applications of RO}
	\begin{itemize}
		\item	demineralization of industrial water before ion exchange
		\item	not primary aim, but RO membranes retain $>$ 300 Dalton organics %Perry's 8ed; 20-45
		\item	ultrahigh-purity water
		 	\begin{itemize}
		 		\item	laboratories
		 		\item	kidney dialysis
		 		\item	microelectronic manufacturing
		 		\item	pharmaceutical manufacturing (purified water)
		 	\end{itemize}
		 \item	tomato, citrus, and apple juice dewatering [$\sim$ 4.5 c/L; 1995]
		 \item	dealcoholization of wine and beer to retain flavour in the retentate
		 \item	other: keep antifreeze, paint, dyes, PAH, pesticides in retentate; discharge permeate to municipal wastewater
	\end{itemize}
\end{frame}

\begin{frame}\frametitle{Videos to watch}
	In your own time, please watch:

	\begin{itemize}
		\item	\href{http://www.youtube.com/watch?v=YlMGZWmh\_Mw}{http://www.youtube.com/watch?v=YlMGZWmh\_Mw}: how spiral membranes are made
		\item	\href{http://www.youtube.com/watch?v=M3mpJysa6zQ}{http://www.youtube.com/watch?v=M3mpJysa6zQ}: novel way of recovering pressure energy
	\end{itemize}
\end{frame}

\begin{frame}\frametitle{Salt-water reverse osmosis example}
	\includegraphics[width=\textwidth]{\imagedir/separations/membranes/ide-tech.com-1D13_0-low-res.jpg}
	\begin{itemize}
		\item	Larnaca, Cyprus [island state near Greece/Turkey]
		\item	Desalination plant: Build, Own, Operate, and Transfer
		\item	21.5 million $\text{m}^3$ per year
		\item	Seawater intake $\rightarrow$ flocculation and filtration [why?] $\rightarrow$ RO $\rightarrow$ chemical dosing $\rightarrow$ chlorination
		\item	Energy recovery of $\Delta P$ {\tiny (see YouTube video mentioned earlier)}
		% Chemical dosing: neutralize scale-forming salts
		% Chlorination: kill bacteria
	\end{itemize}
	\see{ide-tech.com}
\end{frame}

\begin{frame}\frametitle{RO costs [Perry's; 8ed], 1992}
	\begin{columns}[t]
		\column{0.90\textwidth}
			\begin{center}
				\includegraphics[height=0.92\textheight]{\imagedir/separations/membranes/desalination-costs-Perrys-8ed-Table-20-23.png}
			\end{center}
		\column{0.30\textwidth}
			Household RO\\
			cost:
			\begin{itemize}
				\item	\$ 0.015 to \$0.07/L % Perrys 8ed; p 20-45
			\end{itemize}
	\end{columns}
\end{frame}

% Still review Ghosh and Dickson's notes
\begin{frame}\frametitle{Transport modelling of RO}
	{\color{myOrange}{Symbolically:}}
	{ \small
	\[
		J = \frac{(\text{permeability})(\text{driving force})}{\text{thickness}} = \displaystyle (\text{permeance})(\text{driving force}) = \displaystyle \frac{\text{driving force}}{\text{resistance}}
	\]
	}
	\begin{itemize}
		\item	permeability = $f(\text{membrane properties}, \text{diffusivity}, \text{other physical properties})$
		\item	permeance: easier to calculate:
		\begin{itemize}
			\item	given the driving force: easy to measure
			\item	given the flux: easy to measure
		\end{itemize}
		\item	units are always case specific and must be self-consistent [check!]
	\end{itemize}
\end{frame}

\begin{frame}\frametitle{Simplified RO modelling}
	\begin{itemize}
		\item	We don't consider ``cake build-up'': we assume that solid particles are mostly removed in an upstream separation step
		\item	So $\Delta P$ overcomes osmotic pressure and membrane resistance
	\end{itemize}
	\vspace{-6pt}
	\begin{enumerate}
		\item	\textbf{Solvent} flux
			\[	J_v = J_\text{solv} = \frac{(\Delta P - \Delta \pi)}{R_{m,v} + \cancelto{0}{R_{cp,v}}} = \frac{P_\text{solv}}{\ell_M}(\Delta P - \Delta \pi) \] \\
			\[	R_{m,v} = f(\text{\small membrane's thickness=$\ell_M$, diffusivity of solvent in membrane})\]
		\item	\textbf{Solute} (\emph{salt}) flux
			\[
				J_\text{salt} = \displaystyle \frac{(\text{permeability})(\text{driving force})}{\text{resistance}} = \displaystyle \frac{P_\text{salt}}{\ell_M} \left(C_w - C_p\right)
			\]
			\begin{itemize}
				\item	Our assumption: $C_\text{wall} \approx C_\text{bulk}$ = bulk solute conc$^\text{n}$ = $C_\text{p}$
				\begin{itemize}
					\item	how would you enforce this reasonable assumption?
				\end{itemize}
				\item	Crudely assume: $C_\text{bulk}  \approx C_\text{feed}$ [back-of-envelope calculation]
				\item	$P_\text{solv}$ = permeability of the solvent [notation: $P_\text{solv} \equiv P_\text{w}$]
				\item	$C_p$ = concentration of solute in the permeate
				\item	$P_\text{salt}$ = permeability of the salt
			\end{itemize}
	\end{enumerate}
	%\textbf{Note}: all units must be checked for consistency
\end{frame}

% 19 October 2012: Friday tutorial-style lecture
\begin{frame}\frametitle{Example to try at home}
	Brackish water of 1.8wt\% NaCl at 25$^{\circ}\text{C}$ and 1000 psia is fed to a spiral wound membrane.

	\vspace{6pt}
	Conditions on the permeate side are 0.05 wt\% NaCl, at same temperature, but 50 psia.

	\vspace{6pt}
	The permeance of water has been established as $1.1 \times 10^{-4} \text{~kg.s}^{-1}\text{.m}^{-2}\text{.atm}^{-1}$ [how would you get this number?] and $16 \times 10^{-8} \text{m.s}^{-1}$ for salt is determined experimentally.

	\begin{enumerate}
		\item	Calculate the flux of water in LMH.
		\item	What is the flux of salt through the membrane?
		\item	How do these fluxes compare?
		\item	Calculate the rejection coefficient for salt.
		\item	Calculate the separation factor.
	\end{enumerate}
\end{frame}

\begin{frame}\frametitle{Some questions to consider}
	\begin{enumerate}
		\item	What happens, in terms of osmosis, on a really hot day to fluid flow in a tree?
		\item	Is $P_\text{solv}$ going to change if we use a different solute?
		\item	If we double the pressure drop, will we double the solvent flux?
		\item	Why did we not take osmotic pressure in account for microfiltration and ultrafiltration?
		\item	In RO: what will be the expected effect of increasing operating temperature?
	\end{enumerate}
\end{frame}

\begin{frame}\frametitle{Some old and new terminology}
	Recall from ultrafiltration:
	\begin{itemize}
		\item	$R = 1 - \displaystyle\frac{C_P}{C_F}$

		\item	This {\color{purple}{rejection coefficient}} also applies to reverse osmosis.

		\item	A new term = {\color{purple}{cut}} = {\color{purple}{conversion}} = {\color{purple}{recovery}} = $\theta = \displaystyle \frac{Q_\text{P}}{Q_\text{F}}$ is between 40 and 50\% typically
	\end{itemize}
\end{frame}

\begin{frame}\frametitle{Another example: calculating permeances}
	At 25 $^\circ$C in a lab membrane with area $A = 2 \times 10^{-3}$ $\text{m}^{2}$ we feed a solution of 10 kg NaCl per m$^3$ solution so well mixed that essentially it has the same strength leaving.

	\vspace{12pt}
	The permeate is measured as 0.39 kg NaCl per $\text{m}^{3}$ solution at a rate of $1.92 \times 10^{-8}~\text{m}^3\text{.s}^{-1}$ when applying a constant pressure difference of 54.42 atm.

	\vspace{12pt}
	Calculate the permeance constants for solvent and salt (these were previously given, this example shows how to calculate them experimentally), as well as the rejection coefficient.
\end{frame}

\begin{frame}\frametitle{Relaxing the assumption of $C_\text{R}$ = $C_\text{feed}$}
	\begin{enumerate}
		\item	Usually we specify the desired cut, $\theta = \displaystyle \frac{Q_\text{P}}{Q_\text{F}}$
		\item	$Q_\text{F} C_\text{F} = Q_\text{R} C_\text{R} + Q_\text{P} C_\text{P}$
		\item	$Q_\text{F} = Q_\text{R} + Q_\text{P}$
		\item	$1 = \displaystyle \frac{Q_\text{R}}{Q_\text{F}} + \theta$
		\item	$C_\text{F} = (1 - \theta)C_\text{R} + \theta C_\text{P}$ from equation (2) and (4)
		\item	$J_\text{solv}C_\text{P} = A_\text{solv}(\Delta P - \Delta \pi)C_\text{P}$ = salt flux leaving in permeate
		\item	$J_\text{salt} = A_\text{salt}(C_\text{R} - C_\text{P})$ = salt flux into membrane
	\end{enumerate}
	\hrule
	\begin{itemize}
		\item	Specify $C_\text{F}$ and $\theta$
		\item	Guess $C_\text{P}$ value [how?]
		\item	Calculate $C_\text{R}$ from equation 5
		\item	Calculate $J_\text{solv} C_\text{P}$ from equation 6, noting however that $J_\text{solv} = f(\pi_\text{R}, \pi_\text{P})$. So recalculate $\pi_\text{R}$ and $\pi_\text{P}$
		\item	Note then that equation 6 and 7 must be equal
		\item	Solve eqn 7 for $C_\text{P}$ and use that as a revised value to iterate.
	\end{itemize}
\end{frame}

% ---> Cover this new slide 2013
\begin{frame}\frametitle{Alternative to the above}
	\begin{itemize}\item	Specify $C_\text{F}$ and $\theta$
		\item	Guess $C_\text{R}$ value [what's a reasonable value?]
		\item	Calculate $C_\text{P}$ from equation 5
		\item	If your calculated value of $C_\text{P}$ is negative or exceeds $C_\text{F}$, then repeat your guess for $C_\text{R}$, until you get a $C_\text{P}$ that lies between 0 and $C_\text{F}$ and double check also that the rejection coefficient from this $C_\text{P}$ is reasonable, around 90 to 99%.
		\item	Calculate $J_\text{solv} C_\text{P}$ from equation 6, noting however that $J_\text{solv} = f(\pi_\text{R}, \pi_\text{P})$. So recalculate $\pi_\text{R}$ and $\pi_\text{P}$
		\item	Note then that equation 6 and 7 must be equal
		\item	Solve eqn 7 for $C_\text{P}$ and use that as a revised value to iterate.
		%Now carry on with the rest of the steps as described in the previous slide. It's interesting how simply flipping what you guess first leads to much faster convergence.
		%If you iterate and get a negative value for CP or CR, it simply means that you must decrease your guess for that term, since you obviously can't have a negative concentration.
		%And a final hint: this question is much better to solve on a computer, with goal seek, than by hand. There is tremendous sensitivity to initial guesses, so solving by hand will take too long.
	\end{itemize}
\end{frame}

\begin{frame}\frametitle{Nanofiltration (NF)}

	\begin{center}
		\includegraphics[width=\textwidth]{\imagedir/separations/membranes/membrane-application-areas-CRv2-5ed-p438.png}
	\end{center}
	\see{Richardson and Harker, p 438}

	\vspace{24pt}

	Nanofiltration removes particles intermediate to UF and RO.
	\begin{itemize}
		\item	only difference:  $\Delta P$ is 0.3 to 3 MPa for NF
		\item	same particle sizes removed as RO: 8 to 50 \AA
	\end{itemize}
\end{frame}

\begin{frame}\frametitle{Wastewater applications: membrane bioreactor}

	%Membrane Processes These processes use a selectively permeable membrane to separate pollutants from water. Most of the membranes are formulated from complex organics that polymerize during membrane preparation. This allows the membrane to be tailored to discriminate by molecular size or by degree of hydrogen bonding potential. Ultrafiltration membranes discriminate by molecular size or weight, while reverse osmosis membranes discriminate by hydrogen bonding characteristics. The permeability of these membranes is low: from 0.38-3.8 m/day (10-100 gal/d/ft 2). The apparatus in which they are used must provide a high surface area per unit volume.

	%Membrane Bioreactors (MBRs): Membrane bioreactors are a technology that combines biological degradation of waste products with membrane filtratio

	\begin{center}
		\includegraphics[width=\textwidth]{\imagedir/separations/membranes/MBRvsASP-Schematic-Wikipedia.png}
	\end{center}
	\see{\href{http://en.wikipedia.org/wiki/File:MBRvsASP\_Schematic.jpg}{http://en.wikipedia.org/wiki/File:MBRvsASP\_Schematic.jpg}}

	% Perry's chapter 22.5.8
	\begin{itemize}
		\item	Use a membrane instead of a traditional settler to remove impurities and solids
		\item	Bleed the sludge from retentate (feed-and-bleed)
	\end{itemize}
\end{frame}

\begin{frame}\frametitle{Research topics in membranes}
	\begin{itemize}
		\item	Materials of construction
		\begin{itemize}
			\item	carbon nanotubes to improve flux %  (Shol and Johnson)
		\end{itemize}

		\item	Energy recovery (ongoing)
	\end{itemize}
\end{frame}

% \begin{frame}\frametitle{UF: continuous operation}
% 	% eqn 20-84 Perry
% \end{frame}
\begin{frame}\frametitle{Introduction to dialysis}
	Picture from Seader
	Log-mean concentration difference
	Sized like a heat-exchanger in counter-current mode
\end{frame}

\begin{frame}\frametitle{Gas Separation Membranes}
	\begin{itemize}
		\item
	\end{itemize}
\end{frame}

\begin{frame}\frametitle{Dialysis}
	Uhlmanns2-pdf, p 467
\end{frame}

\begin{frame}\frametitle{Economics and operating costs}
	Schweitzer, p 2-69
\end{frame}

\begin{frame}\frametitle{Waste water flowsheet options}
	Schweitzer, p 2-77
\end{frame}

\begin{frame}\frametitle{Other factors influencing UF flux}
	Ghosh book:
	\begin{itemize}
		\item	p 219
		\item	p 226
	\end{itemize}
\end{frame}

\begin{frame}\frametitle{UF example}
	Batch equation:
	Ghosh book p. 231
\end{frame}

\begin{frame}\frametitle{UF: material of construction: ....}
	Perry: TABLE 20-24
\end{frame}

\begin{frame}\frametitle{Factors to picking a membrane}
	% Perry's: chapter 20.4
	\begin{itemize}
		\item	size rating (MWCO) from vendor [rough guide only]
		\item	selectivity (quantified by a separation factor)
		\item	permeability
		\item	mechanical robustness (to allow module fabrication and withstand operating conditions)
		\item	chemical robustness (to fabrication materials, process fluids, cleaners, and sanitizers)
		\item	low fouling
		\item	high flux
		\item	low cost
		\item	and manufacturer's consistency in producing membrane
	\end{itemize}
\end{frame}

\begin{frame}\frametitle{UF: Membrane selection}
	Millipore catalog
	Membrane geometry: Geankoplis, p 893 describes trade-off for geometry
\end{frame}

\begin{frame}\frametitle{Geometries}
	TABLE 20-17 Commercial TFF Modules
	Very useful table
\end{frame}

% \begin{frame}\frametitle{Making membrane}
% 	% Perry: 20.4
%
% 	Composition and Structure Commercial membranes consist primarily of polymers and some ceramics. Other membrane types include sintered metal, glass, and liquid film. Polymeric membranes are formed by precipitating a 5 to 25 wt \% casting solution (lacquer) into a film by solvent evaporation (air casting), liquid extraction (immersion casting), or cooling (melt casting or thermally induced phase separation). Membranes can be cast as flat sheets on a variety of supports or as fibers through a die. Ceramic membranes are formed by depositing successive layers of smaller and smaller inorganic particles on a monolith substrate, to create smaller and smaller interstices or “pores” between particles. These layers are then sintered below the melting point to create a rigid structure.
%
% 	Membranes may be surface-modified to reduce fouling or improve chemical resistance. This can involve adding surface-modifying agents directly to the lacquer or modifying the cast membrane through chemical or physical treatment. Membranes can also be formed by selective etching or track-etching (radiation treatment followed by etching). Stretching is used to change pore morphology.
%
% 	Liquid film membranes consist of immiscible solutions held in membrane supports by capillary forces. The chemical composition of these solutions is designed to enhance transport rates of selected components through them by solubility or coupled chemical reaction.
% \end{frame}

%Geankop: Fig 7.1.3
%Why have sweep: increase $k_c2$ coefficient
%Film on wall effect
%Surface adsorption or materials and fouling

\begin{frame}\frametitle{Effect of pressure difference}
	Ghosh fig 11.18
\end{frame}

\begin{frame}\frametitle{Velocity function}
	Green book, or Ghosh 11.20
\end{frame}

\begin{frame}\frametitle{Feed concentration function}
	Ghosh 11.20
\end{frame}

\begin{frame}\frametitle{Pumps used for microfiltration}
	* Pumps
	* Capital cost
	* Operating costs
\end{frame}

\begin{frame}\frametitle{Understanding mass transfer}
	* Geankop book
	* Ghosh: p 214
\end{frame}

\begin{frame}\frametitle{Bubble point test}
	Bio applications: sterilize membranes
	Bubble-point test: Appendix A of Schweitzer, p 2-89
\end{frame}

\begin{frame}\frametitle{Geometries and flow patterns}
	Wankat p 537
\end{frame}

\begin{frame}\frametitle{Principles used to separate: DISCUSS THESE AS NEEDED, ie. when you are introducing microfiltration vs dialysis}
	* Principle(s) used:
	 - diffusion differences (different solubilities)
	 - size differences
	 - electric attraction
\end{frame}


\end{comment}

