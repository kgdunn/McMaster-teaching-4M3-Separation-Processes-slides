% Perry: 18.9. SELECTION OF A SOLIDS-LIQUID SEPARATOR
% See ``Uhlmann - Overview of Separation Processes.pdf'': selection among alternatives, section 6
% think I also saw something along these lines in Perrys
% Also see Svarovsky

\begin{frame}\frametitle{Final exam details}
	\begin{itemize}
		\item	On Monday, 10 December 2012
		\item	12:30 to 15:30
		\item	Building T-28
		\item	It will likely be around 100 marks
		\item	Recall, the number of questions is immaterial
		\item	The question style will be mostly conceptual
	\end{itemize}
	\vspace{24pt}
	\emph{I've not set the final exam yet}
\end{frame}

\begin{frame}\frametitle{}
	\begin{exampleblock}{What's in the exam?}
		Everything that was covered in class time
	\end{exampleblock}
	\vspace{24pt}
	\begin{itemize}
		\item	Guest lecture
		\item	Course presentations
		\item	My lectures
		\item	``Interactive tutorial-type'' classes
	\end{itemize}
	
\end{frame}

\begin{frame}\frametitle{How to do well in the final exam}
	\begin{itemize}
		\item	Review the midterm without solutions: \emph{it should be easy now}
		\item	Redo assignment questions you go wrong, but understand \textbf{why} you initially got the question wrong
	\end{itemize}
\end{frame}

\begin{frame}\frametitle{}
	\begin{itemize}
		\item	
		* Understand the concepts being learned. My courses are not about applying the correct equation and solving.
		* As you've seen, there are only 10 or so main equations we have learned. Understanding how to use these equations, and how to interpret them, is important.
		* Check that your answers are reasonable (can you really have a flow rate of 1050 \(\text{m}^3.\text{s}^{-1}\) through a pipe?)
		* Read the questions carefully: they are usually worded precisely. The biggest point where students loose marks is to answer only part of the question.
		* Questions that you did on computer in the assignments: make sure you can repeat them by hand. Obviously not where you have to draw an entire plot, but make sure the calculations to draw that plot can be done for at least one or two points on the curve.
		* Review the midterm. All the questions from there should be easy and straightforward.
		* Review and repeat all assignment questions that you do not understand. Do not rely on the assignment solutions: none of the final exam questions are going to be from the assignments (even with different values).
		* I cannot emphasize this strongly enough, even though experience has shown me that ''most of you will disregard this advice'': '''treat the exam as a closed-book test: have a formula sheet for the equations, and understand all the concepts without referring to a textbook'''.  Textbooks and other papers should be used to refer to as a backup only.

	\end{itemize}
\end{frame}

\begin{frame}\frametitle{What can I do to prepare?}
	Some tips from the educational research area:
	\begin{itemize}
		\item	Don't look at a question/topic and say: ``Yeah, I can do that''. Prove that you can.
		\item	Review questions from Geankoplis and Seader \emph{et al.}
		\item	Study with little distraction: cellphone, music, email, 
	\end{itemize}
\end{frame}

\begin{frame}\frametitle{To bring to the exam}
	\begin{itemize}
		\item	Any notes, assignments, midterms, etc
		\item	\textbf{Psychrometric chart}: available on course website. \emph{It will not be provided}, and it will be required.
	\end{itemize}
\end{frame}

\begin{frame}\frametitle{Some more context of the 4M3}
	\todo{Mind map here}
	% Slide 31 in week 1
\end{frame}

\begin{frame}\frametitle{Topics we didn't get around to covering}
	\begin{itemize}
		\item	Crystallization
		\item	Chromatography
		\item	Ion-exchange (identical principle to adsorption)
	\end{itemize}
\end{frame}

\begin{frame}\frametitle{Common themes in all sections}
	\begin{itemize}
		\item	Separation factor
		\item	Recovery
		\item	Concentration of recovered compound
		\item	Separating agents: mass and energy
		\item	Multiple ways to achieve a separation
	\end{itemize}
\end{frame}

\begin{frame}\frametitle{Take the following into account}
	For each separator we looked at:
	\begin{itemize}
		\item	Understand the physical principle used in the separation
		\item	Which phases are present and being separated?
		\item	What affects the unit's cost?
		\item	Factors most related to troubleshooting problems with the unit
		\item	How would you optimize an existing unit?
		\item	Can you repurpose a unit for a similar, but different use?
	\end{itemize}
\end{frame}

\begin{frame}\frametitle{Mechanical separations}
	Sedimentation, screens, centrifuges, cyclones - 
\end{frame}

\begin{frame}\frametitle{Membranes}
	Membranes, reverse osmosis, and bioseparations
\end{frame}

\begin{frame}\frametitle{Liquid-liquid extraction}

\end{frame}

\begin{frame}\frametitle{Adsorption}
	
\end{frame}

\begin{frame}\frametitle{Drying}
	
\end{frame}