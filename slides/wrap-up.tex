% Perry: 18.9. SELECTION OF A SOLIDS-LIQUID SEPARATOR
% See ``Uhlmann - Overview of Separation Processes.pdf'': selection among alternatives, section 6
% think I also saw something along these lines in Perrys
% Also see Svarovsky

\begin{frame}\frametitle{Final exam details}
	\begin{itemize}
		\item	On Monday, 10 December 2012
		\item	12:30 to 15:30
		\item	Building T-28
		\item	It will likely be around 100 marks
		\item	As we discussed previously, the number of questions you see is immaterial
		\item	The question style will be mostly conceptual
		\item	You can be sure there will be 100\% coverage of the topics
	\end{itemize}
	\vspace{24pt}
	\textbf{Please note}: \emph{I've not set the final exam yet}, so please do not interpret anything from this lecture as being more/less important.
\end{frame}

\begin{frame}\frametitle{}
	\begin{exampleblock}{What's in the exam?}
		Everything that was covered in class time
	\end{exampleblock}
	\vspace{24pt}
	\begin{itemize}
		\item	Guest lecture
		\item	Course presentations
		\item	My lectures
		\item	``Interactive tutorial-type'' classes (i.e. Friday classes)
	\end{itemize}	
\end{frame}

\begin{frame}\frametitle{To bring to the exam}
	\begin{itemize}
		\item	\textbf{Psychrometric chart}: available on course website. \emph{It will not be provided}, and it will be required.
		\item	Any notes, assignments, midterms, \emph{etc} that you will feel are helpful
		\item	Any textbooks and printed materials
	\end{itemize}
	Only limitation: no iPads, tablets, electronic devices
\end{frame}

\begin{frame}\frametitle{How to do well in the final exam}
	\begin{itemize}
		\item	{\color{myRed}{Repeat}} the midterm without solutions: \emph{{\color{myOrange}{it should be easy now}}}
		\item	Redo assignment questions you got wrong
		\item	More importantly: understand \textbf{why} you initially got the question wrong
			\begin{itemize}
				\item	what concept did you misunderstand?
				\item	take time to review that concept(s) again
			\end{itemize}
		\item	Review questions from Geankoplis and from Seader \emph{et al.}
		\item	I will be posting practice questions to the course website
	\end{itemize}
\end{frame}

\begin{frame}\frametitle{What can I do to prepare?}
	\textbf{{\color{myGreen}{\emph{Some tips from the educational research area:}}}}
	\begin{itemize}
		\item	Don't just look at a question/topic and say: ``Yeah, I can do that''. Prove that you can.
		\item	While you are learning, check/ask yourself whether you actually understand that topic.
		\item	Can you explain the concept to a study partner without looking at the notes?
		\item	Can you explain \textbf{the approach} you would take to solve a problem?
	\end{itemize}
	\textbf{{\color{myRed}{Poor students do this*}}}:
	\begin{itemize}
		\item	Distractions while studying: music, cellphones and email/website checking, TV on in background
		\item	Skip over parts you don't understand
		\item	After finishing, try to repeat text literally
	\end{itemize}
	\textbf{{\color{myRed}{*}}}{\tiny Meneghetti et al. ``Strategic knowledge and consistency in students with good and poor study skills'', \emph{European Journal of Cognitive Psychology}, \textbf{19}, 2007.}
\end{frame}

\begin{frame}\frametitle{Why study this course: Separation Processes?}
	\begin{itemize}
		\item	Can't beat Nature: ``Second Law of Thermodynamics''
		\item	There are multiple ways to achieve a required separation
		\item	50\% to 90\% of capital investment on petroleum and other chemical-reaction based flowsheets 
		\item	Expense often in proportion to the level of purity (called the separation factor) %\see{Treybal, p 2}
		\item	60 to 100\% of the ongoing operating costs in chemical plants  % Reference for this? Uhlmann overview PDF?
		\item	These systems are all around us
		\begin{itemize}
			\item	leaching (coffee)
			\item	centrifugation and drying (washing and tumble drying clothes)
			\item	absorption (your lungs)
			\item	membranes (your skin, kidneys)
			\item	adsorption (water filter)
		\end{itemize}
	\end{itemize}
\end{frame}

\begin{frame}\frametitle{Some more context around the 4M3 course}
	\begin{center}
		\includegraphics[width=\textwidth]{\imagedir/separations/overview-class/Separation-Processes-Mindmap-updated.png}
	\end{center}
\end{frame}

\begin{frame}\frametitle{Can't remember what was covered when?}
	\begin{exampleblock}{}
		\href{http://learnche.mcmaster.ca/4M3/Course\_videos\_from\_2012}{http://learnche.mcmaster.ca/4M3/Course\_videos\_from\_2012}
	\end{exampleblock}
	\begin{center}
		\includegraphics[width=\textwidth]{\imagedir/separations/admin/course-videos-2012.png}
	\end{center}
\end{frame}

\begin{frame}\frametitle{Mechanical separations}
	
	\begin{tabular}{cll}
		\textbf{Part} & \textbf{Topic} & \textbf{Week number}\\\hline
		1	&	Sedimentation	& {\color{Brown}{\texttt{02A, 02B, 02C}}}\\
		2	&	Screens			& {\color{Brown}{\texttt{03A}}}\\
		3	&	Centrifuges  	& {\color{Brown}{\texttt{03B, 03C, 04A}}}\\
		4	&	Cyclones		& {\color{Brown}{\texttt{04B}}} 
	\end{tabular}
\end{frame}

{\usebackgroundtemplate{\vbox to \paperheight{\vfil\hbox to \paperwidth{\hfil      \includegraphics[width=1.3\paperwidth]{\imagedir/separations/sedimentation/circular-tank-sedimentation-map.png}  \hfil}\vfil}}
\begin{frame}\frametitle{Sedimentation: {\color{Brown}{\texttt{02A, 02B, 02C}}}}

\end{frame}}

{\usebackgroundtemplate{\vbox to \paperheight{\vfil\hbox to \paperwidth{\hfil      \includegraphics[width=\paperwidth]{\imagedir/separations/screens/differential-cumulative-background.png}  \hfil}\vfil}}
\begin{frame}\frametitle{Screens: {\color{Brown}{\texttt{03A}}}}
	\begin{itemize}
		\item	Sphericity
		\item	Equivalent diameters
			\begin{itemize}
				\item	volume
				\item	surface area
				\item	area to volume ratio
				\item	settling velocity
			\end{itemize}
		\item	Mesh sizes
		\item	Differential and cumulative analysis
	\end{itemize}
\end{frame}}

{\usebackgroundtemplate{\vbox to \paperheight{\vfil\hbox to \paperwidth{\hfil      \includegraphics[width=\paperwidth]{\imagedir/separations/centrifuge/centrifuge-background.png}  \hfil}\vfil}}
\begin{frame}\frametitle{Centrifuges: {\color{Brown}{\texttt{03B, 03C, 04A}}}}
	\begin{itemize}
		\item	Many applications
		\item	Tubular bowl and disk bowl
	\end{itemize}
\end{frame}}

{\usebackgroundtemplate{\vbox to \paperheight{\vfil\hbox to \paperwidth{\hfil      \includegraphics[width=\paperwidth]{\imagedir/separations/cyclones/cyclone-base-diagram.png}  \hfil}\vfil}}
\begin{frame}\frametitle{Cyclones: {\color{Brown}{\texttt{04B}}}}
	% \begin{itemize}
	% 	\item	General principle
	% 	\item	Grade efficiency curves
	% 	\item	Circuits of cyclones: parallel/series/recycle
	% \end{itemize}
\end{frame}}

{\usebackgroundtemplate{\vbox to \paperheight{\vfil\hbox to \paperwidth{\hfil      \includegraphics[width=\paperwidth]{\imagedir/separations/membranes/membrane-background.png}  \hfil}\vfil}}
\begin{frame}\frametitle{Membranes: {\color{Brown}{\texttt{04C} to \texttt{07C}}}}
	\textbf{We studied}:
	\begin{itemize}
		\item	Microfiltration
		\item	Ultrafiltration
		\item	Reverse osmosis
	\end{itemize}
	
	\vspace{3cm}
	\emph{{\color{myGreen}{Some things to consider}}}:
	\begin{itemize}
		\item	What are typical LMHs, $\Delta P$ and particle sizes retained?
		\item	When can we set $C_p \approx 0$?
		\item	When can we disregard membrane or cake resistance?
		\item	How are permeances calculated?
	\end{itemize}
\end{frame}}

{\usebackgroundtemplate{\vbox to \paperheight{\vfil\hbox to \paperwidth{\hfil      \includegraphics[width=\paperwidth]{\imagedir/separations/liquid-liquid-extraction/LLE-background.png}  \hfil}\vfil}}
\begin{frame}\frametitle{Liquid-liquid extraction {\color{Brown}{\texttt{08A} to \texttt{09B}}}}
	{\color{myOrange}{\textbf{Plenty of new concepts}}}
	\begin{itemize}
		\item	ternary diagrams
		\item	lever rule
		\item	mixer-settlers
		\item	tie lines
		\item	equilibrium
		\item	solute, solvent, carrier
		\item	extract, raffinate, distribution coefficient $D_\text{A} = \displaystyle \frac{y_\text{E,A}}{x_\text{R,A}}$
		\item	recovery and concentration
		\item	units in sequence
		\item	countercurrent units
		\item	operating point $P$
	\end{itemize}
\end{frame}}

{\usebackgroundtemplate{\vbox to \paperheight{\vfil\hbox to \paperwidth{\hfil      \includegraphics[width=\paperwidth]{\imagedir/separations/adsorption/molecular-sieve-background.png}  \hfil}\vfil}}
\begin{frame}\frametitle{Adsorption {\color{Brown}{\texttt{10B} to \texttt{11A}}}}
	\begin{itemize}
		\item	Langmuir and Freundlich isotherms
		\item	Breakthrough
		\item	MTZ
		\item	$L_\text{UNB}$
		\item	Bed mass balance
	\end{itemize}
\end{frame}}

{\usebackgroundtemplate{\vbox to \paperheight{\vfil\hbox to \paperwidth{\hfil      \includegraphics[width=\paperwidth]{\imagedir/separations/supercritical-fluid-extraction/flowsheet-Chad-Kyle-original.png}  \hfil}\vfil}}
\begin{frame}\frametitle{Supercritical fluid extraction {\color{Brown}{\texttt{13A}}}}
	\vfill
	\vspace{7cm}
	Including all other presentations
\end{frame}}

{\usebackgroundtemplate{\vbox to \paperheight{\vfil\hbox to \paperwidth{\hfil      \includegraphics[width=\paperwidth]{\imagedir/separations/drying/psychrometric-background.png}  \hfil}\vfil}}
\begin{frame}\frametitle{Drying {\color{Brown}{\texttt{13A} and \texttt{13B}}}}
	\begin{itemize}
		\item	heat transfer
		\item	mass transfer
		\item	using psychrometric charts
		\item	$T_\text{db}$ and $T_\text{wb}$
		\item	heat and mass flux equivalence
	\end{itemize}
\end{frame}}

\begin{frame}\frametitle{Common themes in all sections}
	\begin{itemize}
		\item	Separation factor = $S_{ij} = \displaystyle \frac{x_{i,1} / x_{j,1}}{x_{i,2} / x_{j,2}}$
		\item	Concentration of recovered compound in stream $i$
		\item	Recovery = $\displaystyle\frac{\text{mass of desired compound recovered in stream~} i}{\text{mass of desired compound {\color{myRed}{in the feed}}}}$
		\item	Separating agents: mass (MSA) and energy (ESA)
		\item	Which phases are involved?
	\end{itemize}
\end{frame}

\begin{frame}\frametitle{Take the following into account}
	For each separator we looked at, {\color{myOrange}{please aim to}}:
	\vspace{12pt}
	\begin{itemize}
		\item	understand the \textbf{physical principle} used in the separation
		\item	know which \textbf{phases} are present and being separated?
		\item	determine what affects the \textbf{unit's cost}?
		\item	\textbf{identify variables} when troubleshooting problems with the unit
		\item	optimize an existing unit: increase throughput, boost recovery, aka ``intensification''
		\item	\textbf{repurpose} an existing unit for a similar, but different use.
	\end{itemize}
\end{frame}

\begin{frame}\frametitle{Other tips}
	\begin{itemize}
		\item	Understand the concepts being learned. My courses are not about applying the correct equation and solving.
		\item	Read the questions carefully: they are worded precisely. Answer all parts of the questions.
		\item	None of the final exam questions are going to be from the assignments (even with different values).
		\item	Check that your answers are reasonable (can you really have a flow rate of 1050 \(\text{m}^3.\text{s}^{-1}\) through a pipe?)
		\item	Computer questions in assignments: make sure you can repeat them by hand, where reasonable.
	\end{itemize}
	\begin{exampleblock}{{\color{myRed}{Most important advice}}}
		\begin{itemize}
			\item	Treat the exam as a closed-book test: have a formula sheet for the equations, and understand all the concepts without referring to a textbook or notes
			\item	Textbooks and other papers should be used to refer to as a backup only.
		\end{itemize}
	\end{exampleblock}
\end{frame}

\begin{frame}\frametitle{Thank you}
	\begin{itemize}
		\item	It's been a long semester, \textbf{\emph{really busy}}
		\item	You've been the guinea pigs for this 4M3 overhaul.
		\item	In addition to it being my first time teaching this topic.
		\item	Endured late midterms -- 6 weeks -- \emph{yikes!} Sorry about that.
		\item	But you have helped me tremendously with feedback about the notes and good questions in class and by email.
		\item	Further comments? \href{https://evals.mcmaster.ca}{https://evals.mcmaster.ca} or in person.
	\end{itemize}
	\vspace{12pt}
	\textbf{Thank you.}
\end{frame}