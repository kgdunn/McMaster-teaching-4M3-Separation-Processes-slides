% Perry: 18.9. SELECTION OF A SOLIDS-LIQUID SEPARATOR
% See ``Uhlmann - Overview of Separation Processes.pdf'': selection among alternatives, section 6
% think I also saw something along these lines in Perrys
% Also see Svarovsky

\begin{frame}\frametitle{Final exam details}
	\begin{itemize}
		\item	On Monday, 10 December 2012
		\item	12:30 to 15:30
		\item	Building T-28
		\item	It will likely be around 100 marks
		\item	As we discussed previously, the number of questions you see is immaterial
		\item	The question style will be mostly conceptual
		\item	You can be sure there will be 100\% coverage of the topics
	\end{itemize}
	\vspace{24pt}
	\textbf{Please note}: \emph{I've not set the final exam yet}, so please do not interpret anything from this lecture as being more/less important.
\end{frame}

\begin{frame}\frametitle{}
	\begin{exampleblock}{What's in the exam?}
		Everything that was covered in class time
	\end{exampleblock}
	\vspace{24pt}
	\begin{itemize}
		\item	Guest lecture
		\item	Course presentations
		\item	My lectures
		\item	``Interactive tutorial-type'' classes (i.e. Friday classes)
	\end{itemize}	
\end{frame}

\begin{frame}\frametitle{To bring to the exam}
	\begin{itemize}
		\item	\textbf{Psychrometric chart}: available on course website. \emph{It will not be provided}, and it will be required.
		\item	Any notes, assignments, midterms, \emph{etc} that you will feel are helpful
		\item	Any textbooks and printed materials
	\end{itemize}
	Only limitation: no iPads, tablets, electronic devices
\end{frame}

\begin{frame}\frametitle{How to do well in the final exam}
	\begin{itemize}
		\item	Review the midterm without solutions: \emph{{\color{myOrange}{it should be easy now}}}
		\item	Redo assignment questions you got wrong
		\item	More importantly: understand \textbf{why} you initially got the question wrong
			\begin{itemize}
				\item	what concept did you misunderstand?
				\item	take time to review that concept(s) again
			\end{itemize}
		\item	Review questions from Geankoplis and from Seader \emph{et al.}
	\end{itemize}
\end{frame}

\begin{frame}\frametitle{What can I do to prepare?}
	\textbf{{\color{myGreen}{\emph{Some tips from the educational research area:}}}}
	\begin{itemize}
		\item	Don't look at a question/topic and say: ``Yeah, I can do that''. Prove that you can.
		\item	While you are learning check/ask yourself whether you actually understand that topic.
		\item	Can you explain the concept to a study partner without looking at the notes?
		\item	Can you explain \textbf{the approach} you would take to solve a problem?
	\end{itemize}
	\textbf{{\color{myRed}{Poor students do this*}}}:
	\begin{itemize}
		\item	Distractions: study with music, cellphones and email/website checking, TV on in background
		\item	Skip over parts you don't understand
		\item	After finishing, try to repeat text literally
	\end{itemize}
	\textbf{{\color{myRed}{*}}}{\tiny Meneghetti et al. ``Strategic knowledge and consistency in students with good and poor study skills'', \emph{Eur. J. Cognitive Psych.}, \textbf{19}, 2007.}
\end{frame}

\begin{frame}\frametitle{Why study this Separation Processes?}
	\begin{itemize}
		\item	Can't beat Nature: ``Second Law of Thermodynamics''
		\item	There are multiple ways to achieve a required separation
		\item	50\% to 90\% of capital investment on petroleum and other chemical-reaction based flowsheets 
		\item	Expense often in proportion to the level of purity (called the separation factor) %\see{Treybal, p 2}
		\item	60 to 100\% of the ongoing operating costs in chemical plants  % Reference for this? Uhlmann overview PDF?
		\item	These systems are all around us
		\begin{itemize}
			\item	leaching (coffee)
			\item	centrifugation and drying (washing and tumble drying clothes)
			\item	absorption (your lungs)
			\item	membranes (your skin, kidneys)
			\item	adsorption (water filter)
		\end{itemize}
	\end{itemize}
\end{frame}

\begin{frame}\frametitle{Some more context around the 4M3 course}
	\begin{center}
		\includegraphics[width=\textwidth]{\imagedir/separations/overview-class/Separation-Processes-Mindmap-updated.png}
	\end{center}
\end{frame}

\begin{frame}\frametitle{Can't remember what was covered when?}
	\begin{exampleblock}{}
		\href{http://learnche.mcmaster.ca/4M3/Course\_videos\_from\_2012}{http://learnche.mcmaster.ca/4M3/Course\_videos\_from\_2012}
	\end{exampleblock}
	\begin{center}
		\includegraphics[width=\textwidth]{\imagedir/separations/admin/course-videos-2012.png}
	\end{center}
\end{frame}


\begin{frame}\frametitle{Mechanical separations}
	
	\begin{tabular}{cll}
		\textbf{Part} & \textbf{Topic} & \textbf{Week number}\\\hline
		1	&	Sedimentation	& {\color{Brown}{\texttt{02A, 02B, 02C}}}\\
		2	&	Screens			& {\color{Brown}{\texttt{03A}}}\\
		3	&	Centrifuges  	& {\color{Brown}{\texttt{03B, 03C, 04A}}}\\
		4	&	Cyclones		& {\color{Brown}{\texttt{04B}}} 
	\end{tabular}
\end{frame}

{\usebackgroundtemplate{\vbox to \paperheight{\vfil\hbox to \paperwidth{\hfil      \includegraphics[width=1.3\paperwidth]{\imagedir/separations/sedimentation/circular-tank-sedimentation-map.png}  \hfil}\vfil}}
\begin{frame}\frametitle{Sedimentation: {\color{Brown}{\texttt{02A, 02B, 02C}}}}

\end{frame}}

{\usebackgroundtemplate{\vbox to \paperheight{\vfil\hbox to \paperwidth{\hfil      \includegraphics[width=\paperwidth]{\imagedir/separations/screens/differential-cumulative-background.png}  \hfil}\vfil}}
\begin{frame}\frametitle{Screens: {\color{Brown}{\texttt{03A}}}}
	\begin{itemize}
		\item	Sphericity
		\item	Equivalent diameters
		\item	Mesh sizes
		\item	Differential and cumulative analysis
	\end{itemize}
\end{frame}}

{\usebackgroundtemplate{\vbox to \paperheight{\vfil\hbox to \paperwidth{\hfil      \includegraphics[width=\paperwidth]{\imagedir/separations/centrifuge/centrifuge-background.png}  \hfil}\vfil}}
\begin{frame}\frametitle{Centrifuges: {\color{Brown}{\texttt{03B, 03C, 04A}}}}
	\begin{itemize}
		\item	Many applications
		\item	Tubular bowl and disk bowl
	\end{itemize}
\end{frame}}

{\usebackgroundtemplate{\vbox to \paperheight{\vfil\hbox to \paperwidth{\hfil      \includegraphics[width=\paperwidth]{\imagedir/separations/cyclones/cyclone-base-diagram.png}  \hfil}\vfil}}
\begin{frame}\frametitle{Cyclones: {\color{Brown}{\texttt{04B}}}}
	% \begin{itemize}
	% 	\item	General principle
	% 	\item	Grade efficiency curves
	% 	\item	Circuits of cyclones: parallel/series/recycle
	% \end{itemize}
\end{frame}}

{\usebackgroundtemplate{\vbox to \paperheight{\vfil\hbox to \paperwidth{\hfil      \includegraphics[width=\paperwidth]{\imagedir/separations/membranes/membrane-background.png}  \hfil}\vfil}}
\begin{frame}\frametitle{Membranes: {\color{Brown}{\texttt{04C} to \texttt{07C}}}}
	\textbf{We studied}:
	\begin{itemize}
		\item	Microfiltration
		\item	Ultrafiltration
		\item	Reverse osmosis
	\end{itemize}
	
	\vspace{3cm}
	\emph{{\color{myGreen}{Some things to consider}}}:
	\begin{itemize}
		\item	What are typical LMHs, $\Delta P$ and particle sizes retained?
		\item	When can we set $C_p \approx 0$
		\item	When can we disregard membrane or cake resistance?
		\item	How are permeances calculated?
	\end{itemize}
\end{frame}}

{\usebackgroundtemplate{\vbox to \paperheight{\vfil\hbox to \paperwidth{\hfil      \includegraphics[width=\paperwidth]{\imagedir/separations/liquid-liquid-extraction/LLE-background.png}  \hfil}\vfil}}
\begin{frame}\frametitle{Liquid-liquid extraction}
	{\color{myOrange}{\textbf{Plenty of new concepts}}}
	\begin{itemize}
		\item	ternary diagrams
		\item	lever rule
		\item	mixer-settlers
		\item	tie lines
		\item	equilibrium
		\item	solute, solvent, carrier
		\item	extract, raffinate, distribution coefficient $D_\text{A} = \displaystyle \frac{y_\text{E,A}}{x_\text{R,A}}$
		\item	recovery and concentration
		\item	units in sequence
		\item	countercurrent units
	\end{itemize}
\end{frame}
}

\begin{frame}\frametitle{Adsorption}
	
\end{frame}

\begin{frame}\frametitle{Drying}
	
\end{frame}

\begin{frame}\frametitle{Common themes in all sections}
	\begin{itemize}
		\item	Separation factor
		\item	Recovery of desired compound
		\item	Concentration of recovered compound
		\item	Separating agents: mass (MSA) and energy (ESA)
		\item	Which phases are involved?
	\end{itemize}
\end{frame}

\begin{frame}\frametitle{Take the following into account}
	For each separator we looked at:
	\begin{itemize}
		\item	Understand the physical principle used in the separation
		\item	Which phases are present and being separated?
		\item	What affects the unit's cost?
		\item	Factors most related to troubleshooting problems with the unit
		\item	How would you optimize an existing unit?
		\item	Can you repurpose a unit for a similar, but different use?
	\end{itemize}
\end{frame}


\begin{frame}\frametitle{}
	\begin{itemize}
		\item	
		* Understand the concepts being learned. My courses are not about applying the correct equation and solving.
		* As you've seen, there are only 10 or so main equations we have learned. Understanding how to use these equations, and how to interpret them, is important.
		* Check that your answers are reasonable (can you really have a flow rate of 1050 \(\text{m}^3.\text{s}^{-1}\) through a pipe?)
		* Read the questions carefully: they are usually worded precisely. The biggest point where students loose marks is to answer only part of the question.
		* Questions that you did on computer in the assignments: make sure you can repeat them by hand. Obviously not where you have to draw an entire plot, but make sure the calculations to draw that plot can be done for at least one or two points on the curve.
		* Review the midterm. All the questions from there should be easy and straightforward.
		* Review and repeat all assignment questions that you do not understand. Do not rely on the assignment solutions: none of the final exam questions are going to be from the assignments (even with different values).
		* I cannot emphasize this strongly enough, even though experience has shown me that ''most of you will disregard this advice'': '''treat the exam as a closed-book test: have a formula sheet for the equations, and understand all the concepts without referring to a textbook'''.  Textbooks and other papers should be used to refer to as a backup only.

	\end{itemize}
\end{frame}

\begin{frame}\frametitle{Thank you}
	\begin{itemize}
		\item	It's been a long semester, \textbf{\emph{really busy}}
		\item	You've been the guinea pigs for the 4M3 overhaul
		\item	Endured late midterm grading -- 6 weeks: \emph{yikes!}
		\item	But you have helped me tremendously with feedback about the notes and good questions in class/email
	\end{itemize}
\end{frame}