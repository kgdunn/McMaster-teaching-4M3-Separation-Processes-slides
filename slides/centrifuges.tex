Disk stack centrifuge: http://www.youtube.com/watch?v=bzXUiLajVlg

Perry's: 18.8.6. CONTINUOUS FILTERING CENTRIFUGES

Where processing conditions and objectives allow, continuous filtering centrifuges offer the combination of high processing capacities and good wash capabilities. Inherently they are less flexible than batch filtering centrifuges, primarily constrained by much shorter retention time, and in some cases liquid handling capacity requires upstream preconcentration of the slurry. Fines loss to the filtrate is also greater with continuous designs compared to batch.

COSTS: also see chapter 18.8 in Perry's


SEDIMENTATION CENTRIFUGES (Perry's 18.8.3)

When a spherical particle of diameter d settles in a viscous liquid under earth gravity g, the terminal velocity V s is determined by the weight of the particle-balancing buoyancy and the viscous drag on the particle in accordance to Stokes’ law. In a rotating flow, Stokes’ law is modified by the “centrifugal gravity” G = Ω 2r, thus

(18-103)

In order to have good separation or high settling velocity, a combination of the following conditions is generally sufficient:

1. High centrifuge speed

2. Large particle size

3. Large density difference between solid and liquid

4. Large separation radius

5. Low liquor viscosity

Among the five parameters, the settling velocity is very sensitive to change in speed and particle size. It varies as the square of both parameters. The maximum achievable rotational speed of a centrifuge is normally dictated by the stresses exerted by the processing medium on the bowl and the stresses of the bowl on periphery equipment, most notably the drive system, which consists of a gear unit or hydraulic pump. If the particles in the feed slurry are too small to be separated in the existing G-field, coagulation and flocculation by polymers are effective ways to create larger agglomerated particles for settling. Unlike separation under a constant gravitational field, the settling velocity under a centrifugal field increases linearly with the radius. The greater the radius at which the separation takes place in a given centrifuge at a given rotational speed, the better the separation. Sedimentation of particles is favorable in a less viscous liquid. Some processes are run under elevated temperature where liquid viscosity drops to a fraction of its original value at room temperature.

