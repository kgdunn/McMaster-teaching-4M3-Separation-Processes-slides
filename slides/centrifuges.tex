% 20 September 2012

% 2013: do a 2-phase system example; e.g. biodiesel project of Sguigna and Shawn Thompson, project 2012.

% 2013: make this clear: Centrifuges can be divided into two categories that depends on whether the desired product is the solid or the liquid. If the interest is in recovering the solids, a filtering centrifuge would be an ideal choice of use. Since this application required the separation of pulp from the liquid juice where the desired product is the juice, a sedimentation centrifuge is the optimal selection for the separation mechanism to provide a clear supernatant. From Orange Juice project report, 2012, Kasim and Najib.

% 2013: compare different centrifuge types: see report of Cenedese and Abbasi: nice table showing the advantages and disadvantages.

% 2013: See if anything useful in http://cpe.njit.edu/dlnotes/CHE685/Cls10-1.pdf  downloaded as "njit-edu - CHE685 - notes - Cls10-1.pdf" on your computer

\begin{frame}\frametitle{References}
	\begin{itemize}
		\item	Geankoplis, ``Transport Processes and Separation Process Principles'', 3rd or 4th edition, chapter 14.
		\item	Richardson and Harker, ``Chemical Engineering, Volume 2'', 5th edition, chapter 9.
		\item	\href{http://accessengineeringlibrary.com/browse/perrys-chemical-engineers-handbook-eighth-edition}{Perry's Chemical Engineers' Handbook}, 8th edition, chapter 18.8.
		\item	Svarovsky, ``Solid Liquid Separation'', 3rd or 4th edition. %Well written and straightforward to understand.
		\item	Seader et al. ``Separation Process Principles", page 800 to 802 in 3rd edition.
		\item	Schweitzer, ``Handbook of Separation Techniques for Chemical Engineers'', chapter 4.5.
	\end{itemize}
\end{frame}

\begin{frame}\frametitle{Why consider centrifuges?}
	\begin{itemize}
		\item	When gravity (freely available) is not fast enough
		\item	Decrease the separation time and increase {\color{myGreen}{\emph{separation factor}}}
		\item	Much smaller piece of equipment
		\item	Achieve separations not possible by gravity:
			\begin{itemize}
				\item	overcome Brownian limits
				\item	overcome convection currents
				\item	overcome stabilizing forces that hold an emulsion together
			\end{itemize}
	\end{itemize}
	\vspace{12pt}
	\begin{exampleblock}{}
		{\color{myGreen}{Why not just apply flocculation?}}
	\end{exampleblock}
	% If we flocculated, we would create large solid flocs, they settle out faster, smaller area required
	% Disadvantages: operating cost of the flocculant, adding an MSA that you cannot remove easily afterwards
	% The flocculant is going to be present in the overflow and/or overflow
	% e.g. would not separate cream from milk with a flocculant; you are interested in both cream and milk
	%      similar idea for bioprocess separations
\end{frame}

\begin{frame}\frametitle{Terminology}
	\begin{columns}[t]
		\column{0.50\textwidth}
			\begin{center}
				\includegraphics[width=\textwidth]{\imagedir/separations/centrifuge/Tabletop_centrifuge-wikipedia.jpg}
			\end{center}
			\see{\href{http://en.wikipedia.org/wiki/File:Tabletop\_centrifuge.jpg}{http://en.wikipedia.org/wiki/File:Tabletop\_centrifuge.jpg}}
		\column{0.50\textwidth}
			\begin{center}
				\includegraphics[width=\textwidth]{\imagedir/separations/centrifuge/lab-centrifuge-Ghosh-book.png}
			\end{center}
			\small
			\begin{itemize}
				\item	{\color{purple}\textbf{Suspension}}: the mixed material added into the centrifuge tube
				\item	{\color{purple}\textbf{Pellet}} or {\color{purple}\textbf{precipitate}}: hard-packed concentration of particles after centrifugation
				\item	{\color{purple}\textbf{Supernatant}}: clarified liquid above the precipitate
			\end{itemize}
	\end{columns}

	% RPM: Revolutions Per Minute (Speed).
	% 	Rmax: Maximum radius from the axis of rotation in centimeters.
	% 	Rmin: Minimum radius from the axis of rotation in centimeters.
	% 	RCF: Relative centrifugal Force. RCF = 11.17 x Rmax (RPM/1000)2
	% 	K-factor: Pelleting efficiency of a rotor. Smaller the K-factor, better the pelleting efficiency.
	%
	% 	S-value: the sedimentation coefficient is a number that gives information about the molecular weight and shape of the particle. S-value is expressed in Svedberg units. The larger the S-value, the faster the particle separates.
	% 	For more information about sedimentation coefficients, please refer to the section on references and suggested readings in this article.
	%
	% 	Pelleting time: time taken to pellet a given particle. T = K/S where T= pellet time in hours. K = K-factor of the rotor, and S = sedimentation coefficient (see Ghosh book for $S$)
\end{frame}

\begin{frame}\frametitle{Uses}
	Used since 1700's:
	\begin{enumerate}
		\item	separate particles from fluid based on density
		\item	separates immiscible fluids (liquid and even gases) of different densities
		\item	to enhance drainage of fluid from particles for drying
		\item	enhance mass transfer {\scriptsize (look at centrifugal packed bed contactors in your own time)}
	\end{enumerate}

	\vspace{12pt}
	Examples:
	\begin{itemize}
		\item	Cream from milk (milk is an emulsion)
		\item	Clarification: juice, beer (yeast removal), essential oils
		\item	Widely used in {\color{myGreen}{bioseparations}}: blood, viruses, proteins
		\item	Remove sand and water from heavy oils
	\end{itemize}
\end{frame}

\begin{frame}\frametitle{Interesting use: gas-gas separation}
	\begin{itemize}
		\item	\small Uranium enrichment in a \href{http://en.wikipedia.org/wiki/Zippe-type_centrifuge}{Zippe-type centrifuge}:  U-235 is only 1.26\% less dense than U-238: requires counter-current cascade
	\end{itemize}
	\begin{center}
		\includegraphics[height=0.75\textheight]{\imagedir/separations/centrifuge/Gas_centrifuge_cascade-wikipedia.jpg}
	\end{center}
	\vspace{-12pt}
	\see{http://en.wikipedia.org/wiki/File:Gas\_centrifuge\_cascade.jpg}
\end{frame}

\begin{frame}\frametitle{Principle of operation}

	\begin{itemize}
		\item	items being separated must have a \textbf{{\color{purple}density difference}}
		\item	it is not a difference in the mass, only density
			\begin{itemize}
				\item	\href{http://www.youtube.com/watch?v=M1vWI0t-XD8}{Video of emulsion separation} at high G's
			\end{itemize}
		\item	centrifugal force acts outward direction = $ma = m (r\omega^2)$
		\begin{itemize}
			\item	$m$ = particle's mass [kg]
			\item	$r$ = radial distance from center point [m]
			\item	$\omega$ = angular velocity = $\dfrac{d\theta}{dt}$  [$\text{rad.s}^{-1}$]
			\item	recall $2\pi~\text{rad.s}^{-1}$ = 1Hz
			\item	and $1~\text{rad.s}^{-1} \approx 9.55$ revolutions per minute [rpm]
			\item	$G = \displaystyle \frac{m r\omega^2}{mg} = \displaystyle \frac{r\omega^2}{g}$
		\end{itemize}
	\end{itemize}
	\vspace{12pt}
	\small
	\begin{tabular}{l|l|l}

		\textbf{Example}							& \textbf{Revolutions per minute}  & \textbf{G's}\\		 \hline
		Car going round and round 	    			& 10 to 15   			  & 1 to 2	\\
		Washing machine at home 					& 1500       			  & 625 (r=0.25m)	\\
		Industrial centrifuge						& $<15 000$ 			  & 25000 (r=0.1m)	\\
		Laboratory centrifuge 						& 30,000 to 100,000       & 100,000 to 800,000\\   % called ultracentrifugation
		Zippe-type centrifuge*						& 90,000                  & $\sim 1 \times 10^6$ \\ \hline
	\end{tabular}
	* tangential velocity $> \text{Mach 2} \sim 700 \text{m.s}^{-1}$
\end{frame}

\begin{frame}\frametitle{Laboratory centrifuges}

	\begin{center}
		\includegraphics[width=0.7\textwidth]{\imagedir/separations/centrifuge/lab-centrifuge-Ghosh.png}
	\end{center}
	\vspace{-12pt}
	Main selection factors:
	\begin{enumerate}
		\item	duration = $t$ [use minutes in the equation below]
		\item	maximum rotational speed = $\text{RPM}_\text{max}$ 	% there is a "warm-up" period to reach RPM_max, so the time
		 															% calculated below is an estimate when operating at RPM_max
	\end{enumerate}
	\vspace{-12pt}
	\begin{columns}[t]
		\column{0.40\textwidth}
			\[
				t = \displaystyle \frac{k}{S}
			\]

			\vspace{24pt}
			{\small e.g. $S_{20} = 6.43$ for collagen}
		\column{0.60\textwidth}
			\begin{itemize}
				\item	$S$ = \href{http://en.wikipedia.org/wiki/Svedberg}{Svedberg coefficient} of the material (from tables, experiments)
				\item	$r_\text{max}$ and $r_\text{min}$ as shown above [cm]
				\item	$k = 2.53 \times 10^{11} \left(\displaystyle \frac{\ln\left( r_\text{max} -r_\text{min} \right)}{\text{RPM}_\text{max}^2} \right)$
			\end{itemize}
	\end{columns}
\end{frame}

\begin{frame}\frametitle{Tubular bowl centrifuge}
	\begin{itemize}
		\item	Most commonly used for small particle separation
		\item	Fluid and suspended solids are fed at the center
		\item	A vertical wall of fluid is formed. \href{http://www.youtube.com/watch?v=cUhgKFV5Ri4}{Useful video to see this}.
		\item	Feed is continually added, forcing fluid out the top, over the retaining wall. Solids accumulate inside the bowl.
	\end{itemize}
	\begin{center}
		\includegraphics[width=\textwidth]{\imagedir/separations/centrifuge/trajectory-Geankoplis-fig14-4-2-coloured.jpg}
	\end{center}
	\vspace{-24pt}	
	\see{Geankoplis, Fig 14.4-2; \emph{draw top view}}
\end{frame}

\begin{frame}\frametitle{Recall particles in a fluid: Stokes' law}
	Let's understand how the solid particles move:
	\begin{exampleblock}{Recall  if Re $< 1$}
		\[
			v_\text{TSV}^\text{grav} = \frac{D_p^2 \left( \rho_p - \rho_f \right) g }{18 \mu_f} \label{TSV-low-Re}
		\]
	\end{exampleblock}
	In a centrifuge, we have simply replaced $g$ with a centrifugal force, $r \omega^2$ (gravity is negligible)
	\begin{exampleblock}{}
		\[
			v_\text{horiz}^\text{cent} = \frac{dr}{dt} = \frac{D_p^2\left(\rho_p - \rho_f\right)r\omega^2}{18 \mu_f}
		\]
	\end{exampleblock}
	\begin{itemize}
		\item	The particle is also forced in the \emph{vertical} direction of fluid flow at a constant upward velocity, so its net trajectory is curved.
		\item	In centrifuges: particles are likely to have Re $< 1$ (why?)
	\end{itemize}
\end{frame}

% Changes to make for 2013: IMPLEMENTED
%
% * Q_cut vs Q_max notation: in fact, Q_cut > Q_max, so Q_max is bad notation
% * Also, call it \Sigma_cut and \Sigma_max (but don't use max anymore). These correspond to their flowrates. Add this on the page "Cut-size diameter"
% * See red notebook for \Sigma_cut and \Sigma_max
% * Add an example (e.g. the last example)
%
% For reference: Q_max: integrating from r_1 to r_2 within length 0 to h to find the t_T time
%                Q_cut: integrating from r_1 to midpoint, within length 0 to h, to find the t_T time
% Make this point clear
% Emphasize that we could have used any other point, e.g. 0.8(r_2 - r_1): just that 50% is used by convention, everything beyond 50% is presumed to slip out in the overflow
% Accounts for uncertainties in our flows, physical properties and idealities assumed with Stokes' law


\begin{frame}\frametitle{Theoretical trajectories: tubular bowl centrifuge}
	\begin{center}
		\includegraphics[width=1\textwidth]{\imagedir/separations/centrifuge/trajectory-Geankoplis-fig14-4-2-coloured.jpg}
	\end{center}
	Integrate from $t=0$ where $r=r_1$ to the outlet, where we require the particle to be exactly at $r=r_2$ within a time of $t=t_*$ seconds:
	\[
		t_* = \frac{18 \mu_f}{D_p^2\left(\rho_p - \rho_f\right)\omega^2} \ln \frac{r_2}{r_1}
	\]
\end{frame}

\begin{frame}\frametitle{Theoretical trajectories: tubular bowl centrifuge}
	Consider a particle moving with too slow a horizontal velocity (e.g. centrifuge is too slow).
	\begin{itemize}
		\item	Within the time from $t=0$ to $t=t_*$, this particle is moving too slowly, and will not reach the wall at $r_2$ 
		\item	This particle is then assumed to have left in the supernatant (liquid discharge)
	\end{itemize}
	\begin{center}
		\includegraphics[width=0.5\textwidth, trim=0cm 0cm 20cm 4.7cm, clip=true]{\imagedir/separations/centrifuge/trajectory-Geankoplis-fig14-4-2-coloured.jpg}
	\end{center}
	\vspace{-6pt}
	$t=t_*$ gives a \textbf{bound} on the time it should take a particle to reach the wall at $r_2$, starting at $r_1$.
\end{frame}

\begin{frame}\frametitle{Calculating the centrifuge's throughput, $Q$}
	Once we know how long a particle should be in the centrifuge, we can calculate a feed flowrate, $Q$.
	The volume of fluid in the centrifuge is $V = \pi \left(r_2^2 - r_1^2\right) h$. Calculate the volumetric flow rate
	\[
		Q_* = \frac{V}{t_*} = \frac{D_p^2\left(\rho_p - \rho_f\right)\omega^2}{18 \mu_f \ln (r_2/r_1)} \pi \left(r_2^2 - r_1^2\right) h \qquad[\text{m}^3.\text{s}^{-1}]
	\]
	\begin{itemize}
		\item	What happens if we operate a flow rate slower/faster than this $Q_*$?
		\item	\emph{Alternative interpretation}: for a given flow $Q_*$, find the largest particle diameter that will arrive exactly at $r_2$ at height $h$. Particles with smaller $D_p$ are expected to leave in supernatant.
		\item	Obviously this is excessive: we have the horizontal discharge weir to \emph{retain particles} that might not have reached $r_2$ at height $h$
		\item	$r_2$ remains fixed for an purchased and installed centrifuge (design parameter)
	\end{itemize}
\end{frame}

\begin{frame}\frametitle{Cut-size diameter}
	So to prevent excessive over design, we rather find the halfway mark between $r_1$ and $r_2$, and solve the same equations to find the time, called $t_\text{cut}$, for a particle to reaches this {\color{purple}cut point}:
	\[
		Q_\text{cut} = \frac{V}{t_\text{cut}} = \frac{D_{p,\text{cut}}^2\left(\rho_p - \rho_f\right)\omega^2}{18 \mu_f \ln \left[2r_2/(r_1+r_2)\right]} \pi \left(r_2^2 - r_1^2\right) h
	\]
	\begin{itemize}
		\item	we design for the cut-point volumetric flow rate $Q_\text{cut}$
		\item	and can then solve for the cut point diameter, $D_{p,\text{cut}}$
		\item	all other terms in the equation are known/set
		\item	We can also design for a given diameter, and solve for the $Q_\text{cut}$.
	\end{itemize}
	\vspace{6pt}
	\hrule
	\vspace{6pt}
	{\small \textbf{Note}: We could use any reasonable point between $r_1$ and $r_2$. The 50\% point is convention. It accounts for uncertainties in our flows, physical properties and idealities assumed with Stokes' law.}
\end{frame}

\begin{frame}\frametitle{Example}
	A lab scale tubular bowl centrifuge has the following characteristics:
	\begin{itemize}
		\item	$r_1 = 16.5$ mm and $r_2 = 22.2$ mm
		\item	bowl height of 115 mm
		\item	800 revolutions per second
	\end{itemize}

	It is being used to separate bacteria from a fermentation broth experiment.
	
	
	If the broth has the following properties:
	\begin{itemize}
		\item	$\rho_f = 1010~\text{kg.m}^{-3}$   \hfill {\color{myOrange}{$\leftarrow$ note how close these are}}
		\item	$\rho_p = 1040~\text{kg.m}^{-3}$
		\item	$\mu_f = 0.001~\text{kg.m}^{-1}\text{.s}^{-1}$
		\item	$D_{p,\text{min}} = 0.7$ \micron   \hfill {\color{myOrange}{$\leftarrow$ note how small}}
	\end{itemize}
	\begin{enumerate}
		\item	How many G's is the particle experiencing at $r_2$?
		\item	Calculate both $Q_*$ and the more realistic $Q_\text{cut}$.
		\item	Verify whether Stokes' law applies.
		\item	What would be the area of the sedimentation vessel that would operate at this $Q_\text{cut}$? \emph{Hint}: recall that $A = \dfrac{Q}{v_{TSV}}$.
	\end{enumerate}
		
		%
		%\item	
		%\item	\emph{Later}: calculate the $\Sigma$ factor for this centrifuge based on the cut-size flowrate.
\end{frame}

\begin{frame}\frametitle{Example}
	\begin{enumerate}
		\item	Illustrate the trajectory taken by a particle reaching the cut-point within time $t_\text{cut}$
		\item	\emph{In the same duration of time}, what trajectory will a smaller particle have taken?
	\end{enumerate}
	%\vfill
	\vspace{10cm}
	%s
\end{frame}

\begin{frame}\frametitle{Sigma theory for centrifuges}
	Take the previous equation for $Q_\text{cut}$, multiply numerator and denominator by $2g$, then substitute Stokes' law for particles settling under gravity:
	\[
		v_\text{TSV}^\text{grav} = \frac{\left( \rho_p - \rho_f \right) g D_p^2}{18 \mu_f}
	\]
	we obtain:
	\[
		Q_\text{cut} =  \left( \frac{\left( \rho_p - \rho_f \right) g D_{p,\text{cut}}^2}{18 \mu_f}  \right) \cdot ( \Sigma ) =  v_\text{TSV}^\text{grav} \cdot \Sigma
	\]
	\[
		\Sigma = \frac{\omega^2 \left[\pi h \left(r_2^2 - r_1^2\right)\right]}{g \ln \left[2r_2/(r_1+r_2) \right]}
	\]

	\vspace{12pt}
	$\Sigma = f(r_1, r_2, h, \omega)$
	%\[
	%	\Sigma \approx \frac{\omega^2}{2g} \pi h \left(1.5r_2^2 + 0.5r_1^2 \right)
	%\]
	%which is a simplification of the true $\Sigma$, but only has about 4\% error.
\end{frame}

\begin{frame}\frametitle{Why use the Sigma term?}
	\begin{itemize}
		\item	$\Sigma = f(r_1, r_2, h, \omega)$
		\item	it is only a function of the centrifuge's characteristics; not the particle or fluid
		\item	$\Sigma$ has units of $\text{m}^2$: {\color{purple}$\Sigma$ is the equivalent surface area} required for sedimentation by \emph{gravity}
		\item	Centrifuge A: $Q_{\text{cut},A} = v_\text{TSV}^\text{grav} \cdot \Sigma_A$
		\item	Centrifuge B: $Q_{\text{cut},B} = v_\text{TSV}^\text{grav} \cdot \Sigma_B$
		\item	\[
					\frac{Q_{\text{cut},A}}{Q_{\text{cut},B}}  = \frac{\Sigma_A}{\Sigma_B}
				\]
		\item	Used for scale-up \textbf{of the same feed}, i.e. the same $v_\text{TSV}^\text{grav}$
		\item	Used for scale-up \textbf{within the same types of equipment}
		\item	$\Sigma$ equation is different for other centrifuge types
		\item	\emph{Question}: if I know $\Sigma_A$ for a given centrifuge and for a given feed; can I calculate the performance, $Q_{\text{cut},B}$, for a different feed stream?
	\end{itemize}
\end{frame}

\begin{frame}\frametitle{More on the tubular bowl centrifuge}
	\begin{itemize}
		\item	Batch operation: stop to clean out solids; restart again; use paper on wall to assist solids removal [$\sim$ 15 min turnaround]
		\item	Contamination possible, \emph{not always} suitable for bioseparations
		\item	A high L/D aspect ratio is used (around 8), as it is more stable to operate
		\item	Minimize D; very high wall stresses are developed at higher diameters
		\item	Can be used for fluid-fluid separation
	\end{itemize}
	\begin{columns}[c]
		\column{0.60\textwidth}
			\begin{center}
				\includegraphics[width=0.7\textwidth]{\imagedir/separations/centrifuge/liquid-liquid-Geankoplis-fig14-4-3.jpg}
			\end{center}
		\column{0.40\textwidth}
				$\displaystyle \frac{\rho_H}{\rho_L} = \displaystyle \frac{r_2^2 - r_1^2}{r_2^2 - r_4^2}$
	\end{columns}
\end{frame}

\begin{frame}\frametitle{Disk-bowl (disk stack) centrifuges}
	\begin{center}
		\includegraphics[width=.75\textwidth]{\imagedir/separations/centrifuge/disk-bowl-centrifuge-Geankoplis-fig14-4-4.jpg}
	\end{center}
	\see{Geankoplis, Fig 14.4-4}

	\vspace{0pt}
	Video to illustrate operation: \href{http://www.youtube.com/watch?v=YMbaBLpInrc}{http://www.youtube.com/watch?v=YMbaBLpInrc}

	\vspace{2pt}
	{\tiny Another video: \href{http://www.youtube.com/watch?v=bzXUiLajVlg}{http://www.youtube.com/watch?v=bzXUiLajVlg}}

\end{frame}

\begin{frame}\frametitle{Disk-bowl centrifuges}
	\begin{itemize}
		\item	Recall: $Q = V / t_*$ (the $t_*$ will be different for disk-bowl compared to tubular bowl)
		\item	If we increase rate of fluid feed, we get higher throughput, $Q$
		\item	Adding angled disks gives a greater surface area, hence greater volume treated, without increasing bowl diameter
		\item	Widely used in bioseparations: no contamination (aseptic)
		\item	Also for: fish oil, fruit juice, beverage clarification
		\item	3-phases separation: e.g. sand, oil, water mixtures
	\end{itemize}
\end{frame}

\begin{frame}\frametitle{Disk-bowl centrifuges}
	\begin{itemize}
		\item	Disks angled at 35 to 50$^\circ$; $\sim$~50 to 150 disks per unit
		\item	Typically between 0.15 to 1.0m in diameter; with rotational speeds of 0 to 12,000 rpm
		\item	Typically used to treat up to 15\% solids in feed stream
		\item	Can be operated continuously (infrequent cleaning of disks)
	\end{itemize}
	\vspace{-12pt}
	\begin{columns}[c]
		\column{0.50\textwidth}
			$\Sigma = \displaystyle \frac{2\pi\omega^2 N(r_1^3 - r_2^3)}{3g \tan \theta}$
			\vspace{24pt}
			\begin{itemize}
				\item	$N$ = number of disk plates
				\item	$\theta$ = angle of disks
				\item	$r_1$ = outer cone diameter
				\item	$r_2$ = inner cone diameter
			\end{itemize}
		\column{0.55\textwidth}
			\begin{center}
				\includegraphics[width=.7\textwidth]{\imagedir/separations/centrifuge/disk-bowl-notation-Svarovsky.jpg}
			\end{center}
	\end{columns}
\end{frame}

% \begin{frame}\frametitle{Decanter centrifuges}
% 	Good for large amount of solids 
% 	Used when mixed with slurry.  
% 	Energy saving operation can be implemented.  	High maintenance costs 
% 	Overall higher operating costs
% 	Subject to mechanical failure.
% 	
% 	Oil refinery continuous separators
% 	
% \end{frame}

% \begin{frame}\frametitle{Capital costs}
% 	COSTS: also see chapter 18.8 in Perry's
% \end{frame}
% 
% \begin{frame}\frametitle{Performance}
% 	\see{Perry, 8ed, section 18.8.3}
% 
% 	In order to have good separation or high settling velocity, a combination of the following conditions is generally sufficient:
% 
% 	1. High centrifuge speed
% 
% 	2. Large particle size
% 
% 	3. Large density difference between solid and liquid
% 
% 	4. Large separation radius
% 
% 	5. Low liquor viscosity
% 
% 	% Among the five parameters, the settling velocity is very sensitive to change in speed and particle size. It varies as the square of both parameters. The maximum achievable rotational speed of a centrifuge is normally dictated by the stresses exerted by the processing medium on the bowl and the stresses of the bowl on periphery equipment, most notably the drive system, which consists of a gear unit or hydraulic pump. If the particles in the feed slurry are too small to be separated in the existing G-field, coagulation and flocculation by polymers are effective ways to create larger agglomerated particles for settling. Unlike separation under a constant gravitational field, the settling velocity under a centrifugal field increases linearly with the radius. The greater the radius at which the separation takes place in a given centrifuge at a given rotational speed, the better the separation. Sedimentation of particles is favorable in a less viscous liquid. Some processes are run under elevated temperature where liquid viscosity drops to a fraction of its original value at room temperature.
% 
% \end{frame}
% 
% \begin{frame}\frametitle{Flowsheet layout}
% 	Series or parallel? Svarovsky, p 274
% \end{frame}

\begin{frame}\frametitle{Scroll centrifuges}
	The scroll allows for continuous removal of solids:
	\begin{center}
		\includegraphics[width=\textwidth]{\imagedir/separations/centrifuge/Sedicanter-centrifuge-Perry-Fig-18-159.png}
	\end{center}
	\see{Perry, fig 18-159}
	\begin{itemize}
		\item	Sedicanter: biotechnology, vitamin, soy, and yeast separations.
	\end{itemize}
\end{frame}

\begin{frame}\frametitle{Scroll centrifuges}
	\begin{center}
		\includegraphics[width=\textwidth]{\imagedir/separations/centrifuge/Sorticanter-centrifuge-Perry-Fig-18-160.png}
	\end{center}
	\see{Perry, fig 18-160}
	\begin{itemize}
		\item	Sorticanter: used for plastics recycling
		\item	General scroll centrifuges: used in oil-sands separations
	\end{itemize}
\end{frame}

\begin{frame}\frametitle{Sequencing of centrifuges}
	\begin{center}
		\includegraphics[width=\textwidth]{\imagedir/separations/centrifuge/oil-sands-mining-Westfalia.png}
	\end{center}
	\see{\href{http://www.westfalia-separator.com/products/innovations/oil-sand-bitumen-process.html}{http://www.westfalia-separator.com/products/innovations/oil-sand-bitumen-process.html}}
\end{frame}

\begin{frame}\frametitle{Safety}
	\begin{itemize}
		\item	careful selection of materials of construction: corrosion and withstand high forces
		\item	heat removal might be required (some units come with integrated refrigeration)
		\item	rotational equipment requires careful balance
		\item	digital control is critical
		\begin{itemize}
			\item	\href{http://en.wikipedia.org/wiki/Programmable_logic_controller}{PLC}: programmable logical controllers
			\item	\href{http://en.wikipedia.org/wiki/SCADA}{SCADA}: supervisory control and data acquisition
			\item	safety interlocks
			\item	cameras are increasingly used to monitor sediment buildup: auto-stop and clean
		\end{itemize}
		\item	flammable fluids (e.g. solvents): nitrogen blanket
	\end{itemize}
\end{frame}

\begin{frame}\frametitle{Choosing a centrifuge unit}
	\begin{center}
		\includegraphics[width=\textwidth]{\imagedir/separations/centrifuge/centrifuge-choice-Schweitzer.png}
	\end{center}
	\see{Schweitzer, p 4-58}
\end{frame}

\begin{frame}\frametitle{Selecting a centrifuge}
	Based on required performance
	\begin{center}
		\includegraphics[width=\textwidth]{\imagedir/separations/centrifuge/centrifuge-selection-CRv6-p420.png}
	\end{center}
\end{frame}

\begin{frame}\frametitle{Design a centrifuge for beer clarification}
	Design a separation plant to remove suspended yeast cells from beer.

	\vspace{12pt}
	Beer is produced in batches of 100 $\text{m}^3$, with 4 batches per day.

	\vspace{12pt}
	Some data:
	\begin{itemize}
		\item	Density of beer: 1020 $\text{kg.m}^{-3}$
		\item	Density of yeast cells: 1075 $\text{kg.m}^{-3}$
		\item	Yeast cell diameters: 4 to 6 \micron
		\item	11.5 metric tonnes of yeast are suspended in each 100 $\text{m}^3$ fermenter
		\item	Aseptic operation is vital
	\end{itemize}
\end{frame}

\begin{frame}\frametitle{Further practice questions}
	\begin{enumerate}
		\item	In a test particles of density $2800~\text{kg.m}^{-3}$ and of size 5 \micron, equivalent spherical diameter, were separated from suspension in water fed at a volumetric throughput rate of $0.25~\text{m}^{3}\text{.s}^{-1}$. Calculate the value of the capacity factor, $\Sigma$. {\color{myOrange}\tiny{[\emph{Ans}: $\Sigma = 1.02 \times 10^4~\text{m}^2$]}}
		
		\vspace{12pt}
		\item	What will be the corresponding size cut for a suspension of coal particles in oil fed at the rate of $0.04~\text{kg.s}^{-3}$? The density of coal is $1300~\text{kg.m}^{-3}$ and the density of the oil is $850~\text{kg.m}^{-3}$ and its viscosity is $0.01~\text{N.s.m}^{-2}$. {\color{myOrange}\tiny{[\emph{Ans}: $D_\text{p,cut} = 4$\micron]}}
		
		\vspace{12pt}
		\item	Is Stokes' law applicable?  {\color{myOrange}\tiny{[\emph{Ans}: Calculate the $v_\text{TSV}^\text{cent}$ and confirm if Re$<1$]}}
	\end{enumerate}
	\see{Richardson and Harker, v2, 5th ed, p482-483}
\end{frame}
