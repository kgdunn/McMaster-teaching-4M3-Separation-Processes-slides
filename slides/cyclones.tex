% Was supposed to be 21 September 2012; realistically, only 24 September
\begin{comment}
\begin{frame}\frametitle{}
	See: http://www.nzifst.org.nz/unitoperations/, UNIT OPERATIONS IN FOOD PROCESSING

	http://www.nzifst.org.nz/unitoperations/mechseparation3.htm

	Cyclones are often used for the removal from air streams of particles of about 10 mm or more diameter. They are also used for separating particles from liquids and for separating liquid droplets from gases. The cyclone is a settling chamber in the form of a vertical cylinder, so arranged that the particle-laden air spirals round the cylinder to create centrifugal forces which throw the particles to the outside walls. Added to the gravitational forces, the centrifugal action provides reasonably rapid settlement rates. The spiral path, through the cyclone, provides sufficient separation time. A cyclone is illustrated in Fig. 10.2(a).


	Figure 10.2 Cyclone separator: (a) equipment (b) efficiency of dust collection


	Stokes' Law shows that the terminal velocity of the particles is related to the force acting. In a centrifugal separator, such as a cyclone, for a particle, rotating round the periphery of the cyclone:
	           Fc = (mv2)/r                                                                                                         (10.4)

	where Fc is the centrifugal force acting on the particle, m is the mass of the particle, v is the tangential velocity of the particle and r is the radius of the cyclone.

	This equation shows that the force on the particle increases as the radius decreases, for a fixed velocity. Thus, the most efficient cyclones for removing small particles are those of smallest diameter. The limitations on the smallness of the diameter are the capital costs of small diameter cyclones to provide sufficient output, and the pressure drops.

	The optimum shape for a cyclone has been evolved mainly from experience and proportions similar to those indicated in Fig. 10.2(a) have been found effective. The efficient operation of a cyclone depends very much on a smooth double helical flow being produced and anything which creates a flow disturbance or tends to make the flow depart from this pattern will have considerable and adverse effects upon efficiency. For example, it is important that the air enters tangentially at the top. Constricting baffles or lids should be avoided at the outlet for the air.

	The efficiency of collection of dust in a cyclone is illustrated in Fig. 10.2(b). Because of the complex flow, the size cut of particles is not sharp and it can be seen that the percentage of entering particles which are retained in the cyclone falls off for particles below about 10 mm diameter. Cyclones can be used for separating particles from liquids as well as from gases and also for separating liquid droplets from gases.
\end{frame}
\end{comment}

\begin{frame}\frametitle{Administrative}
	Please ensure assignment is a ``pure'' Google Doc
	\begin{itemize}
		\item	Please do not share a Word or PDF file
		\item	We cannot comment on those
		\item	File name: \texttt{2012-4M3-Lastname1-Lastname2-Assignment-NN}
	\end{itemize}
\end{frame}

\begin{frame}\frametitle{The cyclone}
	\vspace{-24pt}
	\begin{columns}[b]
		\column{0.50\textwidth}
			\begin{center}
				\includegraphics[width=0.70\textwidth]{\imagedir/separations/cyclones/hydrocyclone-CRv2-5ed-p50.png}
				Hydrocyclone
			\end{center}
		\column{0.50\textwidth}
			\begin{center}
				Cyclone
				\vspace{12pt}
				
				\includegraphics[width=\textwidth]{\imagedir/separations/cyclones/cyclone-separator-CRv2-5ed-p74.png}
			\end{center}
	\end{columns}
\end{frame}

\begin{frame}\frametitle{Uses}
	\begin{itemize}
		\item	dust removal (principal application)
		\item	mist (droplets) removed from air streams
		\item	recovery of spray-dried particles
	\end{itemize}
\end{frame}

\begin{frame}\frametitle{Principle of operation}
	\begin{itemize}
		\item	Same principle as a centrifuge
		\item	Except {\color{myOrange}{no moving parts}}!
		\item	Very low operating costs: essentially only pay for $\Delta P$
		\item	Particle sizes 5 \micron and higher are effectively removed
	\end{itemize}
	
	\begin{columns}[t]
		\column{0.60\textwidth}
			\todo{picture here}
		\column{0.40\textwidth}
			\begin{itemize}
				\item	Cylindrical section
				\item	Conical section
				\item	Vortex finder
				\item	Spigot (where solids leave)
			\end{itemize}
	\end{columns}	
\end{frame}

\begin{frame}\frametitle{General path of travel in a cyclone}
	\begin{columns}[c]
		\column{0.60\textwidth}
			\begin{center}
				\includegraphics[height=.9\textheight]{\imagedir/separations/cyclones/flow-pattern-CRv6-p451.png}
			\end{center}
		\column{0.40\textwidth}
			However, it is more complex than this. 
			
			See, for example, \href{http://www.youtube.com/watch?v=CWLARs\_dJO0}{this video of a PET scan} of a radioactive isotope labelled particle $^{18}F$
			
			\vspace{12pt}
			Vortex and tangential forces formed by the fluid
	\end{columns}
\end{frame}

\begin{frame}\frametitle{Velocity profile}
	\begin{center}
		\includegraphics[width=\textwidth]{\imagedir/separations/cyclones/cyclone-velocity-profile-CRv2-5ed-p53.png}
	\end{center}
\end{frame}

\begin{frame}\frametitle{Grade efficiency}
	
\end{frame}

\begin{frame}\frametitle{Pressure drop}
	p 453 C+R v6
\end{frame}

\begin{frame}\frametitle{Design considerations: HYDROcyclone}
	C+R v2, 5ed, p54
	Design Considerations
	Hydrocyclones are 20 to500 mm in diameter, with the smaller units giving a much better separation. Typical values of length to diameter ratios range from about 5 to 20. Because of the very high shearing stresses which are set up, flocs will be broken down and the suspension in the secondary vortex will be completely deflocculated, irrespective of its condition on entry. Generally, hydrocylones are not effective in removing particles smaller than about 2to 3 \micron.
%	Because separating power is greatest in hydrocyclones of small diameter, the cut size being approximately proportional to the diameter of the cylindrical shell raised to the power of 1.5, it is common practice to operate banks of small hydrocyclones in parallel inside a large containing vessel. Furthermore, this procedure also makes scale-up easier to carry out. With units connected only in parallel (or as single units), however, it is not possible to control the compositions of the overflow and underflow independently, and therefore some form of series-parallel operation is often employed. This is an important consideration when the hydrocyclone is used for thickening a suspension. In this case, there is the requirement to produce an underflow of the appropriate solids concentration and for the overflow to be particle-free, and these two conditions cannot usually be satisfied in a single unit. In thickening, the particle concentrations are high and little classification by size occurs.
%	In general, the performance of the hydrocyclone is improved by increasing the operating pressure, and the principal control variable is the size of the orifice on the underflow discharge. Several theoretical and practical studies have been made in an attempt to present a sound basis for design, and these have been described BRADLEY(29) and SAVROVSKY(30,35) among others, although they are generally not entirely satisfactory, and some design charts and formulae have been given by ZANKER(40). In practice, tests with the actual materials to be used are desirable for the evaluation for the various parameters. Since systems are usually scaled-up by increasing the number of units in parallel, it is seldom necessary to carry out tests on large units.
%	The optimum design of a hydrocyclone for a given function depends upon reconciling a number of conflicting factors and reference should be made to specialist publications. Because it is simple in construction and has no moving parts, maintenance costs are low. The chief problem arises from the abrasive effect of the solids; materials of construction, such as polyurethane, show less wear than metals and ceramics.
\end{frame}

\begin{frame}\frametitle{Design}
	p 455 C+R v6
	General design procedure
	1. Select either the high-efficiency or high-throughput design, depending on the perfor- mance required.
	2. Obtain an estimate of the particle size distribution of the solids in the stream to be treated.
	3. Estimate the number of cyclones needed in parallel.
	4. Calculate the cyclone diameter for an inlet velocity of 15 m/s (50 ft/s). Scale the
	other cyclone dimensions from Figures 10.44a or 10.44b.
	5. Calculate the scale-up factor for the transposition of Figures 10.45a or 10.45b.
	6. Calculate the cyclone performance and overall efficiency (recovery of solids). If
	unsatisfactory try a smaller diameter.
	7. Calculate the cyclone pressure drop and, if required, select a suitable blower.
	8. Cost the system and optimise to make the best use of the pressure drop available,
	or, if a blower is required, to give the lowest operating cost.
\end{frame}

\begin{frame}\frametitle{}
	C+R v6, p404
	Liquid cyclones can be used for the classification of solid particles over a size range from 5 to 100\micron. Commercial units are available in a wide range of materials of
	
	EQUIPMENT SELECTION, SPECIFICATION AND DESIGN 405
	construction and sizes; from as small as 10 mm to up to 30 m diameter. The separating efficiency of liquid cyclones depends on the particle size and density, and the density and viscosity of the liquid medium.
\end{frame}

\begin{frame}\frametitle{Merits and disadvantages}
	p 212 Svarovsky
\end{frame}

\begin{frame}\frametitle{Design}
	d50 design: C+R v6, p 423
\end{frame}

\begin{frame}\frametitle{Design}
	p 426 C+R v6
\end{frame}

\begin{frame}\frametitle{Design}
	p 450 C+R v6
\end{frame}

\begin{frame}\frametitle{Effect of solids flow rate}
	http://www.youtube.com/watch?v=BicR3JGlE5M
\end{frame}

\begin{frame}\frametitle{Cyclone as a classifier, not a separator}
	We seldom have pure streams leaving a cyclone
	
	Fig 6.6 Svarovsky
\end{frame}

\begin{frame}\frametitle{Circuits of cyclones}
	How is the flowsheet designed?

	Fig 6.15
	Fig 6.18
\end{frame}

\begin{frame}\frametitle{}
	Figre 6.19
\end{frame}

\begin{frame}\frametitle{Typical dimensions}
	How are they designed?
	Scaled-up?	
\end{frame}

\begin{frame}\frametitle{Operational issue}
	Pressure drop
	Electrical costs to operation
\end{frame}

\begin{frame}\frametitle{Capital costs}
	Is a function of ...
	\begin{itemize}
		\item	pressure rating (ASME conformance)
	\end{itemize}
\end{frame}

\begin{frame}\frametitle{References}
	* Svarovsky3, Ch 6
	* C+R v2, page 73
\end{frame}

\begin{frame}\frametitle{References}
	\begin{itemize}
		\item	Richardson and Harker, ``Chemical Engineering, Volume 2'', 5th edition, Chapter XXXX
		\item	Sinnot, ``Chemical Engineering, Volume 6'', XXXXth edition, Chapter XXXX
		\item	Perry's, ``Chemical Engineers' Handbook'', 8th edition, Chapter XXXX
		\item	Svarovsky, ``Solid Liquid Separation'', 3rd or 4th edition, Chapter 6. Clearly written.
		\item	Seader et al. ``Separation Process Principles", page XXXXX in 3rd edition 
		\item	Schweitzer, ``Handbook of Separation Techniques for Chemical Engineers'', Chapter XXXX
	\end{itemize}
\end{frame}

\begin{frame}\frametitle{Selection}
	\begin{center}
		\includegraphics[width=\textwidth]{\imagedir/separations/mechanical-selection/selecting-based-on-particle-size-Svarovsky-4ed-p41.png}
	\end{center}
\end{frame}

\begin{frame}\frametitle{Circuit selection}
	Crushing circuit: to separate fines from coarse: recycle coarse material for recrushing. Could also use a screen. Why use a cyclone instead?
	Also: could use elutriation with air rather than screens or cyclones. Why would you use elutriation?

	Use: to clean dust from air; generally 5 \micron and larger
	Use: mists separated from gas
	U
	A cyclone is a centrifugal elutriator, although it is not usually so regarded. The cyclosizer is a series of inverted cyclones with added apex chambers through which water flows. Suspension is fed into the largest cyclone, and particles are separated into different size ranges.	
\end{frame}
