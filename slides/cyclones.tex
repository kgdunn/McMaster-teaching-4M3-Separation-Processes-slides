% For 2013
% \begin{frame}\frametitle{Venturi systems}
% 	\todo{Add this topic in 2013: does it fit here?}
% \end{frame}

\begin{frame}\frametitle{References*}
	\begin{itemize}
		\item	Svarovsky, ``Solid Liquid Separation'', 3rd or 4th edition, chapter 6.% Clearly written.
		\item	Richardson and Harker, ``Chemical Engineering, Volume 2'', 5th edition, chapter 1.
		\item	Sinnot, ``Chemical Engineering, Volume 6'', 4th edition, chapter 10.
		\item	\href{http://accessengineeringlibrary.com/browse/perrys-chemical-engineers-handbook-eighth-edition}{Perry's Chemical Engineers' Handbook}, 8th edition, chapter 17.2, ``Gas-Solid Separations''
		\item	Schweitzer, ``Handbook of Separation Techniques for Chemical Engineers'', chapter 4-135.
	\end{itemize}
	
	\vspace{1cm}
	\small{$^\ast$ Most of the illustrations are taken from these references.}
\end{frame}

\begin{frame}\frametitle{The (hydro)cyclone}
	\vspace{-24pt}
	\begin{columns}[b]
		\column{0.50\textwidth}
			\begin{center}
				\includegraphics[width=0.70\textwidth]{\imagedir/separations/cyclones/hydrocyclone-CRv2-5ed-p50.png}
			\end{center}
			
		\column{0.50\textwidth}
			\begin{center}
				\vspace{12pt}
				\includegraphics[width=\textwidth]{\imagedir/separations/cyclones/cyclone-separator-CRv2-5ed-p74.png}
			\end{center}
	\end{columns}
\end{frame}

\begin{frame}\frametitle{Uses}
	Wide variety of uses:
	\begin{itemize}
		\item	dust removal (principal application) in many industries
			\begin{itemize}
				\item	cement industry
				\item	sawmills
				\item	catalyst particle recovery in reactors
			\end{itemize}
		\item	mist (droplets) removed from air streams
		\item	recovery of spray-dried particles
		\item	separating immiscible liquids (different densities)
		\item	dewater suspensions: concentrate the product
		\item	remove dissolved gases from liquid stream
		\item	solids-solids separation: very common in mining
	\end{itemize}

	\vspace{12pt}
	\begin{exampleblock}{}
		Where possible, consider a cyclone before a centrifuge for solid-fluid separations.
	\end{exampleblock}
\end{frame}

\begin{frame}\frametitle{Alternatives}
	A number of alternatives exist; based on the principle of removing the particle's momentum relative to the fluid's momentum. {\color{myBlue}{Other options?}}

	\begin{center}
		\includegraphics[width=0.75\textwidth]{\imagedir/separations/cyclones/cyclone-alternative-CRv2-5ed-p81.png}
	\end{center}
	\see{Richardson and Harker, 5ed, p81}
	% Spray washer and scrubbers
\end{frame}

% \begin{frame}\frametitle{Alternatives}
% 	% Electrostatic precipitator
% 	% Bag house
% 	% Venturi
%
% 	See projects from 2013:
% 	electrostatic: electrical costs and safety and more capital intensive; low pressure drop
% 	bag house: warm-up period as the cloth clogs; so low efficiency initially, but cheaper
% 	how could you combine these two?
% 	# See project from Mistry and Salih: good overall description of the units, adv/disadv
%
% 	Venturi: accelerate gas through a throat, with water droplets; force collision
% \end{frame}


\begin{frame}\frametitle{Cyclone operation}
	\begin{center}
		\includegraphics[width=0.62\textwidth]{\imagedir/separations/cyclones/cyclone-Brown-and-Associates-p119.png}
	\end{center}
	\see{Brown and Associates, Unit Operations, p 119}
\end{frame}

\begin{frame}\frametitle{General path of travel in a cyclone {\color{myGreen}{{\scriptsize low viscosity, low solids concentration}}}}
	\begin{columns}[c]
		\column{0.60\textwidth}
			\begin{center}
				\includegraphics[height=.9\textheight]{\imagedir/separations/cyclones/flow-pattern-CRv6-p451.png}
			\end{center}
		\column{0.40\textwidth}
			Generally, flow pattern is more complex than this.

			\vspace{12pt}
			See, for example, \href{http://www.youtube.com/watch?v=CWLARs\_dJO0}{this video of a PET scan} of a radioactive isotope labelled particle $^{18}F$

			\vspace{12pt}
			\begin{itemize}
				\item	Vortex and tangential forces formed by the fluid
			\end{itemize}
	\end{columns}
\end{frame}

\begin{frame}\frametitle{Principle of operation}
	\begin{itemize}
		\item	Same principle as a centrifuge: \textbf{{\color{myGreen}{density difference required}}}
		\item	{\color{myOrange}{No moving parts!}} and {\color{myOrange}{no consumable components!}}
		\item	Low operating costs: essentially only pay for $\Delta P$
		\item	Operated at many temperatures and pressures  % Perry's
		\item	As small as 1 to 2cm to 10m in diameter % Perry's
		\item	Very low capital costs: can be made from many materials
		\item	Particle sizes 5\micron\, and higher are effectively removed  % Perry's
		\item	Even different particle shapes (due to different settling velocities) can be separated
		\item	Forces acting on particles: between 5 (large cyclones) and 2500 G (small cyclones)  % Perry's
	\end{itemize}
	\vspace{8pt}
	Videos:
	\begin{itemize}
		\item	\href{http://www.youtube.com/watch?v=2bUlytvimy4}{http://www.youtube.com/watch?v=2bUlytvimy4}
		\item	\href{http://www.youtube.com/watch?v=GxA49uVP2Ns}{http://www.youtube.com/watch?v=GxA49uVP2Ns}
		\item	\href{http://www.youtube.com/watch?v=BicR3JGlE5M}{http://www.youtube.com/watch?v=BicR3JGlE5M}
		\item	\href{http://www.youtube.com/watch?v=QfTZUMq-LGI}{http://www.youtube.com/watch?v=QfTZUMq-LGI}
		\item	and many other videos of people making their own cyclones.
	\end{itemize}
\end{frame}

\begin{frame}\frametitle{Velocity profile: \emph{very complex}}
	\begin{columns}[c]
		\column{0.60\textwidth}
			\begin{center}
				\includegraphics[width=1.05\textwidth]{\imagedir/separations/cyclones/cyclone-velocity-profile-CRv2-5ed-p53.jpg}
			\end{center}
		\column{0.5\textwidth}
			{\color{myOrange}{\textbf{3 directions of travel:}}}
			\begin{enumerate}
				\item	LZVV = locus of zero \textbf{vertical} velocity (axial~$\updownarrow$)
				\item	\textbf{radial} velocity is small ($\longleftrightarrow$)
				\item	\textbf{tangential} velocity
					\begin{itemize}
						\item	$v_t r^n$ = constant
						\item	true at all heights inside cyclone
					\end{itemize}
			\end{enumerate}
			\hrule
			\begin{itemize}
				\item	centrifugal force (acts $\longrightarrow$)
				\item	drag force (acts $\longleftarrow$)
				\item	if $F_\text{centrifugal} > F_\text{drag}$\\ particle moves towards wall
				\item	then pulled down in axial stream and exits in underflow
			\end{itemize}
	\end{columns}
\end{frame}

\begin{frame}\frametitle{Velocity profiles}
	\begin{exampleblock}{}
		The above description is extremely simplistic; velocity profiles cannot be theoretically derived for most practical cases.
	\end{exampleblock}

	\begin{itemize}
		\item	it is \textbf{not} \emph{gravity} that removes the heavier particles in underflow
		\item	it is the slower, boundary layer flow at the walls and air flow out of the spigot
		\item	particles rotate at a radius where centrifugal force is balanced by drag force (recall {\color{purple}elutriation} concept)
		\item	larger, denser particles move selectively towards the wall
		\item	residence time must be long enough to achieve equilibrium orbits; spiral patterns help
		\item	all of this comes down to a careful balance of radial and tangential velocities
		\item	velocities: these are our degrees of freedom to adjust the cyclone's performance
		%\item	models use Stokes' law: does this seem right?
		%\item	entry velocities: 5 m.s$^{-1}$ or often greater for solid-gas separation
	\end{itemize}
	% Good description: Svarovsky, 3rd ed, p 207.
\end{frame}

\begin{frame}\frametitle{Evaluating a cyclone's performance}
	\begin{center}
		\includegraphics[width=\textwidth]{\imagedir/separations/cyclones/cyclone-psd-Svarovsky.png}
	\end{center}

	\vspace{-64pt}
	Mass balance: $\text{M} = \text{M}_\text{f} + \text{M}_\text{c}$
	\begin{itemize}
		\item	both overall balance
		\item	and within each size fraction
	\end{itemize}
	
	\vfill
	\see{Svarovsky, 3ed, p210}	% We never have pure streams leaving a cyclone
\end{frame}

\begin{frame}\frametitle{{\color{myOrange}{Concept}}: Grade efficiency}
	\begin{exampleblock}{Total efficiency}
		\[E_T = \displaystyle \frac{\text{M}_\text{c}}{\text{M}} = 1 - \displaystyle \frac{\text{M}_\text{f}}{\text{M}}\]
	\end{exampleblock}
	\begin{itemize}
		\item	not too much to interpret here: it is just a definition
		\item	0\% efficiency: all mass is being sent to overflow (fines) stream
		\item	100\% efficiency: all mass to underflow (coarse) stream
	\end{itemize}
	\vspace{12pt}
	More useful though is
	\begin{exampleblock}{Grade efficiency}
		\[G(x) = \displaystyle \frac{(\text{M}_\text{c})(\text{fraction of size $x$ in stream C, coarse stream})}{(\text{M})(\text{fraction of size $x$ in feed})}\]
	\end{exampleblock}
	\begin{itemize}
		\item	calculated for a given particle size fraction $x$; repeat at all $x$'s
	\end{itemize}
\end{frame}

\begin{frame}\frametitle{What is ``fraction of size $x$'' again?}
	\begin{center}
		\includegraphics[width=0.7\textwidth]{\imagedir/separations/screens/particle-size-distribution.png}
	\end{center}
	Percentage area under the (differential) curve, at size fraction $x$.
\end{frame}

\begin{frame}\frametitle{Back to grade efficiency}
	\begin{exampleblock}{Grade efficiency equation}
		$G(x) = \displaystyle \frac{(\text{M}_\text{c})(\text{fraction of size $x$ in stream C, coarse stream})}{(\text{M})(\text{fraction of size $x$ in feed})}$
	\end{exampleblock}
	\begin{itemize}
		\item	If $G(x) = 0.5$ (50\%): implies half the material (by mass) in size fraction $x$ is leaving in the underflow (coarse)
		\item	and the other half in the overflow; 50-50 (mass) split in the two outlets for particles of size $x$. Called the ``{\color{purple}{cut size}}'', $x_{50}$
		\item	If $G(x) = 1.0$: implies the particle size that gets captured 100\%  in the coarse (underflow) stream
		\item	Where $G(x)$ reaches 1.0 means the $x=$ \emph{largest particle size} we expect to ever see in {\color{myOrange}overflow} {\small (see next slide)}
		\item	\adv What would $G(x \rightarrow 0)$ = 10\% mean? \\{\scriptsize [i.e. the $G(x)$ curves don't always reach 0\%]}
	\end{itemize}
\end{frame}

\begin{frame}\frametitle{Grade efficiency curve}
	Calculate efficiency at each size fraction, $x$, and plot it:
	\begin{center}
		\includegraphics[width=0.85\textwidth]{\imagedir/separations/cyclones/grade-efficiency-curve-CRv2-5ed-p78.png}
	\end{center}
	{\color{myOrange}\textbf{Which} is a more desirable cyclone} from a separation efficiency perspective? In general, {\color{myOrange}\textbf{what shape}} would be the most desirable?
\end{frame}

\begin{frame}\frametitle{Day-to-day operation}
	% C+R v6, p 453
	\begin{itemize}
		\item	most important factor: pressure drop = $\Delta P$ = difference between inlet and \emph{overflow} (fines) pressures and typically we have $\Delta P \sim 500$ to $1500$~Pa
		\item	increase $\Delta P$, increases efficiency, $E_T$,  and recovery into the coarse stream %(Svarovsky)
		\item	$\Delta P \propto \rho_f \qquad  \Delta P \propto v_\text{in}^2~\text{and}~v_\text{overflow}^2 \qquad  \Delta P \propto \displaystyle\frac{1}{d_\text{under}}$
		\item	$v_\text{in}$ = entry velocity and $d_\text{under}$ = diameter of underflow
		\item	efficiency drops off at high solids concentration: try to operate as dilutely as possible if requiring high solids recovery
		\item	leave the underflow opening diameter, $d_\text{under}$, as an physically adjustable variable: it is hard to predict its size from theory
		\item	air leaks at this point are disastrous for efficiency \see{Perry, Ch 17.2, 8ed}
	\end{itemize}
	% Perry 17.2: A cyclone will operate equally well on the suction or pressure side of a fan if the dust receiver is airtight. Probably the greatest single cause of poor cyclone performance, however, is the leakage of air into the dust outlet of the cyclone. A slight air leak at this point can result in a tremendous drop in collection efficiency, particularly with fine dusts.
\end{frame}

\begin{frame}\frametitle{Operational advantages and disadvantages}
	{\color{myGreen}{Advantages}}
	\begin{itemize}
		\item	cost of operation: related to $\Delta P$ (i.e. electrical cost only)
		\item	cheap capital cost to build cyclones
		\item	small size
		\item	mounted in any orientation (except for very large units)
		\item	versatile: multiple uses
	\end{itemize}

	Balanced by some {\color{myOrange}{disadvantages}}:
	\begin{itemize}
		\item	subject to abrasion
		\item	cannot use a flocculated feed: high shear forces break flocs up
		\item	limits on their efficiency curves
		\item	requires consistent feed rate and concentration to maintain efficiency i.e. not suitable for variable (volumetric) feeds
		\begin{itemize}
			\item	{\color{myBlue}{counteract}}: use many small cyclones in parallel; bring them online as needed
		\end{itemize}
	\end{itemize}
\end{frame}

\begin{frame}\frametitle{Selection of cyclones, sedimentation or centrifuges}
	\begin{center}
		\includegraphics[width=\textwidth]{\imagedir/separations/mechanical-selection/selecting-based-on-particle-size-Svarovsky-4ed-p41.png}
	\end{center}
\end{frame}

\begin{frame}\frametitle{How to select/model cyclones}
	{\color{myOrange}Given the complex fluid patterns, cyclone selection is best done with the vendor.}
	
	\vspace{12pt}
	There are some guiding equations though.
	\begin{exampleblock}{}
		\[\text{{\color{myOrange}Eu}} = \dfrac{\text{pressure forces}}{\text{inertial forces}} = \dfrac{\Delta P}{\rho_f v^2 / 2}\]
	\end{exampleblock}
	
	\vspace{12pt}
	For a cyclone, the characteristic velocity, {\color{blue}$v = \dfrac{4Q}{\pi D_\text{cyc}^2}$ }
	\[
		\begin{array}{rcll}
			\Delta P&=& \text{pressure drop from inlet to \emph{overflow}} 				&[\text{Pa}]\\
			v 	    &=& \text{characteristic velocity (\textbf{not} inlet velocity)}  	&[\text{m.s}^{-1}]\\
			\rho_f  &=& \text{density of fluid}  										&[\text{kg.m}^{-3}]\\
			Q   	&=& \text{volumetric feed flow rate}								&[\text{m}^{3}\text{.s}^{-1}]\\
			D_\text{cyc}&=& \text{cylindrical section diameter of cyclone}				&[\text{m}] \\
		\end{array}
	\]
	
	\vspace{12pt}
	$0.02 < D_\text{cyc} < 5.0~\text{m}$ are typical values
	
\end{frame}

\begin{frame}\frametitle{{\color{myOrange}Eu}ler number for cyclones}
	\begin{itemize}
		\item	It is relatively constant, under different flow conditions, for a given cyclone
		\item	e.g. ``this cyclone has an Euler number of 540''

		\item	provided solids concentration remains around or below $1~\text{g.m}^{-3}$  % Svarovsky paper, http://www.svarovsky.org/fps2/GASCYC.pdf
		\item	{\color{myOrange}Eu} can be easily calculated found from clean air at ambient conditions 
	\end{itemize}
	\see{Svarovsky}  % http://www.svarovsky.org/fps2/GASCYC.pdf
	
	\vspace{12pt}
	If you can't get/calculate it, then use this:
	\begin{exampleblock}{}
		\[\text{{\color{myOrange}Eu}} = \pi^2\left(\dfrac{D_\text{cyc}}{L}\right)\left(\dfrac{D_\text{cyc}}{K}\right)\left(\dfrac{D_\text{cyc}}{M}\right)^2\]
	\end{exampleblock}
	\[
		\begin{array}{rcll}
			L 			&=& \text{width of rectangular inlet} 				&[\text{m}]\\
			K 			&=& \text{height of rectangular inlet} 				&[\text{m}]\\
			M			&=& \text{diameter of overflow (gas) outlet}		& [\text{m}]
		\end{array}
	\]
\end{frame}

\begin{frame}\frametitle{Predicting cut size}
	The cyclone's cut size, $x_{50}$, can be predicted from the Stokes number. This is a great way to scale-up through geometrically similar cyclones:
	\begin{exampleblock}{}
		\[{\color{myOrange}\text{Stk}_{50}} = \dfrac{x_{50}^2 \,\,\rho_S \,\,v}{18\,\, \mu_f \,\,D_\text{cyc}}\]
	\end{exampleblock}
	\[
		\begin{array}{rcll}
			x_{50}		&=& \text{cut size} 				&[\text{m}]\\
			\rho_S		&=& \text{solids density} 			&[\text{kg.m}^{-3}]\\
			v 			&=& \text{characteristic velocity} 	&[\text{m.s}^{-1}]\\
			\mu_f	    &=& \text{fluid viscosity}        	&[\text{Pa.s}]\\
			{\color{myOrange}\text{Stk}_{50}} &=& \text{Stokes number}  & [-]
		\end{array}
	\]
	\textbf{Note}: 
	\begin{itemize}
		\item	this only predicts the cut-size, not the shape of the grade efficiency curve
		\item	as with Eu, the ${\color{myOrange}\text{Stk}_{50}}$ must be calculated on an actual feed
		\item	it is relatively constant for changing conditions
	\end{itemize}
\end{frame}

\begin{frame}\frametitle{Example (based on data by Svarovsky)}
	% Svarovsky paper, http://www.svarovsky.org/fps2/GASCYC.pdf
	
	Outline the process/plan to solve this problem (do calculations at home!)
	
	\vspace{12pt}
	\textbf{What diameter of cyclone} do we need to treat $0.177~\text{m}^3\text{.s}^{-1}$ of feed, given:
	\begin{itemize}
		\item	$\mu_f = 1.8~\times 10^{-5}$ Pa.s
		\item	$\rho_f = 1.2~\text{kg.m}^{-3}$
		\item	$\rho_S	= 2500~\text{kg.m}^{-3}$
		\item	$\Delta P = 1650$ Pa (cannot be exceeded)
		\item	$x_{50}$ desired is 0.8 \micron
		\item	Eu = 700
		\item	$\text{Stk}_{50} = 6.5 \times 10^{-5}$
	\end{itemize}
	\emph{Hint}: if we use 1 cyclone, the pressure drop will be too high; so we must split the feed into multiple, \emph{parallel} cyclones. So then,
	\textbf{how many cyclones}, and of \textbf{what diameter} should we use?
	
	\vspace{8pt}
	{\scriptsize {\color{myOrange}[\emph{Ans}: 5 cyclones, $D_\text{cyc}=0.15$m]}}
\end{frame}

\begin{frame}\frametitle{Circuits of separators}
	{\scriptsize \textbf{The remaining slides can be applied to any separation system, though most commonly used for cyclones and other solid-fluid separations.}}
	\vspace{12pt}
	\hrule
	\vspace{6pt}
	When one unit is not enough...
	\begin{columns}[t]
		\column{0.60\textwidth}
			\begin{itemize}
				\item	we need a lower cut size
				\item	need a sharper cut (slope of grade efficiency curve at $x_\text{cut}$)
				\item	we need high concentrations
				\item	use lower velocities to reduce abrasion on equipment, but this will change efficiency, so then ...
			\end{itemize}
		\column{0.40\textwidth}
			\begin{center}
				\includegraphics[width=\textwidth]{\imagedir/separations/cyclones/grade-efficiency-curve.png}
			\end{center}
	\end{columns}
	\vspace{24pt}
	\seefull{The rest of this section is from Svarovsky, 4ed, chapter 16}
\end{frame}

\begin{frame}\frametitle{Units in series: overflow}
	\begin{columns}[t]
		\column{0.7\textwidth}
		Grade efficiency curve for the entire sequence
			\begin{center}
				\includegraphics[width=\textwidth]{\imagedir/separations/cyclones/cyclone-efficiency-series-overflow-Svarovsky-4ed-p480.png}
			\end{center}
		\column{0.40\textwidth}
		\begin{itemize}
			\item	cut size becomes smaller with more units in series
			\item	cut size sharpness (steepness of curve) increases
			\item	but there are diminishing returns after 3 to 4 units
		\end{itemize}
	\end{columns}
	\vspace{12pt}
	$G(x\rightarrow 0)=10\%$: implies that 10\% of the smallest size fractions are always found in the coarse underflow: we cannot remove these fines
\end{frame}

\begin{frame}\frametitle{Recycle around a unit}
	\begin{columns}[c]
		\column{0.60\textwidth}
			\begin{center}
				\includegraphics[width=\textwidth]{\imagedir/separations/cyclones/cyclone-efficiency-recycle-Svarovsky-4ed-p483.png}
			\end{center}
		\column{0.40\textwidth}
			\begin{itemize}
				\item	dilutes feed, which improves efficiency
				\item	decreases cut size for increasing recycle ratio: $Q/q$
				\item	again diminishing returns after a ratio of 3 is exceeded
			\end{itemize}
	\end{columns}
\end{frame}

\begin{frame}\frametitle{Units in series: underflow}
	\begin{columns}[t]
		\column{0.7\textwidth}
			\begin{center}
				\includegraphics[width=\textwidth]{\imagedir/separations/cyclones/cyclone-efficiency-series-underflow-Svarovsky-4ed-p488.png}
			\end{center}
		\column{0.40\textwidth}
		\begin{itemize}
			\item	we get worse efficiency
			\item	is this useful for anything?
		\end{itemize}
	\end{columns}
\end{frame}

\begin{frame}\frametitle{Recycle in the underflow}
	\begin{columns}[t]
		\column{0.7\textwidth}
			\begin{center}
				\includegraphics[width=\textwidth]{\imagedir/separations/cyclones/cyclone-efficiency-recycle-underflow-Svarovsky-4ed-p485.png}
			\end{center}
		\column{0.40\textwidth}
		\begin{itemize}
			\item	Best of both worlds?
		\end{itemize}
	\end{columns}
\end{frame}

% \begin{frame}\frametitle{Example}
% 
% 	feed A -> underflowA -> feedB -> underflowB leaves circuit
% 	feed B -> overflowB joins underflow from C
% 	overflowA -> feedC
% 	underflowC + overflowB added to feed of A in a recycle stream
% 	overflowC -> leaves circuit as a mainly air stream
% 	underflowB -> leaves circuit as a high percent solids stream
% 
% 	unit A = ``rougher''
% 	unit B = ``cleaner''
% 	unit C = ``scavenger''
% 
% 	Question: purpose of units A, B, and C
% 
% \end{frame}

% \begin{frame}\frametitle{Design considerations: HYDROcyclone}
% 	C+R v2, 5ed, p54
% 	Design Considerations
% 	Hydrocyclones are 20 to500 mm in diameter, with the smaller units giving a much better separation. Typical values of length to diameter ratios range from about 5 to 20. Because of the very high shearing stresses which are set up, flocs will be broken down and the suspension in the secondary vortex will be completely deflocculated, irrespective of its condition on entry. Generally, hydrocylones are not effective in removing particles smaller than about 2to 3 \micron.
% %	Because separating power is greatest in hydrocyclones of small diameter, the cut size being approximately proportional to the diameter of the cylindrical shell raised to the power of 1.5, it is common practice to operate banks of small hydrocyclones in parallel inside a large containing vessel. Furthermore, this procedure also makes scale-up easier to carry out. With units connected only in parallel (or as single units), however, it is not possible to control the compositions of the overflow and underflow independently, and therefore some form of series-parallel operation is often employed. This is an important consideration when the hydrocyclone is used for thickening a suspension. In this case, there is the requirement to produce an underflow of the appropriate solids concentration and for the overflow to be particle-free, and these two conditions cannot usually be satisfied in a single unit. In thickening, the particle concentrations are high and little classification by size occurs.
% %	In general, the performance of the hydrocyclone is improved by increasing the operating pressure, and the principal control variable is the size of the orifice on the underflow discharge. Several theoretical and practical studies have been made in an attempt to present a sound basis for design, and these have been described BRADLEY(29) and SAVROVSKY(30,35) among others, although they are generally not entirely satisfactory, and some design charts and formulae have been given by ZANKER(40). In practice, tests with the actual materials to be used are desirable for the evaluation for the various parameters. Since systems are usually scaled-up by increasing the number of units in parallel, it is seldom necessary to carry out tests on large units.
% %	The optimum design of a hydrocyclone for a given function depends upon reconciling a number of conflicting factors and reference should be made to specialist publications. Because it is simple in construction and has no moving parts, maintenance costs are low. The chief problem arises from the abrasive effect of the solids; materials of construction, such as polyurethane, show less wear than metals and ceramics.
% \end{frame}

% \begin{frame}\frametitle{Design}
% 	p 455 C+R v6
% 	General design procedure
% 	1. Select either the high-efficiency or high-throughput design, depending on the performance required.
% 	2. Obtain an estimate of the particle size distribution of the solids in the stream to be treated.
% 	3. Estimate the number of cyclones needed in parallel.
% 	4. Calculate the cyclone diameter for an inlet velocity of 15 m/s (50 ft/s). Scale the
% 	other cyclone dimensions from Figures 10.44a or 10.44b.
% 	5. Calculate the scale-up factor for the transposition of Figures 10.45a or 10.45b.
% 	6. Calculate the cyclone performance and overall efficiency (recovery of solids). If
% 	unsatisfactory try a smaller diameter.
% 	7. Calculate the cyclone pressure drop and, if required, select a suitable blower.
% 	8. Cost the system and optimise to make the best use of the pressure drop available,
% 	or, if a blower is required, to give the lowest operating cost.
% \end{frame}

% \begin{frame}\frametitle{Design: typical dimensions and options}
% 	C+R v6, p404
% 	Liquid cyclones can be used for the classification of solid particles over a size range from 5 to 100\micron. Commercial units are available in a wide range of materials of
% 	
% 	EQUIPMENT SELECTION, SPECIFICATION AND DESIGN 405
% 	construction and sizes; from as small as 10 mm to up to 30 m diameter. The separating efficiency of liquid cyclones depends on the particle size and density, and the density and viscosity of the liquid medium.
% \end{frame}

% \begin{frame}\frametitle{Operational}
% 	C+R v2: p 77
% 
% 	A small inlet and outlet therefore result in the separation of smaller particles but, as the pressure drop over the separator varies with the square of the inlet velocity and the square of the outlet velocity(56), the practical limit is set by the permissible drop in pressure. The depth and diameter of the body should be as large as possible because the former determines the radial component of the gas velocity and the latter controls the tangential component at any radius. In general, the larger the particles, the larger should be the diameter of the separator because the greater is the radius at which they rotate. The larger the diameter, the greater too is the inlet velocity which can be used without causing turbulence within the separator. The factor which ultimately settles the maximum size is, of course, the cost. Because the separating power is directly related to the throughput of gas, the cyclone separator is not very flexible though its efficiency can be improved at low throughputs by restricting the area of the inlet with a damper and thereby increasing the velocity. Generally, however, it is better to use a number of cyclones in parallel and to keep the load on each approximately the same whatever the total throughput.
% 
% \end{frame}

% \begin{frame}\frametitle{how is it designed: simple nomographs, equations, etc}
% 	For Hydrocyclones only:
% 	d50 design: C+R v6, p 423 and p 426 and 	p 450 C+R v6
% \end{frame}

% \begin{frame}\frametitle{Effect of solids flow rate}
% 	http://www.youtube.com/watch?v=BicR3JGlE5M
% \end{frame}

% \begin{frame}\frametitle{Circuit selection}
% 	Crushing circuit: to separate fines from coarse: recycle coarse material for recrushing. Could also use a screen. Why use a cyclone instead?
% 	Also: could use elutriation with air rather than screens or cyclones. Why would you use elutriation?
% 
% 	Use: to clean dust from air; generally 5 \micron and larger
% 	Use: mists separated from gas
% 	U
% 	A cyclone is a centrifugal elutriator, although it is not usually so regarded. The cyclosizer is a series of inverted cyclones with added apex chambers through which water flows. Suspension is fed into the largest cyclone, and particles are separated into different size ranges.
% \end{frame}

% \begin{frame}\frametitle{Circuits of cyclones}
% 	How is the flowsheet designed?
% 	Fig 6.15
% 	Fig 6.18
% 	Figre 6.19
% \end{frame}
