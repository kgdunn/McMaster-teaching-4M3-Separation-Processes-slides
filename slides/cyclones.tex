\begin{frame}\frametitle{}
	Crushing circuit: to separate fines from coarse: recycle coarse material for recrushing. Could also use a screen. Why use a cyclone instead?
	Also: could use elutriation with air rather than screens or cyclones. Why would you use elutriation?

	Use: to clean dust from air; generally 5 \micron and larger
	Use: mists separated from gas
	U
	A cyclone is a centrifugal elutriator, although it is not usually so regarded. The cyclosizer is a series of inverted cyclones with added apex chambers through which water flows. Suspension is fed into the largest cyclone, and particles are separated into different size ranges.	
\end{frame}

\begin{frame}\frametitle{}
	See: http://www.nzifst.org.nz/unitoperations/, UNIT OPERATIONS IN FOOD PROCESSING
	
	http://www.nzifst.org.nz/unitoperations/mechseparation3.htm

	Cyclones are often used for the removal from air streams of particles of about 10 mm or more diameter. They are also used for separating particles from liquids and for separating liquid droplets from gases. The cyclone is a settling chamber in the form of a vertical cylinder, so arranged that the particle-laden air spirals round the cylinder to create centrifugal forces which throw the particles to the outside walls. Added to the gravitational forces, the centrifugal action provides reasonably rapid settlement rates. The spiral path, through the cyclone, provides sufficient separation time. A cyclone is illustrated in Fig. 10.2(a).


	Figure 10.2 Cyclone separator: (a) equipment (b) efficiency of dust collection


	Stokes' Law shows that the terminal velocity of the particles is related to the force acting. In a centrifugal separator, such as a cyclone, for a particle, rotating round the periphery of the cyclone:
	           Fc = (mv2)/r                                                                                                         (10.4)

	where Fc is the centrifugal force acting on the particle, m is the mass of the particle, v is the tangential velocity of the particle and r is the radius of the cyclone.

	This equation shows that the force on the particle increases as the radius decreases, for a fixed velocity. Thus, the most efficient cyclones for removing small particles are those of smallest diameter. The limitations on the smallness of the diameter are the capital costs of small diameter cyclones to provide sufficient output, and the pressure drops.

	The optimum shape for a cyclone has been evolved mainly from experience and proportions similar to those indicated in Fig. 10.2(a) have been found effective. The efficient operation of a cyclone depends very much on a smooth double helical flow being produced and anything which creates a flow disturbance or tends to make the flow depart from this pattern will have considerable and adverse effects upon efficiency. For example, it is important that the air enters tangentially at the top. Constricting baffles or lids should be avoided at the outlet for the air.

	The efficiency of collection of dust in a cyclone is illustrated in Fig. 10.2(b). Because of the complex flow, the size cut of particles is not sharp and it can be seen that the percentage of entering particles which are retained in the cyclone falls off for particles below about 10 mm diameter. Cyclones can be used for separating particles from liquids as well as from gases and also for separating liquid droplets from gases.
\end{frame}

\begin{frame}\frametitle{}
	See: http://www.nzifst.org.nz/unitoperations/, UNIT OPERATIONS IN FOOD PROCESSING
	
	http://www.nzifst.org.nz/unitoperations/mechseparation3.htm

	Cyclones are often used for the removal from air streams of particles of about 10 mm or more diameter. They are also used for separating particles from liquids and for separating liquid droplets from gases. The cyclone is a settling chamber in the form of a vertical cylinder, so arranged that the particle-laden air spirals round the cylinder to create centrifugal forces which throw the particles to the outside walls. Added to the gravitational forces, the centrifugal action provides reasonably rapid settlement rates. The spiral path, through the cyclone, provides sufficient separation time. A cyclone is illustrated in Fig. 10.2(a).


	Figure 10.2 Cyclone separator: (a) equipment (b) efficiency of dust collection


	Stokes' Law shows that the terminal velocity of the particles is related to the force acting. In a centrifugal separator, such as a cyclone, for a particle, rotating round the periphery of the cyclone:
	           Fc = (mv2)/r                                                                                                         (10.4)

	where Fc is the centrifugal force acting on the particle, m is the mass of the particle, v is the tangential velocity of the particle and r is the radius of the cyclone.

	This equation shows that the force on the particle increases as the radius decreases, for a fixed velocity. Thus, the most efficient cyclones for removing small particles are those of smallest diameter. The limitations on the smallness of the diameter are the capital costs of small diameter cyclones to provide sufficient output, and the pressure drops.

	The optimum shape for a cyclone has been evolved mainly from experience and proportions similar to those indicated in Fig. 10.2(a) have been found effective. The efficient operation of a cyclone depends very much on a smooth double helical flow being produced and anything which creates a flow disturbance or tends to make the flow depart from this pattern will have considerable and adverse effects upon efficiency. For example, it is important that the air enters tangentially at the top. Constricting baffles or lids should be avoided at the outlet for the air.

	The efficiency of collection of dust in a cyclone is illustrated in Fig. 10.2(b). Because of the complex flow, the size cut of particles is not sharp and it can be seen that the percentage of entering particles which are retained in the cyclone falls off for particles below about 10 mm diameter. Cyclones can be used for separating particles from liquids as well as from gases and also for separating liquid droplets from gases.
\end{frame}

\begin{frame}\frametitle{Description}
	Cylindrical section
	Conical section
	
	Vortex finder
	Spigot
\end{frame}

\begin{frame}\frametitle{Principle}
	Same principle as a centrifuge
	But no moving parts
	Vortex and tangential forces formed by the fluid
	dust collection: smallest size possible?
	Liquid droplets from air?	
\end{frame}

\begin{frame}\frametitle{Typical dimensions}
	How are they designed?
	Scaled-up?	
\end{frame}

\begin{frame}\frametitle{Circuits of cyclones}
	How is the flowsheet designed?
	
\end{frame}

\begin{frame}\frametitle{Operational issue}
	Pressure drop
	Electrical costs to operation
\end{frame}

\begin{frame}\frametitle{Capital costs}
	Is a function of ...
	\begin{itemize}
		\item	pressure rating (ASME conformance)
	\end{itemize}
\end{frame}

\begin{frame}\frametitle{References}
	* Svarovsky3, Ch 6
	* C+R v2, page 73
\end{frame}

\begin{frame}\frametitle{Selection}
	\begin{center}
		\includegraphics[width=\textwidth]{\imagedir/separations/mechanical-selection/selecting-based-on-particle-size-Svarovsky-4ed-p41.png}
	\end{center}
\end{frame}
