%!TEX root = ../4M3-course.tex

\begin{frame}\frametitle{Plan for today's class}
	\begin{enumerate}
		\item	Background
		\item	Administrative issues
		\item	Short brainstorming session of topics to cover
		\item	Course content (today, and the next class also)
	\end{enumerate}
\end{frame}

\begin{frame}\frametitle{Credits for course material}

	\begin{itemize}
		\item	Dr. Santiago Faucher
		\begin{itemize}
			\item	Taught the course in 2009, 2010 and 2011
			\item	Course outline and topics covered are similar to his
		\end{itemize}
		\item	Dr. Raja Ghosh, taught 4M3 for a few years prior to that
		\item	Dr. Jim Dickson, taught the course since 1984
	\end{itemize}
\end{frame}

\begin{frame}\frametitle{Background}
	{\color{myGreen}{About myself}}
	\begin{itemize}
		\item	Undergraduate degree from University of Cape Town, 1999
		\item	Masters degree from McMaster, 2002 (not a ``doctor'', please)
		\item	Worked with a number of companies from 2002 to 2011 on data analysis and consulting projects
		\item	Worked at GSK on a 1-year contract until June 2012
		\item	Now working full-time at McMaster since July 2012
		\item	Office is in BSB, room B105
		\item	Arrange a meeting: \url{kevin.dunn@mcmaster.ca}
		\item	Cell: (905) 921 5803
		\item	\sout{extension 27337}
	\end{itemize}
\end{frame}

\begin{frame}\frametitle{Teaching assistant}
	{\color{myGreen}{Dominik Seepersad}}
	\begin{itemize}
		\item	\url{chemac.4m3@gmail.com}
		\item	JHE, room 370
		\item	extension 22008
		\item	Currently doing his M.A.Sc with Tom Adams
		\item	Office hours to be arranged by email with him
	\end{itemize}
\end{frame}

\begin{frame}\frametitle{Video and audio recordings}
	\begin{itemize}
		\item	As long \textbf{as feasible}, I will try to video record all classes
		\item	Might be useful if you miss a class
		\item	Most useful: review after the class
		\item	Quality might not be the best
		\item	Usually available 24 to 48 hours later
		\item	Audio recordings will also be made available, when possible
	\end{itemize}
\end{frame}

\begin{frame}\frametitle{Course website}

	\begin{exampleblock}{}
		\centering
		\href{http://learnche.mcmaster.ca/4M3}{http://learnche.mcmaster.ca/4M3}
	\end{exampleblock}
	\begin{itemize}
		\item	Please check several times per week for announcements {\tiny (top left)}
		\item	Follow the Twitter feed: \href{https://twitter.com/4m3separations}{@4m3separations}
		\item	Slides will be added to the site before class
		\item	Please print slides and bring to class
		\item	Assignments and solutions will be posted there
	\end{itemize}
\end{frame}

\begin{frame}\frametitle{References and readings}

	\textbf{No required textbook}
	\begin{columns}[t]
		\column{0.80\textwidth}
			\emph{Recommended}: Seader, Henley and Roper, ``Separation Process Principles'' (3rd edition)
		\column{0.10\textwidth}
			\vspace{-1cm}
			\begin{center}
				\includegraphics[width=1.5\textwidth]{\imagedir/separations/textbook-covers/Seader-et-al.jpg}
			\end{center}
	\end{columns}

	\vspace{12pt}
	\begin{columns}[t]
		\column{0.80\textwidth}
			\emph{Recommended}: Geankoplis, ``Transport Processes and Separation Process Principles'', (3rd or 4th edition)
		\column{0.10\textwidth}
			\vspace{-1cm}
			\begin{center}
				\includegraphics[width=1.5\textwidth]{\imagedir/separations/textbook-covers/Geankoplis.jpg}
			\end{center}
	\end{columns}

	\vspace{12pt}
	\emph{Recommended}: \href{http://accessengineeringlibrary.com/browse/perrys-chemical-engineers-handbook-eighth-edition}{Perry's ``Chemical Engineers' Handbook''}, any edition. Please make full use of the library's subscription: \\
	\href{http://accessengineeringlibrary.com/browse/perrys-chemical-engineers-handbook-eighth-edition}{\tiny http://accessengineeringlibrary.com/browse/perrys-chemical-engineers-handbook-eighth-edition}

	\vspace{24pt}
	Other references on course website

\end{frame}

\begin{frame}\frametitle{Course feedback via Learning website}
	\begin{itemize}
		\item	I might not have explained something clearly;
		\item	you didn't get a chance to ask a question, \emph{etc}
	\end{itemize}
	\href{http://learnche.mcmaster.ca/feedback-questions}{http://learnche.mcmaster.ca/feedback-questions}
	\vspace{12pt}
	\hrule
	\begin{center}
		\includegraphics[width=0.65\textwidth]{\imagedir/teaching/anonymous-feedback.png}
	\end{center}
	\hrule
\end{frame}

\begin{frame}\frametitle{Expectations outside class}
	\begin{itemize}
		\item	You can expect TA and I to answer emails promptly
		\item	If you have questions
			\begin{enumerate}
				\item	Please email the TA with CC to me \hfill {\tiny{\color{myOrange}{$\longleftarrow$ hopefully this solves your problem}}}
				\item	if not, set up meeting with TA or myself
			\end{enumerate}
		\item	Please email from your McMaster address (filtering)
	\end{itemize}
\end{frame}

\begin{frame}\frametitle{Grading}
	What we look for in the grading is demonstration that you/group:
	\begin{enumerate}
		\item	understand the concept
		\item	have the ability to apply the concept to new instances
		\item	think creatively about problems
		\item	apply a systematic problem-solving strategy
		\begin{itemize}
			\item	\texttt{Define, Explore, Plan, Do, Check, Generalize}
		\end{itemize}
		\item	accuracy.
	\end{enumerate}
\end{frame}

\begin{frame}\frametitle{Grading}
	\begin{tabular}{ll}\\
		Assignments (about 5)       	& 20\% \\
	    Written midterm        			& 15\% \\
	    Quest tests						& 8\% \\
	    Project      					& 12\% \\
	    Final exam 						& 45\% \\
	\end{tabular}

	\vspace{12pt}
	\vspace{12pt}

	\begin{itemize}
		\item	\emph{Grading allocation is subject to change}
		\item	Course letters will be assigned using standard system
		\item	Minimum requirements to pass:
		\begin{itemize}
			\item	50\% or more in the final exam
			\item	Must submit a course project
		\end{itemize}
	\end{itemize}
\end{frame}

\begin{frame}\frametitle{Midterms and exam}
	\begin{itemize}
		\item	Written midterm: 22 October, 18:30
		\begin{itemize}
			\item	Optional, no make-up
		\end{itemize}

		\item	Quest tests
		\begin{itemize}
			\item	Short duration, computer-based tests
			\item	Quick answers, to help you stay on top of the material
		\end{itemize}

		\item	Final exam
		\begin{itemize}
			\item	Cumulative of all material
		\end{itemize}
	\end{itemize}

	\vspace{12pt}
	All tests and exams:
	\begin{itemize}
		\item	open notes -- any form of paper
		\item	any calculator
		% \item	no e-books \hfill {\color{myOrange}{\scriptsize (ideas on how to handle this? Please reply via feedback form)}}
	\end{itemize}
\end{frame}

\begin{frame}\frametitle{Project}
	\begin{exampleblock}{}
		\textbf{{\color{myBlue}{AIM}}}: a short report on a selected separation process (choice of 3 or 4 units)
	\end{exampleblock}
	\begin{itemize}
		\item	Details to come later on the report's scope
		\item	Only electronic hand-in will be accepted
		\item	Important dates:

		\vspace{12pt}
		\begin{tabular}{ll}
			Topic selection & 04 October, or earlier\\
			Outline due & 15 October\\
			Project due & 12 November\\
			% Presentations & week of 19 to 23 November\\
		\end{tabular}
	\end{itemize}
\end{frame}

\begin{frame}\frametitle{Group-based assignments}
	\begin{itemize}
		\item	``Appropriate'' group work is highly encouraged
		\begin{itemize}
			\item	32\% of course
		\end{itemize}
		\item	Learn with each other: \textbf{groups of 2}, no larger, no exceptions
	\end{itemize}
\end{frame}

\begin{frame}\frametitle{Group-based assignments}
	\begin{itemize}
		\item	\textbf{Optimal group work}: \emph{an example of one approach}
			\begin{itemize}
				\item	Sarah and Brad work on an assignment
				\item	Both Sarah and Brad do {\color{myRed}{\textbf{all questions}}} in draft: quick notes at home, on the bus, \emph{etc}, $\pm 4$ days before assignment due
				\pause
				\item	Meet in the library next day and go over each other's notes
				\item	Explain to the other why you disagree
				\item	e.g. Sarah sees a mistaken interpretation in Brad's work
				\begin{itemize}
					\item	She explains why it is a mistake to Brad: Sarah learns
					\item	Brad also learns: he's heard this in class, and from Sarah now
					\item	If neither can resolve it? speak with TA or Kevin
				\end{itemize}
				\pause
				\item	Write up a joint solution from both group members' notes
				\begin{itemize}
					\item	e.g. Sarah does Q1 and 2, Brad does Q3
				\end{itemize}
				\item	Both review it before submitting
			\end{itemize}
		\vspace{2pt}\hrule\vspace{2pt}
		\pause
		\item	Other approaches are possible: your group decides
		\item	{\color{myOrange}{What doesn't work}:} Sarah does Q1 and Q2, Brad does Q3; staple and submit
		\begin{itemize}
			\item	Neither learns the other material
		\end{itemize}
	\end{itemize}
\end{frame}

\begin{frame}\frametitle{Over to you ...}
	Work on the hand-out in groups of 3 or 4
	\begin{itemize}
		\item	Identify separation processes that begin with each letter
	\end{itemize}
	\begin{center}
		\includegraphics[width=1.0\textwidth]{\imagedir/separations/admin/active-identification-units.png}
	\end{center}
\end{frame}

