% 11 September 2012
\begin{frame}\frametitle{Admin issues}
	\begin{itemize}
		\item	Group size: 2 (preferred), but groups of 3 will be allowed if you email me with a convincing explanation
		\item	Midterm: 12 October 2012, 18:30 to 21:30; split venue
		\item	Assignment 1 is posted on website
	\end{itemize}
\end{frame}

\begin{frame}\frametitle{Mechanical separations}
	We will start with this topic
	\begin{itemize}
		\item	It's easy to understand!
		\item	Requires only a knowledge of basic physics (e.g. 1st year physics)
		\item	It introduces a number of important principles we will re-use later
		\item	Mechanical separations remain some of the most widely used steps in many flowsheets. Why?
		\begin{itemize}
			\item	reliable units
			\item	relatively inexpensive to maintain and operate
			\item	we can often achieve a very high \emph{separation factor} (that's desirable!)
		\end{itemize}
	\end{itemize}
\end{frame}

% 2013: specify the units in the equation: any units, as long as consistent, but emphasize this
\begin{frame}\frametitle{Separation factor}
	As mentioned, we will introduce a number of important principles we will re-use later.
	
	\begin{exampleblock}{Separation factor}
		\begin{columns}[c]
			\column{0.40\textwidth}
				$S_{ij} = \displaystyle \frac{x_{i,1} / x_{j,1}}{x_{i,2} / x_{j,2}}$
			\column{0.40\textwidth}
				\begin{center}
					\includegraphics[width=\textwidth]{\imagedir/separations/overview-class/separation-factor.png}
				\end{center}
		\end{columns} 
	\end{exampleblock}
	
	\begin{itemize}
		\item	select $i$ and $j$ so that $S_{ij} \geq 1$
		\item	units of $x$ terms in the above equation can be mass or mole fractions (or flows) 		
	\end{itemize}
	
	Based on this definition: we can see why solid-fluid separations often have high separation factors
\end{frame}

\begin{frame}\frametitle{Separating agents: MSA and ESA}
	\begin{exampleblock}
		{\small A material, force, or energy source applied to the feed for separation }
	\end{exampleblock}
	\vspace{6pt}
	i.e. what you add to get a separation. MSA = mass separating agent and ESA = energy separating agent
	\vspace{6pt}	
	\begin{itemize}
		\item	heat (ESA)
		\item	liquid solvent (MSA)
		\item	pressure (ESA)
		\item	\pause\iftoggle{instructor}{vacuum}{}
		\item	\iftoggle{instructor}{membrane}{}
		\item	\iftoggle{instructor}{filter media}{}
		\item	\iftoggle{instructor}{electric field}{}
		\item	\iftoggle{instructor}{flow}{}
		\item	\iftoggle{instructor}{temperature gradient}{}
		\item	\iftoggle{instructor}{concentration gradient}{}
		\item	\iftoggle{instructor}{gravitational field (natural, or artificially created)}{}
		\item	\iftoggle{instructor}{adsorbent}{}
		\item	\iftoggle{instructor}{absorbent}{}
	\end{itemize}
\end{frame}

\begin{frame}\frametitle{Units we will consider in depth}
	Under the title of ``Mechanical Separations'' we will consider:
	\begin{itemize}
		\item	free settling (sedimentation)
		\item	screening of particles 
		\item	centrifuges
		\item	cyclones
	\end{itemize}
	
	\vspace{12pt}
	There are also others that go in this category. Deserving a quick mention are:
	\begin{itemize}
		\item	magnetic separation
		\item	electrostatic precipitation
	\end{itemize}
\end{frame}

\begin{frame}\frametitle{{\color{myGreen}{Quick mention:}} Magnetic separation}
	\begin{itemize}
		\item	used mainly in the mineral processing industries
		\item	high throughputs: up to 3000 kg/hour per meter of rotating drum
		\item	e.g. remove iron from feed
		\item	Also used in food and drug industries at multiple stages to ensure product integrity		
	\end{itemize}
	
	\see{Sinnott, 4ed, v6, Ch10}
	\begin{center}
		\includegraphics[width=0.68\textwidth]{\imagedir/separations/overview-class/magnetic-separation.png}
	\end{center}
\end{frame}

\begin{frame}\frametitle{{\color{myGreen}{Quick mention:}} Electrostatic separators}
	
	\begin{itemize}
		\item	depends on differences in conductivity of the material
		\item	materials passes through a high-voltage field while on a rotating drum
		\item	the drum is earthed
		\item	some of the particles acquire a charge and adhere stronger to the drum surface
		\item	they are carried further than the other particles, creating a split
	\end{itemize}
	
	\see{Sinnott, 4ed, v6, Ch10}
	\begin{center}
		\includegraphics[width=0.50\textwidth]{\imagedir/separations/overview-class/electrostatic-separation.png}
	\end{center}
\end{frame}