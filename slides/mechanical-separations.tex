\begin{frame}\frametitle{Mechanical separations}
	We will start with this topic
	\begin{itemize}
		\item	It's easy to understand!
		\item	Requires only a knowledge of basic physics (e.g. 1st year physics)
		\item	It introduces a number of important principles we will re-use later
		\item	Mechanical separations remain some of the most widely used steps in many flowsheets. Why?
		\begin{itemize}
			\item	reliable units
			\item	relatively inexpensive to maintain and operate
			\item	we can often achieve an \textbf{infinite} \emph{separation factor} (that's desirable!)
		\end{itemize}
	\end{itemize}
\end{frame}

\begin{frame}\frametitle{Separation factor}
	As mentioned, we will introduce a number of important principles we will re-use later.
	
	\begin{exampleblock}{Separation factor}
		\todo{define it here}
	\end{exampleblock}
	
	Based on this definition: we can see why solid-fluid separations often have a nearly infinite separation factor
\end{frame}

\begin{frame}\frametitle{Units we will consider in depth}
	Under the title of ``Mechanical Separations'' we will consider:
	\begin{itemize}
		\item	free settling (sedimentation)
		\item	screening of particles 
		\item	centrifuges
		\item	cyclones
	\end{itemize}
	
	\vspace{12pt}
	There are also others that go in this category. Deserving a quick mention are:
	\begin{itemize}
		\item	magnetic separation
		\item	electrostatic precipitation
	\end{itemize}
\end{frame}

\begin{frame}\frametitle{{\color{myGreen}{Quick mention:}} Magnetic separation}
	\begin{itemize}
		\item	used mainly in the mineral processing industries
		\item	high throughputs: up to 3000 kg/hour per meter of rotating drum
		\item	e.g. remove iron from feed
		\item	Also used in food and drug industries at multiple stages to ensure product integrity		
	\end{itemize}
	
	\todo{figure here}
\end{frame}

\begin{frame}\frametitle{{\color{myGreen}{Quick mention:}} Electrostatic separators}
	
	\begin{itemize}
		\item	depends on differences in conductivity of the material
		\item	materials passes through a high-voltage field while on a rotating drum
		\item	the drum is earthed
		\item	some of the particles acquire a charge and adhere stronger to the drum surface
		\item	they are carried further than the other particles, creating a split
	\end{itemize}
	
	\todo{figure here}
\end{frame}