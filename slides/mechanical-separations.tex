\begin{frame}\frametitle{Mechanical separations}
	We will start with this topic
	\begin{itemize}
		\item	It's easy to understand!
		\item	Requires only a knowledge of basic physics (e.g. 1st year physics)
		\item	It introduces a number of important principles we will re-use later
		\item	Mechanical separations remain some of the most widely used steps in many flowsheets. Why?
		\begin{itemize}
			\item	reliable units
			\item	relatively inexpensive to maintain and operate
			\item	we can often achieve an \textbf{infinite} \emph{separation factor} (that's desirable!)
		\end{itemize}
	\end{itemize}
\end{frame}

\begin{frame}\frametitle{Separation factor}
	As mentioned, we will introduce a number of important principles we will re-use later.
	
	\begin{exampleblock}{Separation factor}
		\todo{define it here}
	\end{exampleblock}
	
	Based on this definition: we can see why solid-fluid separations have an infinite separation factor
\end{frame}


\begin{frame}\frametitle{Units we will consider}
	Under the title of ``Mechanical Separations'' we will consider:
	\begin{itemize}
		\item	free settling
		\item	etc
	\end{itemize}
	
	\vspace{12pt}
	There are others though:
	\begin{itemize}
		\item	demisters
		\item	magnetic separation
		\item	electrostatic precipitation
	\end{itemize}
\end{frame}
