\usepackage[latin1]{inputenc}  % allows direct input of Unicode characters
% %\usepackage[utf8x]{inputenc}   % allows direct input of Unicode characters

\usepackage{amsmath,amssymb,amsfonts,euscript,mathrsfs,wasysym,textcomp}
\usepackage{array}
\usepackage{multirow}
\usepackage{multimedia}
\usepackage{fancybox}
\usepackage{psfrag}
\usepackage{listings}
\usepackage{hyperref}
\usepackage[normalem]{ulem}  % For strikeout text: \sout{text goes here}
\usepackage{datetime} % For better date/time display
\usepackage{comment}  % To block out parts of the notes
\usepackage[multidot]{grffile} % Handles file names with spaces and dots in them
%\usepackage{rotating} % Rotated text

\usepackage{etoolbox}  % For selective compiling: figure sources might be on different machines
\providetoggle{instructor}
%\providetoggle{student}
\providetoggle{unixmac}
\settoggle{instructor}{true}    %\iftoggle{instructor}{}{}
%\settoggle{student}{false}      %\iftoggle{student}{}{}
\settoggle{unixmac}{true}
\iftoggle{unixmac}
{
	\newcommand{\imagedir}{/Users/kevindunn/Sync/Figures}
}{
	\newcommand{\imagedir}{C:/Figures}
}

\usetheme{default} 
\setbeamertemplate{navigation symbols}{}          % suppress all navigation symbols
\setbeamertemplate{blocks}[rounded][shadow=true]  % use rounded blocks (boxes), with shadows

\definecolor{todoGreen}{rgb}{0.0, 0.9, 0.0}
\definecolor{myGreen}{rgb}{0.,0.4,0.}
\definecolor{myOrange}{rgb}{1.,0.5,0.}
\definecolor{myBlue}{rgb}{0.0,0.1,0.9}
\definecolor{myRed}{rgb}{1.0,0.0,0.0}
\definecolor{myLightGrey}{rgb}{0.8,0.8,0.8}
\definecolor{Brown}{cmyk}{0,0.81,1,0.60}
\definecolor{OliveGreen}{cmyk}{0.64,0,0.95,0.40}
\definecolor{CadetBlue}{cmyk}{0.62,0.57,0.23,0}

% Show page numbers and dates on slides
% KGD: removed footer completely: 01 Oct 2012: Presentation Zen: no redundant information
%\setbeamertemplate{footline}{\begin{beamercolorbox}[right]{section in head/foot}{\color{myLightGrey}{\tiny \copyright Kevin Dunn, \today}} ~~~ \vskip5pt \end{beamercolorbox}}

\setbeamercovered{transparent=4} 

\hypersetup{colorlinks        = true,    
 			linkcolor         = blue,    
}
% 			linkbordercolor   = {1 0 0},    
% 			urlcolor          = cyan,    
% 			bookmarks         = {true,},    
% 			bookmarksopen     = {true,},    
% 			bookmarksnumbered = {false,},    
% 			pdftitle          = {Kevin Dunn},
% 			pdfsubject        = {Kevin Dunn},
% 			pdfauthor         = {Kevin Dunn},
% 			pdfproducer       = {LaTeX, wiki2beamer, beamer, BeamerPDF},    
% 			pdfkeywords       = http://learnche.mcmaster.ca/,
% 		}
\usepackage{pgfpages}

\makeatletter
\def\hlinewd#1{%
\noalign{\ifnum0=`}\fi\hrule \@height #1 %
\futurelet\reserved@a\@xhline}
\makeatother

% Some definitions
\newcommand{\todo}[1]{{\center{\color{todoGreen} #1}}}
\newcommand{\q}{{\textbf{Q}}}
\newcommand{\adv}{{\small {\color{Brown} (advanced)}}}
\newcommand{\extra}{{\small {\color{Brown} (extra)}}}
\newcommand{\lit}[1]{\href{http://literature.connectmv.com/item/#1}{http://literature.connectmv.com/item/#1}}
\newcommand{\lititem}[2]{\href{http://literature.connectmv.com/item/#1}{#2 {\tiny (http://literature.connectmv.com/item/#1)}}}
\newcommand{\liturl}[2]{\href{#1}{#2 {\tiny ~(#1)}}}
\newcommand{\micron}{$\mu $m}
\newcommand{\see}[1]{{\tiny [{\color{myBlue}{#1}}]}}
\newcommand{\seefull}[1]{{\color{myBlue}{#1}}}