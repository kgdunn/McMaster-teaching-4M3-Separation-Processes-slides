\documentclass[handout, 11pt]{beamer}
%\documentclass[11pt]{beamer}

%\begin{exampleblock}{Title here}
%	\centering{Text here}
%\end{exampleblock}
% \tiny\scriptsize\footnotesize\small\normalsize\large\Large\LARGE\huge\Huge

\usepackage[latin1]{inputenc}  % allows direct input of Unicode characters
% %\usepackage[utf8x]{inputenc}   % allows direct input of Unicode characters

\usepackage{amsmath,amssymb,amsfonts,euscript,mathrsfs,wasysym,textcomp}
\usepackage{array}
\usepackage{multirow}
\usepackage{multimedia}
\usepackage{fancybox}
\usepackage{psfrag}
\usepackage{listings}
\usepackage{hyperref}
\usepackage[normalem]{ulem}  % For strikeout text: \sout{text goes here}
\usepackage{datetime} % For better date/time display
\usepackage{comment}  % To block out parts of the notes
\usepackage[multidot]{grffile} % Handles file names with spaces and dots in them
%\usepackage{rotating} % Rotated text

\usepackage{etoolbox}  % For selective compiling: figure sources might be on different machines
\providetoggle{instructor}
%\providetoggle{student}
\providetoggle{unixmac}
\settoggle{instructor}{true}    %\iftoggle{instructor}{}{}
%\settoggle{student}{false}      %\iftoggle{student}{}{}
\settoggle{unixmac}{true}
\iftoggle{unixmac}
{
	\newcommand{\imagedir}{/Users/kevindunn/Sync/Figures}
}{
	\newcommand{\imagedir}{C:/Figures}
}

\usetheme{default} 
\setbeamertemplate{navigation symbols}{}          % suppress all navigation symbols
\setbeamertemplate{blocks}[rounded][shadow=true]  % use rounded blocks (boxes), with shadows

\definecolor{todoGreen}{rgb}{0.0, 0.9, 0.0}
\definecolor{myGreen}{rgb}{0.,0.4,0.}
\definecolor{myOrange}{rgb}{1.,0.5,0.}
\definecolor{myBlue}{rgb}{0.0,0.1,0.9}
\definecolor{myRed}{rgb}{1.0,0.0,0.0}
\definecolor{myLightGrey}{rgb}{0.8,0.8,0.8}
\definecolor{Brown}{cmyk}{0,0.81,1,0.60}
\definecolor{OliveGreen}{cmyk}{0.64,0,0.95,0.40}
\definecolor{CadetBlue}{cmyk}{0.62,0.57,0.23,0}
\definecolor{purple}{rgb}{0.70,0.22,0.92}

% Show page numbers and dates on slides
% KGD: removed footer completely: 01 Oct 2012: Presentation Zen: no redundant information
%\setbeamertemplate{footline}{\begin{beamercolorbox}[right]{section in head/foot}{\color{myLightGrey}{\tiny \copyright Kevin Dunn, \today}} ~~~ \vskip5pt \end{beamercolorbox}}

\setbeamercovered{transparent=4} 

\hypersetup{colorlinks        = true,    
 			linkcolor         = blue,    
}
% 			linkbordercolor   = {1 0 0},    
% 			urlcolor          = cyan,    
% 			bookmarks         = {true,},    
% 			bookmarksopen     = {true,},    
% 			bookmarksnumbered = {false,},    
% 			pdftitle          = {Kevin Dunn},
% 			pdfsubject        = {Kevin Dunn},
% 			pdfauthor         = {Kevin Dunn},
% 			pdfproducer       = {LaTeX, wiki2beamer, beamer, BeamerPDF},    
% 			pdfkeywords       = http://learnche.mcmaster.ca/,
% 		}
\usepackage{pgfpages}

\makeatletter
\def\hlinewd#1{%
\noalign{\ifnum0=`}\fi\hrule \@height #1 %
\futurelet\reserved@a\@xhline}
\makeatother

% Some definitions
\newcommand{\todo}[1]{{\center{\color{todoGreen} #1}}}
\newcommand{\q}{{\textbf{Q}}}
\newcommand{\adv}{{\small {\color{Brown} (advanced)}}}
\newcommand{\extra}{{\small {\color{Brown} (extra)}}}
\newcommand{\lit}[1]{\href{http://literature.connectmv.com/item/#1}{http://literature.connectmv.com/item/#1}}
\newcommand{\lititem}[2]{\href{http://literature.connectmv.com/item/#1}{#2 {\tiny (http://literature.connectmv.com/item/#1)}}}
\newcommand{\liturl}[2]{\href{#1}{#2 {\tiny ~(#1)}}}
\newcommand{\micron}{$\mu $m}
\newcommand{\see}[1]{{\tiny [{\color{myBlue}{#1}}]}}
\newcommand{\seefull}[1]{{\color{myBlue}{#1}}}

\newdateformat{mydate}{\THEYEAR} % \twodigit{\THEDAY}-\twodigit{\THEMONTH}-\THEYEAR
\mydate
\title[]{\LARGE Separation Processes}
\subtitle[]{\Large ChE 4M3 \\ \vspace{0.5cm} \includegraphics[width=0.2\textwidth]{\imagedir/teaching/logos/4M3-logo-2012.png} \vspace{-1.5cm} }\author[]{}
\institute[]{}
\date[]{\copyright~ Kevin Dunn, \today \\ \vspace{1cm}{\footnotesize {\tt kevin.dunn@mcmaster.ca}\\ \href{http://learnche.mcmaster.ca/4M3}{http://learnche.mcmaster.ca/4M3}\\ \vspace{1cm}}

{\footnotesize Overall revision number: \hgversion (\monthname~\THEYEAR)} %\twodigit{\THEDAY}-\twodigit{\THEMONTH}-
}

\global\let\hgversion

\ifeof18
	\newcommand{\hgversion}{Unknown version}
\else
    % Windows users: download http://en.wikipedia.org/wiki/UnxUtils to get "grep" and "cut"
	\immediate\write18{hg log -l1|grep changeset| awk '{print $2}' | cut -f1 -d":" > hg_version.txt}
	\immediate\openin 10=hg_version.txt
	\immediate\read 10 to \localline
	\global\let\hgversion\localline
	\closein10\relax
	% http://stackoverflow.com/questions/2671079/how-can-i-save-shell-output-to-a-variable-in-latex
	% http://stackoverflow.com/questions/2485651/print-current-mercurial-revision-hash
	% http://tex.stackexchange.com/questions/16790/write18-capturing-shell-script-output-as-command-variable
\fi

\begin{document}
	
\begin{frame} \titlepage \end{frame}

\begin{frame}\frametitle{Copyright, sharing, and attribution notice}

	{\footnotesize This work is licensed under the Creative Commons Attribution-ShareAlike 3.0 Unported License. To view a copy of this license, 
	please visit \href{http://creativecommons.org/licenses/by-sa/3.0/}{http://creativecommons.org/licenses/by-sa/3.0/}}
	\vspace{-1.0cm}
	\begin{flushright}
		\includegraphics[width=0.2\textwidth]{\imagedir/common/creative-commons-by-sa.png}
	\end{flushright}	
	\vspace{-0.2cm}
	\begin{exampleblock}{}
		{\small This license allows you: }
		\begin{itemize}
			\item	{\color{myGreen}{\textbf{to share}}} - to copy, distribute and transmit the work
			\item	{\color{myOrange}{\textbf{to adapt}}} - but you must distribute the new result under the same or similar license to this one
			\item	{\color{myRed}{\textbf{commercialize}}} - you \underline{\emph{are allowed}} to use this work for commercial purposes 
			\item	{\color{blue}{\textbf{attribution}}} - but you must attribute the work as follows:
			\begin{itemize}
				\item	``Portions of this work are the copyright of Kevin Dunn'', \emph{or}
				\item	``This work is the copyright of Kevin Dunn'' \\{\tiny (when used without modification)}
			\end{itemize} 
		\end{itemize}
	\end{exampleblock}
\end{frame}

\begin{frame}\frametitle{}
	We appreciate:
	\begin{itemize}
		%\addtolength{\itemsep}{-0.2\baselineskip}
		\item	if you let us know about {\color{myRed}{\textbf{any errors}}} in the slides
		\item	{\color{myGreen}{\textbf{any suggestions to improve the notes}}}
		%\item	emailing us to ask about different licensing terms
	\end{itemize}
	\vskip24pt
	
	\begin{exampleblock}{}
		All of the above can be done by writing to
		\begin{center}
			{\Large \tt kevin.dunn@mcmaster.ca}
		\end{center}
		or anonymous messages can be sent to Kevin Dunn at
		\begin{center}
			{\Large \href{http://learnche.mcmaster.ca/feedback-questions}{http://learnche.mcmaster.ca/feedback-questions}}
		\end{center}
	\end{exampleblock}
	\vskip12pt
	{\scriptsize If reporting errors/updates, please quote the current revision number: \textbf{\tt \hgversion}}
\end{frame}

\input{\jobname}

\end{document}

